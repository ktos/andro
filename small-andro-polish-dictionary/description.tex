\section[Ogólne uwagi]{Ogólne uwagi o języku androyasańskim}

\begin{spacing}{1.1}
Język androyasański może być zaliczony w~dużej mierze do języków 
izolujących i~analitycznych, ze pewną formą aglutynacji w~słowotwórstwie.

\subsection{Fonetyka, alfabet i~transkrypcja}

Występują następujące samogłoski: a, ɛ, i, ɔ, u, ʏ, oraz 18 spółgłosek:

\begin{table}[ht]
\centering
\caption{Spółgłoski w~języku androyasańskim}
\begin{tabular}{lcccc}\toprule
                  & Wargowe & Przedniojęzykowe & Podniebienne & Tylnojęzykowe \\\midrule
Nosowe            & m       & n                &              & ŋ             \\\midrule
Zwarte            & p b     & t d              &              & k g           \\\midrule
Szczelinowe       & f v     & s z~ʐ            &              & x             \\\midrule
Półotwarte        &         &                  & j            &               \\\midrule
Drżące            &         & r                &              &               \\\midrule
Boczne półotwarte &         & l                &              &               \\\bottomrule
\end{tabular}
\label{tab:consonants}
\end{table}

Oprócz spółgłosek zaprezentowanych w~tabeli powyżej, występuje 
również zwarto-szczelinowe \xm{ʈ͡ʂ} oraz \xm{w}.

Występuje szereg dyftongów: \xm{ɛi}, \xm{ɛɔ}, \xm{ɔa}, \xm{ia}, \xm{iɔ},
\xm{uɛ}, \xm{au}, \xm{ai}, \xm{aɛ}, \xm{uɔ}, \xm{ɔʏ}, \xm{ɔi}, ale pomimo
nagromadzenia fonemów w~stylu \xm{ri}, \xm{hi}, czy \xm{pi} palatalizacja
praktycznie nie występuje (a przynajmniej nie powinna).

W poszczególnych dialektach możliwe jest jednak pojawianie się innych głosek
oraz innych zjawisk fonetycznych -- są to miedzy innymi głoski \xt{ʁ} czy też
przydech \xt{ɡʱ} lub \xt{kʱ}.

Głoski \xt{ʐ} oraz \xt{x} mogą być również wymawiane jako \xt{ʒ} oraz \xt{h},
odpowiednio. Jest to traktowane normalnie w dialektach i~jako błąd w~\emph{ardo
andro}. W dialekcie południowym użytkownicy często wymawiają głoskę \xt{ɨ}
zamiast \xt{ʏ}.

\subsubsection{Fonotaktyka}

Z uwagi na historię języka, pełny zestaw reguł tworzenia sylaby jest dość
skomplikowany:

\begin{itemize}
	\item (C)V(C1), gdzie C1 to: \xm{d}, \xm{j}, \xm{k}, \xm{l}, \xm{m}, \xm{n}, \xm{r}, \xm{s} lub \xm{t} (\xm{a}, \xm{ba}, \xm{bad}),
	\item (C)V(V)(C1), gdzie jedyne dopuszczalne dyftongi to: \xm{ɛa}, \xm{ɛi}, \xm{ɛɔ}, \xm{ɔa}, \xm{ɔɛ}, \xm{ɔi}, \xm{uɛ}, \xm{au}, \xm{aɛ}, \xm{uɔ}, \xm{ua}, \xm{ui}, \xm{ia}, \xm{iɔ}, \xm{aɔ} oraz \xm{ai} (\xm{ɛa}, \xm{ɛad}, \xm{bɛad}),
	\item CRV(C1), gdzie R to wyłącznie \xm{r}, \xm{j} lub \xm{l} (\xm{bra}, \xm{brad}) ale klaster CR nie może mieć postaci \xm{rr}, \xm{ll}, \xm{jj}; natomiast V oznacza wszystkie samogłoski poza \xm{ʏ},
	\item (C)Vŋt, (\xm{aŋt}, \xm{baŋt}),
	\item (C)Vrn, (\xm{arn}, \xm{barn}),
	\item stV(C1), (\xm{sta}, \xm{stad}).
\end{itemize}

\subsubsection{Transkrypcja}

Jak wspomniano wcześniej, w tym słowniku wykorzystywana jest transkrypcja Ziri. 
Zapis z~wykorzystaniem transkrypcji Ziri odbywa się w~sposób zaprezentowany w~
tabelach poniżej:

\begin{table}[ht]
	\centering
	\caption{Fonemy}
	\begin{tabular}{ll} \toprule
		Samogłoski & a, ɛ, i, ɔ, u, ʏ \\
		Spółgłoski & m, n, ŋ, p, b, t, d, k, g, s, z, ʐ, x, j, r, l, w, v, f, ʈ͡ʂ \\\bottomrule
	\end{tabular}
	\label{tab:phonemes}
\end{table}

\begin{table}[ht]
\centering
\caption{Znaki odpowiadające fonemom}
\begin{tabular}{ll} \toprule
	Samogłoski & a~e i~o u yi \\
	Spółgłoski & m, n, n, p, b, t, d, k, g, s, z, j, h, y, r, l, w, v, f, ch \\\bottomrule
\end{tabular}
\label{tab:chars}
\end{table}

W transkrypcji Ziri zapis \xo{-nt} określa głoski \xm{-ŋt}, jako że nosowe
\xm{n} nie występuje praktycznie w~innym kontekście. Do zapisu przydechu używa
się dwóch znaków -- \xm{ɡʱ} staje się \xo{gh}. Głoska \xm{ʁ} zapisywana jest
identycznie jak \xm{r}, czyli \xo{r}.

\note{Jak wspomniano wcześniej, głoski \xm{ɡʱ} oraz \xm{ʁ} nie występują
w~\emph{ardo andro}, jedynie w~niektórych dialektach.}\skipline

Zapis \xo{aa}, \xo{uu} i~\xo{oo} nie jest powtórzeniem -- najczęściej jest to
rozziew (hiat), kiedy dwie samogłoski występują w~oddzielnych sylabach (albo
stanowią oddzielne sylaby) i~powinny być czytane oddzielnie, np. \emph{baán}
\xm{ba.ˈan} (czas). Zazwyczaj w~sytuacji potencjalnego rozziewu występuje też
zmiana akcentu. Szczególnym przypadkiem jest \xo{ii}, który jest wydłużeniem --
\xm{i:} -- ale występuje tylko w nielicznych słowach pochodzenia obcego.
W~dialekcie południowym zdarza się, że użytkownicy nie traktują \xo{aa} i
podobnych jako rozziewu, a jako wydłużenie samogłoski (\xm{a:}) i jest to błąd w
\emph{ardo andro}, traktowane normalnie w dialekcie.

Dyftongi są prezentowane za pomocą makronu nad oboma elementami dyftongu, np.
\xm{ɛɔ} jest zapisywane jako \xo{e͞o}. Akcent na pierwszą sylabę nie jest
oznaczany, akcent na inne sylaby jest realizowany akutem nad pierwszym znakiem
sylaby akcentowanej.

O ile w transkrypcji będziemy się posługiwać polskimi znakami przestankowymi
oraz koncepcją wielkich liter na początku zdania i~przy nazwach własnych, to nie
jest to stosowane w~tradycyjnym alfabecie fonetycznym \emph{chiwo}, który nie
rozróżnia wielkości znaków.

\subsection{Akcent wewnątrz słów i~zdań}

Akcent kładziony jest zazwyczaj na pierwszą sylabę, najczęstszym wyjątkiem jest
fakt, że słowo składa się z~przedrostka -- np. słowo \emph{uf́riti}
\xm{u.ˈfri.ti} (gasić) jest w~gruncie rzeczy słowem \emph{friti} \xm{ˈfri.ti}
(świecić, palić) z przedrostkiem \xo{u-}. Drugi wyjątek to słowa zapożyczone
z~innych języków.

Nie występuje praktycznie akcent zdaniowy.

\subsection{Rzeczowniki}
Rzeczownik odmienia się przez rodzaje (męski, żeński -- nie zawsze występuje,
istnieją rzeczowniki tylko i~wyłącznie w~rodzaju żeńskim lub tylko i~wyłącznie
w~rodzaju męskim) oraz liczby (pojedyncza, mnoga -- nie zawsze występuje,
istnieją słowa tylko w~liczbie pojedynczej lub tylko mnogiej). Końcówka liczby
mnogiej to zazwyczaj \xo{-s}, \xo{-os}, \xo{-ji} lub \xo{-jis}, końcówka rodzaju
żeńskiego to zazwyczaj \xo{-a}.

\note{Poza rodzajem męskim i żeńskim w bardzo rzadkich przypadkach konieczne
jest rozróżnienie pomiędzy obiektami ożywionymi (\An{}) i nieożywionymi
(\Nan{}). Do rzeczowników oznaczonych jako ożywione zalicza się: ludzi, rośliny
oraz zwierzęta.}

Rzeczowniki odczasownikowe tworzone są zazwyczaj od formy czasu przeszłego
(\emph{kanti} -- śpiewać, \emph{kant} -- śpiew). Rzeczowniki określające osoby
(zawody) najczęściej mają końcówkę \xo{-er} (\emph{kanter} -- śpiewak,
\emph{rufaler} -- rolnik, \emph{suier} -- marynarz).

\glossex{noḱarler}{no-ḱarl-er}{NEG-death-man}{,,nieśmiertelny człowiek''}

\subsubsection{Partykuła \emph{ja}}

Nie są stosowane rodzajniki określone lub nieokreślone, ale istnieje możliwość
wskazania na konkretny obiekt za pomocą partykuły \emph{ja}.

\glossex{Ja muche esi ruko}{Ja muche esi ruk-o}{DEM cat be.PRS black-ADJ}{,,Ten kot jest czarny'' \\ ,,Ten konkretny kot jest czarny''}

Rzeczowniki nie ulegają odmianie przez przypadki, do oznaczania których używa
się głównie partykuł, patrz sekcję poświęconą szykowi zdania.

\subsection{Zaimki osobowe}

\begin{multicols}{2}

\dictwordb{mi}[ˈmi]
\dictterm{pro}{ja (\Fsg{})}

\dictwordb{ti}[ˈti]
\dictterm{pro}{ty (\Ssg{})}

\dictwordb{egi}[ˈɛg.i]
\dictterm{pro}{on, ona (\Tsg{})}
\note{Zaimek egi jest niezależny od płci. Spotykany zazwyczaj tylko w~dialekcie
Republiki Nennek.} 

\dictwordb{egli}[ˈɛg.li]
\dictterm{pro}{on (\Tsg{}.\M{})}

\dictwordb{egla}[ˈɛg.la]
\dictterm{pro}{ona (\Tsg{}.\F{})}

\dictwordb{che}[ˈʈ͡ʂɛ]
\dictterm{pro}{to (\Tsg{}.\Nan{}}
\note{Zaimek che stosowany jest wyłącznie do obiektów nieożywionych.}

\dictwordb{epié}[ɛ.pi.ˈɛ]
\dictterm{pro}{pan (\Ssg{}.\For{}.\M{} lub \Tsg{}.\For{}.\M{})} 

\note{Zaimki epié oraz epiá są bardzo formalne. Używa się ich tylko w~języku
formalnym, w~odniesieniu do obcych osób albo osób stojących wyżej w~hierarchii,
albo w~sytuacji, kiedy nie znamy imienia osoby, do której chcemy się zwrócić.
Mogą być używane zarówno jako zaimki drugiej, jak i trzeciej osoby.}
\note{Uwaga: zaimków tych nie stosuje się w~powszechnej mowie w~Republice
Nennek.}

\dictwordb{epiá}[ɛ.pi.ˈa]
\dictterm{pro}{pani (\Ssg{}.\For{}.\F{} lub \Tsg{}.\For{}.\F{})}

\dictwordb{noni}[ˈnɔ.ni]
\dictterm{pro}{my (\Fpl{})}

\dictwordb{nodi}[ˈnɔ.di]
\dictterm{pro}{my ekskluzywne}
\note{Oznacza ,,my, ale nie włącznie z~tobą''. Spotykany wyłącznie w dialekcie
pustynnym.}

\dictwordb{toi}[ˈtɔ.i]
\dictterm{pro}{wy (\Spl{})}

\dictwordb{ego͞i}[ˈɛg.ɔi]
\dictterm{pro}{oni, one (\Tpl{})}

\dictwordb{cheí}[ʈ͡ʂɛ.ˈi]
\dictterm{pro}{te rzeczy (\Tpl{}.\Nan{})}

\end{multicols}

Oprócz zaimków osobowych istnieją również zaimki dzierżawcze, określające
posiadanie.

\note{Dawniej w \emph{ardo andro} używało się tej formy zaimka także do
określenia dopełniacza, co jest pozostałością odmiany przez przypadki w językach
starożytnych. Zwyczaj ten pozostał obecnie tylko w bardzo formalnym języku.}.

\begin{multicols}{2}

\dictwordb{myi}[ˈmʏ]
\dictterm{pro}{mój (\Fsg{}.\Poss{}), mnie (\Fsg{}.\Gen{})}

\dictwordb{tyi}[ˈtʏ]
\dictterm{pro}{twój (\Ssg{}.\Poss{}), ciebie (\Ssg{}.\Poss{})}

\note{Zaimki myi oraz tyi zazwyczaj zapisywane są w~postaci ideogramu, czasami
nawet w~tekście w~zapisie fonetycznym \emph{chiwo}.}

\dictwordb{il}[ˈil]
\dictterm{pro}{jego, jej (\Tsg{}.\Poss{} oraz \Tsg{}.\Gen{})}

\dictwordb{chyi}[ˈʈ͡ʂʏ]
\dictterm{pro}{tej rzeczy (\Tsg{}.\Nan{}.\Poss{})}

\dictwordb{epil}[ˈɛ.pil]
\dictterm{pro}{pana, pani, pański (\Ssg{}.\For{}.\Poss{} lub \Tsg{}.\For{}.\Poss{})}

\dictwordb{niyi}[ˈni.ʏ]
\dictterm{pro}{nasze (\Fpl{}.\Poss{}), nas (\Fpl{}.\Gen{})}

\dictwordb{nodyi}[ˈnɔ.dʏ]
\dictterm{pro}{nasze (ekskluzywne)}

\dictwordb{tyoi}[ˈtjɔ.i]
\dictterm{pro}{wasze (\Spl{}.\Poss{}), was (\Spl{}.\Gen{})}

\dictwordb{egyi}[ˈɛ.gʏ]
\dictterm{pro}{ich (\Tpl{}.\Poss{} lub \Tpl{}.\Gen{})}

\dictwordb{chey}[ˈʈ͡ʂɛj]
\dictterm{pro}{tych rzeczy (\Tpl{}.\Nan{}.\Poss{})}

\end{multicols}

\noindent
Zaimek \emph{mi} może być pomijany, tj. poprawne jest zarówno: 

\glossex{Mi pazi muchi.}{mi pazi much-i}{1SG like cat-PL}{,,Ja lubię koty.''}

jak i~po prostu:

\glossex{Pazi muchi.}{pazi much-i}{like cat-PL}{,,Lubię koty.''}

To samo może dotyczyć zaimka \emph{ti}:

\glossex{Pazi muchi.}{ø pazi much-i}{2SG like cat-PL}{,,Lubisz koty.''}

Rozróżnienie obu tych zachowań następuje tylko z wykorzystaniem kontekstu
wypowiedzi.

\subsection{Szyk zdania}

Podstawowy szyk zdania to SVO (Jan kocha Marię), a~zdania podrzędnego to SOV
(Jan Marię kocha), ale w~szyku pytającym to OSV (Marię Jan kocha). Stosowany
jest szyk przymiotnik-rzeczownik, nigdy odwrotnie.

Czasami, w języku formalnym lub poetyckim, pojawia się czasownik na końcu zdania
oznajmującego, tworząc szyk SOV lub nawet OSV, jest to jednak rzadkie
i~zazwyczaj w mowie potocznej utrzymywany jest szyk SVO.

\glossex{Beúsma egi va wesazit.}{beúsma egi va wesazi-t}{legend 3SG PFV become-PST}{,,On stał się legendą.''}

\glossex{Egi beúsma va wesazit.}{egi beúsma va wesazi-t}{3SG legend PFV become-PST}{,,On stał się legendą.''}

Role gramatyczne zasadniczo wyznaczane są przez szyk zdania (\Nom{}--\Acc{}),
jednakże do określania dodatkowych funkcji gramatycznych wykorzystuje się
partykuły: przykładowo partykuła \emph{yi} wyznacza przynależność
(\emph{possesivus}, \Poss{}), natomiast partykuła \emph{chu} wyznacza
dopełniacz (\emph{genitivus}, \Gen{}) -- aczkolwiek dopełniacz może być
rozpoznany również z~kontekstu. Partykuły stosowane są przed częścią zdania do
którego się odnoszą.

\glossex{Mi pazi muchi.}{mi pazi much-i}{1SG like cat-PL}{,,Ja lubię koty.''}

Czasami do podkreślenia jakiejś frazy lub w niektórych dialektach stosuje się do
tego również partykułę \emph{chu}, jednak nie jest to zalecane w \emph{ardo
andro}, aby uniknąć konsternacji w sytuacji kiedy jest to dopełniacz
(\Gen{}).

\glossex{Mi veydi yasaji.}{mi veydi yasa-ji}{1SG see eel-PL}{,,Widzę węgorze.''}

\glossex{Mi veydi chu yasaji.}{mi veydi chu yasa-ji}{1SG see ACC eel-PL}{,,Widzę węgorze.''}

\glossex{Myi keromamerey esi dowo chu yasaji.}{myi keromamerey esi dowo chu yasa-ji}{1SG.POSS hovercraft be.PRS full GEN eel-PL}{,,Mój poduszkowiec jest pełen węgorzy.''}

\glossex{Myi keromamerey esi dowo yasaji.}{myi keromamerey esi dowo yasa-ji}{1SG.POSS hovercraft be.PRS full eel-PL}{,,Mój poduszkowiec jest pełen węgorzy.''}

Do określenia czym została wykonana czynność (narzędnika, \Ins{})
stosowana może być partykuła \emph{da}.

\glossex{Id́ak ostro da keja.}{id́a-k ostro da keja}{open-PST door INS key}{,,Otworzyłem drzwi kluczem.'' \\ ,,Otworzyłem drzwi używając klucza.''}

Natomiast partykuła \emph{in} może być stosowana jako miejscownik
(\Loc{}), na przykład:

\glossex{Mi esi in mibozor.}{,i esi in mibozor}{1SG be.PRS in.LOC school}{,,Jestem w szkole.'' \\ ,,Jestem teraz w szkole.''}

Przymiotnik znajduje się zawsze przed rzeczownikiem, który określa, natomiast
wyrażenie przysłówkowe -- po czasowniku, który określa.

\glossex{Nontriso lanji fesgai gepo.}{no-ntris-o lan-ji fesgai waril}{NEG-interesting-ADJ book-PL read.PRS long.ADV}{,,Nudne książki czytam długo.''}

\subsection{Czasownik}

Czasownik koniuguje przez czasy: istnieje bezokolicznik i~forma czasu
przeszłego. Istniała również dziś niestosowana forma czasu przyszłego. Czas,
aspekt i~tryb wyrażane są za pomocą operatorów (czasowników posiłkowych lub
partykuł) i~ewentualnej koniugacji. Operatory znajdują się przed czasownikiem
lub, w~ekstremalnie formalnym języku albo zdaniach podrzędnych, na końcu zdania.
Czasownik w~bezokoliczniku zawsze kończy się na \xo{-i}, w~formie przeszłej
zazwyczaj na \xo{-t}.

Wymowa bezokolicznika może nie przewidywać dyftongu, i~zazwyczaj go nie
przewiduje, na przykład \emph{chikai} \xm{'ʈ͡ʂi.ka.i} (śmiać się), jednak wielu
użytkowników zlewa tutaj końcówki \xo{-ai}, \xo{-ei}, \xo{-oi}.

\note{Do zasad dobrego wychowania należy poprawne wymawianie czasowników.
Niektórzy użytkownicy przestrzegają tego aż do takiego stopnia, że w~ich ustach
tworzy się coś w~stylu \xt{ˈʈ͡ʂi.ka.ʔi}.}
\skipline

Podstawowy czas i~podstawowa forma czasownika opisuje to, co jest teraz
aktualne, lub to, co jest raczej niezmienne (zazwyczaj w~odniesieniu do obiektów
nieożywionych), może być stosowana do stwierdzeń typu ,,pada deszcz'' i~innych
odnoszących się do stanu rzeczywistości -- są realizowane wtedy tylko przez sam
czasownik i~podmiot domyślny, często określają również niejawnie aspekt
niedokonany.

\glossex{Ti fesgai.}{ti fesgai}{2SG read.PRS}{,,Ty teraz czytasz.''}

\glossex{Che esi karié.}{che esi karié}{DEM.NAN be.PRS beautiful}{,,To jest piękne.'' \\ ,,To coś jest piękne.''}

\note{Należy tutaj przypomnieć, że partykuła \emph{che} może być stosowana tylko do obiektów nieożywionych.}

\glossex{Aḱame osupi.}{aḱame osupi}{rain fall.PRS.IPFV}{,,Pada deszcz.''}

\glossex{Osupi}{osupi}{fall.PRS.IPFV}{,,Pada deszcz.''}

Opis czynności regularnych obowiązuje tylko z~określeniem jednostki czasu w
postaci wyrażenia przysłówkowego:

\glossex{Mi fesgai relita.}{mi fesgai relita}{1SG read.PRS always}{,,Zawsze czytam.''}

\glossex{Mi fesgai eveni tay.}{mi fesgai eveni tay}{1SG read.PRS every day}{,,Czytam każdego dnia.''}

\glossex{Mi koóli ti relita.}{mi koóli ti relita}{1SG love.PRS 2SG always}{,,Kocham cię na zawsze.'' \\ ,,Kocham cię zawsze.''}

Czasowniki modalne (\emph{epi} -- móc, \emph{pazi} -- lubić, \emph{kiruki} --
umieć, \emph{vipini} -- zmuszać, musieć) powodują przeniesienie czasownika
odpowiadającego na koniec zdania, zaburzając nieco jego szyk. Czasownik ten
przyjmuje formę bezokolicznika, a~koniugacja i~miejsce operatorów będą dotyczyły
tylko modalnego. W~podobny sposób wygląda to w~sytuacji niektórych idiomów
czasownikowych.

\glossex{Mi kiruki feni.}{mi kiruki feni}{1SG can swim}{,,Umiem pływać.''}

\glossex{Mi vi kiruket feni, abe va tayet.}{mi vi kiruke-t feni, abe va taye-t}{1SG IPFV can-PST swim.PRS but PFV forget-PST}{,,Umiałem pływać, ale zapomniałem.''}

Czasowniki modalne uruchamiają tryb czynności regularnej i~nie odnosi się to do
tej konkretnej chwili, ale do stwierdzenia faktu (umiem czytać -- kiedyś się
nauczyłem i~od tej pory umiem i~raczej nie zapomnę -- oczywiście, można przestać
coś lubić albo móc, ale będzie to opisane odpowiednim aspektem w~przyszłości).

\glossex{Mi epi chet seysi.}{mi epi chet seysi}{1SG be.able.PRS here.LOC sit.PRS}{,,Mogę tutaj usiąść.''}

\glossex{Chet seysi mi epi?}{chet seysi mi epi}{here.LOC sit.PRS 1SG be.able.PRS}{,,Czy mogę tu usiąść?''}

\glossex{Pazi fesgai.}{pazi fesgai}{like.PRS read.PRS}{,,Lubię czytać.''}

\emph{Vi} to partykuła/operator aspektu niedokonanego. Aspekt niedokonany jest
,,domyślny'' dla czasu teraźniejszego, można czasami jednak jej używać dla
podkreślenia niezakończenia czynności.

\glossex{Ti fesgai.}{ti fesgai}{2SG read.PRS}{,,Ty czytasz.'' \\ ,,Czytasz.''}

\glossex{Ti vi fesgai.}{ti vi fesgai}{2SG IPFV read.PRS}{,,Ty czytasz. [i wciąż nie skończyłeś]''}

\glossex{Ti vi fesgat.}{ti vi fesga-t}{2SG IPFV read-PST}{,,Ty zacząłeś czytać. [w przeszłości i wciąż nie skończyłeś]''}

\glossex{Mi vi koólet egi.}{mi vi koóle-t egi}{1SG IPFV love-PST 3SG}{,,Pokochałem ją/jego w przeszłości. [i wciąż kocham]''}

\emph{Va} to operator aspektu dokonanego. Aspekt dokonany jest uznawany za
domyślny w~formie czasu przeszłego i~za bardzo nie ma sensu w~czasie
teraźniejszym.

\glossex{Ti va fesgat.}{ti va fesga-t}{2SG PFV read-PST}{,,Przeczytałeś.''}

\glossex{Va koólet egla.}{va koóle-t egla}{PFV love-PST 3SG.F}{,,Kochałem ją. [w przeszłości]''}

\glossex{Egi palimit e mi.}{egi palimi-t e mi}{3SG talk-PST and 1SG}{,,On rozmawiał ze mną.''}

Partykuły \emph{va} oraz \emph{vi} mogą być pomijane, jeżeli będą wynikały z~
kontekstu.

Czas przyszły realizowany jest przez partykułę \emph{ze}. Domyślnym aspektem
czasu przyszłego jest aspekt niedokonany, więc partykułę \emph{vi} można
pomijać.

\glossex{Ti ze fesgai.}{ti ze fesgai}{2SG FUT read}{,,Zaczniesz czytać.'' \\ ,,Będziesz czytał.''}

\glossex{Ti ze va fesgai.}{ti ze va fesgai}{2SG FUT PFV read}{,,Przeczytasz.''}

\glossex{Hetay mi ze va vibi rome.}{hetay mi ze va vibi rome\\
		today 1SG FUT PFV eat meat}{,,Dzisiaj zjem mięso.''}

\emph{Vige} to bardzo formalny operator ,,niech się stanie'' (\emph{optativus},
tryb życzący \Opt{}):

\glossex{Vige hallo͞i nome yi Ori!}{Vige hallo͞i nome yi ori}{OPT praise name POSS god}{,,Chwała imieniu Boga!'' \\ ,,Niech będzie chwalone imię Boga!''}

Natomiast \emph{vage} to również operator ,,niech się stanie'', ale wyrażający
coś bardzo konkretnego. W~odróżnieniu od \emph{vige}, który nie
oczekuje ,,fizycznego'' rezultatu. W~odniesieniu do konkretnego odbiorcy
oznacza ,,powinieneś coś zrobić'', ale jest traktowane nieco jak rozkaz,
silniej niż np. \emph{hemi}. Można rozumieć jako jedną z odmian trybu
hortatywnego (\Hort{}).

\glossex{Vage fesgai.}{vage fesgai}{HORT read.PRS}{,,Niech ktoś to przeczyta.'' \\ ,,Powinieneś to przeczytać.'' \\ ,,Przeczytaj to.''}

\glossex{Ti vage fesgai.}{ti vage fesgai}{2SG HORT read.PRS}{,,Powinieneś to przeczytać.'' \\ ,,Przeczytaj to.''}

\glossex{Ti vage fesgai ja lana.}{ti vage fesgai ja lana}{2SG HORT read.PRS DEM book}{,,Powinieneś przeczytać tę książkę.''}

\note{\emph{vage} nie oznacza automatycznie rozkazu, jednak jest stosowane jako
jego grzeczniejszy i bardziej formalny odpowiednik. W szczególności można się z
nim spotkać w szkole czy pracy, kiedy nauczyciel lub przełożony nakazują
wykonanie pewnej pracy.}
\skipline

\emph{Ge} to operator strony biernej.

\glossex{Sotak ge karlet chu polno sotak.}{sotak ge karle-t chu polno sotak}{somebody PASS kill-PST ACC different somebody}{,,Ktoś został zabity przez kogoś innego.''}

Wykorzystana tutaj może być partykuła \emph{chu} określająca dopełnienie, ale
nie jest to wymagane. W zdaniu tym również nie zastosowano partykuły \emph{va},
gdyż aspekt dokonany jest ,,domyślny''.

Partykuła \emph{do} służy jako operator trybu rozkazującego (\Imp{}). Trybu
rozkazującego należy w~miarę możliwości unikać, zazwyczaj przez użycie operatora
\emph{vage} lub \emph{hemi}, jako, że jest uznawany za niegrzeczny.

Zawsze znajduje się na końcu zdania. W~trybie rozkazującym użycie zaimka
osobowego może zostać oczywiście pominięte, zwłaszcza jeżeli zwracamy się
bezpośrednio do odbiorcy polecenia. W niektórych dialektach używany razem z
partykułą \emph{ze}.

\glossex{Toi tori dej́itos do!}{toi tori dej́it-os do}{2PL throw weapon-PL IMP}{,,Rzućcie broń!'' \\ ,,Wy rzućcie broń!''}

\glossex{Tori dej́itos do!}{tori dej́it-os do}{throw weapon-PL IMP}{,,Rzuć broń!''}

\emph{Hemi} to czasownik oznaczający ,,prosić'', jednak może zostać wykorzystany
zamiast operatora \emph{do} do określenia trybu ekshortatywnego. W podobnej roli
można również użyć czasownika \emph{ih́emi}, który jest ,,silniejszy'' i oznacza
,,błaganie''.

\glossex{Mudi mi fayse, hemi.}{mudi mi fayse hemi}{give 1SG drink please.HORT}{,,Podaj mi napój, proszę.'' \\ ,,Czy mógłbyś mi podać napój?''}

\glossex{Karla͞i no mi, ih́emi.}{karla͞i no mi ih́emi}{kill NEG 1SG beg.HORT}{,,Błagam, nie zabijaj mnie.''}

\glossex{Inra͞i ti egi karlet no, ih́emi.}{inra͞i ti egi karlet no ih́emi}{tell 2SG 3SG kill-PST NEG beg.HORT}{,,Błagam, powiedz, że go nie zabiłeś.''}

Z kolei zachęta, ,,zróbmy to'', czyli tryb hortatywny ze wskazaniem, że coś
zostanie wykonane wspólnie przez adresata i podmiot (kohortatywny, \Chr{}),
wykorzystuje partykułę \emph{heme} lub wręcz \emph{hemee}, zwłaszcza w
nieformalnej mowie.

\glossex{Zetay ze edihi heme!}{zetay ze edihi heme}{tomorrow FUT meet CHR}{,,Spotkajmy się jutro!''}

\glossex{Eni nafiye faysi hemee!}{eni nafiye faysi hemee}{go beer drink CHR}{,,Chodźmy na piwo!'' \\ ,,Chodźmy się napić piwa!''}

Czysty tryb przypuszczający (warunkowy, \textsc{cond}) realizowany jest za
pomocą partykuł \emph{miam} oraz \emph{vimi}.

\glossex{Miam mi va fesgat sepo͞e vimi ti diyu fari.}{miam mi va fesga-t sepo͞e vimi ti diyu fari}{if.COND 1SG PFV read-PST early.ADV then.COND 2SG something do}{,,Gdybym wcześniej przeczytał to ty byś coś zrobił.''}

\glossex{Miam mi ze va fesgai vimi ti che fari.}{miam mi ze va fesgai vimi ti fari}{if.COND 1SG FUT PFV read then.COND 2SG DEM do}{,,Gdy przeczytam, to zrób to.''}

Można zastosować partykułę \emph{abe}, aby określić warunek przeciwny:

\glossex{Miam egli ze avi deíto vimi ti lugiti abe reki! }{miam egli ze avi deíto vimi ti lugiti abe reki}{if.COND 3SG.M FUT have weapon then.COND 2SG escape but hit}{,,Jeśli on będzie miał broń, uciekaj, a w przeciwnym razie -- atakuj.''}

Partykuła \emph{vimi} może być pomijana, o ile zastosuje się szyk zdania podrzędnego:

\glossex{Miam ze va fesgai, mi jeoza chu ti ze dari.}{Miam ze va fesgai, mi jeoza chu ti ze dari}{if.COND FUT PFV read 1SG candy ACC 2SG FUT give}{,,Jeśli to przeczytasz, to dam ci cukierka.'' \\ ,,Gdy to przeczytasz, to dam ci cukierka.''}

W odniesieniu do czynności zakończonych będzie to miało znaczenie ,,skoro-to'',
jednak alternatywnie w tej roli można również stosować partykułę \emph{imin}.

\glossex{Miam ze fesgat, jeoza chu ti.}{Miam ze fesga-t, jeoza chu ti}{if.COND PFV read-PST candy ACC 2SG}{,,Skoro przeczytałeś, cukierek dla ciebie''}

\glossex{Imin ze fesgat, jeoza baljezi.}{imin ze fesga-t, jeoza baljezi}{because PFV read-PST candy receive.PRS}{,,Skoro przeczytałeś, dostajesz teraz cukierka.''}

Sama partykuła \emph{vimi} służy, w szyku pytającym, do zadawania pytań ,,czy
nie powinniśmy czegoś zrobić?''.

\glossex{Enitya vimi fesgai alive ?}{enitya vimi fesgai alive}{instruction COND read before.ADV}{,,Nie powinieneś wcześniej przeczytać instrukcji?''}

Przeczenia, negacja czynności realizowana jest z~wykorzystaniem partykuły
\emph{no}, która znajduje się \textbf{po} czasowniku. Możesz spotkać teksty
w~których partykuła \emph{no} znajduje się przed czasownikiem, ale są one zawsze
oznaką niedouczenia autora.

Kwestia podwójnego zaprzeczenia: w Andro poprawne są obydwie formy, tj. poprawne
jest zarówno \emph{ze moĺi karli}, jak i \emph{ze moĺi no karli} (,,nigdy nie
umrę'').

\glossex{Ze karli moĺi.}{ze karli moĺi}{FUT die never.ADV}{,,Nigdy nie umrę!''}

\glossex{Ze karli no moĺi.}{ze karli no moĺi}{FUT die NEG never.ADV}{,,Nigdy nie umrę!''}

\note{W niektórych dialektach partykuła \emph{no} może pojawiać się zawsze na 
końcu zdania.}

\glossex{Ti bugi no.}{ti bugi no}{2SG lie NEG}{,,Ty nie leżysz.''}

\glossex{Tori no dej́itos do!}{tori no dej́it-os do}{throw NEG weapon-PL IMP}{,,Nie rzucaj broni!''}

\glossex{Egi karlet no il rige.}{egi karle-t no il rige}{3SG kill-PST NEG 3SG.POSS lord}{,,On nie zabił swojego władcy.''}

Szyk pytający (OSV):

\examples{Il maŕie͞o egi koóli?}{Czy on kocha swoją żonę?}

\glossex{Il maŕie͞o egi koóli?}{il maŕie͞-o egi koóli}{3SG.POSS spouse-M 3SG love}{,,Czy on/ona kocha swojego męża?''}

\note{\emph{egi} nie wskazuje na płeć, jednak \emph{marié͞o} jest w rodzaju męskim.}

\glossex{Il hima egi esi?}{il hima egi esi}{3SG.POSS girlfriend 3SG be.PRS}{,,Czy ona to jego dziewczyna?''}

\glossex{Sotak epié karlet?}{sotak epié karle-t}{somebody 2SG.FOR.M kill-PST}{,,Czy zabił Pan kogoś?''}

\glossex{Je rujaler epié ze va karla͞i?}{Je rujaler epié ze va karla͞i}{DEM man 2SG.FOR.M FUT PFV kill}{,,Czy zabije Pan tego człowieka?''}

\glossex{Mi epié ze vi karla͞i?}{mi epié ze vi karla͞i}{1SG 2SG.FOR.M FUT IPFV kill}{,,Czy spróbuje Pan mnie zabić?''}

\glossex{Il alye Epiá veyt hetay?}{il alye epiá vey-t hetay}{3SG.POSS friend 2SG.FOR.F see-PST today}{,,Czy widziała Pani dzisiaj swojego przyjaciela?''}

\glossex{Ti va fesgat?}{ti va fesga-t}{2SG PFV read-PST}{,,Przeczytałeś?''}

Szyk pytający z~czasownikiem modalnym zachowuje konwencję czasownika
niemodalnego na końcu zdania.

\glossex{Chet mi epi seysi?}{chet mi epi seysi}{here 1SG be.able sit}{,,Czy mogę tutaj usiąść?''}

\glossex{Chet mi va epit seysi?}{chet mi va epi-t seysi}{here 1SG PFV be.able-PST sit}{,,Czy mogłem tutaj siąść?''}

Szyk pytający jest również uruchamiany automatycznie przez partykuły pytające -- 
\emph{chyi} (czyj), \emph{ko͞e} (jak), \emph{osor} (dlaczego), \emph{so} (co), 
\emph{somar} (gdzie), \emph{soter} (kto), \emph{voli} (kiedy), \emph{wodo} 
(którędy), \emph{yage} (dokąd), \emph{yase} (skąd).

\glossex{Yasu ti iéni?}{yasu ti i-éni}{from.where.Q 2SG VEN-go}{,,Skąd pochodzisz?''}

\glossex{Yasu ti iént?}{yasu ti i-én-t}{from.where.Q 2SG VEN-go-PST}{,,Skąd przyszedłeś?''}

\glossex{Ko͞e tyi ager nomi?}{ko͞e tyi ager nomi}{how.Q 2SG.POSS country to.name}{,,Jak nazywa się twój kraj?''}

\glossex{Ko͞e ti seiti?}{ko͞e ti seiti}{how.Q 2SG feel}{,,Jak się czujesz?''}

\glossex{So tyi vahuryiáysi o ti famei?}{so tyi vahuryiáysi o ti famei}{what.Q 2SG.POSS tatoo for 2SG mean}{,,Co znaczy dla ciebie twój tatuaż?''}

\glossex{Wodo o Polska mi ze cheri?}{wodo o Polska mi ze cheri}{which.way.Q to Poland 1SG FUT cheri}{,,Którędy mam jechać do Polski?''}

Idiomy, takie jak na przykład \emph{kipeni a} (wzorować się na) powodują
przestawienie drugiego elementu (najczęściej partykuły) bliżej dopełnienia, np.:

\glossex{Mi kipeni a myi patal.}{mi kipeni a myi patal}{1SG model on 1SG.POSS father.DIM}{,,Wzoruję się na moim tacie.''}

\glossex{A tyi patal ti kipeni?}{a tyi patal ti kipeni}{on 2SG.POSS father.DIM 2SG model}{,,Wzorujesz się na swoim tacie?''}

Do pytań w~stylu ,,czy'' można również stosować partykułę wzmacniającą
\emph{vay}:

\glossex{Vay a tyi patal ti kipeni?}{vay a tyi patal ti kipeni}{really.Q on 2SG.POSS father.DIM 2SG model}{,,Naprawdę wzorujesz się na swoim tacie?''}

Zdania podrzędne tworzone są automatycznie przez partykuły takie jak \emph{ko͞e} 
(jak), \emph{voli} (kiedy), \emph{imin} (ponieważ), \emph{abe} (ale), 
\emph{chua} (która), \emph{pama} (wtedy), \emph{per} (żeby), a~także \emph{e} 
(i, oraz). Szyk zdania podrzędnego to SOV (Jan Marię kocha).

\glossex{Mi eni o jan, per mi vibo vibi.}{eni o jan per mi vibo vibi}{1SG go to.LOC home so 1SG food eat}{,,Idę do domu aby zjeść jedzenie.''}

\glossex{Cherlok Holmes va choint sosbet kayetor esi no, e egi o jan eni permet.}{cherlok holmes va choin-t sosbet kayetor esi no e o jan eni perme-t}{Sherlock Holmes PFV state-PST suspect murderer be NEG and to home go allow-PST}{,,Sherlock Holmes uznał, że podejrzany nie jest przestępcą i pozwolił mu iść do domu.''}

\glossex{Cherlok Holmes va choint sosbet kayetor esi no, abe egi va est - abe deíto va kopet lipe.}{cherlok holmes va choin-t sosbet kayetor esi no abe egi va est abe deíto va lope-t lipe}{Sherlock Holmes PFV state-PST suspect murderer be NEG but 3SG PFV be but weapon PFV hide-PST well}{,,Sherlock Holmes uznał, że podejrzany nie jest przestępcą, jednak on był, tylko dobrze ukrył broń.''}

Fakt, że ,,egi'' odnosi się do podejrzanego wynika z kontekstu zdania.

Czasownik w~zdaniach podrzędnych może być pomijany, jeżeli wynika z~kontekstu.
Czasownik \emph{esi} (być) może również być pomijany, stąd poprawne są zdania
takie jak:

\glossex{Ari esi wofo, koy vipetode yi muche.}{ari esi wofo koy vipetode yi muche}{sky be.PRS color like bowl POSS cat}{,,Niebo ma kolor taki jak kocia miska.''}

\glossex{Ari arso, koy vipetode yi muche.}{ari arso koy vipetode yi muche}{sky skyblue.ADJ like bowl POSS cat}{,,Niebo ma niebieski kolor, tak jak kocia miska.''}

\glossex{Hetay ari stobo.}{hetay ari stobo}{today sky gray}{,,Niebo jest dzisiaj szare.''}

\subsubsection{Nominalizacja czasownika}

Czasownik może być przekształcony do formy rzeczownikowej poprzez wykorzystanie
partykuły \xo{na}, nominalizatora (\Nmlz{}), który bardzo często jest stosowany w
formie przyrostka.

\glossex{Chiwi ozeyo}{chiwi ozeyo}{write.PRS difficult.ADV}{,,Pisać jest trudno.'' / ,,Pisanie jest trudne.''}

\glossex{Chiwina esi ozeyo}{chiwi-na esi ozeyo}{write-NMLZ be.PRS difficult.ADJ}{,,Pisanie jest trudne.''}

\glossex{Chiwi na esi ozeyo}{chiwi-na esi ozeyo}{write NMLZ be.PRS difficult.ADJ}{,,Pisanie jest trudne.''}

\subsection{Przymiotnik oraz imiesłów przymiotnikowy}

Przymiotniki są tworzone zarówno od rzeczowników, jak i od czasowników.

Przymiotniki są stopniowane w~3 stopniach (w większości), gdzie drugi stopień
zazwyczaj ma końcówkę \xo{-e͞a}, a~trzeci stopień \xo{-e͞am}. Przymiotnik nie
ulega odmianie przez rodzaje gramatyczne ani liczby.

Istnieją przymiotniki, które powstały przez dodanie morfemu \xo{no-} przed
pewnym rdzeniem, wyrażające przeciwność, np. \emph{anper} \xm{ˈan.pɛr} (mokry)
i~\emph{nonper} \xm{ˈnɔn.pɛr} (suchy) oraz przymiotniki z~prefiksem \xo{mo-},
oznaczającym negatywne zestopniowanie, np. \emph{leder} \xm{ˈlɛ.dɛr} (oszczędny)
i~\emph{moleder} \xm{ˈmɔ.lɛ.dɛr} (skąpy).

Pozycja przymiotnika w~zdaniu ma decydujące znaczenie w~określeniu do którego
rzeczownika się odnosi.

\glossex{Mi pazi karié himji.}{mi pazi karié him-ji}{1SG like beautiful woman-PL}{,,Lubię piękne kobiety.''}

\glossex{Mi pazi karié himji e chid kahokapataji.}{mi pazi karié him-ji e chid kahokapata-ji.}{1SG like beautiful woman-PL and fast aeroplane-PL}{,,Lubię piękne kobiety i szybkie samoloty.''}

\glossex{Waril vayarji esi lipe aloser che karié sipalima.}{Waril vayar-ji esi lipe aloser che karié sipalima}{long.ADJ travel-PL be.PRS good.ADJ source GEN beautiful story}{,,Długie podróże są dobrym źródłem dla pięknej opowieści.''}

Przymiotniki mogą pełnić rolę przysłówka, odpowiadając na pytanie ,,jak'' i
określając czasowniki, jednak w odróżnieniu od stosowania ich w roli określenia
rzeczownika stosowane są \textbf{po} czasowniku w szyku.

\glossex{Ko͞e ti seiti? Lipe.}{ko͞e ti seiti lipe}{how.Q 2SG feel good.ADV}{,,Jak się czujesz? Dobrze.''}

Przymiotnik od rzeczownika-nazwy własnej tworzony jest przez przyrostek
dzierżawczy \xo{-yi} (\xt{ʏ}) lub przyrostek \xo{-o}: pierwszy oznacza
przynależność, a drugi -- cechę. Stąd na przykład \emph{zokemo}, metalowy od
\emph{zokem} (cecha odnosząca się do przedmiotu), ale na już na przykład
\emph{erokwyirome}, ,,mięso drobiowe'' to dosłownie ,,mięso (pochodzące) z
kurczaka''. Różnica ta jest bardzo delikatna.

Same nazwy własne, także w~formie przymiotnika odrzeczownikowego, są
w~transkrypcji zawsze zapisywane wielką literą, w~ alfabecie naturalnym zawsze
fonetycznie, nigdy sylabicznie.

Imiesłów przymiotnikowy tworzony jest poprzez zmianę -i na -o w~bezokoliczniku.

\glossex{Kaho beykar kahi in ari.}{kah-o beykar kahi on ari}{fly-ADJ snake fly.PRS on.LOC sky}{,,Latający wąż lata na niebie.''}

\glossex{Andro ya mosto inrat.}{andro ya most-o inrat}{andro TOP create-ADJ language}{,,Andro to stworzony język.''}

Zamiast imiesłowów można też stosować partykułę \emph{chu} lub \emph{che},
przyimek lub zaimek względny (\Rel{}):

\glossex{Pakopo rujalar kant hetay ne͞a ji͞ari͞o in Lublin.}{pakop-o rujalar kan-t hetay ne͞a ji͞ari͞o in Lublin}{worry-ADJ man sing-PST today near.physically garden in.LOC Lublin}{,,Zmartwiony mężczyzna śpiewał dzisiaj niedaleko ogrodu w Lublinie.''}

\glossex{Rujalar chu vi pakot kant hetay ne͞a ji͞ari͞o in Jechuf.}{rujalar chu vi pako-t kan-t hetay ne͞a ji͞ari͞o in Jechuf}{man REL IPFV worry-PST sing-PST today near.physically garden in Rzeszów}{,,Mężczyzna, który się martwił, śpiewał dzisiaj niedaleko ogrodu w Rzeszowie.''}

\glossex{Muche chu altur vi fart.}{muche chu altur vi far-t}{cat REL noise IPFV make-PST}{,,Kot, który hałasował.''}

\glossex{Altur faro muche.}{altur far-o muche}{noise make-ADJ cat}{,,Robiący hałas kot.''}

\glossex{Alturo muche.}{altur-o muche}{noise-ADJ cat}{,,Hałasujący kot.''}

\subsection{Liczebniki}

Liczebniki od 0 do 9 to, po kolei: \emph{wa}, \emph{a}, \emph{ka}, \emph{sa}, 
\emph{ta}, \emph{na}, \emph{cha}, \emph{ma}, \emph{ya}, \emph{ra}.

\begin{table}[ht]
	\centering
	\caption{Podstawowe liczebniki}
	\begin{tabular}{ccc} \toprule
		zero & 0 & wa \\
		jeden & 1 & a \\
		dwa & 2 & ka \\
		trzy & 3 & sa \\
		cztery & 4 & ta \\
		pięć & 5 & na \\
		sześć & 6 & cha \\
		siedem & 7 & ma \\
		osiem & 8 & ya \\
		dziewięć & 9 & ra \\
		dziesięć & 10 & awa \\\bottomrule
	\end{tabular}
	\label{tab:numerals}
\end{table}

Najprostszym sposobem tworzenia liczebnika dla liczb większych od 9 jest
ustawienie w~kolejności zapisu dziesiętnego kolejnych słów określających cyfry,
tj. 17 to \emph{ama}, 123 to \emph{akasa}, a~241 to \emph{kataa}. Dla większych
liczb było by to jednak zbyt problematyczne (np. 1000 to by było
\emph{awawawa}), dlatego można stosować przyrostek <-y>, określający liczbę
powtórzeń, w~stosunku do sylaby określającej liczbę powtórzeń, a~potem cyfrę
powtarzaną, na przykład \emph{aśaywa} -- dosłownie: jeden i~trójka zer. Stąd
milion to \emph{aćhaywa}, a~100023 to \emph{aśaywakasa}. Akcent kładziony jest
zawsze na sylabę, która zawiera przyrostek <-y>.

W sytuacji, kiedy jest kilka sylab z wrostkiem <-y>, akcent kładziony jest na
ostatnią taką sylabę.

\begin{table}[ht]
	\centering
	\caption{Liczebniki wyższe}
	\begin{tabular}{ccc} \toprule
		jedenaście & 11 & aa \\
		dwanaście & 12 & aka \\
		dwadzieścia & 20 & kawa \\
		trzydzieści & 30 & sawa \\
		czterdzieści & 40 & tawa \\
		sto & 100 & aḱaywa  \\
		tysiąc & 1000 & aśaywa \\
		milion & 1000000 & aćhaywa \\\bottomrule
	\end{tabular}
	\label{tab:numerals2}
\end{table}

\subsubsection{Partykuła \emph{jo}}

Niektórzy użytkownicy języka nie lubią formy pozycyjnej, w której wyrazy
potrafią być problematyczne, np. \emph{aaa} jako sto jedenaście. Stąd używana
jest również partykuła \emph{jo}, która oznacza ,,oraz'', i jest stosowana do
dodawania wartości wypowiedzianych liczebników.

Przykładowo, \emph{aśaywa jo kata} to tysiąc oraz dwadzieścia cztery (1024),
\emph{aḱaywa jo a} to dosłownie ,,sto oraz jeden'' (101), natomiast
\emph{aḱaywa jo awa jo a} czyli ,,sto oraz dziesięć oraz jeden'' to 111.

\subsubsection{Liczebniki porządkowe}

Przyrostek <-ti> określa liczebnik porządkowy, tj. \emph{kati} to drugi, a
\emph{sasati} -- trzydziesty trzeci.

\begin{table}[ht]
\centering
\caption{Podstawowe liczebniki porządkowe}
\begin{tabular}{cc} \toprule
	pierwszy & ati \\
	drugi & kati \\
	trzeci & sati \\
	czwarty & tati \\
	piąty & nati \\
	szósty & chati \\
	siódmy & mati \\
	ósmy & yati \\
	dziewiąty & rati \\
	dziesiąty & awati \\\bottomrule
\end{tabular}
\label{tab:numerals3}
\end{table}

\subsubsection{Liczebniki ułamkowe}

Liczebnik ułamkowy tworzy się przez infiks <-je->, na przykład \emph{ajeka} to
jedna druga, \emph{najera} -- pięć dziewiątych, a~\emph{awajerana} to 10/95.

\begin{table}[ht]
	\centering
	\caption{Ułamki}
	\begin{tabular}{cc} \toprule
		połowa & ajeka \\
		jedna trzecia & ajesa \\
		dwie trzecie & kajesa \\
		jedna czwarta & ajeta \\
		trzy czwarte & sajeta \\
		jedna siódma & ajema  \\\bottomrule
	\end{tabular}
	\label{tab:numerals4}
\end{table}

Możliwe jest również użycia słowa \emph{jerya} oznaczającego ,,przecinek'',
w~taki sposób, że:

\glossex{Egi doloze esi a jerya rasa metr.}{Egi doloze esi a jerya rasa metr\\
	3SG.POSS height be.PRS one comma 83 meter}{,,Jego wzrost to 1,83 metra.''}

\subsection{Przyimek dzierżawczy}

Przyimek (partykuła) dzierżawczy \emph{yi} określa posiadanie (\textsc{poss}) --
np. \emph{vipetode yi muche} - kocia miska (miska kota). Możliwe jest jego
połączenie z~obiektem do którego się odnosi jako przyrostka, w stylu
przymiotnika, co wymaga zmiany szyku wyrazów, np. \emph{mucheyi vipetode}.
Koncepcja ta istnieje głównie w~dialektach zachodnich, w~których są silne wpływy
aglutynacyjnego języka kaireńskiego oraz w~tzw. dialekcie pustynnym. Oprócz
tego, stosowane jest to w~sytuacji wielokrotnego zagnieżdżenia dzierżawczego, na
przykład miska matki mojego ojca \emph{vipetode yi myi arśityi natali͞a},
aczkolwiek \emph{vipetode yi natali͞a yi myi arśit} jest oczywiscie równie
poprawne.

\subsection{Luźne dodatkowe notatki}

Partykuła \emph{no} może również służyć do negacji rzeczownika, tworząc
konstrukcję ,,zamiast czegoś'':

\glossex{A femji inji no rujalaros femit}{a fem-ji inji no rujalar-os femi-t}{on branch-PL leave-PL instead.of people-PL hang-PST}{,,Ludzie wisieli na gałęziach zamiast liści.''}

\subsection{Honoryfikacja, imiona, nazwiska, formalność i~grzeczność}

W zależności od regionu możliwe jest, że użytkownicy języka będą bardzo
wyczuleni na kwestie grzecznościowe. Typowym tego typu elementem jest niechęć do
stosowania operatora trybu rozkazującego \emph{do} na rzecz \emph{hemi}, ale
bardzo często możesz również spotkać się z~honoryfikatorami służącymi do
odpowiedniego zwracania się do innych osób.

W And́royas typowym jest używanie pierwszego imienia w~odniesieniu do rozmówcy,
lub podczas określania osoby, chyba, że jest to niemożliwe do jednoznacznej
identyfikacji, wtedy używa się pełnego imienia i~nazwiska. Z drugiej jednak
strony mieszkańcy Cesarstwa są bardzo dumni ze swojej rodziny i~swojego rodowego
nazwiska, stąd przedstawiajac się często to podkreślą przedstawiając się pełnym
imieniem oraz nazwiskiem.

\glossex{Mi nomi Eryus mal Edoraril.}{mi nomi Eryus mal Edoraril}{1SG name.PRS Eryus of.family Edoraril}{,,Nazywam się Eryus mal Edoraril.''}

Warto tutaj zwrócić uwagę, że imiona i~nazwiska w~Cesarstwie są rozdzielane
partykułą \emph{mal}, oznaczającą ,,z rodziny'', np. \emph{Koolder mal
Erlehirni} to Koolder z rodziny Erlehirni. Czasami możliwe jest, że dzieci
dziedziczą nazwiska po obojgu rodziców, stąd występują nazwiska łączone
łącznikiem, takie jak \emph{Alya mal Arkai-Valor}. Istnieją również także
oznaczane za pomocą łącznika gałęzie rodów, które dziś stały się zwykłymi
nazwiskami, np. \emph{Nimu͞e mal Hetasi-Hi}, co oznacza ród Hetasi i jego gałąź
Hi.

\note{Gałęzie rodowe, i co za tym idzie, ich oznaczenia w nazwiskach pojawiały
się w~sytuacji, kiedy nazwisko rodowe przechodziło tylko na pierwsze dziecko,
natomiast kolejne dzieci uzyskiwały nazwiska z~określeniem gałęzi. Zwyczaj ten
zanikł prawie całkowicie około VII wieku po Zjednoczeniu.}
\skipline

Używa się zaimków osobowych \emph{epié} i~\emph{epiá}, które odpowiadają mniej
więcej polskim ,,pan'' i~,,pani''. Używa się ich w~odniesieniu do obcych osób
albo osób stojących wyżej w~hierarchii, albo w~sytuacji, kiedy nie znamy imienia
osoby, do której chcemy się zwrócić. Bardzo często można napotkać ich stosowanie
w postaci przyrostków z~łącznikiem, w~stosunku do imienia, np. mówiąc o kimś
wyżej w~hierarchii możemy powiedzieć \emph{Koolder-epié}. W~podobny sposób
można określać czyjąś funkcję, np. \emph{Furu-falazera} -- dowódca Furu. Czasami
można napotkać formę \emph{falazera-epiá} -- pani dowódca. W~taki sposób można
używać słów takich jak \emph{falazer} (dowódca), \emph{kachister} (nauczyciel),
\emph{meneder} (lekarz) i~innych.

\note{\emph{epié} i~\emph{epiá} praktycznie nie są używane w Republice Nennek,
gdzie preferowane jest zwracanie się imieniem, zaimkiem ogólnym \emph{egi} lub
zależnymi od płci \emph{egli/egla} lub ewentualnie przyrostkiem \emph{-gam},
jeżeli naprawdę chce się podkreślić swoją niższą pozycję wobec rozmówcy.}

\note{W momencie gdy dwoje rozmówców będzie traktować się nawzajem z identycznym
poziomem grzeczności, będą się do siebie zwracać nawzajem ukazując swoją niższą
pozycję, np. nawzajem tytułować siebie z przyrostkiem \emph{-epié}.}
\skipline

Z drugiej strony, możliwe jest, że rozmówca będzie traktować drugą osobę jako
osobę od niego niższą statusem, ukazując swoją wyższą pozycję. Jest to
oczywiście niegrzeczne i stąd bardzo rzadko spotykane. Przyrostkami takimi mogą
być rzeczownik \emph{pezawe} (,,gorszy człowiek'') lub wręcz rzeczownik
\emph{zam} (dosłownie ,,śmieć''). Bardzo pogardliwe, spotkane w sytuacji i
próbach zastraszenia rozmówcy.

\subsubsection{Zdrobnienia i poufałość}

Oczywiście, w codziennych sytuacjach osoby sobie bliskie nie będą używały
określeń stricte formalnych -- do babci raczej wnukowie zwrócą się
\emph{chancha} niż \emph{gruchana}, jeśli są z nią blisko, a do swoich rodziców
\emph{mama} i \emph{patal} bardziej niż \emph{natali͞a} oraz \emph{vapal}.

W podobny sposób stosowane są często zdrobienia i partykuły lub rzeczowniki
z~nimi związane, takie jak \emph{myi}, który może być stosowany jako przyrostek
zdrabniający (\textsc{dim}), np. \emph{pelir-myi} -- ,,mój pieseczek'', czy też
\emph{koól}, stosowany np. \emph{Alya-koóla} -- ,,kochana Alya'', stosowany w
odniesieniu do osoby darzonej uczuciem.

Istnieje również słabsza wersja \emph{koól}, \emph{arey}, przyimek stosowany do
określenia sympatii do drugiej osoby. Może być również stosowany do określenia
sympatii do rzeczy, bez określenia jej posiadania, w przeciwieństwie do
\emph{myi}.

\subsubsection{Rodzina cesarska}

W przypadku rodziny cesarskiej używa się określeń \emph{eyger} oraz
\emph{eygera}, np. \emph{Fayfnira-eygera} -- cesarzowa Fayfnira, ale i~takich
jak \emph{and́royasyikigje͞a} (księżniczka And́royas -- tj. siostry
cesarza/cesarzowej), czy \emph{and́royasyikigeje} (książę And́royas -- bracia
panującego).

Z kolei dzieci panującego często określane są tytułami \emph{icheryikigeje},
\emph{icheryikigeje͞a} lub \emph{icheryihima} -- dosłownie książę lub
księżniczka krwi. Oprócz tego istnieje przyrostek \emph{-hima}, stosowany często
na Wschodzie w~stosunku do całej żeńskiej strony rodu panującego, poza
Cesarzową, który z kolei na Zachodzie często jest używany zamiennie z
\emph{-hina} jako "panna" dla kobiety niezamężnej.

Małżonek panującego może posiadać tytuł zarówno równorzędny -- np. \emph{eyger},
ale i~na przykład \emph{eygeryikigeje}, dosłownie ,,książę cesarstwa'' lub
,,książę cesarzowej''. Dokładne zasady są zależne od aktualnej sytuacji
politycznej.

Stąd aktualnie (w momencie pisania tej książki), mamy:

\begin{itemize}
\item \textbf{Katia-eygera mal Arkai}\\ \xt{ˈka.ti.a ˈɛj.gɛ.ra ˈmal ˈar.ka.i},\\ 
Cesarzowa Katia mal Arkai,
\item \textbf{So'tak-eygeryikigeje mal Valor}\\ \xt{ˈsɔ|.tak ˈɛj.gɛ.rʏ.ki.gɛ.ʐɛ 
ˈmal ˈva.lɔr},\\ Ksiażę Cesarzowej, So'tak mal Valor,
\item \textbf{Alya-icheryikigeje͞a mal Arkai-Valor}\\\xt{ˈal.ja i.ʈ͡ʂe.rʏ.ki.gɛ.ʐɛa ˈmal 
ˈar.ka.i-va.lɔr},\\ Księżniczka Krwi, Alya mal Arkai-Valor,
\item \textbf{Niva-and́royasyikigeje͞a mal Arkai}\\\xt{ˈni.va an.ˈdrɔ.ja.sʏ.ki.gɛ.ʐɛa 
mal ˈar.ka.i},\\ Księżniczka And́royas, Niva mal Arkai,
\item \textbf{Karra-and́royasyikigeje͞a mal Arkai}\\\xt{ˈkar.ra an.ˈdrɔ.ja.sʏ.ki.gɛ.ʐɛa 
mal ˈar.ka.i},\\ Księżniczka And́royas, Karra mal Arkai,
\item \textbf{Jaida-and́royasyikigeje͞a mal Arkai}\\\xt{ˈʐa.i.da an.ˈdrɔ.ja.sʏ.ki.gɛ.ʐɛa
mal ˈar.ka.i},\\ Księżniczka And́royas, Jaida mal Arkai.
\end{itemize}

\note{Należy tutaj zwrócić uwagę na wymowę imienia Jej Wysokości, w~której
głoski /i/ oraz /a/ nie zlewają się w~/ia/, oraz na imię Jego Wysokości, w~
którym występuje pauza pomiędzy sylabami, oznaczana przez <'> w~transkrypcji.
Wynika to z~faktu, że Jego Wysokość pochodzi z~wyspy Rem, gdzie pojawiają
się takie, unikatowe, elementy języka, z~uwagi na wpływ języków krajów
ościennych.}

\subsubsection{Pozostała tytulatura}

Jako monarchia, Cesarstwo posiadało szereg tytułów szlacheckich (np. \emph{lir},
czy \emph{eber}) jednakże od czasów początku Trzeciego Cesarstwa zostały
wycofane z użytku. Wciąż jednak, w ekstremalnie formalnym języku można stosować
te określenia jako przyrostek funkcyjny (na wzór np.~\emph{Koolder-epié} --
\emph{Koolder-lir}), jednak nie są stosowane w drugiej osobie (\textsc{2SG}), a
zamiast tego stosuje się \emph{arḱer/arḱera} dla podkreślenia formalności.

\note{\emph{arḱer/arḱera} nie obowiązuje przy zwracaniu się do członków
	rodziny cesarskiej oraz rodzin królów i książąt, do których należy zwracać
	się zaimkiem \emph{rige/rigea} (,,władca'' / ,,władczyni'')) w mniej
	formalnych sytuacjach i tytułem w bardziej formalnych.}

\note{W przypadku osoby pełniącej funkcję \emph{ajor} również stosuje się tytuł
jako zaimek w drugiej osobie.}

Określenie \emph{arḱer} stosowane było również do zwracania się do osób
pełniących niektóre funkcje, np. burmistrza lub zarządcy miasta, często w formie
przyrostku funkcyjnego, obecnie jest to bardzo rzadkie.

Lista tytułów szlacheckich:

\begin{itemize}
	\item \emph{kyige/kyige͞a} -- król/królowa,
	\item \emph{kigeje/kigeje͞a} -- książę/księżna/księżniczka (tytuły Rodziny
	Cesarskiej)
	\item \emph{ajor/ajora} -- obecnie jest to funkcja administracyjna w
	Cesarstwie, oznaczająca ,,gubernatora'', osobę zarządzającą regionem
	administracyjnym,
	\item \emph{lir/lira} -- hrabia/hrabina,
	\item \emph{eber/ebera} -- baron/baronessa,	
	\item \emph{arḱer/arḱera} -- ogólna forma ,,lord'' (,,lady'').
\end{itemize}

\subsubsection{Zwroty honorowe}

W odróżnieniu od kultur ziemskich, w And́royas nie przyjęła się koncepcja
zwrotów honorowych, np. ,,Jego Najjaśniejsza Wysokość Książę Lichtensteinu''
raczej byłby tytułowany po prostu \emph{Kigeje yi Lihtenchuteyn}, względnie jak
każdy inny tytuł jako przyrostek, np. \emph{Henrik-lihtenchuteynyikigeje}.

Dla podkreślenia ważności osoby o której się mówi, lub do której się zwraca, w
wysoce formalnych zwrotach stosuje się przyrostek \emph{-epié} w stosunku do
funkcji, np. \emph{diosever-epié} -- ,,pan sierżant'', ,,panie sierżancie''.

\note{W odniesieniu do głów państw raczej powinno używać się formy \emph{arḱer},
np. w stosunku do królów, książąt, prezydentów czy Cesarzowej.}

Istnieje jeden zwrot honorowy stosowany do dzisiaj, \emph{ardo arḱeji} --
,,wysocy panowie'', stosowany do grupy wysoko postawionych osób.

\subsection{Dialekty}
Jak wspomniano, ten słownik skupia się na standardowej formie języka, jednak już
mogłeś zauważyć, że wielokrotnie wspominamy o dialektach. Obecnie wyróżnia się 6
głównych dialektów oprócz języka standardowego:

\begin{itemize}
    \item dialekt nennecki, wyróżnia się znacznie mniejszym poziomem
    formalności, w szczególności stosowaniem zaimka \emph{egi} zamiast
    formalnych \emph{epié/epiá}, a czasami nawet zamiast odmiennych przez
    rodzaje gramatyczne \emph{egli/egla}; w dialekcie tym raczej też nie stosuje
    się honoryfikatora \emph{-gam},
    \item dialekt zachodni (lideński, lonoński, izolański), w~którym preferowany
    jest przyrostek dzierżawczy \emph{-yi} zamiast partykuły \emph{yi}, w
    podobny sposób niektóre inne partykuły są używane jako przyrostki, np.
    partykuła czasu przyszłego \emph{ze} -- \emph{Ti ze vibi.} jest często
    zastępowane \emph{Ti vibize.}, występuje też dużo własnego regionalnego
    słownictwa,
    \item dialekt północny (ellański) wyróżniający się pojawianiem się fonemu
    \xt{ʃ} w miejsce \xt{ʐ},
    \item dialekt pustynny (dorelski), z szeregiem własnego słownictwa, użyciem
    \xt{ʁ} zamiast \xt{r} oraz pojawianiem się aspiracji -- \xt{g} oraz \xt{k}
    przed samogłoskami są wymawiane jako \xt{ɡʱ} oraz \xt{kʱ}, odpowiednio;
    jedną z ciekawostek jest również \emph{nodi}, ekskluzywne "my",
    \item dialekt południowy (nomisrański), stosujący długie samogłoski (np.
    /a:/) zamiast rozziewu ze zmianą akcentu (stąd \emph{baán} to raczej
    \xt{ba:n} niż \xt{ba.ˈan}) oraz \xt{ɨ} zamiast \xt{ʏ}, \xt{ʒ} zamiast \xt{ʐ}
    i czasem \xt{h} zamiast \xt{x},
    \item dialekt południowo-wschodni (papityjski), z szeregiem własnego
    słownictwa, oraz wymową \xt{ʐ} i \xt{x} jako \xt{ʒ} oraz \xt{h}, czasami
    także \xt{ʏ} jako \xt{ji},
	\item dialekt remański, z występowaniem pauzy \xt{|} w niektórych słowach,
	w~szczególności spotykane jest to w imionach -- patrz \emph{So'tak}
	\xt{sɔ|.tak}.
\end{itemize}

Dialekt nennecki najpopularniejszy jest w regonie Nennek, ale i pojawia się
w~Rem, Amesrze lub północnym Vagyr. Dialekt zachodni najpopularniejszy jest
w~Lono, ale i Rilli, Dorel oraz w~regionie Wysp Zachodnich. Dialekt pustynny
przede wszystkim można spotkać w~Agavie i~Istapie. Dialekt południowy to domena
południowego Vagyr, natomiast południowo-wschodni najczęściej występuje w Maddo.

\subsubsection{Mowa potoczna}

Wielokrotnie w tym słowniku spotkasz się z określeniami, że pewne określenia,
słowa lub zwroty stosowane są wyłącznie w mowie potocznej, nieformalnej.

Język andro nie używa wielu poziomów formalności, niespecjalnie też rozróżniane
są formy wyrażeń zarezerwowane dla stereotypowej mowy męskiej i~żeńskiej,
społeczeństwo And́royas już od wieków było dość egalitarne pod względem płci.

Spotykane są jednak formy, które najczęściej stosowane są w formie ustnej,
a~które są albo całkowicie nieformalne albo są składową dawniejszych procesów
językowych. Przykładem elementów nieformalnych mogą być:

\begin{itemize}
	\item stosowanie końcówki czasownika <-ee> \xt{ɛ:}, np. \emph{Foree!}
	\xt{ˈfɔ.rɛ:} -- ,,płacę!'', w~szczególności do określenia czynności
	wykonywanej niechętnie, stosowane często w stereotypie ,,twardego
	mężczyzny'',
	\item partykuła \emph{de}, jako wzmocnienie wyrażenia, np. \emph{Mi fesgai
	de!} -- ,,ja przecież czytam!''.
\end{itemize}

\end{spacing}