\section[Ogólne uwagi]{Ogólne uwagi o języku androyasańskim}

\begin{spacing}{1.1}
Język androyasański może być zaliczony w~dużej mierze do języków 
izolujących i~analitycznych, ze pewną formą aglutynacji w~słowotwórstwie.

\subsection{Fonetyka, alfabet i~transkrypcja}

Występują następujące samogłoski: a, ɛ, i, ɔ, u, ʏ, oraz 18 spółgłosek:

\begin{table}[ht]
\centering
\caption{Spółgłoski w~języku androyasańskim}
\begin{tabular}{lcccc}\toprule
                  & Wargowe & Przedniojęzykowe & Podniebienne & Tylnojęzykowe \\\midrule
Nosowe            & m       & n                &              & ŋ             \\\midrule
Zwarte            & p b     & t d              &              & k g           \\\midrule
Szczelinowe       & f v     & s z~ʐ            &              & x             \\\midrule
Półotwarte        &         &                  & j            &               \\\midrule
Drżące            &         & r                &              &               \\\midrule
Boczne półotwarte &         & l                &              &               \\\bottomrule
\end{tabular}
\label{tab:consonants}
\end{table}

Oprócz spółgłosek zaprezentowanych w~tabeli powyżej, występuje 
również zwarto-szczelinowe /t͡ʂ/ oraz /w/.

Występuje szereg dyftongów: /ɛi/, /ɛɔ/, /ɔa/, /ia/, /iɔ/, /uɛ/, /au/, /ai/, 
/aɛ/, /uɔ/, /ɔʏ/, /ɔi/, ale pomimo nagromadzenia morfemów w~stylu /ri/, /hi/, 
czy /pi/ palatalizacja praktycznie nie występuje (a przynajmniej nie powinna).

W poszczególnych dialektach możliwe jest jednak pojawianie się innych głosek
oraz innych zjawisk fonetycznych -- są to miedzy innymi głoski /ʁ/ czy też
przydech /ɡʱ/ lub /kʱ/.

Głoski [ʐ] oraz [x] mogą być również wymawiane jako [ʒ] oraz [h], odpowiednio.
Jest to traktowane normalnie w dialektach i~jako błąd w~\emph{ardo andro}. W
dialekcie południowym użytkownicy często wymawiają głoskę [ɨ] zamiast [ʏ].

\subsubsection{Fonotaktyka}

Z uwagi na historię języka, pełny zestaw reguł tworzenia sylaby jest dość
skomplikowany:

\begin{itemize}
	\item (C)V(C1), gdzie C1 to: /d/, /j/, /k/, /l/, /m/, /n/, /r/, /s/ or /t/ (a, ba, bad),
	\item (C)V(V)(C1), gdzie jedyne dopuszczalne dyftongi to: /ɛa/, /ɛi/, /ɛɔ/, /ɔa/, /ɔɛ/, /ɔi/, /uɛ/, /au/, /aɛ/, /uɔ/, /ua/, /ui/, /ia/, /iɔ/, /aɔ/ oraz /ai/ (ɛa, ɛad, bɛad),
	\item CRV(C1), gdzie R to wyłącznie /r/, /j/ lub /l/ (bra, brad) ale klaster CR nie może mieć postaci /rr/, /ll/, /jj/; natomiast V oznacza wszystkie samogłoski poza /ʏ/,
	\item (C)Vŋt, (aŋt, baŋt),
	\item (C)Vrn, (arn, barn),
	\item stV(C1), (sta, stad).
\end{itemize}

\subsubsection{Transkrypcja}

Jak wspomniano wcześniej, w tym słowniku wykorzystywana jest transkrypcja Ziri. 
Zapis z~wykorzystaniem transkrypcji Ziri odbywa się w~sposób zaprezentowany w~
tabelach poniżej:

\begin{table}[ht]
	\centering
	\caption{Fonemy}
	\begin{tabular}{ll} \toprule
		Samogłoski & a, ɛ, i, ɔ, u, ʏ \\
		Spółgłoski & m, n, ŋ, p, b, t, d, k, g, s, z, ʐ, x, j, r, l, w, v, f, t͡ʂ, \\\bottomrule
	\end{tabular}
	\label{tab:phonemes}
\end{table}

\begin{table}[ht]
\centering
\caption{Znaki odpowiadające fonemom}
\begin{tabular}{ll} \toprule
	Samogłoski & a~e i~o u yi \\
	Spółgłoski & m, n, n, p, b, t, d, k, g, s, z, j, h, y, r, l, w, v, f, ch \\\bottomrule
\end{tabular}
\label{tab:chars}
\end{table}

W transkrypcji Ziri zapis <-nt> określa głoski /-ŋt/, jako że nosowe n nie 
występuje praktycznie w~innym kontekście. Do zapisu przydechu używa się dwóch 
znaków -- /ɡʱ/ staje się <gh>. Głoska /ʁ/ zapisywana jest identycznie jak /r/, 
czyli <r>.

\note{Jak wspomniano wcześniej, głoski /ɡʱ/ oraz /ʁ/ nie występują w~\emph{ardo
andro}, jedynie w~niektórych dialektach.}\skipline

Zapis <aa>, <uu> i~<oo> nie jest powtórzeniem -- najczęściej jest to rozziew
(hiat), kiedy dwie samogłoski występują w~oddzielnych sylabach (albo stanowią
oddzielne sylaby) i~powinny być czytane oddzielnie, np. baán /ba.ˈan/ (czas).
Zazwyczaj w~sytuacji potencjalnego rozziewu występuje też zmiana akcentu.
Szczególnym przypadkiem jest <ii>, który jest wydłużeniem -- /i:/ -- ale
występuje tylko w nielicznych słowach. W~dialekcie południowym zdarza się, że
użytkownicy nie traktują <aa> i podobnych jako rozziewu, a jako wydłużenie
samogłoski (/a:/) i jest to błąd w \emph{ardo andro}, traktowane normalnie w
dialekcie.

Dyftongi są prezentowane za pomocą makronu nad oboma elementami dyftongu, np.
/ɛɔ/ jest zapisywane jako <e͞o>. Akcent na pierwszą sylabę nie jest oznaczany, 
akcent na inne sylaby jest realizowany akutem nad pierwszym znakiem sylaby 
akcentowanej.

O ile w transkrypcji będziemy się posługiwać polskimi znakami przestankowymi
oraz koncepcją wielkich liter na początku zdania i~przy nazwach własnych, to nie
jest to stosowane w~tradycyjnym alfabecie fonetycznym \emph{chiwo}, który nie
rozróżnia wielkości znaków.

\subsection{Akcent wewnątrz słów i~zdań}

Akcent kładziony jest zazwyczaj na pierwszą sylabę, najczęstszym wyjątkiem jest
fakt, że słowo składa się z~przedrostka -- np. słowo \emph{uf́riti /u.ˈfri.ti/
(gasić)} jest w~gruncie rzeczy słowem \emph{friti /ˈfri.ti/ (świecić, palić)} z
przedrostkiem /u/. Drugi wyjątek to słowa zapożyczone z~innych języków.

Nie występuje praktycznie akcent zdaniowy.

\subsection{Rzeczowniki}
Rzeczownik odmienia się przez rodzaje (męski, żeński -- nie zawsze występuje,
istnieją rzeczowniki tylko i~wyłącznie w~rodzaju żeńskim lub tylko i~wyłącznie
w~rodzaju męskim) oraz liczby (pojedyncza, mnoga -- nie zawsze występuje,
istnieją słowa tylko w~liczbie pojedynczej lub tylko mnogiej). Końcówka liczby
mnogiej to zazwyczaj <-s>, <-os>, <-ji> lub <-jis>, końcówka rodzaju żeńskiego
to zazwyczaj <-a>.

Rzeczowniki odczasownikowe tworzone są zazwyczaj od formy czasu przeszłego
(\emph{kanti} -- śpiewać, \emph{kant} -- śpiew). Rzeczowniki określające osoby
najczęściej mają końcówkę <-er> (\emph{kanter} -- śpiewak, \emph{rufaler} --
rolnik, \emph{suier} -- marynarz).

Nie istnieje pojęcia rodzajnika określonego lub nieokreślonego, ale jest
możliwość wskazania na konkretny obiekt za pomocą partykuły <ja>.

\example{Ja muche esi ruko.}{ten (konkretny) kot jest czarny}

Rzeczowniki nie ulegają odmianie przez przypadki, jednak te są wyraźnie
widoczne. Poza mianownikiem, partykuła \emph{yi} wyznacza przynależność
(\emph{possesivus}), natomiast partykuła \emph{chu} wyznacza dopełniacz --
dopełniacz może być rozpoznany z~kontekstu. Biernik wyznaczany jest zawsze
poprzez szyk zdania.

\example{Myi futomamerey esi dowo chu yasaji.}{Mój poduszkowiec jest pełen
[kogo, czego?] węgorzy.}

\subsection{Zaimki osobowe}

\begin{multicols}{2}

\dictwordb{mi}[ˈmi]
\dictterm{pro}{ja (\textsc{1sg})}

\dictwordb{ti}[ˈti]
\dictterm{pro}{ty (\textsc{2sg})}

\dictwordb{egi}[ˈɛg.i]
\dictterm{pro}{on, ona (\textsc{3sg})}
\note{Zaimek egi jest niezależny od płci. Spotykany zazwyczaj tylko w~dialekcie
Republiki Nennek.} 

\dictwordb{egli}[ˈɛg.li]
\dictterm{pro}{on (\textsc{3sg})}

\dictwordb{egla}[ˈɛg.la]
\dictterm{pro}{ona (\textsc{3sg.fem})}

\dictwordb{che}[ˈt͡ʂɛ]
\dictterm{pro}{to (\textsc{3sg})}
\note{Zaimek che stosowany jest wyłącznie do obiektów nieożywionych.}

\dictwordb{epié}[ɛ.pi.ˈɛ]
\dictterm{pro}{pan (on) (\textsc{3sg})}
\note{Zaimki epié oraz epiá są bardzo formalne. Używa się ich tylko w~języku
formalnym, w~odniesieniu do obcych osób albo osób stojących wyżej w~hierarchii,
albo w~sytuacji, kiedy nie znamy imienia osoby, do której chcemy się zwrócić.}
\note{Uwaga: zaimków tych nie stosuje się w~powszechnej mowie w~Republice
Nennek.}

\dictwordb{epiá}[ɛ.pi.ˈa]
\dictterm{pro}{pani (ona) (\textsc{3sg.fem})}

\dictwordb{noni}[ˈnɔ.ni]
\dictterm{pro}{my (\textsc{1pl})}

\dictwordb{nodi}[ˈnɔ.di]
\dictterm{pro}{my ekskluzywne}
\note{Oznacza ,,my, ale nie włącznie z~tobą''. Spotykany najczęściej tylko w
dialekcie pustynnym.}

\dictwordb{toi}[ˈtɔ.i]
\dictterm{pro}{wy (\textsc{2pl})}

\dictwordb{ego͞i}[ˈɛg.ɔi]
\dictterm{pro}{oni, one (\textsc{3pl})}

\dictwordb{cheí}[t͡ʂɛ.ˈi]
\dictterm{pro}{te rzeczy (\textsc{3pl})}

\end{multicols}

Oprócz zaimków osobowych istnieją również zaimki dzierżawcze, określające
posiadanie, jak również dopełniacz zdania w klasycznym \emph{ardo andro}.

\note{W klasycznym \emph{ardo andro} używa się tej formy zaimka do określenia
dopełniacza, co jest pozostałością odmiany przez przypadki w językach
starożytnych. W większości dialektów nie jest błędem użycie tutaj podstawowej
formy zaimka.}.

\begin{multicols}{2}

\dictwordb{myi}[ˈmʏ]
\dictterm{pro}{mój (\textsc{1sg.poss}), mnie (\textsc{1sg.gen})}

\dictwordb{tyi}[ˈtʏ]
\dictterm{pro}{twój (\textsc{2sg.poss}), ciebie (\textsc{2sg.gen})}

\note{Zaimki myi oraz tyi zazwyczaj zapisywane są w~postaci ideogramu, czasami
nawet w~tekście w~zapisie fonetycznym \emph{chiwo}.}

\dictwordb{il}[ˈil]
\dictterm{pro}{jego, jej (\textsc{3sg.poss} oraz \textsc{3sg.gen})}

\dictwordb{chyi}[ˈt͡ʂʏ]
\dictterm{pro}{tej rzeczy}

\dictwordb{epil}[ˈɛ.pil]
\dictterm{pro}{pana, pani, pański}

\dictwordb{niyi}[ˈni.ʏ]
\dictterm{pro}{nasze (\textsc{1pl.poss}), nas (\textsc{1pl.gen})}

\dictwordb{nodyi}[ˈnɔ.dʏ]
\dictterm{pro}{nasze (ekskluzywne)}

\dictwordb{tyoi}[ˈtjɔ.i]
\dictterm{pro}{wasze (\textsc{2pl.poss}), was (\textsc{2pl.gen})}

\dictwordb{egyi}[ˈɛ.gʏ]
\dictterm{pro}{ich (\textsc{3sg.poss} lub \textsc{3sg.gen})}

\dictwordb{chey}[ˈt͡ʂɛj]
\dictterm{pro}{tych rzeczy}

\end{multicols}

\noindent
Zaimek \emph{mi} może być pomijany, tj. poprawne jest zarówno: 

\example{Mi pazi muchi.}{lubię koty,}

\noindent
jak i~po prostu

\example{Pazi muchi.}{lubię koty.}

\subsection{Szyk zdania}

Podstawowy szyk zdania to SVO (Jan kocha Marię), a~zdania podrzędnego to SOV
(Jan Marię kocha), ale w~szyku pytającym to OSV (Marię Jan kocha). Szyk postaci
element definiujący-element definiowany jest obowiązkowy.

Zdania podrzędne mają szyk SOV.

Czasami, w języku formalnym lub bardziej poetyckim, istnieje zmiana szyku zdania
oznajmującego na SOV, dla podkreślenia sytuacji, np.:

\example{Beúsma egi va wesazit.}{On stał się legendą.}

\subsection{Czasownik}

Czasownik koniuguje przez czasy: istnieje bezokolicznik i~forma czasu
przeszłego. Istniała również dziś niestosowana forma czasu przyszłego. Czas,
aspekt i~tryb wyrażane są za pomocą operatorów (czasowników posiłkowych lub
partykuł) i~ewentualnej koniugacji. Operatory znajdują się przed czasownikiem
lub, w~ekstremalnie formalnym języku albo zdaniach podrzędnych, na końcu zdania.
Czasownik w~bezokoliczniku zawsze kończy się na <-i>, w~formie przeszłej
zazwyczaj na <-t>.

Wymowa bezokolicznika może nie przewidywać dyftongu, i~zazwyczaj go nie
przewiduje, na przykład \emph{chikai /'t͡ʂi.ka.i/ (śmiać się)}, jednak wielu
użytkowników zlewa tutaj końcówki <-ai>, <-ei>, <-oi>.

\note{Do zasad dobrego wychowania należy poprawne wymawianie czasowników.
Niektórzy użytkownicy przestrzegają tego aż do takiego stopnia, że w~ich ustach
tworzy się coś w~stylu /ˈt͡ʂi.ka.ʔi/.}
\skipline

Podstawowy czas i~podstawowa forma czasownika opisuje to, co jest teraz
aktualne, lub to, co jest raczej niezmienne (zazwyczaj w~odniesieniu do obiektów
nieożywionych).

\examples{Ti fesgai.}{Ty czytasz.}
\example{Che esi karié.}{To [coś] jest piękne.}

Opis czynności regularnych obowiązuje tylko z~określeniem jednostki czasu lub
czasownikiem modalnym:

\examples{Mi fesgai relita.}{Zawsze czytam.}

\examples{Mi fesgai eveni tay.}{Czytam każdego dnia.}

\example{Mi koóli ti relita.}{Kocham cię na zawsze.}

Czasowniki modalne (\emph{epi} -- móc, \emph{pazi} -- lubić, \emph{kiruki} --
umieć, \emph{vipini} -- zmuszać, musieć) powodują przeniesienie czasownika
odpowiadającego na koniec zdania, zaburzając nieco jego szyk. Czasownik ten
przyjmuje formę bezokolicznika, a~koniugacja i~miejsce operatorów będą dotyczyły
tylko modalnego. W~podobny sposób wygląda to w~sytuacji niektórych idiomów
czasownikowych.

\examples{Mi kiruki feni.}{Umiem pływać.}

\example{Mi vi kiruket feni, abe va tayet.}{Umiałem pływać, ale zapomniałem.}

Czasowniki modalne uruchamiają tryb czynności regularnej i~nie odnosi się to do
tej konkretnej chwili, ale do stwierdzenia faktu (umiem czytać -- kiedyś się
nauczyłem i~od tej pory umiem i~raczej nie zapomnę -- oczywiście, można przestać
coś lubić albo móc, ale będzie to opisane odpowiednim aspektem w~przyszłości).

Stwierdzenia ,,pada deszcz'' i~inne odnoszące się do stanu rzeczywistości są
realizowane tylko przez sam czasownik i~podmiot domyślny.

\examples{Mi epi chet seysi.}{Mogę tutaj usiąść.}

\example{Mi pazi fesgai.}{Lubię czytać [w tej chwili].}

\emph{Vi} to partykuła/operator aspektu niedokonanego. Aspekt niedokonany
jest,,domyślny'' dla czasu teraźniejszego.

\examples{Ti vi fesgai.}{Czytasz, ale nie skończyłeś.}

\examples{Ti vi fesgat.}{Zacząłeś czytać [w przeszłości].}

\example{Mi vi koolt egi.}{Kochałem ją/jego [i nie przestałem].}

\emph{Va} to operator aspektu dokonanego. Aspekt dokonany jest uznawany za
domyślny w~formie czasu przeszłego i~za bardzo nie ma sensu w~czasie
teraźniejszym.

\examples{Ti va fesgat.}{Przeczytałeś.}

\examples{Mi va koolt egi.}{Kochałem ją/jego [ale przestałem].}

\example{Egi palimit a͞u mi.}{On rozmawiał ze mną.}

Partykuły \emph{va} oraz \emph{vi} mogą być pomijane, jeżeli będą wynikały z~
kontekstu.

\emph{Vige} to bardzo formalny operator ,,niech się stanie'':

\example{Vige hallo͞i nomi yi Ori!}{Niech będzie chwała imieniu Boga!}

Natomiast \emph{vage} to również operator ,,niech się stanie'', ale wyrażający
coś bardzo konkretnego. W~odróżnieniu od \emph{vige}, który nie
oczekuje,,fizycznego'' rezultatu. W~odniesieniu do konkretnego odbiorcy
oznacza,,powinieneś coś zrobić'' i~może służyć do grzecznego wyrażania prośby.

\examples{Vage fesgai.}{Niech ktoś przeczyta. [ale jeżeli z~kontekstu wynika to 
może być skierowane bezpośrednio do odbiorcy]}

\examples{Ti vage fesgai.}{Powinieneś przeczytać. [jawnie bezpośrednio do 
odbiorcy]}

\example{Ti vage fesgai ja lana.}{Ty powinieneś przeczytać tę książkę.}

\emph{Ge} to operator strony biernej.

\example{Sotak ge karlet [chu] sotak.}{Ktoś został zabity przez kogoś.} 

Wykorzystana tutaj może być partykuła \emph{chu} określająca dopełnienie, ale
nie jest to wymagane, kiedy kontekst zdania jest jasny.

\examples{Sotak karlet sotak.}{Ktoś zabił kogoś.}

\note{Brak w~tym zdaniu jest operatora \emph{va} z~uwagi na czas przeszły.}
\skipline

\emph{Do} określa operator trybu rozkazującego. Trybu rozkazującego należy
w~miarę możliwości unikać, zazwyczaj przez użycie operatora \emph{vage} lub
\emph{hemi}, jako, że jest uznawany za niegrzeczny.

Używany tylko w~połączeniu z~formą czasu przeszłego i~zawsze znajduje się na
końcu zdania. W~trybie rozkazującym użycie zaimka osobowego może zostać
pominięte, jeżeli zwracamy się bezpośrednio do odbiorcy polecenia.

\examples{Toi tort dej́itos do!}{Wy rzućcie broń!}

\example{Tort dej́itos do!}{[Ty] rzuć broń!}

\emph{Hemi} to bliźniaczy do operator, ale określający tryb
przypuszczająco-proszący, w~zależności od kontekstu. Istnieje również operator
\emph{ih́emi} oznaczający coś silniejszego, ,,błaganie''.

\examples{Ti tort dej́itos hemi.}{Rzuć proszę broń/czy mógłbyś rzucić broń?}

\example{Karlet no mi, ih́emi.}{Błagam, nie zabijaj mnie.}

Czysty tryb przypuszczający realizowany jest za pomocą operatora \emph{vimi}.

\example{Miam mi va fesgat ti vimi fari diyu.}{Gdybym przeczytał, ty byś coś 
zrobił.}

Przeczenia, negacja czynności realizowana jest z~wykorzystaniem partykuły
\emph{no}, która znajduje się \textbf{po} czasowniku. Możesz spotkać teksty
w~których partykuła \emph{no} znajduje się przed czasownikiem, ale są one zawsze
oznaką niedouczenia autora.

\note{W niektórych dialektach partykuła \emph{no} może pojawiać się zawsze na 
końcu zdania.}

\examples{Ti bugi no.}{Nie leżysz.}

\examples{Tort no dej́itos do!}{Nie rzucaj broni!}

\example{Egi karlet no il rige.}{On nie zabił jego (swojego) władcy.}

Czas przyszły realizowany jest przez partykułę \emph{ze}. Domyślnym aspektem
czasu przyszłego jest aspekt niedokonany, więc partykułę \emph{vi} można
pomijać.

\examples{Ti ze fesgai.}{Zaczniesz czytać.}

\examples{Ti ze va fesgai.}{Przeczytasz.}

\example{Hetay mi ze va vibi rome.}{Dzisiaj [ale w~przyszłości] zjem mięso.}

Szyk pytający (OSV):

\examples{Il maŕie͞o egi koóli?}{Czy on kocha swoją żonę?}

\examples{Il hima egi esi?}{Czy to jego dziewczyna?}

\examples{Sotak Epié karlet?}{Czy zabił Pan kogoś?}

\examples{Sotak Epié ze va karla͞i?}{Czy zabije Pan kogoś? [w nieokreślonej 
przyszłości]}

\examples{Sotak Epié ze vi karla͞i?}{Czy spróbuje Pan kogoś zabić?}

\examples{Il alye Epiá veyt hetay?}{Czy widziała Pani dzisiaj swojego 
przyjaciela?}

\example{Ti va fesgat?}{Przeczytałeś?}

Szyk pytający z~czasownikiem modalnym zachowuje konwencję czasownika
niemodalnego na końcu zdania.

\examples{Chet mi epi seysi?}{Czy mogę tutaj siąść?}

\note{Można by również zapytać \emph{Chet mi vi epi seysi?}, ale prawdopodobnie
zostało by to uznane za pretensjonalne aspirowanie do wyższych sfer.}

\example{Chet mi va ep seysi?}{Czy mogłem tutaj siąść?}

Szyk pytający jest również uruchamiany automatycznie przez partykuły pytające -- 
\emph{chyi} (czyj), \emph{ko͞e} (jak), \emph{osor} (dlaczego), \emph{so} (co), 
\emph{somar} (gdzie), \emph{soter} (kto), \emph{voli} (kiedy), \emph{wodo} 
(którędy), \emph{yage} (dokąd), \emph{yase} (skąd).

\examples{Yasu ti iéni?}{Skąd pochodzisz?}

\examples{Ko͞e tyi ager nomi?}{Jak nazywa się twój kraj?}

\examples{Ko͞e ti seiti?}{Jak się czujesz?}

\examples{So tyi vahuryiáysi o ti famei?}{Co znaczy dla ciebie twój tatuaż?}

\example{Wodo o Polska mi ze fari?}{Którędy do Polski mam jechać?}

Idiomy, takie jak na przykład \emph{kipeni a} (wzorować się na) powodują
przestawienie drugiego elementu (najczęściej partykuły) bliżej dopełnienia, np.:

\examples{Mi kipeni a~myi patal.}{Wzoruję się na moim ojcu.}

\example{A tyi patal ti kipeni?}{Czy wzorujesz się na swoim ojcu?}

Do pytań w~stylu ,,czy'' można również stosować partykułę wzmacniającą
\emph{vay}:

\example{Vay a~tyi patal ti kipeni?}{Czy naprawdę wzorujesz się na swoim ojcu?}

Zdania podrzędne tworzone są automatycznie przez partykuły takie jak \emph{ko͞e} 
(jak), \emph{voli} (kiedy), \emph{imin} (ponieważ), \emph{abe} (ale), 
\emph{chua} (która), \emph{pama} (wtedy), \emph{per} (żeby), a~także \emph{e} 
(i, oraz). Szyk zdania podrzędnego to SOV (Jan Marię kocha).

\examples{Mi fari o jan, per mi vibo vibi.}{Idę do domu, żeby zjeść jedzenie.}

\example{Cherlok Holms va choint sosbet kayetor esi no, e egi o jan fari
permi.}{Sherlock Holmes stwierdził, że podejrzany nie jest przestępcą i~pozwolił
mu iść do domu.}

Egi (on) odnosi się tutaj do podmiotu zdania głównego (Holmes pozwolił). Podmiot
zdania podrzędnego może zostać pominięty, jeżeli wynika z~kontekstu, jak w~
przypadkach powyżej. Zupełnie inaczej jednak to będzie wyglądało przy partykule
\emph{abe}:

\example{Cherlok Holms va choint sosbet kayetor esi no, abe egi va est -- abe
deíto lipe va kopet.}{Sherlock Holmes stwierdził, że podejrzany nie jest
przestępcą, ale on [podejrzany!] był, ale [podejrzany!] ukrył dobrze broń.}

Czasownik w~zdaniach podrzędnych może być pomijany, jeżeli wynika z~kontekstu.

\example{Ari esi arso, ko͞e vipetode yi muche esi.}{Niebo jest niebieskie, tak,
jak miska kota [jest].}

\subsection{Przymiotnik oraz imiesłów przymiotnikowy}

Przymiotniki są stopniowane w~3 stopniach (w większości), gdzie drugi stopień
zazwyczaj ma końcówkę <-e͞a>, a~trzeci stopień <-e͞am>. Przymiotnik nie ulega
odmianie przez rodzaje gramatyczne ani liczby.

Istnieją przymiotniki, które powstały przez dodanie <no-> przed pewnym rdzeniem,
wyrażające przeciwność, np. \emph{anper /ˈan.pɛr/ (mokry)} i~\emph{nonper
/ˈnɔn.pɛr/ (suchy)} oraz przymiotniki z~prefiksem <mo->, oznaczającym negatywne
zestopniowanie, np. \emph{leder /ˈlɛ.dɛr/ (oszczędny)} i~\emph{moleder
/ˈmɔ.lɛ.dɛr/ (skąpy)}.

Pozycja przymiotnika w~zdaniu ma decydujące znaczenie w~określeniu do czego się
odnosi.

\examples{Mi pazi karié himji.}{Lubię piękne kobiety.}

\examples{Mi pazi voli himji karié esi.}{Lubię, kiedy kobiety są piękne.}

\examples{Ji hima esi karié.}{Ta kobieta jest piękna.}

\example{Intrise vayarji esi lipe aloser che karié sipalima.}{Interesujące 
podróże są dobrym źródłem [dla] pięknej opowieści.}

Przymiotnik może pełnić rolę przysłówka, odpowiadając na pytanie ,,jak''.

\example{Ko͞e ti seiti? Lipe.}{Jak się czujesz? Dobrze.}

Przymiotnik od rzeczownika-nazwy własnej tworzony jest przez przyrostek
dzierżawczy <-yi> (/ʏ/). Same nazwy własne, także w~formie przymiotnika
odrzeczownikowego, są w~transkrypcji zawsze zapisywane wielką literą, w~
alfabecie naturalnym zawsze fonetycznie, nigdy sylabicznie.

Imiesłów przymiotnikowy tworzony jest poprzez zmianę -i na -o w~bezokoliczniku.

\examples{kahi}{latać}
\examples{kaho}{latający}
\examples{kaho beykar}{latający wąż}
\example{mosto inrat}{stworzony język -- conlang}

Ale zamiast imiesłowów można też stosować partykułę \emph{chu}:

\examples{Pakopo rujalar kant hetay ne͞a ji͞ari͞o in Lublin.}{Zmartwiony
mężczyzna śpiewał dzisiaj obok ogrodu w~Lublinie.}

\examples{Rujalar chu vi pakot kant hetay ne͞a ji͞ari͞o in Jechuf.}{Mężczyzna, 
który się martwił, śpiewał dzisiaj obok ogrodu w~Rzeszowie.}

\examples{Muche che altur vi fart}{Kot, który robił hałas}

\examples{Altur faro muche}{Kot robiący hałas}

\example{Alturo muche}{Hałasujący kot}

\subsection{Liczebniki}

Liczebniki od 0 do 9 to, po kolei: \emph{wa}, \emph{a}, \emph{ka}, \emph{sa}, 
\emph{ta}, \emph{na}, \emph{cha}, \emph{ma}, \emph{ya}, \emph{ra}.

\begin{table}[ht]
	\centering
	\caption{Podstawowe liczebniki}
	\begin{tabular}{ccc} \toprule
		zero & 0 & wa \\
		jeden & 1 & a \\
		dwa & 2 & ka \\
		trzy & 3 & sa \\
		cztery & 4 & ta \\
		pięć & 5 & na \\
		sześć & 6 & cha \\
		siedem & 7 & ma \\
		osiem & 8 & ya \\
		dziewięć & 9 & ra \\
		dziesięć & 10 & awa \\\bottomrule
	\end{tabular}
	\label{tab:numerals}
\end{table}

Najprostszym sposobem tworzenia liczebnika dla liczb większych od 9 jest
ustawienie w~kolejności zapisu dziesiętnego kolejnych słów określających cyfry,
tj. 17 to \emph{ama}, 123 to \emph{akasa}, a~241 to \emph{kataa}. Dla większych
liczb było by to jednak zbyt problematyczne (np. 1000 to by było
\emph{awawawa}), dlatego można stosować przyrostek <-y>, określający liczbę
powtórzeń, w~stosunku do sylaby określającej liczbę powtórzeń, a~potem cyfrę
powtarzaną, na przykład \emph{aśaywa} -- dosłownie: jeden i~trójka zer. Stąd
milion to \emph{aćhaywa}, a~100023 to \emph{aśaywakasa}. Akcent kładziony jest
zawsze na sylabę, która zawiera przyrostek <-y>.

W sytuacji, kiedy jest kilka sylab z wrostkiem <-y>, akcent kładziony jest na
ostatnią taką sylabę.

\begin{table}[ht]
	\centering
	\caption{Liczebniki wyższe}
	\begin{tabular}{ccc} \toprule
		jedenaście & 11 & aa \\
		dwanaście & 12 & aka \\
		dwadzieścia & 20 & kawa \\
		trzydzieści & 30 & sawa \\
		czterdzieści & 40 & tawa \\
		sto & 100 & aḱaywa  \\
		tysiąc & 1000 & aśaywa \\
		milion & 1000000 & aćhaywa \\\bottomrule
	\end{tabular}
	\label{tab:numerals2}
\end{table}

\subsubsection{Partykuła \emph{jo}}

Niektórzy użytkownicy języka nie lubią formy pozycyjnej, w której wyrazy
potrafią być problematyczne, np. \emph{aaa} jako sto jedenaście. Stąd używana
jest również partykuła \emph{jo}, która oznacza ,,oraz'', i jest stosowana do
dodawania wartości wypowiedzianych liczebników.

Przykładowo:

\examples{aśaywa jo kata}{tysiąc oraz dwadzieścia cztery = 1024}

\example{aḱaywa jo a}{sto oraz jeden = 101}

\example{aḱaywa jo awa jo a}{sto oraz dziesięć oraz jeden = 111}

\subsubsection{Liczebniki porządkowe}

Przyrostek <-ti> określa liczebnik porządkowy, tj. \emph{kati} to drugi, a
\emph{sasati} -- trzydziesty trzeci.

\begin{table}[ht]
\centering
\caption{Podstawowe liczebniki porządkowe}
\begin{tabular}{cc} \toprule
	pierwszy & ati \\
	drugi & kati \\
	trzeci & sati \\
	czwarty & tati \\
	piąty & nati \\
	szósty & chati \\
	siódmy & mati \\
	ósmy & yati \\
	dziewiąty & rati \\
	dziesiąty & awati \\\bottomrule
\end{tabular}
\label{tab:numerals3}
\end{table}

\subsubsection{Liczebniki ułamkowe}

Liczebnik ułamkowy tworzy się przez infiks <-je->, na przykład \emph{ajeka} to
jedna druga, \emph{najera} -- pięć dziewiątych, a~\emph{awajerana} to 10/95.

\begin{table}[ht]
	\centering
	\caption{Ułamki}
	\begin{tabular}{cc} \toprule
		połowa & ajeka \\
		jedna trzecia & ajesa \\
		dwie trzecie & kajesa \\
		jedna czwarta & ajeta \\
		trzy czwarte & sajeta \\
		jedna siódma & ajema  \\\bottomrule
	\end{tabular}
	\label{tab:numerals4}
\end{table}

Możliwe jest również użycia słowa \emph{jerya} oznaczającego ,,przecinek'',
w~taki sposób, że:

\example{a jerya rasa}{jeden przecinek osiem trzy = 1,83}

\subsection{Honoryfikacja, imiona, nazwiska, formalność i~grzeczność}

W zależności od regionu możliwe jest, że użytkownicy języka będą bardzo
wyczuleni na kwestie grzecznościowe. Typowym tego typu elementem jest niechęć do
stosowania operatora trybu rozkazującego \emph{do} na rzecz \emph{hemi}, ale
bardzo często możesz również spotkać się z~honoryfikatorami służącymi do
odpowiedniego zwracania się do innych osób.

W And́royas typowym jest używanie pierwszego imienia w~odniesieniu do rozmówcy,
lub podczas określania osoby, chyba, że jest to niemożliwe do jednoznacznej
identyfikacji, wtedy używa się pełnego imienia i~nazwiska. Z drugiej jednak
strony mieszkańcy Cesarstwa są bardzo dumni ze swojej rodziny i~swojego rodowego
nazwiska, stąd przedstawiajac się często to podkreślą przedstawiając się pełnym
imieniem oraz nazwiskiem.

\example{Mi nomi Eryus mal Edoraril.}{Nazywam się Eryus mal Edoraril.}

Warto tutaj zwrócić uwagę, że imiona i~nazwiska w~Cesarstwie są rozdzielane
partykułą \emph{mal}, oznaczającą ,,z rodziny'', np. \emph{Koolder mal
Erlehirni} to Koolder z rodziny Erlehirni. Czasami możliwe jest, że dzieci
dziedziczą nazwiska po obojgu rodziców, stąd występują nazwiska łączone
łącznikiem, takie jak \emph{Alya mal Arkai-Valor}. Istnieją również także
oznaczane za pomocą łącznika gałęzie rodów, które dziś stały się zwykłymi
nazwiskami, np. \emph{Nimu͞e mal Hetasi-Hi}, co oznacza ród Hetasi i jego gałąź
Hi.

\note{Gałęzie rodowe, i co za tym idzie, ich oznaczenia w nazwiskach pojawiały
się w~sytuacji, kiedy nazwisko rodowe przechodziło tylko na pierwsze dziecko,
natomiast kolejne dzieci uzyskiwały nazwiska z~określeniem gałęzi. Zwyczaj ten
zanikł prawie całkowicie około VII wieku po Zjednoczeniu.}
\skipline

Używa się zaimków osobowych \emph{epié} i~\emph{epiá}, które odpowiadają mniej
więcej polskim ,,pan'' i~,,pani''. Używa się ich w~odniesieniu do obcych osób
albo osób stojących wyżej w~hierarchii, albo w~sytuacji, kiedy nie znamy imienia
osoby, do której chcemy się zwrócić. Bardzo często można napotkać ich stosowanie
w postaci przyrostków z~łącznikiem, w~stosunku do imienia, np. mówiąc o kimś
wyżej w~hierarchii możemy powiedzieć \emph{Koolder-epié}. W~podobny sposób
można określać czyjąś funkcję, np. \emph{Furu-falazera} -- dowódca Furu. Czasami
można napotkać formę \emph{falazera-epiá} -- pani dowódca. W~taki sposób można
używać słów takich jak \emph{falazer} (dowódca), \emph{kachister} (nauczyciel),
\emph{meneder} (lekarz) i~innych.

\note{\emph{epié} i~\emph{epiá} praktycznie nie są używane w Republice Nennek,
gdzie preferowane jest zwracanie się imieniem, zaimkiem ogólnym \emph{egi} lub
zależnymi od płci \emph{egli/egla} lub ewentualnie przyrostkiem \emph{-gam},
jeżeli naprawdę chce się podkreślić swoją niższą pozycję wobec rozmówcy.}

\note{W momencie gdy dwoje rozmówców będzie traktować się nawzajem z identycznym
poziomem grzeczności, będą się do siebie zwracać nawzajem ukazując swoją niższą
pozycję, np. nawzajem tytułować siebie z przyrostkiem \emph{-epié}.}
\skipline

Z drugiej strony, możliwe jest, że rozmówca będzie traktować drugą osobę jako
osobę od niego niższą statusem, ukazując swoją wyższą pozycję. Jest to
oczywiście niegrzeczne i stąd bardzo rzadko spotykane. Przyrostkami takimi mogą
być rzeczownik \emph{pezawe} (,,gorszy człowiek'') lub wręcz rzeczownik
\emph{zam} (dosłownie ,,śmieć''). Bardzo pogardliwe, spotkane w sytuacji i
próbach zastraszenia rozmówcy.

\subsubsection{Zdrobnienia i poufałość}

Oczywiście, w codziennych sytuacjach osoby sobie bliskie nie będą używały
określeń stricte formalnych -- do babci raczej wnukowie zwrócą się
\emph{chancha} niż \emph{gruchana}, jeśli są z nią blisko, a do swoich rodziców
\emph{mama} i \emph{patal} bardziej niż \emph{natali͞a} oraz \emph{vapal}.

W podobny sposób stosowane są często zdrobienia i partykuły lub rzeczowniki
z~nimi związane, takie jak \emph{myi}, który może być stosowany jako przyrostek
zdrabniający, np. \emph{pelir-myi} -- ,,mój pieseczek'', czy też \emph{koól},
stosowany np. \emph{Alya-koóla} -- ,,kochana Alya'', stosowany w odniesieniu do
osoby darzonej uczuciem.

Istnieje również słabsza wersja \emph{koól}, \emph{arey}, przyimek stosowany do
określenia sympatii do drugiej osoby. Może być również stosowany do określenia
sympatii do rzeczy, bez określenia jej posiadania, w przeciwieństwie do
\emph{myi}.

\subsubsection{Rodzina cesarska}

W przypadku rodziny cesarskiej używa się określeń \emph{eyger} oraz
\emph{eygera}, np. \emph{Fayfnira-eygera} -- cesarzowa Fayfnira, ale i~takich
jak \emph{and́royasyikigje͞a} (księżniczka And́royas -- tj. siostry
cesarza/cesarzowej), czy \emph{and́royasyikigeje} (książę And́royas -- bracia
panującego).

Z kolei dzieci panującego często określane są tytułami \emph{icheryikigeje},
\emph{icheryikigeje͞a} lub \emph{icheryihima} -- dosłownie książę lub
księżniczka krwi. Oprócz tego istnieje przyrostek \emph{-hima}, stosowany często
na Wschodzie w~stosunku do całej żeńskiej strony rodu panującego, poza
Cesarzową.

Małżonek panującego może posiadać tytuł zarówno równorzędny -- np. \emph{eyger},
ale i~na przykład \emph{eygeryikigeje}, dosłownie ,,książę cesarstwa''
lub,,książę cesarzowej''. Dokładne zasady są zależne od aktualnej sytuacji
politycznej.

Stąd aktualnie (w momencie pisania tej książki), mamy:

\begin{itemize}
\item \textbf{Katia-eygera mal Arkai} [ˈka.ti.a ˈɛj.gɛ.ra ˈmal ˈar.ka.i],\\ 
Cesarzowa Katia mal Arkai,
\item \textbf{So'tak-eygeryikigeje mal Valor} [ˈsɔ|.tak ˈɛj.gɛ.rʏ.ki.gɛ.ʐɛ 
ˈmal ˈva.lɔr],\\ Ksiażę Cesarzowej, So'tak mal Valor,
\item \textbf{Alya-icheryikigeje͞a mal Arkai-Valor} [ˈal.ja i.t͡ʂe.rʏ.ki.gɛ.ʐɛa ˈmal 
ˈar.ka.i-va.lɔr],\\ Księżniczka Krwi, Alya mal Arkai-Valor,
\item \textbf{Niva-and́royasyikigeje͞a mal Arkai} [ˈni.va an.ˈdrɔ.ja.sʏ.ki.gɛ.ʐɛa 
mal ˈar.ka.i],\\ Księżniczka And́royas, Niva mal Arkai,
\item \textbf{Karra-and́royasyikigeje͞a mal Arkai} [ˈkar.ra an.ˈdrɔ.ja.sʏ.ki.gɛ.ʐɛa 
mal ˈar.ka.i],\\ Księżniczka And́royas, Karra mal Arkai,
\item \textbf{Jaida-and́royasyikigeje͞a mal Arkai} [ˈʐa.i.da an.ˈdrɔ.ja.sʏ.ki.gɛ.ʐɛa
mal ˈar.ka.i],\\ Księżniczka And́royas, Jaida mal Arkai.
\end{itemize}

\note{Należy tutaj zwrócić uwagę na wymowę imienia Jej Wysokości, w~której
głoski /i/ oraz /a/ nie zlewają się w~/ia/, oraz na imię Jego Wysokości, w~
którym występuje pauza pomiędzy sylabami, oznaczana przez <'> w~transkrypcji.
Wynika to z~faktu, że Jego Wysokość pochodzi z~regionu Maddo, gdzie pojawiają
się takie, unikatowe, elementy języka, z~uwagi na wpływ języków krajów
ościennych.}

\subsection{Uwagi końcowe}

Przyimek dzierżawczy \emph{yi} określa posiadanie -- np. \emph{vipetode yi
muche} - kocia miska (miska kota). Możliwe jest jego połączenie z~obiektem do
którego się odnosi jako przyrostka, zmieniany jest wtedy szyk, np. \emph{mucheyi
vipetode}. Koncepcja ta istnieje głównie w~dialektach zachodnich, w~których są
silne wpływy aglutynacyjnego języka kaireńskiego oraz w~tzw. dialekcie
pustynnym. Oprócz tego, stosowane jest to w~sytuacji wielokrotnego zagnieżdżenia
dzierżawczego, na przykład miska matki mojego ojca \emph{vipetode yi myi arśityi
natali͞a}, aczkolwiek \emph{vipetode yi natali͞a yi myi arśit} jest oczywiscie
równie poprawne. Stosowane jest to również bardzo często w~słowotwórstwie.

\subsection{Dialekty}
Jak wspomniano, ten słownik skupia się na standardowej formie języka, jednak już
mogłeś zauważyć, że wielokrotnie wspominamy o dialektach. Obecnie wyróżnia się 6
głównych dialektów oprócz języka standardowego:

\begin{itemize}
    \item dialekt nennecki, wyróżnia się znacznie mniejszym poziomem
    formalności, w szczególności stosowaniem zaimka \emph{egi} zamiast
    formalnych \emph{epié/epiá}, a czasami nawet zamiast odmiennych przez
    rodzaje gramatyczne \emph{egli/egla}; w dialekcie tym raczej też nie stosuje
    się honoryfikatora \emph{-gam},
    \item dialekt zachodni, w~którym preferowany jest przyrostek dzierżawczy
    \emph{-yi} zamiast partykuły \emph{yi}, w podobny sposób niektóre inne
    partykuły są używane jako przyrostki, np. partykuła czasu przyszłego
    \emph{ze} -- \emph{Ti ze vibi.} jest często zastępowane \emph{Ti vibize.},
    występuje też dużo własnego regionalnego słownictwa,
    \item dialekt północny wyróżniający się pojawianiem się fonemu [ʃ] zamiast
    [ʐ],
    \item dialekt pustynny, z szeregiem własnego słownictwa, użyciem [ʁ] zamiast
    [r] oraz pojawianiem się aspiracji -- [g] oraz [k] przed samogłoskami są
    wymawiane jako [ɡʱ] oraz [kʱ], odpowiednio; jedną z ciekawostek jest również
    \emph{nodi}, ekskluzywne "my",
    \item dialekt południowy, stosujący długie samogłoski (np. /a:/) zamiast
    rozziewu ze zmianą akcentu (stąd \emph{baán} to raczej [ba:n] niż [ba.ˈan])
    oraz [ɨ] zamiast [ʏ], [ʒ] zamiast [ʐ] i czasem [h] zamiast [x],
    \item dialekt południowo-wschodni, z szeregiem własnego słownictwa, oraz
    występowaniem pauzy [|] w niektórych słowach, w szczególności imionach --
    patrz \emph{So'tak} [sɔ|.tak].
\end{itemize}

Dialekt nennecki najpopularniejszy jest w regonie Nennek, ale i pojawia się
w~Rem, Amesrze lub północnym Vagyr. Dialekt zachodni najpopularniejszy jest
w~Lono, ale i Rilli, Dorel oraz w~regionie Wysp Zachodnich. Dialekt pustynny
przede wszystkim można spotkać w~Agavie i~Istapie. Dialekt południowy to domena
południowego Vagyr, natomiast południowo-wschodni najczęściej występuje w Maddo.

\subsubsection{Mowa potoczna}

Wielokrotnie w tym słowniku spotkasz się z określeniami, że pewne określenia,
słowa lub zwroty stosowane są wyłącznie w mowie potocznej, nieformalnej.

Język andro nie używa wielu poziomów formalności, niespecjalnie też rozróżniane
są formy wyrażeń zarezerwowane dla stereotypowej mowy męskiej i~żeńskiej,
społeczeństwo And́royas już od wieków było dość egalitarne pod względem płci.

Spotykane są jednak formy, które najczęściej stosowane są w formie ustnej,
a~które są albo całkowicie nieformalne albo są składową dawniejszych procesów
językowych. Przykładem elementów nieformalnych mogą być:

\begin{itemize}
	\item stosowanie końcówki czasownika <-ee> [ɛ:], np. \emph{Foree!} [ˈfɔ.rɛ:]
	-- ,,płacę!'', w~szczególności do określenia czynności wykonywanej
	niechętnie, stosowane często w stereotypie ,,twardego mężczyzny'',
	\item partykuła \emph{de}, jako wzmocnienie wyrażenia, np. \emph{Mi fesgai
	de!} -- ,,ja przecież czytam!''.
\end{itemize}

\end{spacing}