\section[Ogólne uwagi]{Ogólne uwagi o języku androyasańskim}

\begin{spacing}{1.1}
    Język androyasański może być zaliczony w~dużej mierze do języków
    izolujących i~analitycznych, ze pewną formą aglutynacji w~słowotwórstwie.

    \subsection{Fonetyka, alfabet i~transkrypcja}

    Występują następujące samogłoski: a, ɛ, i, ɔ, u, ʏ, oraz 18 spółgłosek:

    \begin{table}[ht]
        \centering
        \caption{Spółgłoski w~języku androyasańskim}
        \begin{tabular}{lcccc}\toprule
                              & Wargowe & Przedniojęzykowe & Podniebienne & Tylnojęzykowe \\\midrule
            Nosowe            & m       & n                &              & ŋ             \\\midrule
            Zwarte            & p b     & t d              &              & k g           \\\midrule
            Szczelinowe       & f v     & s z~ʐ            &              & x             \\\midrule
            Półotwarte        &         &                  & j            &               \\\midrule
            Drżące            &         & r                &              &               \\\midrule
            Boczne półotwarte &         & l                &              &               \\\bottomrule
        \end{tabular}
        \label{tab:consonants}
    \end{table}

    Oprócz spółgłosek zaprezentowanych w~tabeli powyżej, występuje
    również zwarto-szczelinowe \xm{ʈ͡ʂ} oraz \xm{w}.

    Występuje szereg dyftongów: \xm{ɛi}, \xm{ɛɔ}, \xm{ɔa}, \xm{ia}, \xm{iɔ},
    \xm{uɛ}, \xm{au}, \xm{ai}, \xm{aɛ}, \xm{uɔ}, \xm{ɔʏ}, \xm{ɔi}, ale pomimo
    nagromadzenia fonemów w~stylu \xm{ri}, \xm{hi}, czy \xm{pi} palatalizacja
    praktycznie nie występuje (a przynajmniej nie powinna).

    W poszczególnych dialektach możliwe jest jednak pojawianie się innych głosek
    oraz innych zjawisk fonetycznych -- są to miedzy innymi głoski \xt{ʁ} czy też
    przydech \xt{ɡʱ} lub \xt{kʱ}.

    Głoski \xt{ʐ} oraz \xt{x} mogą być również wymawiane jako \xt{ʒ} oraz \xt{h},
    odpowiednio. Jest to traktowane normalnie w dialektach i~jako błąd w~\emph{ardo
        andro}. W dialekcie południowym użytkownicy często wymawiają głoskę \xt{ɨ}
    zamiast \xt{ʏ}.

    \subsubsection{Fonotaktyka}

    Z uwagi na historię języka, pełny zestaw reguł tworzenia sylaby jest dość
    skomplikowany:

    \begin{itemize}
        \item (C)V(C1), gdzie C1 to: \xm{d}, \xm{j}, \xm{k}, \xm{l}, \xm{m}, \xm{n}, \xm{r}, \xm{s} lub \xm{t} (\xm{a}, \xm{ba}, \xm{bad}),
        \item (C)V(V)(C1), gdzie jedyne dopuszczalne dyftongi to: \xm{ɛa}, \xm{ɛi}, \xm{ɛɔ}, \xm{ɔa}, \xm{ɔɛ}, \xm{ɔi}, \xm{uɛ}, \xm{au}, \xm{aɛ}, \xm{uɔ}, \xm{ua}, \xm{ui}, \xm{ia}, \xm{iɔ}, \xm{aɔ} oraz \xm{ai} (\xm{ɛa}, \xm{ɛad}, \xm{bɛad}),
        \item CRV(C1), gdzie R to wyłącznie \xm{r}, \xm{j} lub \xm{l} (\xm{bra}, \xm{brad}) ale klaster CR nie może mieć postaci \xm{rr}, \xm{ll}, \xm{jj}; natomiast V oznacza wszystkie samogłoski poza \xm{ʏ},
        \item (C)Vŋt, (\xm{aŋt}, \xm{baŋt}),
        \item (C)Vrn, (\xm{arn}, \xm{barn}),
        \item stV(C1), (\xm{sta}, \xm{stad}).
    \end{itemize}

    \subsubsection{Transkrypcja}

    Jak wspomniano wcześniej, w tym słowniku wykorzystywana jest transkrypcja Ziri.
    Zapis z~wykorzystaniem transkrypcji Ziri odbywa się w~sposób zaprezentowany w~
    tabelach poniżej:

    \begin{table}[ht]
        \centering
        \caption{Fonemy}
        \begin{tabular}{ll} \toprule
            Samogłoski & a, ɛ, i, ɔ, u, ʏ                                            \\
            Spółgłoski & m, n, ŋ, p, b, t, d, k, g, s, z, ʐ, x, j, r, l, w, v, f, ʈ͡ʂ \\\bottomrule
        \end{tabular}
        \label{tab:phonemes}
    \end{table}

    \begin{table}[ht]
        \centering
        \caption{Znaki odpowiadające fonemom}
        \begin{tabular}{ll} \toprule
            Samogłoski & a~e i~o u yi                                                \\
            Spółgłoski & m, n, n, p, b, t, d, k, g, s, z, j, h, y, r, l, w, v, f, ch \\\bottomrule
        \end{tabular}
        \label{tab:chars}
    \end{table}

    W transkrypcji Ziri zapis \xo{-nt} określa głoski \xm{-ŋt}, jako że nosowe
    \xm{n} nie występuje praktycznie w~innym kontekście. Do zapisu przydechu używa
    się dwóch znaków -- \xm{ɡʱ} staje się \xo{gh}. Głoska \xm{ʁ} zapisywana jest
    identycznie jak \xm{r}, czyli \xo{r}.

    \note{Jak wspomniano wcześniej, głoski \xm{ɡʱ} oraz \xm{ʁ} nie występują
        w~\emph{ardo andro}, jedynie w~niektórych dialektach.}\skipline

    Zapis \xo{aa}, \xo{uu} i~\xo{oo} nie jest powtórzeniem -- najczęściej jest to
    rozziew (hiat), kiedy dwie samogłoski występują w~oddzielnych sylabach (albo
    stanowią oddzielne sylaby) i~powinny być czytane oddzielnie, np. \emph{baán}
    \xm{ba.ˈan} (czas). Zazwyczaj w~sytuacji potencjalnego rozziewu występuje też
    zmiana akcentu. Szczególnym przypadkiem jest \xo{ii}, który jest wydłużeniem --
    \xm{i:} -- ale występuje tylko w nielicznych słowach pochodzenia obcego.
    W~dialekcie południowym zdarza się, że użytkownicy nie traktują \xo{aa} i
    podobnych jako rozziewu, a jako wydłużenie samogłoski (\xm{a:}) i jest to błąd w
    \emph{ardo andro}, traktowane normalnie w dialekcie.

    Dyftongi są prezentowane za pomocą makronu nad oboma elementami dyftongu, np.
    \xm{ɛɔ} jest zapisywane jako \xo{e͞o}. Akcent na pierwszą sylabę nie jest
    oznaczany, akcent na inne sylaby jest realizowany akutem nad pierwszym znakiem
    sylaby akcentowanej.

    O ile w transkrypcji będziemy się posługiwać polskimi znakami przestankowymi
    oraz koncepcją wielkich liter na początku zdania i~przy nazwach własnych, to nie
    jest to stosowane w~tradycyjnym alfabecie fonetycznym \emph{chiwo}, który nie
    rozróżnia wielkości znaków.

    \subsection{Akcent wewnątrz słów i~zdań}

    Akcent kładziony jest zazwyczaj na pierwszą sylabę, najczęstszym wyjątkiem jest
    fakt, że słowo składa się z~przedrostka -- np. słowo \emph{uf́riti}
    \xm{u.ˈfri.ti} (gasić) jest w~gruncie rzeczy słowem \emph{friti} \xm{ˈfri.ti}
    (świecić, palić) z przedrostkiem \xo{u-}. Drugi wyjątek to słowa zapożyczone
    z~innych języków.

    Nie występuje praktycznie akcent zdaniowy.

    \subsection{Rzeczowniki}
    Rzeczownik odmienia się przez rodzaje (męski, żeński -- nie zawsze występuje,
    istnieją rzeczowniki tylko i~wyłącznie w~rodzaju żeńskim lub tylko i~wyłącznie
    w~rodzaju męskim) oraz liczby (pojedyncza, mnoga -- nie zawsze występuje,
    istnieją słowa tylko w~liczbie pojedynczej lub tylko mnogiej). Końcówka liczby
    mnogiej to zazwyczaj \xo{-s}, \xo{-os}, \xo{-ji} lub \xo{-jis}, końcówka rodzaju
    żeńskiego to zazwyczaj \xo{-a}.

    \note{Poza rodzajem męskim i żeńskim w bardzo rzadkich przypadkach konieczne
        jest rozróżnienie pomiędzy obiektami ożywionymi (\An{}) i nieożywionymi
        (\Inan{}). Do rzeczowników oznaczonych jako ożywione zalicza się: ludzi, rośliny
        oraz zwierzęta.}

    Rzeczowniki odczasownikowe tworzone są zazwyczaj od formy czasu przeszłego
    (\emph{kanti} -- śpiewać, \emph{kant} -- śpiew). Rzeczowniki określające osoby
    (zawody) najczęściej mają końcówkę \xo{-er} (\emph{kanter} -- śpiewak,
    \emph{rufaler} -- rolnik, \emph{suier} -- marynarz).

    \subsection{Zaimki osobowe}

    \begin{multicols}{2}

        \dictwordb{mi}[ˈmi]
        \dictterm{pro}{ja (\Fsg{})}

        \dictwordb{ti}[ˈti]
        \dictterm{pro}{ty (\Ssg{})}

        \dictwordb{egi}[ˈɛg.i]
        \dictterm{pro}{on, ona (\Tsg{})}
        \note{Zaimek egi jest niezależny od płci. Spotykany zazwyczaj tylko w~dialekcie
            Republiki Nennek.}

        \dictwordb{egli}[ˈɛg.li]
        \dictterm{pro}{on (\Tsg{}.\M{})}

        \dictwordb{egla}[ˈɛg.la]
        \dictterm{pro}{ona (\Tsg{}.\F{})}

        \dictwordb{che}[ˈʈ͡ʂɛ]
        \dictterm{pro}{to (\Tsg{}.\Inan{}}
        \note{Zaimek che stosowany jest wyłącznie do obiektów nieożywionych.}

        \dictwordb{epié}[ɛ.pi.ˈɛ]
        \dictterm{pro}{pan (\Ssg{}.\Frm{}.\M{} lub \Tsg{}.\Frm{}.\M{})}

        \note{Zaimki epié oraz epiá są bardzo formalne. Używa się ich tylko w~języku
            formalnym, w~odniesieniu do obcych osób albo osób stojących wyżej w~hierarchii,
            albo w~sytuacji, kiedy nie znamy imienia osoby, do której chcemy się zwrócić.
            Mogą być używane zarówno jako zaimki drugiej, jak i trzeciej osoby.}
        \note{Uwaga: zaimków tych nie stosuje się w~powszechnej mowie w~Republice
            Nennek.}

        \dictwordb{epiá}[ɛ.pi.ˈa]
        \dictterm{pro}{pani (\Ssg{}.\Frm{}.\F{} lub \Tsg{}.\Frm{}.\F{})}

        \dictwordb{noni}[ˈnɔ.ni]
        \dictterm{pro}{my (\Fpl{})}

        \dictwordb{nodi}[ˈnɔ.di]
        \dictterm{pro}{my ekskluzywne}
        \note{Oznacza ,,my, ale nie włącznie z~tobą''. Spotykany wyłącznie w dialekcie
            pustynnym.}

        \dictwordb{toi}[ˈtɔ.i]
        \dictterm{pro}{wy (\Spl{})}

        \dictwordb{ego͞i}[ˈɛg.ɔi]
        \dictterm{pro}{oni, one (\Tpl{})}

        \dictwordb{cheí}[ʈ͡ʂɛ.ˈi]
        \dictterm{pro}{te rzeczy (\Tpl{}.\Inan{})}

    \end{multicols}

    Oprócz zaimków osobowych istnieją również zaimki dzierżawcze, określające
    posiadanie.

    \note{Dawniej w \emph{ardo andro} używało się tej formy zaimka także do
        określenia dopełniacza, co jest pozostałością odmiany przez przypadki w językach
        starożytnych. Zwyczaj ten pozostał obecnie tylko w bardzo formalnym języku.}.

    \begin{multicols}{2}

        \dictwordb{myi}[ˈmʏ]
        \dictterm{pro}{mój (\Fsg{}.\Poss{}), mnie (\Fsg{}.\Gen{})}

        \dictwordb{tyi}[ˈtʏ]
        \dictterm{pro}{twój (\Ssg{}.\Poss{}), ciebie (\Ssg{}.\Poss{})}

        \note{Zaimki myi oraz tyi zazwyczaj zapisywane są w~postaci ideogramu, czasami
            nawet w~tekście w~zapisie fonetycznym \emph{chiwo}.}

        \dictwordb{il}[ˈil]
        \dictterm{pro}{jego, jej (\Tsg{}.\Poss{} oraz \Tsg{}.\Gen{})}

        \dictwordb{chyi}[ˈʈ͡ʂʏ]
        \dictterm{pro}{tej rzeczy (\Tsg{}.\Inan{}.\Poss{})}

        \dictwordb{epil}[ˈɛ.pil]
        \dictterm{pro}{pana, pani, pański (\Ssg{}.\Frm{}.\Poss{} lub \Tsg{}.\Frm{}.\Poss{})}

        \dictwordb{niyi}[ˈni.ʏ]
        \dictterm{pro}{nasze (\Fpl{}.\Poss{}), nas (\Fpl{}.\Gen{})}

        \dictwordb{nodyi}[ˈnɔ.dʏ]
        \dictterm{pro}{nasze (ekskluzywne)}

        \dictwordb{tyoi}[ˈtjɔ.i]
        \dictterm{pro}{wasze (\Spl{}.\Poss{}), was (\Spl{}.\Gen{})}

        \dictwordb{egyi}[ˈɛ.gʏ]
        \dictterm{pro}{ich (\Tpl{}.\Poss{} lub \Tpl{}.\Gen{})}

        \dictwordb{chey}[ˈʈ͡ʂɛj]
        \dictterm{pro}{tych rzeczy (\Tpl{}.\Inan{}.\Poss{})}

    \end{multicols}

    \subsection{Mowa potoczna}

    Wielokrotnie w tym słowniku spotkasz się z określeniami, że pewne określenia,
    słowa lub zwroty stosowane są wyłącznie w mowie potocznej, nieformalnej.

    \emph{andro} nie używa wielu poziomów formalności, niespecjalnie też rozróżniane
    są formy wyrażeń zarezerwowane dla stereotypowej mowy męskiej i~żeńskiej,
    społeczeństwo And́royas już od wieków było dość egalitarne pod względem płci.

    Spotykane są jednak formy, które najczęściej stosowane są w formie ustnej,
    a~które są albo całkowicie nieformalne albo są składową dawniejszych procesów
    językowych. Przykładem elementów nieformalnych mogą być:

    \begin{itemize}
        \item stosowanie końcówki czasownika <-ee> \xt{ɛ:}, np. \emph{Foree!}
              \xt{ˈfɔ.rɛ:} -- ,,płacę!'', w~szczególności do określenia czynności
              wykonywanej niechętnie, stosowane często w stereotypie ,,twardego
              mężczyzny'',
        \item partykuła \emph{de}, jako wzmocnienie wyrażenia, np. \emph{Mi fesgai
                  de!} -- ,,ja przecież czytam!''.
    \end{itemize}

\end{spacing}