\newpage

\section{Przedmowa}

\begin{spacing}{1.1}

\emph{Kiiway!} Cieszę się, że czytasz tę książkę! Nazywam się Koolder mal
Erlehirni i~ta książka to owoc wytężonej pracy ostatnich kilku lat.
Przygotowałem ten dokument w~taki sposób, aby oprócz bycia tylko słownikiem
zawierał również szereg uwag dotyczących samego języka \emph{andro}, jak
również różnorodne notatki dotyczące kultury naszego kraju, które mogą być
przydatne w~zrozumieniu tekstów, które będziesz tłumaczyć.

Część z~tych materiałów wymaga pewnego przygotowania -- znajomości pewnych pojęć
z dziedziny językoznawstwa, jednak nie jest to absolutnie wymagane.

Słownik zaczyna się od wyjaśnienia pojęć i~oznaczeń, które będziesz spotykać
dalej w~treści, oraz z~krótkiego opisu historii języka, wprowadzenia do jego
ortografii, fonetyki i~gramatyki.

W słowniku będzie stosowana tylko transkrypcja, nie będą (prawie) pojawiały się
teksty zapisane naszym tradycyjnym alfabetem fonetycznym, ani sylabariuszami,
ale w~miarę obycia z~transkrypcją zdasz sobie sprawę, że bardzo dokładnie
odzwierciedla nasze pismo, więc w~dalszej kolejności będziesz w~stanie przejść
do bardziej zaawansowanych materiałów dotyczących języka \emph{andro}.

\bigskip

Powodzenia!

\section[Skróty i~oznaczenia]{Skróty i~oznaczenia używane w~słowniku}

Każde słowo, które napotkasz w~słowniku, jest zapisane dokładnie w~ten sam
sposób.

Tekst pogrubiony oznacza słowo w~transkrypcji. Następnie zapisana jest wymowa za
pomocą standardowego Międzynarodowego Alfabetu Fonetycznego, IPA. Kolejny będzie
skrót, który oznacza z~jaką częścią mowy masz do czynienia:

\begin{table}[h]
\begin{tabular}{ll}
\emph{n}    & rzeczownik           \\
\emph{v}    & czasownik            \\
\emph{adj}  & przymiotnik          \\
\emph{pro}  & przyimek albo zaimek \\
\emph{part} & partykuła           
\end{tabular}
\end{table}

W zależności od części mowy napotkasz kolejne oznaczenia. W~przypadku
rzeczownika możesz napotkać na przykład takie hasło:

\dictwordb{archit}[ˈar.t͡ʂit]
\dictterm{n}{(\textsc{pl} architji \xm{ˈar.t͡ʂit.ʐi}) (\textsc{fem} archita \xm{ˈar.t͡ʂi.ta}) przodek}
\skipline

Oznaczenie \textsc{pl} określa tutaj jaka jest forma mnoga danego rzeczownika.
Oznaczenie \textsc{fem} określa jaka jest forma w~rodzaju żeńskim. Możesz
napotkać również słowa pozbawione formy mnogiej oraz takie, które mają
oznaczenie (\textsc{fem}) i~występują tylko w~rodzaju żeńskim, na przykład:

\dictwordb{hu͞ekapa}[ˈxuɛ.ka.pa]
\dictterm{n}{(\textsc{fem}) komputer}
\skipline

W przypadku czasownika, możesz zobaczyć oznaczenie \textsc{pst}, które pokazuje
formę czasu przeszłego dla czasownika.

\dictwordb{kuanti}[ˈku.an.ti]
\dictterm{v}{(\textsc{pst} kuant \xm{ˈku.aŋt}) polować}
\skipline

Dla przymiotników przedstawiane są formy \textsc{comp} oraz \textsc{supl}, które
oznaczają odpowiednio stopień wyższy i~najwyższy. Oczywiście nie wszystkie
przymiotniki będą posiadały takie formy, stąd te oznaczenia nie będą występować
zawsze.

\dictwordb{wa͞ime}[ˈwai.mɛ]
\dictterm{adj}{(\textsc{comp} wa͞ime͞a \xm{ˈwai.mɛa}, \textsc{supl} wa͞ime͞am \xm{ˈwai.mɛam}) szeroki}
\skipline

Partykuły i~przyimki nie posiadają zazwyczaj dodatkowych skrótowców w~swoich
opisach.

\note{Napotkasz również tekst zapisany w~taki sposób. Są to dodatkowe uwagi na temat poprzedzającego słowa, definiujące jego źródłosłów lub kontekst kulturowy. Zwracaj uwagę na takie opisy, gdyż oprócz ciekawych informacji możesz napotkać sposób użycia w~mowie formalnej lub potocznej.}
\skipline

W części poświęconej gramatyce języka \emph{andro} będziesz napotykać różnorodne
przykłady zdań, w których stosowane będą glosy (ang. ,,glossing''), adnotacje,
które pozwolą po kolei na analizę konkretnych elementów gramatyki lub nawet
morfemów znaczeniowych.

Pierwsza linia to zapis w transkrypcji w języku \emph{andro}, kolejne dwie linie
to zapis w transkrypcji wraz z symbolami glosy, oraz właściwa glosa, używająca
zasad zgodnie z Leipzig Glossing Rules, wykorzystująca język angielski jako
metajęzyk opisu. Poniżej znajdzie się jedno lub więcej tłumaczeń na naturalny
język polski.

\begin{exe}
	\ex
    \trans Mi esi Koolder
	\gll  Mi esi Koolder \\
	  1SG be.PRS Koolder \\
	\glt  ,,Jestem Koolder'' \\ ,,Nazywam się Koolder''
  \end{exe}

\begin{exe}
	\ex
	\trans Ja muche esi ruko
	\gll  Ja muche esi ruk-o \\
	  DEM cat be.PRS black-ADJ \\
	\glt  ,,Ten kot jest czarny'' \\ ,,Ten konkretny kot jest czarny''
\end{exe}

\begin{exe}
	\ex
	\trans Nowirklo recha
	\gll  No-wirkl-o recha \\
	  NEG-importance-ADJ thing \\
	\glt  ,,Nieważna rzecz'' \\ ,,Nieistotna rzecz''
\end{exe}

\end{spacing}