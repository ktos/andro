\chapter{Morphology}
\label{ch:morphology}

In this chapter, the basic parts of speech and their rules for declination or
conjugation are presented. This section does not define the derivational
morphology, such as prefixes or compound words, which are described in a
separate chapter on \nameref{ch:derivational}, however some suffixes presented
here are actually the part of derivational morphology. The main goal of this
chapter is however to describe different parts of speech and their grammatical
properties.

\section{Nouns}

Nouns in \andro have two grammatical properties: number and class. Apart from
common nouns, there are also proper nouns (e.g. names) and nominalizations.

\subsection{Classes}

There are two main divisions of grammatical classes in \andro: the grammatical
gender and animacy. There are two grammatical genders: masculine (\M{}) and
feminine (\F{}), however because masculine is used in most of the contexts (like
in plural number), some linguists are proposing alternative names: feminine and
non-feminine. In this book, ``masculine'' will be used, however please note that
it does not automatically define the noun as being related to the male part of
the society.

Each noun has an assigned grammatical gender, and in most cases it is masculine.
Many nouns are also able to be transformed to a feminine grammatical gender, by
using \xo{-a} suffix, which is especially visible in case of names of different
job names.

\glossex{Edihit rufaler.}{edihi-t rufaler}{meet-PST farmer}{``I met a farmer.''}
\glossex{Edihit rufalera.}{edihi-t rufaler-a}{meet-PST farmer-F}{``I met a woman farmer.''}

As mentioned earlier, the masculine form is the default one, it may be also used
to describe a woman. So the first example may also mean that the farmer is a
woman, while the second one is specifying it directly. This ambiguity is one of
the reasons of naming the masculine grammatical gender as ``non-feminine''. In
fact, when describing people, one may use masculine grammatical gender to
effectively hide talking about the physical gender.

As for animated (\An{}) and inanimate nouns (\Inan{}), that classification is
visible only in very specific cases, related mostly to demonstrative pronouns,
so they will be mostly described in the later sections. This concept is almost
absent as it is a remnant of more complex grammatical classes system which was
available in Lono, one of the languages from the Keilic family.

The animate class refers, generally, to entities that are considered alive --
humans, animals and plants, however also is used with elements (fire, water,
earth, air and electricity) especially when talking about destructive force
(e.g. earthquake or tornado).

\subsection{Numbers}

Similarly to the typical Indo-European languages, \andro does recognize two
numbers: singular and plural. The number is usually marked by a morphological
suffix, modifying the basic version of the noun, which is in most cases
singular. Usually, nouns are countable, and some uncountable nouns are usually
expressed in the plural number only.

\glossex{Veydi sekot.}{veydi sekot}{see shadow}{``I see a shadow.''}
\glossex{Veydi sekotos.}{veydi sekot-os}{see shadow-PL}{``I see shadows.''}

Typical suffixes marking the plural number are: \xo{-ji}, \xo{-os} and \xo{-s},
sometimes also \xo{-i}. There is no regularity on creating a plural number from
a singular verb, apart from a few specific cases.

In case of words relating to humans, ending with \xo{-er}, the typical plural
number is achieved with \xo{-os} suffix.

As for the relation between grammatical gender and number there is an
interesting trick, because plural number is always masculine. There is no way to
express plural noun with feminine grammatical class.

\section{Pronouns}

\andro is rich in different kind of pronouns -- starting with the personal ones,
but also a set of possessive pronouns, demonstrative pronouns, interrogative
pronouns, relative and reflexive.

\subsection{Personal and possessive pronouns}

Similarly to the nouns, pronouns are inflected by grammatical class (gender and
animacy) and number. Personal pronouns are also the first element in which the
formality appears. \andro does not include many levels of formality, yet formal
pronouns are very important part of everyday life.

\begin{table}[]
  \caption{Personal and possessive pronouns}
  \label{tab:pronouns}
  \begin{tabular}{lll}
    \textbf{Person}    & \textbf{Pronoun} & \textbf{Possessive pronoun} \\
    \Fsg{}             & mi               & myi                         \\
    \Ssg{}             & ti               & tyi                         \\
    \Ssg{}.\M{}.\Frm{} & epié             & epil                        \\
    \Ssg{}.\F{}.\Frm{} & epiá             & epil                        \\
    \Tsg{}             & egi              & il                          \\
    \Tsg{}.\M{}        & egli             & il                          \\
    \Tsg{}.\F{}        & egla             & il                          \\
    \Tsg{}.\Inan{}     & che              & chyi                        \\
    \Tsg{}.\M{}.\Frm{} & epié             & epil                        \\
    \Tsg{}.\F{}.\Frm{} & epiá             & epil                        \\
    \Fpl{}             & noni             & niyi                        \\
    \Fpl{}.\Excl{}     & nodi             & nodyi                       \\
    \Spl{}             & toi              & tyoi                        \\
    \Tpl{}             & ego͞i             & egyi                        \\
    \Tpl{}.\Inan{}     & chei             & chey
  \end{tabular}
\end{table}

Possessive pronouns may be used in two cases: to specify possession, e.g. ``his
car'' or, sometimes, when marking the genitive case (\Gen{}). This is a remnant
of classical Old Nennekan in which there was an inflection of pronouns and nouns
on specific cases. While it is not an mistake to use regular form of pronoun in
this case, classical \randro{ardo andro} will present this behavior, especially
in older texts.

\glossex{Miam mi asati no, vimi toi kayetor ze kani!}{miam mi asati no vimi toi kayetor ze kani}{if.COND 1SG detain NEG then.COND 2PL criminal FUT catch}{``If you haven't stopped me, you would catch the criminal!''}

In possessive pronouns, one may notice a huge amount of \xo{-yi} suffix, which
is actually a suffix based on a possessive particle \randro{yi}, a very
important element which will be described in the section
\nameref{sec:possessive}.

As visible in table~\ref{tab:pronouns}, there is a small difference between
nouns and pronouns in case of specyfing the grammatical gender. The pronoun
\randro{egi} is used specifically for ambiguity -- if one does not want to
describe the third person gender, or the third person gender is not important,
or third person is identifying themselves as neither masculine nor feminine,
this one may be used. So the difference is that usually \randro{egli} is
actually stating masculine gender, in contract to nouns. \randro{egi} was very
regional, mostly used in the Republic of Nennek until late nine century where it
started to be popular in other regions of And́royas.

There is no such contrast in the case of formal pronouns \randro{epié} and
\randro{epiá}. They both may be used as second person and third person pronouns,
but -- similarly to the nouns -- ``masculine'' \randro{epié} may be also used as
general non-feminine pronoun.

Formal pronouns should be used when referring to strangers, when referring to
the people which names are not yet known or when referring to the people which
are higher in the social hierachy (e.g. the company boss, the client as the
shop's worker, etc.).

\glossex{Epil nome ya rede hemi?}{epil nome ya rede hemi}{3SG.FRM.POSS name TOP again EXH}{``What was your name again, sir?''}

\glossex{Epié in Nowaja ati razi?}{epié in Nowaja a-ti razi}{2SG.FRM in.LOC Nowaja one-ORD time}{``Are you first time in Nowaja?''}

\subsection{Demonstrative and relative pronouns}

Demonstrative pronouns in \andro feature similar morphological properties as
personal pronouns and nouns: gramatical class and animacy, but also relative
distance between the speaker, the listener and the object.

The basic two are \randro{je} and \randro{ja}, \Dem{} and \Dem{}.\F{},
respectively, while the feminine form is used rarely, and \randro{je} is used
when referring to any kind of object. Both of them are used only for animated beings,
and used in conjunction with third person pronoun when referring to humans.

% TODO: example of je with personal pronoun

\begin{table}[]
  \caption{Demonstrative pronouns}
  \label{tab:demonstrative}
  \begin{tabular}{llll}
    \textbf{Pronoun}          & \textbf{Speaker} & \textbf{Listener} & \textbf{Animate} \\
    \randro{je} / \randro{ja} & any              & any               & yes              \\
    \randro{heje}             & close            & close             & no               \\
    \randro{aje}              & far              & close             & no               \\
    \randro{dite}             & far              & far               & no               \\
    \randro{che}              & close            & far               & no               \\
    \randro{niger}            & far              & far               & no
  \end{tabular}
\end{table}

\randro{heje} as demonstrative pronoun used to mark an object (animate or
inanimate) which is close to both speaker and listener. \randro{aje} is used in
case of an inanimate object which is close to listener than to speaker.

For objects closer to the speaker than the listener, there is \randro{che}, used
for inanimate objects only, for inanimate objects far from both speakers,
\randro{dite} may be used.

As may be noticed, there is lack of demonstrative pronouns marking animate
beings apart from \randro{je} and \randro{ja}.

\randro{niger} is a interesting case: it is used almost exclusively in the
South-Eastern dialect and is used to mark objects far from the speaker and the
listener, similarly to \randro{dite}, but usually is used for things which are
not in the line of sight -- while \randro{dite} for far things, yet still
visible.

\subsection{Relative pronouns}

When building more complex sentences with subsentences, there may be a need to
talk about the objects in the parent sentence. To refer to that, relative
(\Rel{}) pronouns may be used. In \andro, most demonstrative pronouns may be
used as relative pronouns, apart from \randro{je} and \randro{ja}, in case of
which third person personal pronouns would be preferred (or sometimes ommited at
all, see syntax examples).

However, there is one generic relative pronoun which may be used for inanimate
objects: \randro{cheí} -- \Rel{}.\Inan{}. Almost a homonym of \randro{chei}
(\Tpl{}.\Inan{}), differentiated only by accent, \randro{cheí} is used only in a
relative context.

\section{Verbs}

The verb conjugates only through tenses: there is an infinitive/present form and
a~form of the past tense. Future tense, aspect and mood are expressed by
particles.

The verb in~the infinitive always ends in ~xo{-i}, in~the past form
usually with ~xo{-t}.



Wymowa bezokolicznika może nie przewidywać dyftongu, i~zazwyczaj go nie
przewiduje, na przykład \emph{chikai} \xm{'ʈ͡ʂi.ka.i} (śmiać się), jednak wielu
użytkowników zlewa tutaj końcówki \xo{-ai}, \xo{-ei}, \xo{-oi}.

\note{Do zasad dobrego wychowania należy poprawne wymawianie czasowników.
  Niektórzy użytkownicy przestrzegają tego aż do takiego stopnia, że w~ich ustach
  tworzy się coś w~stylu \xt{ˈʈ͡ʂi.ka.ʔi}.}
\skipline

\subsection{Nominalization}

Czasownik może być przekształcony do formy rzeczownikowej poprzez wykorzystanie
partykuły \xo{na}, nominalizatora (\Nmlz{}), który bardzo często jest stosowany w
formie przyrostka.

\glossex{Chiwi ozeyo}{chiwi ozeyo}{write.PRS difficult.ADV}{,,Pisać jest trudno.'' / ,,Pisanie jest trudne.''}

\glossex{Chiwina esi ozeyo}{chiwi-na esi ozeyo}{write-NMLZ be.PRS difficult.ADJ}{,,Pisanie jest trudne.''}

\glossex{Chiwi na esi ozeyo}{chiwi-na esi ozeyo}{write NMLZ be.PRS difficult.ADJ}{,,Pisanie jest trudne.''}

\section{Adjectives and adverbs}

Przymiotniki są tworzone zarówno od rzeczowników, jak i od czasowników.

Przymiotniki są stopniowane w~3 stopniach (w większości), gdzie drugi stopień
zazwyczaj ma końcówkę \xo{-e͞a}, a~trzeci stopień \xo{-e͞am}. Przymiotnik nie
ulega odmianie przez rodzaje gramatyczne ani liczby.

Istnieją przymiotniki, które powstały przez dodanie morfemu \xo{no-} przed
pewnym rdzeniem, wyrażające przeciwność, np. \emph{anper} \xm{ˈan.pɛr} (mokry)
i~\emph{nonper} \xm{ˈnɔn.pɛr} (suchy) oraz przymiotniki z~prefiksem \xo{mo-},
oznaczającym negatywne zestopniowanie, np. \emph{leder} \xm{ˈlɛ.dɛr} (oszczędny)
i~\emph{moleder} \xm{ˈmɔ.lɛ.dɛr} (skąpy).

Pozycja przymiotnika w~zdaniu ma decydujące znaczenie w~określeniu do którego
rzeczownika się odnosi.

\glossex{Mi pazi karié himji.}{mi pazi karié him-ji}{1SG like beautiful woman-PL}{,,Lubię piękne kobiety.''}

\glossex{Mi pazi karié himji e chid kahokapataji.}{mi pazi karié him-ji e chid kahokapata-ji.}{1SG like beautiful woman-PL and fast aeroplane-PL}{,,Lubię piękne kobiety i szybkie samoloty.''}

\glossex{Waril vayarji esi lipe aloser che karié sipalima.}{Waril vayar-ji esi lipe aloser che karié sipalima}{long.ADJ travel-PL be.PRS good.ADJ source GEN beautiful story}{,,Długie podróże są dobrym źródłem dla pięknej opowieści.''}

Przymiotniki mogą pełnić rolę przysłówka, odpowiadając na pytanie ,,jak'' i
określając czasowniki, jednak w odróżnieniu od stosowania ich w roli określenia
rzeczownika stosowane są \textbf{po} czasowniku w szyku.

\glossex{Ko͞e ti seiti? Lipe.}{ko͞e ti seiti lipe}{how.Q 2SG feel good.ADV}{,,Jak się czujesz? Dobrze.''}

Przymiotnik od rzeczownika-nazwy własnej tworzony jest przez przyrostek
dzierżawczy \xo{-yi} (\xt{ʏ}) lub przyrostek \xo{-o}: pierwszy oznacza
przynależność, a drugi -- cechę. Stąd na przykład \emph{zokemo}, metalowy od
\emph{zokem} (cecha odnosząca się do przedmiotu), ale na już na przykład
\emph{erokwyirome}, ,,mięso drobiowe'' to dosłownie ,,mięso (pochodzące) z
kurczaka''. Różnica ta jest bardzo delikatna.

Same nazwy własne, także w~formie przymiotnika odrzeczownikowego, są
w~transkrypcji zawsze zapisywane wielką literą, w~ alfabecie naturalnym zawsze
fonetycznie, nigdy sylabicznie.

Imiesłów przymiotnikowy tworzony jest poprzez zmianę -i na -o w~bezokoliczniku.

\glossex{Kaho beykar kahi in ari.}{kah-o beykar kahi on ari}{fly-ADJ snake fly.PRS on.LOC sky}{,,Latający wąż lata na niebie.''}

\glossex{Andro ya mosto inrat.}{andro ya most-o inrat}{andro TOP create-ADJ language}{,,Andro to stworzony język.''}

Zamiast imiesłowów można też stosować partykułę \emph{chu} lub \emph{che},
przyimek lub zaimek względny (\Rel{}):

\glossex{Pakopo rujalar kant hetay ne͞a ji͞ari͞o in Lublin.}{pakop-o rujalar kan-t hetay ne͞a ji͞ari͞o in Lublin}{worry-ADJ man sing-PST today near.physically garden in.LOC Lublin}{,,Zmartwiony mężczyzna śpiewał dzisiaj niedaleko ogrodu w Lublinie.''}

\glossex{Rujalar chu vi pakot kant hetay ne͞a ji͞ari͞o in Jechuf.}{rujalar chu vi pako-t kan-t hetay ne͞a ji͞ari͞o in Jechuf}{man REL IPFV worry-PST sing-PST today near.physically garden in Rzeszów}{,,Mężczyzna, który się martwił, śpiewał dzisiaj niedaleko ogrodu w Rzeszowie.''}

\glossex{Muche chu altur vi fart.}{muche chu altur vi far-t}{cat REL noise IPFV make-PST}{,,Kot, który hałasował.''}

\glossex{Altur faro muche.}{altur far-o muche}{noise make-ADJ cat}{,,Robiący hałas kot.''}

\glossex{Alturo muche.}{altur-o muche}{noise-ADJ cat}{,,Hałasujący kot.''}

\subsection{Comparatives}
\subsection{Superlatives}

\subsection{Adverbs}

\section{Numbers}

Liczebniki od 0 do 9 to, po kolei: \emph{wa}, \emph{a}, \emph{ka}, \emph{sa},
\emph{ta}, \emph{na}, \emph{cha}, \emph{ma}, \emph{ya}, \emph{ra}.

\begin{table}[ht]
  \centering
  \caption{Podstawowe liczebniki}
  \begin{tabular}{ccc} \toprule
    zero     & 0  & wa  \\
    jeden    & 1  & a   \\
    dwa      & 2  & ka  \\
    trzy     & 3  & sa  \\
    cztery   & 4  & ta  \\
    pięć     & 5  & na  \\
    sześć    & 6  & cha \\
    siedem   & 7  & ma  \\
    osiem    & 8  & ya  \\
    dziewięć & 9  & ra  \\
    dziesięć & 10 & awa \\\bottomrule
  \end{tabular}
  \label{tab:numerals}
\end{table}

Najprostszym sposobem tworzenia liczebnika dla liczb większych od 9 jest
ustawienie w~kolejności zapisu dziesiętnego kolejnych słów określających cyfry,
tj. 17 to \emph{ama}, 123 to \emph{akasa}, a~241 to \emph{kataa}.

\begin{table}[ht]
  \centering
  \caption{Liczebniki wyższe}
  \begin{tabular}{ccc} \toprule
    jedenaście   & 11      & aa      \\
    dwanaście    & 12      & aka     \\
    dwadzieścia  & 20      & kawa    \\
    trzydzieści  & 30      & sawa    \\
    czterdzieści & 40      & tawa    \\
    sto          & 100     & aḱaywa  \\
    tysiąc       & 1000    & aśaywa  \\
    milion       & 1000000 & aćhaywa \\\bottomrule
  \end{tabular}
  \label{tab:numerals2}
\end{table}

Niektórzy użytkownicy języka nie lubią formy pozycyjnej, w której wyrazy
potrafią być problematyczne, np. \emph{aaa} jako sto jedenaście. Stąd używana
jest również partykuła \emph{jo}, która oznacza ,,oraz'', i jest stosowana do
dodawania wartości wypowiedzianych liczebników.

Przykładowo, \emph{aśaywa jo kata} to tysiąc oraz dwadzieścia cztery (1024),
\emph{aḱaywa jo a} to dosłownie ,,sto oraz jeden'' (101), natomiast
\emph{aḱaywa jo awa jo a} czyli ,,sto oraz dziesięć oraz jeden'' to 111


\subsection{Large numbers}

Dla większych
liczb było by to jednak zbyt problematyczne (np. 1000 to by było
\emph{awawawa}), dlatego można stosować przyrostek <-y>, określający liczbę
powtórzeń, w~stosunku do sylaby określającej liczbę powtórzeń, a~potem cyfrę
powtarzaną, na przykład \emph{aśaywa} -- dosłownie: jeden i~trójka zer. Stąd
milion to \emph{aćhaywa}, a~100023 to \emph{aśaywakasa}. Akcent kładziony jest
zawsze na sylabę, która zawiera przyrostek <-y>.

W sytuacji, kiedy jest kilka sylab z wrostkiem <-y>, akcent kładziony jest na
ostatnią taką sylabę.



\subsection{Ordinal numbers}

Przyrostek <-ti> określa liczebnik porządkowy, tj. \emph{kati} to drugi, a
\emph{sasati} -- trzydziesty trzeci.

\begin{table}[ht]
  \centering
  \caption{Podstawowe liczebniki porządkowe}
  \begin{tabular}{cc} \toprule
    pierwszy  & ati   \\
    drugi     & kati  \\
    trzeci    & sati  \\
    czwarty   & tati  \\
    piąty     & nati  \\
    szósty    & chati \\
    siódmy    & mati  \\
    ósmy      & yati  \\
    dziewiąty & rati  \\
    dziesiąty & awati \\\bottomrule
  \end{tabular}
  \label{tab:numerals3}
\end{table}

\subsection{Fractions}

Liczebnik ułamkowy tworzy się przez infiks <-je->, na przykład \emph{ajeka} to
jedna druga, \emph{najera} -- pięć dziewiątych, a~\emph{awajerana} to 10/95.

\begin{table}[ht]
  \centering
  \caption{Ułamki}
  \begin{tabular}{cc} \toprule
    połowa        & ajeka  \\
    jedna trzecia & ajesa  \\
    dwie trzecie  & kajesa \\
    jedna czwarta & ajeta  \\
    trzy czwarte  & sajeta \\
    jedna siódma  & ajema  \\\bottomrule
  \end{tabular}
  \label{tab:numerals4}
\end{table}

Możliwe jest również użycia słowa \emph{jerya} oznaczającego ,,przecinek'',
w~taki sposób, że:

\glossex{Egi doloze esi a jerya rasa metr.}{Egi doloze esi a jerya rasa metr\\
  3SG.POSS height be.PRS one comma 83 meter}{,,Jego wzrost to 1,83 metra.''}

\section{Prepositions}

\section{Particles}