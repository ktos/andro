\chapter{Morphology}
\label{ch:morphology}

In this chapter, the basic parts of speech and their rules for declination or
conjugation are presented. This section does not define the derivational
morphology, such as prefixes or compound words, which are described in a
separate chapter on \nameref{ch:derivational}, however some suffixes presented
here are actually the part of derivational morphology. The main goal of this
chapter is however to describe different parts of speech and their grammatical
properties.

\section{Nouns}
\label{sec:morph-nouns}

Nouns in \andro have two grammatical properties: number and class. Apart from
common nouns, there are also proper nouns (e.g. names) and nominalizations.

\subsection{Classes}

There are two main divisions of grammatical classes in \andro: the grammatical
gender and animacy. There are two grammatical genders: masculine (\M{}) and
feminine (\F{}), however because masculine is used in most of the contexts (like
in plural number), some linguists are proposing alternative names: feminine and
non-feminine. In this book, ``masculine'' will be used, however please note that
it does not automatically define the noun as being related to the male part of
the society.

Each noun has an assigned grammatical gender, and in most cases it is masculine.
Many nouns are also able to be transformed to a feminine grammatical gender, by
using \xo{-a} suffix, which is especially visible in case of names of different
job names.

\glossex{Edihit rufaler.}{edihi-t rufaler}{meet-PST farmer}{``I met a farmer.''}
\glossex{Edihit rufalera.}{edihi-t rufaler-a}{meet-PST farmer-F}{``I met a (female) farmer.''}

As mentioned earlier, the masculine form is the default one, it may be also used
to describe a woman. So the first example may also mean that the farmer is a
woman, while the second one is specifying it directly. This ambiguity is one of
the reasons of naming the masculine grammatical gender as ``non-feminine''. In
fact, when describing people, one may use masculine grammatical gender to
effectively hide talking about the physical gender.

As for animated (\An{}) and inanimate nouns (\Inan{}), that classification is
visible only in very specific cases, related mostly to demonstrative pronouns,
so they will be mostly described in the later sections. This concept is almost
absent as it is a remnant of more complex grammatical classes system which was
available in Lono, one of the languages from the Keilic family.

The animate class refers, generally, to entities that are considered alive --
humans, animals and plants, however also is used with elements (fire, water,
earth, air and electricity) especially when talking about destructive force
(e.g. earthquake or tornado); and all kinds of machines which are running ``on
their own'', like engines.

\note{Artificial Intelligence beings, as well as robots, are also considered
    animate.}

\subsection{Numbers}

Similarly to the typical Indo-European languages, \andro does recognize two
numbers: singular and plural. The number is usually marked by a morphological
suffix, modifying the basic version of the noun, which is in most cases
singular. Usually, nouns are countable, and some uncountable nouns are usually
expressed in the plural number only.

\glossex{Veydi sekot.}{veydi sekot}{see shadow}{``I see a shadow.''}
\glossex{Veydi sekotos.}{veydi sekot-os}{see shadow-PL}{``I see shadows.''}

Typical suffixes marking the plural number are: \xo{-ji}, \xo{-os} and \xo{-s},
sometimes also \xo{-i}. There is no regularity on creating a plural number from
a singular verb, apart from a few specific cases.

In case of words relating to humans, ending with \xo{-er}, the typical plural
number is achieved with \xo{-os} suffix.

As for the relation between grammatical gender and number there is an
interesting trick, because plural number is always masculine. There is no way to
express plural noun with feminine grammatical class.

\section{Pronouns}
\label{sec:morph-pronouns}

\andro is rich in different kind of pronouns -- starting with the personal ones,
but also a set of possessive pronouns, demonstrative pronouns, interrogative
pronouns, relative and reflexive.

\subsection{Personal and possessive pronouns}

Similarly to the nouns, pronouns are inflected by grammatical class (gender and
animacy) and number. Personal pronouns are also the first element in which the
formality is visible. \andro does not include many levels of formality, yet
formal pronouns are an important part of everyday life.

\begin{table}[]
    \caption{Personal and possessive pronouns}
    \label{tab:pronouns}
    \begin{tabular}{lll}
        \textbf{Person}    & \textbf{Pronoun} & \textbf{Possessive pronoun} \\
        \Fsg{}             & mi               & myi                         \\
        \Ssg{}             & ti               & tyi                         \\
        \Ssg{}.\M{}.\Frm{} & epié             & epil                        \\
        \Ssg{}.\F{}.\Frm{} & epiá             & epil                        \\
        \Tsg{}             & egi              & il                          \\
        \Tsg{}.\M{}        & egli             & il                          \\
        \Tsg{}.\F{}        & egla             & il                          \\
        \Tsg{}.\Inan{}     & che              & chyi                        \\
        \Tsg{}.\M{}.\Frm{} & epié             & epil                        \\
        \Tsg{}.\F{}.\Frm{} & epiá             & epil                        \\
        \Fpl{}             & noni             & niyi                        \\
        \Fpl{}.\Excl{}     & nodi             & nodyi                       \\
        \Spl{}             & toi              & tyoi                        \\
        \Tpl{}             & ego͞i             & egyi                        \\
        \Tpl{}.\Inan{}     & chei             & chey
    \end{tabular}
\end{table}

Possessive pronouns are used in two cases: to specify possession, e.g. ``his
car'' or, rarely, when marking the genitive (\Gen{}) or dative (\Dat{}) case.
This is a remnant of classical Old Nennekan in which there was an inflection of
pronouns and nouns on specific cases. While it is not an mistake to use regular
form of pronoun in this case, classical \ardo will present this behavior,
especially in older texts, much rarely in the informal context.

\glossex{Miam mi asati no, vimi toi kayetor ze kani!}{miam mi asati no vimi toi kayetor ze kani}{if.COND 1SG detain NEG then.COND 2PL criminal FUT catch}{``If you haven't stopped me, you would catch the criminal!''}

In possessive pronouns, one may notice a huge amount of \xo{-yi} suffix, which
is actually a suffix based on a possessive particle \randro{yi}, a very
important element which will be described in the section
\nameref{sec:cases}.

As visible in table~\ref{tab:pronouns}, there is a small difference between
nouns and pronouns in case of specifying the grammatical gender. The pronoun
\randro{egi} is used specifically for ambiguity -- if one does not want to
describe the third person gender, or the third person gender is not important,
or third person is identifying themselves as neither masculine nor feminine,
this one may be used. So the difference is that usually \randro{egli} is
actually stating masculine gender, in contract to nouns. \randro{egi} was very
regional, and were mostly used in the Republic of Nennek until late nine
century, where it started to be popular in other regions of And́royas.

There is no such contrast in the case of formal pronouns \randro{epié} and
\randro{epiá}. They both may be used as second and third person pronouns,
but -- similarly to the nouns -- ``masculine'' \randro{epié} may be also used as
general non-feminine pronoun.

Formal pronouns should be used when referring to strangers, when referring to
the people which names are not yet known or when referring to the people which
are higher in the social hierarchy (e.g. the company boss, the client as the
shop's worker, etc.).

\glossex{Epil nome ya rede hemi?}{epil nome ya rede hemi}{3SG.FRM.POSS name TOP again EXH}{``What was your name again, sir?''}

\glossex{Epié in Nowaja ati razi?}{epié in Nowaja a-ti razi}{2SG.FRM in.LOC Nowaja one-ORD time}{``Are you first time in Nowaja?''}

\subsection{Demonstrative and relative pronouns}
\label{sec:dem}

Demonstrative pronouns in \andro feature similar morphological properties as
personal pronouns and nouns: grammatical class and animacy, but also relative
distance between the speaker, the listener and the object.

The basic two are \randro{je} and \randro{ja}, \Dem{} and \Dem{}.\F{},
respectively. It is worth noting that \randro{ja} is not used at all in some
dialects, most notably Nennekan.

\glossex{Je muche esi ruó.}{je muche esi ruo}{DEM cat COP black}{``This cat is black.''}

\begin{table}[]
    \caption{Demonstrative pronouns}
    \label{tab:demonstrative}
    \begin{tabular}{llll}
        \textbf{Pronoun} & \textbf{Speaker} & \textbf{Listener} & \textbf{Animate} \\
        \randro{je}                                                                \\ \randro{ja} & any              & any               & yes              \\
        \randro{heje}    & close            & close             & no               \\
        \randro{aje}     & far              & close             & no               \\
        \randro{dite}    & far              & far               & no               \\
        \randro{che}     & close            & far               & no               \\
        \randro{niger}   & far              & far               & no
    \end{tabular}
\end{table}

\randro{heje} as demonstrative pronoun used to mark an object (animate or
inanimate) which is close to both speaker and listener. \randro{aje} is used in
case of an inanimate object which is close to listener than to speaker.

\note{One may wonder -- \randro{je} and \randro{heje} sound a bit similar, so
    why only \randro{je} is marking animacy? In the Proto-Lono language,
    \randro{*heje} and \randro{*aje} were used for animated beings, but during the
    language evolution, their meaning changed.}

For objects closer to the speaker than the listener, there is \randro{che}, used
for inanimate objects only, for inanimate objects far from both speakers,
\randro{dite} may be used.

As may be noticed, there is a lack of demonstrative pronouns marking animated
beings apart from \randro{je} and \randro{ja}, and they must be used for both
close and far beings.

% TODO: czy jest jakaś metoda na powiedzenie "Tamten człowiek"?

\randro{niger} is an interesting case: it is used almost exclusively in the
South-Eastern dialect, and it is used to mark objects far from the speaker and
the listener, similarly to \randro{dite}, but usually is used for things which
are not in the line of sight -- while \randro{dite} for far things, yet still
visible.

\section{Verbs}
\label{sec:morph-verbs}

The verb conjugates only through tenses: there is an infinitive\\present form and
the~form of the past tense. Future tense, aspect and mood are expressed by
particles.

The verb in~the infinitive always ends in \xo{-i}. This causes an interesting
problem -- because pronunciation of the infinitive form usually does not include
a diphtong, e.g. \tandro{chikai}{to laugh} is \xm{'ʈ͡ʂi.ka.i}, many users are
pronouncing the ending \xo{-ai} as \xt{ay} or \xt{aʲi} (and similar phonemic
change appears for \xo{-ei} or \xo{-oi}).

\note{In \ardo, and in families aspiring to the ``higher'' classes, there was a
    huge emphasis on proper pronunciation of infinitive verbs, as this was one of
    the features distinguishing from the lower social classes. That caused some
    people pronouncing with a visible pause, change of accent or even glottal stop
    \xt{ˈʈ͡ʂi.ka.ʔi}.}

% TODO: opisać formę czasu przeszłego

\subsection{Nominalization}

A verb can be changed into a noun by the usage of \randro{na} particle, the
nominalizer (\Nmlz{}). It could be used in two alternative variants: as a
\textbf{head-final} particle, but also as a suffix to the original verb.
Nominalized verb is the name of the act of doing something.

\glossex{Chiwi ozeyo}{chiwi ozeyo}{write.PRS difficult.ADV}{``It is hard to write''.}

\glossex{Chiwina esi ozeyo}{chiwi-na esi ozeyo}{write-NMLZ COP difficult.ADJ}{``Writing is hard.''}

\glossex{Chiwi na esi ozeyo}{chiwi na esi ozeyo}{write NMLZ COP difficult.ADJ}{``Writing is hard.''}

\section{Adjectives and adverbs}
\label{sec:morph-adjectives}

There is no huge distinction between adjectives and adverbs in \andro, most
linguists are using term ``adjective'' for both of them, as many may modify both
noun and verb. From the morphological point of view, the most important is that
both of them are not inflected by number or grammatical class, but by their
degree.

There are three degrees: basic, comparative (\Comp{}) and superlative (\Supl{}).
In most cases, basic form of the adjective is ending with \xo{-o}, while
comparative form is ending with \xo{-e͞a}, and superlative -- \xo{-e͞am}.
Comparative form is obviously used in comparisons.

\glossex{Zetay ze esi votepo͞e o hetay.}{zetay ze esi votepo-e o hetay}{tomorrow FUT COP hot-COMP than today}{``Tomorrow there will be hotter than today.''}

Adjectives may be created out of nouns and verbs. In case of nouns, it may be
created using either \xo{-yi}, a possessive suffix, or \xo{-o}, a feature
suffix. The difference can be very nuanced.

\glossex{Zokemo taris.}{zokem-o taris}{metal-ADJ blade}{``A metal blade.''}

\glossex{Erokwyirome.}{erokw-yi-rome}{chicken-POSS-meat}{``Chicken meat.''}

In the presented example, \randro{erokwyirome} is using the \xo{-yi} marker for
\randro{erokwa}, ``chicken'', saying that the meat was actually possessed by the
chicken. In case of the metal blade, \randro{zokemo} is used to say ``made from
the metal''.

As of verbs, only \xo{-o} suffix is used -- the basic form of the verb (ending
with \xo{-i}) has their suffix changed into \xo{-o} to create a new adjective.

\glossex{Ruó ekwen esi riguto.}{ruo ekwen esi rigut-o}{black horse COP lose-ADJ}{``The black horse is a losing one.''}

\glossex{Kaho beykar kahi in ari.}{kah-o beykar kahi on ari}{fly-ADJ snake fly.PRS on.LOC sky}{``The flying snake is flying in the sky.''}

\glossex{Andro ya mosto inrat.}{andro ya most-o inrat}{andro TOP create-ADJ language}{``Andro is a created language.''}

\section{Numbers}
\label{sec:morph-numbers}

\andro is using a decimal, positional, numbering system. The basic numbers from
0 to 9 are: \emph{wa}, \emph{a}, \emph{ka}, \emph{sa}, \emph{ta}, \emph{na},
\emph{cha}, \emph{ma}, \emph{ya}, \emph{ra}.

\begin{table}[ht]
    \centering
    \caption{Basic numbers}
    \begin{tabular}{ccc} \toprule
        zero  & 0  & wa  \\
        one   & 1  & a   \\
        two   & 2  & ka  \\
        three & 3  & sa  \\
        four  & 4  & ta  \\
        five  & 5  & na  \\
        six   & 6  & cha \\
        seven & 7  & ma  \\
        eight & 8  & ya  \\
        nine  & 9  & ra  \\
        ten   & 10 & awa \\\bottomrule
    \end{tabular}
    \label{tab:numerals}
\end{table}

The twist is, however, in the convention for creating bigger numbers, because
the most basic form is a positional one: just the numbers, in the order of
decimals, from the highest number.

\glossex{ama}{a-ma}{one-seven}{``Seventeen.''}

\glossex{akasa}{a-ka-sa}{one-two-three}{``One hundred and twenty three.''}

\glossex{kataá}{ka-ta-a}{two-four-one}{``Two hundred forty one.''}

In case of numbers ending with one, \randro{a}, accent is moving to the last
syllable for easier recognition -- for example, \xm{ka.ta.ˈa}. You may notice
this way is horribly inefficient for larger numbers, e.g. a million would be
\randro{awawawawawawa}, while even simple ``one hundred and eleven'' --
\randro{aaá}. In case of duplication of a same number, an \xo{-y-} infix is
used, meaning the ``number of times''.

\glossex{kaśywata}{ka-sa-y-wa-ta}{two-three-times-zero-four}{``Two, three times zero and four.'' \\ ``Twenty thousand four.'' \\ ``20004''}

In the case of usage of \xo{-y-} infix, the accent is defaulting to the syllable
with the infix, if there is no duplication of numbers -- in the former, the
accent is switching to the last syllable again. If there are multiple \xo{-y-}
infixes, the accent is on the last infixed syllable.

\begin{table}[ht]
    \centering
    \caption{Higher numbers}
    \begin{tabular}{ccc} \toprule
        eleven    & 11      & aa      \\
        twelve    & 12      & aka     \\
        twenty    & 20      & kawa    \\
        thirty    & 30      & sawa    \\
        forty     & 40      & tawa    \\
        a hundred & 100     & aḱaywa  \\
        thousand  & 1000    & aśaywa  \\
        million   & 1000000 & aćhaywa \\\bottomrule
    \end{tabular}
    \label{tab:numerals2}
\end{table}

This system is far from being ideal and is popular for the numbers up to a
thousand. In the case of even higher numbers, the users are usually switching to
the usage of the \randro{jo} particle, meaning ``and'', but the same particle
may be used also for smaller numbers.

\glossex{aśaywa jo kata}{a-śa-y-wa jo ka-ta}{one-three-times-zero CONJ two-four}{``A thousand and twenty-four.'' \\ ``1024''}

\glossex{aḱaywa jo awa jo a}{a-ḱa-y-wa jo a-wa jo a}{one-two-times-zero CONJ one-zero CONJ one}{``A hundred and ten and one.'' \\ ``A hundred eleven.'' \\ ``111''}

\note{It is being said sometimes that everybody in And́royas is using a scientific notation for large numbers, saying things like ``one and nine zeroes''.}

\subsection{Ordinal numbers}

Ordinal numbers are being created from the regular numbers with \xo{-ti} suffix.

\begin{table}[ht]
    \centering
    \caption{Basic ordinals}
    \begin{tabular}{cc} \toprule
        first   & ati   \\
        second  & kati  \\
        third   & sati  \\
        fourth  & tati  \\
        fifth   & nati  \\
        sixth   & chati \\
        seventh & mati  \\
        eighth  & yati  \\
        nineth  & rati  \\
        tenth   & awati \\\bottomrule
    \end{tabular}
    \label{tab:numerals3}
\end{table}

\subsection{Fractions}

Fractional numbers are created using the \xo{-je-} infix, marking the division.

\glossex{ajeka}{a-je-ka}{one-divided-two}{``A half.'' \\ ``1/2''}
\glossex{najera}{na-je-ra}{five-divided-nine}{``Five of nine.'' \\ ``5/9''}

\begin{table}[ht]
    \centering
    \caption{Fractions}
    \begin{tabular}{cc} \toprule
        a half                 & ajeka  \\
        one third              & ajesa  \\
        two thirds             & kajesa \\
        one fourth (a quarter) & ajeta  \\
        three fourths          & sajeta \\
        one seventh            & ajema  \\\bottomrule
    \end{tabular}
    \label{tab:numerals4}
\end{table}

A very common is also using a \tandro{jerya}{comma}.

\glossex{Egi doloze esi a jerya rasa meterdi.}{egi doloze esi a jerya ra-sa meter-di}{3SG.POSS height COP one comma eight-three meter-PL}{``His height is 1.83 meters.''}

\section{Particles}
\label{sec:morph-parts}

\andro has many particles, and is using them for most grammatical functions --
they are used as case markers (for example instrumental case particle
\randro{da}), negation (\randro{no}), interrogation, possession, tense, mood,
aspect and many more.

From the morphological point of view, the most important note is that particles
are not inflected. Some, may be used in the form of suffix, e.g. \randro{na},
the nominalizer or \randro{yi}, the possessive particle.