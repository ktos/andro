\chapter{Morphology}
\label{ch:morphology}

\section{Nouns}
\subsection{Genders}
\subsection{Numbers}

\section{Pronouns}



\subsection{Personal pronouns}
\subsection{Possesive pronouns}

\section{Verbs}

The verb conjugates only through tenses: there is an infinitive/present form and
a~form of the past tense. Future tense, aspect and mood are expressed by
particles.

The verb in~the infinitive always ends in ~xo{-i}, in~the past form
usually with ~xo{-t}.



Wymowa bezokolicznika może nie przewidywać dyftongu, i~zazwyczaj go nie
przewiduje, na przykład \emph{chikai} \xm{'ʈ͡ʂi.ka.i} (śmiać się), jednak wielu
użytkowników zlewa tutaj końcówki \xo{-ai}, \xo{-ei}, \xo{-oi}.

\note{Do zasad dobrego wychowania należy poprawne wymawianie czasowników.
  Niektórzy użytkownicy przestrzegają tego aż do takiego stopnia, że w~ich ustach
  tworzy się coś w~stylu \xt{ˈʈ͡ʂi.ka.ʔi}.}
\skipline

\subsection{Nominalization}

Czasownik może być przekształcony do formy rzeczownikowej poprzez wykorzystanie
partykuły \xo{na}, nominalizatora (\Nmlz{}), który bardzo często jest stosowany w
formie przyrostka.

\glossex{Chiwi ozeyo}{chiwi ozeyo}{write.PRS difficult.ADV}{,,Pisać jest trudno.'' / ,,Pisanie jest trudne.''}

\glossex{Chiwina esi ozeyo}{chiwi-na esi ozeyo}{write-NMLZ be.PRS difficult.ADJ}{,,Pisanie jest trudne.''}

\glossex{Chiwi na esi ozeyo}{chiwi-na esi ozeyo}{write NMLZ be.PRS difficult.ADJ}{,,Pisanie jest trudne.''}

\section{Adjectives}

Przymiotniki są tworzone zarówno od rzeczowników, jak i od czasowników.

Przymiotniki są stopniowane w~3 stopniach (w większości), gdzie drugi stopień
zazwyczaj ma końcówkę \xo{-e͞a}, a~trzeci stopień \xo{-e͞am}. Przymiotnik nie
ulega odmianie przez rodzaje gramatyczne ani liczby.

Istnieją przymiotniki, które powstały przez dodanie morfemu \xo{no-} przed
pewnym rdzeniem, wyrażające przeciwność, np. \emph{anper} \xm{ˈan.pɛr} (mokry)
i~\emph{nonper} \xm{ˈnɔn.pɛr} (suchy) oraz przymiotniki z~prefiksem \xo{mo-},
oznaczającym negatywne zestopniowanie, np. \emph{leder} \xm{ˈlɛ.dɛr} (oszczędny)
i~\emph{moleder} \xm{ˈmɔ.lɛ.dɛr} (skąpy).

Pozycja przymiotnika w~zdaniu ma decydujące znaczenie w~określeniu do którego
rzeczownika się odnosi.

\glossex{Mi pazi karié himji.}{mi pazi karié him-ji}{1SG like beautiful woman-PL}{,,Lubię piękne kobiety.''}

\glossex{Mi pazi karié himji e chid kahokapataji.}{mi pazi karié him-ji e chid kahokapata-ji.}{1SG like beautiful woman-PL and fast aeroplane-PL}{,,Lubię piękne kobiety i szybkie samoloty.''}

\glossex{Waril vayarji esi lipe aloser che karié sipalima.}{Waril vayar-ji esi lipe aloser che karié sipalima}{long.ADJ travel-PL be.PRS good.ADJ source GEN beautiful story}{,,Długie podróże są dobrym źródłem dla pięknej opowieści.''}

Przymiotniki mogą pełnić rolę przysłówka, odpowiadając na pytanie ,,jak'' i
określając czasowniki, jednak w odróżnieniu od stosowania ich w roli określenia
rzeczownika stosowane są \textbf{po} czasowniku w szyku.

\glossex{Ko͞e ti seiti? Lipe.}{ko͞e ti seiti lipe}{how.Q 2SG feel good.ADV}{,,Jak się czujesz? Dobrze.''}

Przymiotnik od rzeczownika-nazwy własnej tworzony jest przez przyrostek
dzierżawczy \xo{-yi} (\xt{ʏ}) lub przyrostek \xo{-o}: pierwszy oznacza
przynależność, a drugi -- cechę. Stąd na przykład \emph{zokemo}, metalowy od
\emph{zokem} (cecha odnosząca się do przedmiotu), ale na już na przykład
\emph{erokwyirome}, ,,mięso drobiowe'' to dosłownie ,,mięso (pochodzące) z
kurczaka''. Różnica ta jest bardzo delikatna.

Same nazwy własne, także w~formie przymiotnika odrzeczownikowego, są
w~transkrypcji zawsze zapisywane wielką literą, w~ alfabecie naturalnym zawsze
fonetycznie, nigdy sylabicznie.

Imiesłów przymiotnikowy tworzony jest poprzez zmianę -i na -o w~bezokoliczniku.

\glossex{Kaho beykar kahi in ari.}{kah-o beykar kahi on ari}{fly-ADJ snake fly.PRS on.LOC sky}{,,Latający wąż lata na niebie.''}

\glossex{Andro ya mosto inrat.}{andro ya most-o inrat}{andro TOP create-ADJ language}{,,Andro to stworzony język.''}

Zamiast imiesłowów można też stosować partykułę \emph{chu} lub \emph{che},
przyimek lub zaimek względny (\Rel{}):

\glossex{Pakopo rujalar kant hetay ne͞a ji͞ari͞o in Lublin.}{pakop-o rujalar kan-t hetay ne͞a ji͞ari͞o in Lublin}{worry-ADJ man sing-PST today near.physically garden in.LOC Lublin}{,,Zmartwiony mężczyzna śpiewał dzisiaj niedaleko ogrodu w Lublinie.''}

\glossex{Rujalar chu vi pakot kant hetay ne͞a ji͞ari͞o in Jechuf.}{rujalar chu vi pako-t kan-t hetay ne͞a ji͞ari͞o in Jechuf}{man REL IPFV worry-PST sing-PST today near.physically garden in Rzeszów}{,,Mężczyzna, który się martwił, śpiewał dzisiaj niedaleko ogrodu w Rzeszowie.''}

\glossex{Muche chu altur vi fart.}{muche chu altur vi far-t}{cat REL noise IPFV make-PST}{,,Kot, który hałasował.''}

\glossex{Altur faro muche.}{altur far-o muche}{noise make-ADJ cat}{,,Robiący hałas kot.''}

\glossex{Alturo muche.}{altur-o muche}{noise-ADJ cat}{,,Hałasujący kot.''}

\subsection{Comparatives}
\subsection{Superlatives}

\section{Adverbs}

\section{Numbers}

Liczebniki od 0 do 9 to, po kolei: \emph{wa}, \emph{a}, \emph{ka}, \emph{sa},
\emph{ta}, \emph{na}, \emph{cha}, \emph{ma}, \emph{ya}, \emph{ra}.

\begin{table}[ht]
  \centering
  \caption{Podstawowe liczebniki}
  \begin{tabular}{ccc} \toprule
    zero     & 0  & wa  \\
    jeden    & 1  & a   \\
    dwa      & 2  & ka  \\
    trzy     & 3  & sa  \\
    cztery   & 4  & ta  \\
    pięć     & 5  & na  \\
    sześć    & 6  & cha \\
    siedem   & 7  & ma  \\
    osiem    & 8  & ya  \\
    dziewięć & 9  & ra  \\
    dziesięć & 10 & awa \\\bottomrule
  \end{tabular}
  \label{tab:numerals}
\end{table}

Najprostszym sposobem tworzenia liczebnika dla liczb większych od 9 jest
ustawienie w~kolejności zapisu dziesiętnego kolejnych słów określających cyfry,
tj. 17 to \emph{ama}, 123 to \emph{akasa}, a~241 to \emph{kataa}.

\begin{table}[ht]
  \centering
  \caption{Liczebniki wyższe}
  \begin{tabular}{ccc} \toprule
    jedenaście   & 11      & aa      \\
    dwanaście    & 12      & aka     \\
    dwadzieścia  & 20      & kawa    \\
    trzydzieści  & 30      & sawa    \\
    czterdzieści & 40      & tawa    \\
    sto          & 100     & aḱaywa  \\
    tysiąc       & 1000    & aśaywa  \\
    milion       & 1000000 & aćhaywa \\\bottomrule
  \end{tabular}
  \label{tab:numerals2}
\end{table}

Niektórzy użytkownicy języka nie lubią formy pozycyjnej, w której wyrazy
potrafią być problematyczne, np. \emph{aaa} jako sto jedenaście. Stąd używana
jest również partykuła \emph{jo}, która oznacza ,,oraz'', i jest stosowana do
dodawania wartości wypowiedzianych liczebników.

Przykładowo, \emph{aśaywa jo kata} to tysiąc oraz dwadzieścia cztery (1024),
\emph{aḱaywa jo a} to dosłownie ,,sto oraz jeden'' (101), natomiast
\emph{aḱaywa jo awa jo a} czyli ,,sto oraz dziesięć oraz jeden'' to 111


\subsection{Large numbers}

Dla większych
liczb było by to jednak zbyt problematyczne (np. 1000 to by było
\emph{awawawa}), dlatego można stosować przyrostek <-y>, określający liczbę
powtórzeń, w~stosunku do sylaby określającej liczbę powtórzeń, a~potem cyfrę
powtarzaną, na przykład \emph{aśaywa} -- dosłownie: jeden i~trójka zer. Stąd
milion to \emph{aćhaywa}, a~100023 to \emph{aśaywakasa}. Akcent kładziony jest
zawsze na sylabę, która zawiera przyrostek <-y>.

W sytuacji, kiedy jest kilka sylab z wrostkiem <-y>, akcent kładziony jest na
ostatnią taką sylabę.



\subsection{Ordinal numbers}

Przyrostek <-ti> określa liczebnik porządkowy, tj. \emph{kati} to drugi, a
\emph{sasati} -- trzydziesty trzeci.

\begin{table}[ht]
  \centering
  \caption{Podstawowe liczebniki porządkowe}
  \begin{tabular}{cc} \toprule
    pierwszy  & ati   \\
    drugi     & kati  \\
    trzeci    & sati  \\
    czwarty   & tati  \\
    piąty     & nati  \\
    szósty    & chati \\
    siódmy    & mati  \\
    ósmy      & yati  \\
    dziewiąty & rati  \\
    dziesiąty & awati \\\bottomrule
  \end{tabular}
  \label{tab:numerals3}
\end{table}

\subsection{Fractions}

Liczebnik ułamkowy tworzy się przez infiks <-je->, na przykład \emph{ajeka} to
jedna druga, \emph{najera} -- pięć dziewiątych, a~\emph{awajerana} to 10/95.

\begin{table}[ht]
  \centering
  \caption{Ułamki}
  \begin{tabular}{cc} \toprule
    połowa        & ajeka  \\
    jedna trzecia & ajesa  \\
    dwie trzecie  & kajesa \\
    jedna czwarta & ajeta  \\
    trzy czwarte  & sajeta \\
    jedna siódma  & ajema  \\\bottomrule
  \end{tabular}
  \label{tab:numerals4}
\end{table}

Możliwe jest również użycia słowa \emph{jerya} oznaczającego ,,przecinek'',
w~taki sposób, że:

\glossex{Egi doloze esi a jerya rasa metr.}{Egi doloze esi a jerya rasa metr\\
  3SG.POSS height be.PRS one comma 83 meter}{,,Jego wzrost to 1,83 metra.''}

