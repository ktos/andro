This book is my latest attempt to write a complete description of the grammar of
\andro, a fictional language (or \emph{conlang}), which I have been developing
for over ten years as of now. \andro was created for my worldbuilding
initiative, which I've been working on since forever and probably will take a
bit longer.

Unfortunately, neither I am a linguist, nor English is not my native language,
which means some parts of this book may be simply wrong in English or may use
confusing terminology. I would love to improve it, you can give me a Pull
Request on GitHub.

This book has been written almost from scratch, as most of my notes were very
disorganized and incomplete, some only in my head, some only in my native
language. Another problem is that this document is a ``living one'', some things
may change without a notice in newer versions if I decide on changing or
improving some element of the language. But do not worry, the changes won't be
completely changing everything!

As you may notice, the book is written from the point of view of foreign
linguist, trying to describe his native language to us. Please remember that
everything in here is a work of pure fiction and imagination. No real-world
language have been used specifically as a source of vocabulary, but some words
and constructs will bear very visible Polish, English, Japanese or other
inspiration. All mentions of Earth, And́royas, Maŕid and other places, people
and history make sense only in-universe. \andro was meant as a form of artistic
expression and a way to learn something new. The goals were to create something
that is reasonably naturalistic, with its own unique vibe, being a bit familiar
as well as alien at the same time.

Everything in here is licensed under \textsc{CC-BY 4.0}. If you would like to
use \andro in any of your own creations, feel free to do that, but please
attribute myself and possibly this repository. Game, book, movie? Absolutely no
problem, it would be amazing to see \andro anywhere, but remember that many
parts of the language make sense only including the culture of its speakers.

\begin{flushright}\itshape\footnotesize
    Lublin, January 2022
\end{flushright}