\chapter{Derivational morphology}
\label{ch:derivational}

morfem -la- oznacza negatywne nastawienie, np. zir -- zirala -
jaskółka-samotność, czy sini -- silani - wychodzić-opuszczać

no- negacja
i- "przy" - z zewnątrz (venitive)
si- "na zewnątrz"
u- "zakończenie czynności"
un- "na zewnątrz"
a- "prze"
me - "do środka"

From the derivational morphology point of view, many of the adjectives may be
modified using the \xo{no-} prefix, which is identical to the negation marker.

\glossex{Muche esi anper.}{muche esi anper}{cat COP wet}{``The cat is wet.''}

\glossex{Muche esi nonper.}{muche esi no-nper}{cat COP NEG-wet}{``The cat is dry.''}

Similarly,

Istnieją przymiotniki, które powstały przez dodanie morfemu \xo{no-} przed
pewnym rdzeniem, wyrażające przeciwność, np. \emph{anper} \xm{ˈan.pɛr} (mokry)
i~\emph{nonper} \xm{ˈnɔn.pɛr} (suchy) oraz przymiotniki z~prefiksem \xo{mo-},
oznaczającym negatywne zestopniowanie, np. \emph{leder} \xm{ˈlɛ.dɛr} (oszczędny)
i~\emph{moleder} \xm{ˈmɔ.lɛ.dɛr} (skąpy).