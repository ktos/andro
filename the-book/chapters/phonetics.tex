\chapter{Phonetics and phonology}
\label{ch:phonetics}

This chapter will present charts depicting the phoneme inventory of \andro, as
well as provide with a small description of the very hard phonotactics of the
language.

\section{Phonology}

At 6 vowels and 20 consonants in the \randro{ardo andro}, the inventory is not
too extensive, and is pretty common and understandable for foreign speakers.
Vowels are presented in figure~\ref{fig:vowels} while consonants in
table~\ref{tab:consonants} and other sounds in table~\ref{tab:consonants2}. The
one interesting feature one may notice is the slight assymetry in both vowels
and consonants inventories. In case of consonants, the visible lacking sound is
retroflex \xm{ʂ} -- while there are both \xm{s} and \xm{z} and most of other
pairs, the \xm{ʐ} is lonely, without pair. Its pairing is actually a voiceless
retroflex affricate \xm{ʈ͡ʂ}.

\begin{figure}
    \begin{center}
        {\Large
            \begin{vowel}
                \putcvowel{i}{1}
                \putcvowel{ɛ}{3}
                \putcvowel{a}{4}
                \putcvowel{ɔ}{6}
                \putcvowel{u}{8}
                \putcvowel{ʏ}{13}
            \end{vowel}
        }
    \end{center}
    \label{fig:vowels}
    \caption{Vowel inventory}
\end{figure}

\begin{table}
    \caption{Consonant inventory}
    \begin{tabular}{lccccccc}
        \\
        \\
                                     & \textbf{Bilabial} & \textbf{Labio-dental} & \textbf{Dental} & \textbf{Alveolar} & \textbf{Retroflex} & \textbf{Palatal} & \textbf{Velar} \\
        \textbf{Plosive}             & p b               &                       & t d             &                   &                    &                  & k g            \\
        \textbf{Nasal}               & m                 &                       & n               &                   &                    &                  & ŋ              \\
        \textbf{Trill}               &                   &                       &                 & r                 &                    &                  &                \\
        \textbf{Fricative}           &                   & f v                   &                 & s z               & ʐ                  &                  & x              \\
        \textbf{Approximant}         &                   &                       &                 &                   &                    & j                &                \\
        \textbf{Lateral Approximant} &                   &                       &                 & l                 &                    &                  &                \\
    \end{tabular}
    \label{tab:consonants}
\end{table}

\begin{table}
    \caption{Remaining sounds}
    \begin{tabular}{lc}
        \textbf{Voiced labial-velar approximant} & w  \\
        \textbf{Voiceless retroflex affricate}   & ʈ͡ʂ
    \end{tabular}
    \label{tab:consonants2}
\end{table}

There is a serious number of diphtongs: \xm{ɛi}, \xm{ɛɔ}, \xm{ɔa}, \xm{ia},
\xm{iɔ}, \xm{uɛ}, \xm{au}, \xm{ai}, \xm{aɛ}, \xm{uɔ}, \xm{ɔʏ} and \xm{ɔi}.
Diphtongs are however pretty rare, usually a hiatus, the syllable break between
two vowels exists. The length of the vowel is not used apart from one: \xm{i:}.
The longer version of \xm{i} is found in words borrowed from some foreign
languages. In the Southern dialect, some speakers are using longer version of
\xm{a}, \xm{a:} instead of a hiatus.

While there is a lot of phonemes similar to \xm{ri}, \xm{hi}, or \xm{pi}
palatalization is avoided in \randro{ardo andro}, while is absolutely normal in
dialects.

As mentioned earlier, dialects may introduce other phonemes -- \xt{ʁ} or
aspiration, \xt{ɡʱ} or \xt{kʱ}, \xt{ʒ} and \xt{h} in the place of \xt{ʐ} oraz
\xt{x} and so on. This is however not documented, similarly to the allophony --
the \tandro{Eygepa Aliona yi Inrat}{Royal Council of Language} never decided on
full phonemic description of \andro, officially ,,due to its extensive and
variant usage''.

\subsection{Ziri Transcription}

The \randro{Ziri} transcription, mentioned earlier, is using a Latin script,
with very close relationship between phonemes and assigned characters, using
digraphs for two specific phonemes, which is a inherited concept from the
previous transcription system.

\begin{table}[ht]
    \centering
    \caption{Transcription for vowels}
    \begin{tabular}{llccccc}
        \textbf{Phoneme}   & a & ɛ & i & ɔ & u & ʏ  \\
        \textbf{Character} & a & e & i & o & u & yi
    \end{tabular}
    \label{tab:phonemes}
\end{table}

\begin{table}[ht]
    \centering
    \caption{Transcription for consonants}
    \begin{tabular}{lcccccccccc}
        \textbf{Phoneme}   & p & b & t & d & k & g & m & n & ŋ  & f \\
        \textbf{Character} & p & b & t & d & k & g & m & n & n  & f \\
        \textbf{Phoneme}   & v & s & z & ʐ & x & j & l & w & ʈ͡ʂ     \\
        \textbf{Character} & v & s & z & j & h & y & l & w & ch
    \end{tabular}
    \label{tab:chars}
\end{table}

As one may notice, both \xm{n} and \xm{ŋ} are represented as \xo{n}. It's due to
a fact that \xo{-nt} is used for \xm{-ŋt}, because nasal \xm{ŋ} is almost
non-existant in any other context.

\note{ In case of aspiration, \randro{Ziri} allows for digraphs -- \xm{ɡʱ} is
    \xo{gh}, while \xm{ʁ} is exactly the same as \xm{r} -- \xo{r}. As mentioned
    earlier, these phonemes exist only in dialects.}
\skipline

The writing \xo{aa}, \xo{uu} and~\xo{oo} are usually used in case of hiatus,
except for \xo{ii}, which is realized as a long vowel \xm{i:}.

Diphtongs, as they are more rare than hiatus, are marked. \randro{Ziri} is using
a macron on both vowels included in a diphtong, e.g. \xm{ɛɔ} is \xo{e͞o}. It's a
very unique feature of transcription and it's actually a borrowing from the
\randro{chiwo} writing system, a native writing system for \andro. Accent, which
is described in the following section, is unmarked when typical and marked using
accute on the first character of the accented syllable, e.g. \tandro{baán}{time}
-- \xo{baán} -- \xm{ba.ˈan}.

While in the transcription punctuation marks and capital letters will be used,
they are not used in writing system \randro{chiwo}, which is case insensitive.

\section{Syllable structure}

As mentioned, \andro evolved from a language created on a base of a traders
pidgin. The languages from which it was created were ,,incompatible'' -- the Old
Nennekan had a huge number of consonant clusters, while Western languages
(especially Lono) were usually following very clear CV syllable structure.

The language was later adapted to better fit phonotactics of its neighbouring
languages, for better adaptation in the Western society and -- which was also
very important -- to fit into different writing systems. This caused the
phonotactics to be one of the most complex parts of the language.

The six syllabic structures are possible in \andro:

\begin{itemize}
    \item (C)V(C1), where C1 is: \xm{d}, \xm{j}, \xm{k}, \xm{l}, \xm{m}, \xm{n}, \xm{r}, \xm{s} or \xm{t} (e.g.: \xm{a}, \xm{ba}, \xm{bad}),
    \item (C)V(V)(C1), where not all diphtongs are possible, only: \xm{ɛa}, \xm{ɛi}, \xm{ɛɔ}, \xm{ɔa}, \xm{ɔɛ}, \xm{ɔi}, \xm{uɛ}, \xm{au}, \xm{aɛ}, \xm{uɔ}, \xm{ua}, \xm{ui}, \xm{ia}, \xm{iɔ}, \xm{aɔ} oraz \xm{ai} (e.g.: \xm{ɛa}, \xm{ɛad}, \xm{bɛad}),
    \item CRV(C1), where R may be exclusively \xm{r}, \xm{j} or \xm{l} (e.g.: \xm{bra}, \xm{brad}) but CR cluster cannot produce \xm{rr}, \xm{ll}, \xm{jj}; while V is for every vowel apart from \xm{ʏ},
    \item (C)Vŋt, (e.g.: \xm{aŋt}, \xm{baŋt}),
    \item (C)Vrn, (e.g.: \xm{arn}, \xm{barn}),
    \item stV(C1), (e.g.: \xm{sta}, \xm{stad}).
\end{itemize}

This is resulting in 7520 possible syllables. In \andro, most of the words (at
the moment of writing this text about 50\%) is 2-3 syllables long, however there
are words long up to 7 syllables. The number of possible syllables is one of the
reasons that homonyms are very rare in \andro.

\section{Stress and intonation}

The stress based on difference in the loudness of syllables (dynamic stress) is
used in \andro. Some speakers, especially whose first language is one of the
Keilic language group, are using a combination of loudness and pitch, but it is
a regional variant.

In most cases the stressed syllable is the first one, and if so, it's unmarked
in the transcription. However, in case of morphological prefix, stress is common
on the second syllable. For example, \randro{uf́riti} \xm{u.ˈfri.ti} (to
extinguish fire) is morphologically a compound word created from the verb
\randro{friti} \xm{ˈfri.ti} (to give light, to burn) with a morphological prefix
\xo{u-}, so the accent is on the first syllable of the ,,original'' word. Such
change, movement of stress from the first syllable is also prevalent in case of
hiatus, for example in \randro{baán} or \tandro{laák}{tree}.

The stress change also appears in some of the compound words, where is moved to
the first syllable of the second word. A notable example is
\tandro{unatalit́ay}{birthday} \xm{u.na.ta.li.ˈtaj}, where first component also
features a morphological prefix \xo{u-}.

As visible from the number of possible syllables, \andro does not have many
homonyms. However, if there are any, they are usually differentiated by stress.
Such pair may be: \tandro{sio}{external}, pronounced \xm{ˈsi.ɔ} and its homonym
\tandro{sió}{white}, where stress is on the ultimate syllable, \xm{si.ˈɔ}. Such
pairs are very rare.

As for intonation in a sentence, there are no clear rules.