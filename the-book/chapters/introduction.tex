\chapter{Introduction}
\label{ch:introduction}

\begin{flushright}\small
    \randro{\larger Miam rujaler chetni navicha vimi architji ze one aśulgina iéni.}\\
    % \ayr{\larger prony AdnYaaNF si miNF thnojyaaNF, EdreNF voj kotnsF.}\\
    % \fw{Paronaya adanyāng si ming tahanoyyāng, edareng voy kotanas.}\\
    ``When a man is facing the trial, the ancestors will come with help.''\\
    --- The First Great Saying\footnotemark
\end{flushright}\bigskip

\footnotetext{Or, how is it called in \andro, \randro{Ati Gruwe Lannake}.}

\tandro{Kiiway!}{Welcome!} I am very glad you are reading this book. My name is
Koolder mal Erlehirni and this book is the beautiful fruit of my hard work
started over five years ago.

In the beginning, we started preparing a simple Andro to English dictionary,
with only a few grammar notes for understanding the overall concept of the
language, however during that time a few notes expanded into large library.

I have prepared this text, similarly to the Andro to English dictionary, with
you -- citizens of the Earth -- in mind. So apart from notes about our language,
there will be notes about our culture.

Some parts of this book may require some knowledge of linguistic terms.

\bigskip

\tandro{Heme ofari!}{Let's start!}

\section{Conventions}

\subsection{Romanization}

While Andro as a language is using different writing systems, on which you may
read more in \nameref{ch:appendixa}, in this book mainly the romanization system
called \randro{Ziri} will be used. You may read more on \randro{Ziri} and the
sounds of the language in the chapter on \nameref{ch:phonetics}.

The text written \xo{like this} means the particular set of characters, it will
be mostly used in describing what latin characters are used to write the Andro
sounds.

A format like \xm{this format} will be marking phonetic transcription, all
sounds will be presented according to the International Phonetic Alphabet. Due
to differences in dialects and overall not such importance of exact phonemic
values, it will be mostly used to present sounds of the language. Phonemic
markers \xt{like this} will be used sporadically to emphase pronounciation of
the particular word.

A text marked \randro{in this way} will be always the example of the Andro
language.

\subsection{Glossing}

All important examples of the Andro language will be glossed using a interlinear
gloss as presented below.

\glossex{Saḱaywa yaro lot ti vivi per ti ruókeyi yarji eédi?}{sa-ḱa-y-wa yar-o lot ti vivi per ti ruóke-yi yar-ji eédi}{3-2-times-0 year-ADJ life 2SG to.intend in.order.to 2SG crow-POSS year-PL get.to.know}{``Are you going to live three hundred years to learn the age of the crow?''}

The first line is the text in (romanized) Andro, the second line is providing
glosses according to Leipzig Glossing Rules, while the third component is the
text translated to English, sometimes with alternative translation available.
Glossing abbreviations like \Dem{}, \Nan{}, \Rel{}, \Refl{} or \Top{} will be
always presented in \textsc{small caps}, the list of all glossing abbreviations
and their definitions is available at the end of this book.

Smaller examples will be translated in the footnotes only, like
\tandro{che}{this}.