\begin{center}\small
% \ayr{\larger prony AdnYaaNF si miNF thnojyaaNF, EdreNF voj kotnsF.}\\
% \fw{Paronaya adanyāng si ming tahanoyyāng, edareng voy kotanas.}\\
`He who cannot write believes it not to be toil.'\\
--- Anonymous\footnotemark
\end{center}\bigskip

\footnotetext{In the original Latin, \textit{Quia qui nescit scribere putat
hoc esse nullum laborem}.}

\tandro{Kiiway!}{Welcome!} I am very glad you are reading this book. My name is
Koolder mal Erlehirni and this book is the beautiful fruit of my hard work
started over five years ago.

In the beginning, we started preparing a simple Andro-English dictionary, with
only a few grammar notes for understanding the overall concept of the language,
however during that time a small number of notes expanded into large library.

I have prepared this text, similarly to the Andro-English dictionary, with you -
citizens of the Earth - in mind. So apart from notes about our language, there
will be notes about our culture.

Some of parts of this book may require some knowledge of linguistic terms.

\bigskip

\tandro{Heme ofari!}{Let's start!}

\section{Conventions}

\subsection{Romanization}

While Andro as a language is using different writing systems, on which you may
read more in Appendix A, in this book the romanization system called
\textit{Ziri} will be used.

\subsection{Glossing}

\glossex{Saḱaywa yaro lot ti vivi per ti ruókeyi yarji eédi?}{sa-ḱa-y-wa yar-o lot ti vivi per ti ruóke-yi yar-ji eédi}{3-2-times-0 year-ADJ life 2SG to.intend in.order.to 2SG crow-POSS year-PL get.to.know}{,,Are you going to live three hundred years to learn the age of the crow?''}