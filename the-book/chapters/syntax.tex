\chapter{Syntax}
\label{ch:syntax}

\section{Basic sentence structures}
\label{sec:basic}

Podstawowy szyk zdania to SVO (Jan kocha Marię), a~zdania podrzędnego to SOV
(Jan Marię kocha), ale w~szyku pytającym to OSV (Marię Jan kocha). Stosowany
jest szyk przymiotnik-rzeczownik, nigdy odwrotnie.

Czasami, w języku formalnym lub poetyckim, pojawia się czasownik na końcu zdania
oznajmującego, tworząc szyk SOV lub nawet OSV, jest to jednak rzadkie
i~zazwyczaj w mowie potocznej utrzymywany jest szyk SVO.

\glossex{Beúsma egi va wesazit.}{beúsma egi va wesazi-t}{legend 3SG PFV become-PST}{,,On stał się legendą.''}

\glossex{Egi beúsma va wesazit.}{egi beúsma va wesazi-t}{3SG legend PFV become-PST}{,,On stał się legendą.''}

Role gramatyczne zasadniczo wyznaczane są przez szyk zdania (\Nom{}--\Acc{}),
jednakże do określania dodatkowych funkcji gramatycznych wykorzystuje się
partykuły: przykładowo partykuła \emph{yi} wyznacza przynależność
(\emph{possesivus}, \Poss{}), natomiast partykuła \emph{chu} wyznacza
dopełniacz (\emph{genitivus}, \Gen{}) -- aczkolwiek dopełniacz może być
rozpoznany również z~kontekstu. Partykuły stosowane są przed częścią zdania do
którego się odnoszą.

\glossex{Mi pazi muchi.}{mi pazi much-i}{1SG like cat-PL}{,,Ja lubię koty.''}

Czasami do podkreślenia jakiejś frazy lub w niektórych dialektach stosuje się do
tego również partykułę \emph{chu}, jednak nie jest to zalecane w \emph{ardo
  andro}, aby uniknąć konsternacji w sytuacji kiedy jest to dopełniacz
(\Gen{}).

\glossex{Mi veydi yasaji.}{mi veydi yasa-ji}{1SG see eel-PL}{,,Widzę węgorze.''}

\glossex{Mi veydi chu yasaji.}{mi veydi chu yasa-ji}{1SG see ACC eel-PL}{,,Widzę węgorze.''}

\glossex{Myi keromamerey esi dowo chu yasaji.}{myi keromamerey esi dowo chu yasa-ji}{1SG.POSS hovercraft be.PRS full GEN eel-PL}{,,Mój poduszkowiec jest pełen węgorzy.''}

\glossex{Myi keromamerey esi dowo yasaji.}{myi keromamerey esi dowo yasa-ji}{1SG.POSS hovercraft be.PRS full eel-PL}{,,Mój poduszkowiec jest pełen węgorzy.''}

Do określenia czym została wykonana czynność (narzędnika, \Ins{})
stosowana może być partykuła \emph{da}.

\glossex{Id́ak ostro da keja.}{id́a-k ostro da keja}{open-PST door INS key}{,,Otworzyłem drzwi kluczem.'' \\ ,,Otworzyłem drzwi używając klucza.''}

Natomiast partykuła \emph{in} może być stosowana jako miejscownik
(\Loc{}), na przykład:

\glossex{Mi esi in mibozor.}{,i esi in mibozor}{1SG be.PRS in.LOC school}{,,Jestem w szkole.'' \\ ,,Jestem teraz w szkole.''}

Przymiotnik znajduje się zawsze przed rzeczownikiem, który określa, natomiast
wyrażenie przysłówkowe -- po czasowniku, który określa.

\glossex{Nontriso lanji fesgai gepo.}{no-ntris-o lan-ji fesgai waril}{NEG-interesting-ADJ book-PL read.PRS long.ADV}{,,Nudne książki czytam długo.''}



\subsection{Modal verbs}

Czasowniki modalne (\emph{epi} -- móc, \emph{pazi} -- lubić, \emph{kiruki} --
umieć, \emph{vipini} -- zmuszać, musieć) powodują przeniesienie czasownika
odpowiadającego na koniec zdania, zaburzając nieco jego szyk. Czasownik ten
przyjmuje formę bezokolicznika, a~koniugacja i~miejsce operatorów będą dotyczyły
tylko modalnego. W~podobny sposób wygląda to w~sytuacji niektórych idiomów
czasownikowych.

\glossex{Mi kiruki feni.}{mi kiruki feni}{1SG can swim}{,,Umiem pływać.''}

\glossex{Mi vi kiruket feni, abe va tayet.}{mi vi kiruke-t feni, abe va taye-t}{1SG IPFV can-PST swim.PRS but PFV forget-PST}{,,Umiałem pływać, ale zapomniałem.''}

Czasowniki modalne uruchamiają tryb czynności regularnej i~nie odnosi się to do
tej konkretnej chwili, ale do stwierdzenia faktu (umiem czytać -- kiedyś się
nauczyłem i~od tej pory umiem i~raczej nie zapomnę -- oczywiście, można przestać
coś lubić albo móc, ale będzie to opisane odpowiednim aspektem w~przyszłości).

\glossex{Mi epi chet seysi.}{mi epi chet seysi}{1SG be.able.PRS here.LOC sit.PRS}{,,Mogę tutaj usiąść.''}

\glossex{Chet seysi mi epi?}{chet seysi mi epi}{here.LOC sit.PRS 1SG be.able.PRS}{,,Czy mogę tu usiąść?''}

\glossex{Pazi fesgai.}{pazi fesgai}{like.PRS read.PRS}{,,Lubię czytać.''}

\subsection{Negations}

Przeczenia, negacja czynności realizowana jest z~wykorzystaniem partykuły
\emph{no}, która znajduje się \textbf{po} czasowniku. Możesz spotkać teksty
w~których partykuła \emph{no} znajduje się przed czasownikiem, ale są one zawsze
oznaką niedouczenia autora.

Kwestia podwójnego zaprzeczenia: w Andro poprawne są obydwie formy, tj. poprawne
jest zarówno \emph{ze moĺi karli}, jak i \emph{ze moĺi no karli} (,,nigdy nie
umrę'').

\glossex{Ze karli moĺi.}{ze karli moĺi}{FUT die never.ADV}{,,Nigdy nie umrę!''}

\glossex{Ze karli no moĺi.}{ze karli no moĺi}{FUT die NEG never.ADV}{,,Nigdy nie umrę!''}

\note{W niektórych dialektach partykuła \emph{no} może pojawiać się zawsze na
  końcu zdania.}

\glossex{Ti bugi no.}{ti bugi no}{2SG lie NEG}{,,Ty nie leżysz.''}

\glossex{Tori no dej́itos do!}{tori no dej́it-os do}{throw NEG weapon-PL IMP}{,,Nie rzucaj broni!''}

\glossex{Egi karlet no il rige.}{egi karle-t no il rige}{3SG kill-PST NEG 3SG.POSS lord}{,,On nie zabił swojego władcy.''}

Partykuła \emph{no} może również służyć do negacji rzeczownika, tworząc
konstrukcję ,,zamiast czegoś'':

\glossex{A femji inji no rujalaros femit}{a fem-ji inji no rujalar-os femi-t}{on branch-PL leave-PL instead.of people-PL hang-PST}{,,Ludzie wisieli na gałęziach zamiast liści.''}

\subsection{Questions}

Szyk pytający (OSV):

\examples{Il maŕie͞o egi koóli?}{Czy on kocha swoją żonę?}

\glossex{Il maŕie͞o egi koóli?}{il maŕie͞-o egi koóli}{3SG.POSS spouse-M 3SG love}{,,Czy on/ona kocha swojego męża?''}

\note{\emph{egi} nie wskazuje na płeć, jednak \emph{marié͞o} jest w rodzaju męskim.}

\glossex{Il hima egi esi?}{il hima egi esi}{3SG.POSS girlfriend 3SG be.PRS}{,,Czy ona to jego dziewczyna?''}

\glossex{Sotak epié karlet?}{sotak epié karle-t}{somebody 2SG.FRM.M kill-PST}{,,Czy zabił Pan kogoś?''}

\glossex{Je rujaler epié ze va karla͞i?}{Je rujaler epié ze va karla͞i}{DEM man 2SG.FRM.M FUT PFV kill}{,,Czy zabije Pan tego człowieka?''}

\glossex{Mi epié ze vi karla͞i?}{mi epié ze vi karla͞i}{1SG 2SG.FRM.M FUT IPFV kill}{,,Czy spróbuje Pan mnie zabić?''}

\glossex{Il alye Epiá veyt hetay?}{il alye epiá vey-t hetay}{3SG.POSS friend 2SG.FRM.F see-PST today}{,,Czy widziała Pani dzisiaj swojego przyjaciela?''}

\glossex{Ti va fesgat?}{ti va fesga-t}{2SG PFV read-PST}{,,Przeczytałeś?''}

Szyk pytający z~czasownikiem modalnym zachowuje konwencję czasownika
niemodalnego na końcu zdania.

\glossex{Chet mi epi seysi?}{chet mi epi seysi}{here 1SG be.able sit}{,,Czy mogę tutaj usiąść?''}

\glossex{Chet mi va epit seysi?}{chet mi va epi-t seysi}{here 1SG PFV be.able-PST sit}{,,Czy mogłem tutaj siąść?''}

Szyk pytający jest również uruchamiany automatycznie przez partykuły pytające --
\emph{chyi} (czyj), \emph{ko͞e} (jak), \emph{osor} (dlaczego), \emph{so} (co),
\emph{somar} (gdzie), \emph{soter} (kto), \emph{voli} (kiedy), \emph{wodo}
(którędy), \emph{yage} (dokąd), \emph{yase} (skąd).

\glossex{Yasu ti iéni?}{yasu ti i-éni}{from.where.Q 2SG VEN-go}{,,Skąd pochodzisz?''}

\glossex{Yasu ti iént?}{yasu ti i-én-t}{from.where.Q 2SG VEN-go-PST}{,,Skąd przyszedłeś?''}

\glossex{Ko͞e tyi ager nomi?}{ko͞e tyi ager nomi}{how.Q 2SG.POSS country to.name}{,,Jak nazywa się twój kraj?''}

\glossex{Ko͞e ti seiti?}{ko͞e ti seiti}{how.Q 2SG feel}{,,Jak się czujesz?''}

\glossex{So tyi vahuryiáysi o ti famei?}{so tyi vahuryiáysi o ti famei}{what.Q 2SG.POSS tatoo for 2SG mean}{,,Co znaczy dla ciebie twój tatuaż?''}

\glossex{Wodo o Polska mi ze cheri?}{wodo o Polska mi ze cheri}{which.way.Q to Poland 1SG FUT cheri}{,,Którędy mam jechać do Polski?''}

\subsection{Idioms}

Idiomy, takie jak na przykład \emph{kipeni a} (wzorować się na) powodują
przestawienie drugiego elementu (najczęściej partykuły) bliżej dopełnienia, np.:

\glossex{Mi kipeni a myi patal.}{mi kipeni a myi patal}{1SG model on 1SG.POSS father.DIM}{,,Wzoruję się na moim tacie.''}

\glossex{A tyi patal ti kipeni?}{a tyi patal ti kipeni}{on 2SG.POSS father.DIM 2SG model}{,,Wzorujesz się na swoim tacie?''}

Do pytań w~stylu ,,czy'' można również stosować partykułę wzmacniającą
\emph{vay}:

\glossex{Vay a tyi patal ti kipeni?}{vay a tyi patal ti kipeni}{really.Q on 2SG.POSS father.DIM 2SG model}{,,Naprawdę wzorujesz się na swoim tacie?''}

\subsection{Relative clauses}

Zdania podrzędne tworzone są automatycznie przez partykuły takie jak \emph{ko͞e}
(jak), \emph{voli} (kiedy), \emph{imin} (ponieważ), \emph{abe} (ale),
\emph{chua} (która), \emph{pama} (wtedy), \emph{per} (żeby), a~także \emph{e}
(i, oraz). Szyk zdania podrzędnego to SOV (Jan Marię kocha).

\glossex{Mi eni o jan, per mi vibo vibi.}{eni o jan per mi vibo vibi}{1SG go to.LOC home so 1SG food eat}{,,Idę do domu aby zjeść jedzenie.''}

\glossex{Cherlok Holmes va choint sosbet kayetor esi no, e egi o jan eni permet.}{cherlok holmes va choin-t sosbet kayetor esi no e o jan eni perme-t}{Sherlock Holmes PFV state-PST suspect murderer be NEG and to home go allow-PST}{,,Sherlock Holmes uznał, że podejrzany nie jest przestępcą i pozwolił mu iść do domu.''}

\glossex{Cherlok Holmes va choint sosbet kayetor esi no, abe egi va est - abe deíto va kopet lipe.}{cherlok holmes va choin-t sosbet kayetor esi no abe egi va est abe deíto va lope-t lipe}{Sherlock Holmes PFV state-PST suspect murderer be NEG but 3SG PFV be but weapon PFV hide-PST well}{,,Sherlock Holmes uznał, że podejrzany nie jest przestępcą, jednak on był, tylko dobrze ukrył broń.''}

Fakt, że ,,egi'' odnosi się do podejrzanego wynika z kontekstu zdania.

Czasownik w~zdaniach podrzędnych może być pomijany, jeżeli wynika z~kontekstu.
Czasownik \emph{esi} (być) może również być pomijany, stąd poprawne są zdania
takie jak:

\glossex{Ari esi wofo, koy vipetode yi muche.}{ari esi wofo koy vipetode yi muche}{sky be.PRS color like bowl POSS cat}{,,Niebo ma kolor taki jak kocia miska.''}

\glossex{Ari arso, koy vipetode yi muche.}{ari arso koy vipetode yi muche}{sky skyblue.ADJ like bowl POSS cat}{,,Niebo ma niebieski kolor, tak jak kocia miska.''}

\glossex{Hetay ari stobo.}{hetay ari stobo}{today sky gray}{,,Niebo jest dzisiaj szare.''}

\section{Copula and topic marker}
\label{sec:copula}

\section{Pronouns}
\label{sec:pronouns}
Zaimek \emph{mi} może być pomijany, tj. poprawne jest zarówno:

\glossex{Mi pazi muchi.}{mi pazi much-i}{1SG like cat-PL}{,,Ja lubię koty.''}

jak i~po prostu:

\glossex{Pazi muchi.}{pazi much-i}{like cat-PL}{,,Lubię koty.''}

To samo może dotyczyć zaimka \emph{ti}:

\glossex{Pazi muchi.}{ø pazi much-i}{2SG like cat-PL}{,,Lubisz koty.''}

Rozróżnienie obu tych zachowań następuje tylko z wykorzystaniem kontekstu
wypowiedzi.

\section{Phrase structures}
\label{sec:phrases}

\subsection{Adjective phrases}
\subsection{Adverbial phrases}
\subsection{Regulars}

Opis czynności regularnych obowiązuje tylko z~określeniem jednostki czasu w
postaci wyrażenia przysłówkowego:

\glossex{Mi fesgai relita.}{mi fesgai relita}{1SG read.PRS always}{,,Zawsze czytam.''}

\glossex{Mi fesgai eveni tay.}{mi fesgai eveni tay}{1SG read.PRS every day}{,,Czytam każdego dnia.''}

\glossex{Mi koóli ti relita.}{mi koóli ti relita}{1SG love.PRS 2SG always}{,,Kocham cię na zawsze.'' \\ ,,Kocham cię zawsze.''}

\subsection{Possessive particle}

The possessive particle \randro{yi} specifies possession

(¯textsc{poss}) --.
ex. \randro{vipetode yi muche} - the cat's bowl (the cat's bowl). It is possible to
combination with the~object to which it refers as a suffix, in the style of an
adjective, which requires a change in word order, e.g. \randro{mucheyi vipetode}.
This concept exists mainly in the~dialects of the West, in~which there are strong influences of
agglutinative Cairean language and in~the so-called desert dialect. In addition to
In addition, it is used in~situations of multiple possessive nesting, for
For example, my father's mother's bowl \randro{vipetode yi myi arśityi natali͞a},
although \randro{vipetode yi natali͞a yi myi arśit} is of course equally
correct.

\section{Tense, aspect and voice markings}
\label{sec:markers}

\subsection{Tenses}

\subsection{Present tense}

Podstawowy czas i~podstawowa forma czasownika opisuje to, co jest teraz
aktualne, lub to, co jest raczej niezmienne (zazwyczaj w~odniesieniu do obiektów
nieożywionych), może być stosowana do stwierdzeń typu ,,pada deszcz'' i~innych
odnoszących się do stanu rzeczywistości -- są realizowane wtedy tylko przez sam
czasownik i~podmiot domyślny, często określają również niejawnie aspekt
niedokonany.

\glossex{Ti fesgai.}{ti fesgai}{2SG read.PRS}{,,Ty teraz czytasz.''}

\glossex{Che esi karié.}{che esi karié}{DEM.NAN be.PRS beautiful}{,,To jest piękne.'' \\ ,,To coś jest piękne.''}

\note{Należy tutaj przypomnieć, że partykuła \emph{che} może być stosowana tylko do obiektów nieożywionych.}

\glossex{Aḱame osupi.}{aḱame osupi}{rain fall.PRS.IPFV}{,,Pada deszcz.''}

\glossex{Osupi}{osupi}{fall.PRS.IPFV}{,,Pada deszcz.''}

Czas przyszły realizowany jest przez partykułę \emph{ze}. Domyślnym aspektem
czasu przyszłego jest aspekt niedokonany, więc partykułę \emph{vi} można
pomijać.

\glossex{Ti ze fesgai.}{ti ze fesgai}{2SG FUT read}{,,Zaczniesz czytać.'' \\ ,,Będziesz czytał.''}

\glossex{Ti ze va fesgai.}{ti ze va fesgai}{2SG FUT PFV read}{,,Przeczytasz.''}

\glossex{Hetay mi ze va vibi rome.}{hetay mi ze va vibi rome\\
  today 1SG FUT PFV eat meat}{,,Dzisiaj zjem mięso.''}

\subsection{Aspects}

\emph{Vi} to partykuła/operator aspektu niedokonanego. Aspekt niedokonany jest
,,domyślny'' dla czasu teraźniejszego, można czasami jednak jej używać dla
podkreślenia niezakończenia czynności.

\glossex{Ti fesgai.}{ti fesgai}{2SG read.PRS}{,,Ty czytasz.'' \\ ,,Czytasz.''}

\glossex{Ti vi fesgai.}{ti vi fesgai}{2SG IPFV read.PRS}{,,Ty czytasz. [i wciąż nie skończyłeś]''}

\glossex{Ti vi fesgat.}{ti vi fesga-t}{2SG IPFV read-PST}{,,Ty zacząłeś czytać. [w przeszłości i wciąż nie skończyłeś]''}

\glossex{Mi vi koólet egi.}{mi vi koóle-t egi}{1SG IPFV love-PST 3SG}{,,Pokochałem ją/jego w przeszłości. [i wciąż kocham]''}

\emph{Va} to operator aspektu dokonanego. Aspekt dokonany jest uznawany za
domyślny w~formie czasu przeszłego i~za bardzo nie ma sensu w~czasie
teraźniejszym.

\glossex{Ti va fesgat.}{ti va fesga-t}{2SG PFV read-PST}{,,Przeczytałeś.''}

\glossex{Va koólet egla.}{va koóle-t egla}{PFV love-PST 3SG.F}{,,Kochałem ją. [w przeszłości]''}

\glossex{Egi palimit e mi.}{egi palimi-t e mi}{3SG talk-PST and 1SG}{,,On rozmawiał ze mną.''}

Partykuły \emph{va} oraz \emph{vi} mogą być pomijane, jeżeli będą wynikały z~
kontekstu.

\subsection{Passive voice}

\emph{Ge} to operator strony biernej.

\glossex{Sotak ge karlet chu polno sotak.}{sotak ge karle-t chu polno sotak}{somebody PASS kill-PST ACC different somebody}{,,Ktoś został zabity przez kogoś innego.''}

Wykorzystana tutaj może być partykuła \emph{chu} określająca dopełnienie, ale
nie jest to wymagane. W zdaniu tym również nie zastosowano partykuły \emph{va},
gdyż aspekt dokonany jest ,,domyślny''.

\subsection{Imperative and Optative constructs}

Partykuła \emph{do} służy jako operator trybu rozkazującego (\Imp{}). Trybu
rozkazującego należy w~miarę możliwości unikać, zazwyczaj przez użycie operatora
\emph{vage} lub \emph{hemi}, jako, że jest uznawany za niegrzeczny.

Zawsze znajduje się na końcu zdania. W~trybie rozkazującym użycie zaimka
osobowego może zostać oczywiście pominięte, zwłaszcza jeżeli zwracamy się
bezpośrednio do odbiorcy polecenia. W niektórych dialektach używany razem z
partykułą \emph{ze}.

\glossex{Toi tori dej́itos do!}{toi tori dej́it-os do}{2PL throw weapon-PL IMP}{,,Rzućcie broń!'' \\ ,,Wy rzućcie broń!''}

\glossex{Tori dej́itos do!}{tori dej́it-os do}{throw weapon-PL IMP}{,,Rzuć broń!''}

\emph{Hemi} to czasownik oznaczający ,,prosić'', jednak może zostać wykorzystany
zamiast operatora \emph{do} do określenia trybu ekshortatywnego. W podobnej roli
można również użyć czasownika \emph{ih́emi}, który jest ,,silniejszy'' i oznacza
,,błaganie''.

\glossex{Mudi mi fayse, hemi.}{mudi mi fayse hemi}{give 1SG drink please.HORT}{,,Podaj mi napój, proszę.'' \\ ,,Czy mógłbyś mi podać napój?''}

\glossex{Karla͞i no mi, ih́emi.}{karla͞i no mi ih́emi}{kill NEG 1SG beg.HORT}{,,Błagam, nie zabijaj mnie.''}

\glossex{Inra͞i ti egi karlet no, ih́emi.}{inra͞i ti egi karlet no ih́emi}{tell 2SG 3SG kill-PST NEG beg.HORT}{,,Błagam, powiedz, że go nie zabiłeś.''}

Z kolei zachęta, ,,zróbmy to'', czyli tryb hortatywny ze wskazaniem, że coś
zostanie wykonane wspólnie przez adresata i podmiot (kohortatywny, \Chr{}),
wykorzystuje partykułę \emph{heme} lub wręcz \emph{hemee}, zwłaszcza w
nieformalnej mowie.

\glossex{Zetay ze edihi heme!}{zetay ze edihi heme}{tomorrow FUT meet CHR}{,,Spotkajmy się jutro!''}

\glossex{Eni nafiye faysi hemee!}{eni nafiye faysi hemee}{go beer drink CHR}{,,Chodźmy na piwo!'' \\ ,,Chodźmy się napić piwa!''}


\emph{Vige} to bardzo formalny operator ,,niech się stanie'' (\emph{optativus},
tryb życzący \Opt{}):

\glossex{Vige hallo͞i nome yi Ori!}{Vige hallo͞i nome yi ori}{OPT praise name POSS god}{,,Chwała imieniu Boga!'' \\ ,,Niech będzie chwalone imię Boga!''}

Natomiast \emph{vage} to również operator ,,niech się stanie'', ale wyrażający
coś bardzo konkretnego. W~odróżnieniu od \emph{vige}, który nie
oczekuje ,,fizycznego'' rezultatu. W~odniesieniu do konkretnego odbiorcy
oznacza ,,powinieneś coś zrobić'', ale jest traktowane nieco jak rozkaz,
silniej niż np. \emph{hemi}. Można rozumieć jako jedną z odmian trybu
hortatywnego (\Hort{}).

\glossex{Vage fesgai.}{vage fesgai}{HORT read.PRS}{,,Niech ktoś to przeczyta.'' \\ ,,Powinieneś to przeczytać.'' \\ ,,Przeczytaj to.''}

\glossex{Ti vage fesgai.}{ti vage fesgai}{2SG HORT read.PRS}{,,Powinieneś to przeczytać.'' \\ ,,Przeczytaj to.''}

\glossex{Ti vage fesgai ja lana.}{ti vage fesgai ja lana}{2SG HORT read.PRS DEM book}{,,Powinieneś przeczytać tę książkę.''}

\note{\emph{vage} nie oznacza automatycznie rozkazu, jednak jest stosowane jako
  jego grzeczniejszy i bardziej formalny odpowiednik. W szczególności można się z
  nim spotkać w szkole czy pracy, kiedy nauczyciel lub przełożony nakazują
  wykonanie pewnej pracy.}

\section{Demonstratives}
\label{sec:demonstratives}

\subsection{Partykuła \emph{ja}}

Nie są stosowane rodzajniki określone lub nieokreślone, ale istnieje możliwość
wskazania na konkretny obiekt za pomocą partykuły \emph{ja}.

\glossex{Ja muche esi ruko}{Ja muche esi ruk-o}{DEM cat be.PRS black-ADJ}{,,Ten kot jest czarny'' \\ ,,Ten konkretny kot jest czarny''}

Rzeczowniki nie ulegają odmianie przez przypadki, do oznaczania których używa
się głównie partykuł, patrz sekcję poświęconą szykowi zdania.

\section{Conditionals}
\label{sec:conditionals}

Czysty tryb przypuszczający (warunkowy, \textsc{cond}) realizowany jest za
pomocą partykuł \emph{miam} oraz \emph{vimi}.

\glossex{Miam mi va fesgat sepo͞e vimi ti diyu fari.}{miam mi va fesga-t sepo͞e vimi ti diyu fari}{if.COND 1SG PFV read-PST early.ADV then.COND 2SG something do}{,,Gdybym wcześniej przeczytał to ty byś coś zrobił.''}

\glossex{Miam mi ze va fesgai vimi ti che fari.}{miam mi ze va fesgai vimi ti fari}{if.COND 1SG FUT PFV read then.COND 2SG DEM do}{,,Gdy przeczytam, to zrób to.''}

Można zastosować partykułę \emph{abe}, aby określić warunek przeciwny:

\glossex{Miam egli ze avi deíto vimi ti lugiti abe reki! }{miam egli ze avi deíto vimi ti lugiti abe reki}{if.COND 3SG.M FUT have weapon then.COND 2SG escape but hit}{,,Jeśli on będzie miał broń, uciekaj, a w przeciwnym razie -- atakuj.''}

Partykuła \emph{vimi} może być pomijana, o ile zastosuje się szyk zdania podrzędnego:

\glossex{Miam ze va fesgai, mi jeoza chu ti ze dari.}{Miam ze va fesgai, mi jeoza chu ti ze dari}{if.COND FUT PFV read 1SG candy ACC 2SG FUT give}{,,Jeśli to przeczytasz, to dam ci cukierka.'' \\ ,,Gdy to przeczytasz, to dam ci cukierka.''}

W odniesieniu do czynności zakończonych będzie to miało znaczenie ,,skoro-to'',
jednak alternatywnie w tej roli można również stosować partykułę \emph{imin}.

\glossex{Miam ze fesgat, jeoza chu ti.}{Miam ze fesga-t, jeoza chu ti}{if.COND PFV read-PST candy ACC 2SG}{,,Skoro przeczytałeś, cukierek dla ciebie''}

\glossex{Imin ze fesgat, jeoza baljezi.}{imin ze fesga-t, jeoza baljezi}{because PFV read-PST candy receive.PRS}{,,Skoro przeczytałeś, dostajesz teraz cukierka.''}

Sama partykuła \emph{vimi} służy, w szyku pytającym, do zadawania pytań ,,czy
nie powinniśmy czegoś zrobić?''.

\glossex{Enitya vimi fesgai alive ?}{enitya vimi fesgai alive}{instruction COND read before.ADV}{,,Nie powinieneś wcześniej przeczytać instrukcji?''}

\section{Conjuctions}
\label{sec:conjunctions}

\section{Comparatives}
\label{sec:comparatives}

\section{Diminutives, familiarity and formality}
\label{sec:diminutives}

\subsection{Honorification, names and surnames}

W zależności od regionu możliwe jest, że użytkownicy języka będą bardzo
wyczuleni na kwestie grzecznościowe. Typowym tego typu elementem jest niechęć do
stosowania operatora trybu rozkazującego \emph{do} na rzecz \emph{hemi}, ale
bardzo często możesz również spotkać się z~honoryfikatorami służącymi do
odpowiedniego zwracania się do innych osób.

W And́royas typowym jest używanie pierwszego imienia w~odniesieniu do rozmówcy,
lub podczas określania osoby, chyba, że jest to niemożliwe do jednoznacznej
identyfikacji, wtedy używa się pełnego imienia i~nazwiska. Z drugiej jednak
strony mieszkańcy Cesarstwa są bardzo dumni ze swojej rodziny i~swojego rodowego
nazwiska, stąd przedstawiajac się często to podkreślą przedstawiając się pełnym
imieniem oraz nazwiskiem.

\glossex{Mi nomi Eryus mal Edoraril.}{mi nomi Eryus mal Edoraril}{1SG name.PRS Eryus of.family Edoraril}{,,Nazywam się Eryus mal Edoraril.''}

Warto tutaj zwrócić uwagę, że imiona i~nazwiska w~Cesarstwie są rozdzielane
partykułą \emph{mal}, oznaczającą ,,z rodziny'', np. \emph{Koolder mal
  Erlehirni} to Koolder z rodziny Erlehirni. Czasami możliwe jest, że dzieci
dziedziczą nazwiska po obojgu rodziców, stąd występują nazwiska łączone
łącznikiem, takie jak \emph{Alya mal Arkai-Valor}. Istnieją również także
oznaczane za pomocą łącznika gałęzie rodów, które dziś stały się zwykłymi
nazwiskami, np. \emph{Nimu͞e mal Hetasi-Hi}, co oznacza ród Hetasi i jego gałąź
Hi.

\note{Gałęzie rodowe, i co za tym idzie, ich oznaczenia w nazwiskach pojawiały
  się w~sytuacji, kiedy nazwisko rodowe przechodziło tylko na pierwsze dziecko,
  natomiast kolejne dzieci uzyskiwały nazwiska z~określeniem gałęzi. Zwyczaj ten
  zanikł prawie całkowicie około VII wieku po Zjednoczeniu.}
\skipline

Używa się zaimków osobowych \emph{epié} i~\emph{epiá}, które odpowiadają mniej
więcej polskim ,,pan'' i~,,pani''. Używa się ich w~odniesieniu do obcych osób
albo osób stojących wyżej w~hierarchii, albo w~sytuacji, kiedy nie znamy imienia
osoby, do której chcemy się zwrócić. Bardzo często można napotkać ich stosowanie
w postaci przyrostków z~łącznikiem, w~stosunku do imienia, np. mówiąc o kimś
wyżej w~hierarchii możemy powiedzieć \emph{Koolder-epié}. W~podobny sposób
można określać czyjąś funkcję, np. \emph{Furu-falazera} -- dowódca Furu. Czasami
można napotkać formę \emph{falazera-epiá} -- pani dowódca. W~taki sposób można
używać słów takich jak \emph{falazer} (dowódca), \emph{kachister} (nauczyciel),
\emph{meneder} (lekarz) i~innych.

\note{\emph{epié} i~\emph{epiá} praktycznie nie są używane w Republice Nennek,
  gdzie preferowane jest zwracanie się imieniem, zaimkiem ogólnym \emph{egi} lub
  zależnymi od płci \emph{egli/egla} lub ewentualnie przyrostkiem \emph{-gam},
  jeżeli naprawdę chce się podkreślić swoją niższą pozycję wobec rozmówcy.}

\note{W momencie gdy dwoje rozmówców będzie traktować się nawzajem z identycznym
  poziomem grzeczności, będą się do siebie zwracać nawzajem ukazując swoją niższą
  pozycję, np. nawzajem tytułować siebie z przyrostkiem \emph{-epié}.}
\skipline

Z drugiej strony, możliwe jest, że rozmówca będzie traktować drugą osobę jako
osobę od niego niższą statusem, ukazując swoją wyższą pozycję. Jest to
oczywiście niegrzeczne i stąd bardzo rzadko spotykane. Przyrostkami takimi mogą
być rzeczownik \emph{pezawe} (,,gorszy człowiek'') lub wręcz rzeczownik
\emph{zam} (dosłownie ,,śmieć''). Bardzo pogardliwe, spotkane w sytuacji i
próbach zastraszenia rozmówcy.

\subsection{Diminutives}

Oczywiście, w codziennych sytuacjach osoby sobie bliskie nie będą używały
określeń stricte formalnych -- do babci raczej wnukowie zwrócą się
\emph{chancha} niż \emph{gruchana}, jeśli są z nią blisko, a do swoich rodziców
\emph{mama} i \emph{patal} bardziej niż \emph{natali͞a} oraz \emph{vapal}.

W podobny sposób stosowane są często zdrobienia i partykuły lub rzeczowniki
z~nimi związane, takie jak \emph{myi}, który może być stosowany jako przyrostek
zdrabniający (\textsc{dim}), np. \emph{pelir-myi} -- ,,mój pieseczek'', czy też
\emph{koól}, stosowany np. \emph{Alya-koóla} -- ,,kochana Alya'', stosowany w
odniesieniu do osoby darzonej uczuciem.

Istnieje również słabsza wersja \emph{koól}, \emph{arey}, przyimek stosowany do
określenia sympatii do drugiej osoby. Może być również stosowany do określenia
sympatii do rzeczy, bez określenia jej posiadania, w przeciwieństwie do
\emph{myi}.

\subsection{Titles}

Jako monarchia, Cesarstwo posiadało szereg tytułów szlacheckich (np. \emph{lir},
czy \emph{eber}) jednakże od czasów początku Trzeciego Cesarstwa zostały
wycofane z użytku. Wciąż jednak, w ekstremalnie formalnym języku można stosować
te określenia jako przyrostek funkcyjny (na wzór np.~\emph{Koolder-epié} --
\emph{Koolder-lir}), jednak nie są stosowane w drugiej osobie (\textsc{2SG}), a
zamiast tego stosuje się \emph{arḱer/arḱera} dla podkreślenia formalności.

\note{\emph{arḱer/arḱera} nie obowiązuje przy zwracaniu się do członków
  rodziny cesarskiej oraz rodzin królów i książąt, do których należy zwracać
  się zaimkiem \emph{rige/rigea} (,,władca'' / ,,władczyni'')) w mniej
  formalnych sytuacjach i tytułem w bardziej formalnych.}

\note{W przypadku osoby pełniącej funkcję \emph{ajor} również stosuje się tytuł
  jako zaimek w drugiej osobie.}

Określenie \emph{arḱer} stosowane było również do zwracania się do osób
pełniących niektóre funkcje, np. burmistrza lub zarządcy miasta, często w formie
przyrostku funkcyjnego, obecnie jest to bardzo rzadkie.

Lista tytułów szlacheckich:

\begin{itemize}
  \item \emph{kyige/kyige͞a} -- król/królowa,
  \item \emph{kigeje/kigeje͞a} -- książę/księżna/księżniczka (tytuły Rodziny
        Cesarskiej)
  \item \emph{ajor/ajora} -- obecnie jest to funkcja administracyjna w
        Cesarstwie, oznaczająca ,,gubernatora'', osobę zarządzającą regionem
        administracyjnym,
  \item \emph{lir/lira} -- hrabia/hrabina,
  \item \emph{eber/ebera} -- baron/baronessa,
  \item \emph{arḱer/arḱera} -- ogólna forma ,,lord'' (,,lady'').
\end{itemize}

\subsection{Honorary phrases}

W odróżnieniu od kultur ziemskich, w And́royas nie przyjęła się koncepcja
zwrotów honorowych, np. ,,Jego Najjaśniejsza Wysokość Książę Lichtensteinu''
raczej byłby tytułowany po prostu \emph{Kigeje yi Lihtenchuteyn}, względnie jak
każdy inny tytuł jako przyrostek, np. \emph{Henrik-lihtenchuteynyikigeje}.

Dla podkreślenia ważności osoby o której się mówi, lub do której się zwraca, w
wysoce formalnych zwrotach stosuje się przyrostek \emph{-epié} w stosunku do
funkcji, np. \emph{diosever-epié} -- ,,pan sierżant'', ,,panie sierżancie''.

\note{W odniesieniu do głów państw raczej powinno używać się formy \emph{arḱer},
  np. w stosunku do królów, książąt, prezydentów czy Cesarzowej.}

Istnieje jeden zwrot honorowy stosowany do dzisiaj, \emph{ardo arḱeji} --
,,wysocy panowie'', stosowany do grupy wysoko postawionych osób.

\subsection{The Royal Family}

W przypadku rodziny cesarskiej używa się określeń \emph{eyger} oraz
\emph{eygera}, np. \emph{Fayfnira-eygera} -- cesarzowa Fayfnira, ale i~takich
jak \emph{and́royasyikigje͞a} (księżniczka And́royas -- tj. siostry
cesarza/cesarzowej), czy \emph{and́royasyikigeje} (książę And́royas -- bracia
panującego).

Z kolei dzieci panującego często określane są tytułami \emph{icheryikigeje},
\emph{icheryikigeje͞a} lub \emph{icheryihima} -- dosłownie książę lub
księżniczka krwi. Oprócz tego istnieje przyrostek \emph{-hima}, stosowany często
na Wschodzie w~stosunku do całej żeńskiej strony rodu panującego, poza
Cesarzową, który z kolei na Zachodzie często jest używany zamiennie z
\emph{-hina} jako "panna" dla kobiety niezamężnej.

Małżonek panującego może posiadać tytuł zarówno równorzędny -- np. \emph{eyger},
ale i~na przykład \emph{eygeryikigeje}, dosłownie ,,książę cesarstwa'' lub
,,książę cesarzowej''. Dokładne zasady są zależne od aktualnej sytuacji
politycznej.

Stąd aktualnie (w momencie pisania tej książki), mamy:

\begin{itemize}
  \item \textbf{Katia-eygera mal Arkai}\\ \xt{ˈka.ti.a ˈɛj.gɛ.ra ˈmal ˈar.ka.i},\\
        Cesarzowa Katia mal Arkai,
  \item \textbf{So'tak-eygeryikigeje mal Valor}\\ \xt{ˈsɔ|.tak ˈɛj.gɛ.rʏ.ki.gɛ.ʐɛ
          ˈmal ˈva.lɔr},\\ Ksiażę Cesarzowej, So'tak mal Valor,
  \item \textbf{Alya-icheryikigeje͞a mal Arkai-Valor}\\\xt{ˈal.ja i.ʈ͡ʂe.rʏ.ki.gɛ.ʐɛa ˈmal
          ˈar.ka.i-va.lɔr},\\ Księżniczka Krwi, Alya mal Arkai-Valor,
  \item \textbf{Niva-and́royasyikigeje͞a mal Arkai}\\\xt{ˈni.va an.ˈdrɔ.ja.sʏ.ki.gɛ.ʐɛa
          mal ˈar.ka.i},\\ Księżniczka And́royas, Niva mal Arkai,
  \item \textbf{Karra-and́royasyikigeje͞a mal Arkai}\\\xt{ˈkar.ra an.ˈdrɔ.ja.sʏ.ki.gɛ.ʐɛa
          mal ˈar.ka.i},\\ Księżniczka And́royas, Karra mal Arkai,
  \item \textbf{Jaida-and́royasyikigeje͞a mal Arkai}\\\xt{ˈʐa.i.da an.ˈdrɔ.ja.sʏ.ki.gɛ.ʐɛa
          mal ˈar.ka.i},\\ Księżniczka And́royas, Jaida mal Arkai.
\end{itemize}

\note{Należy tutaj zwrócić uwagę na wymowę imienia Jej Wysokości, w~której
  głoski /i/ oraz /a/ nie zlewają się w~/ia/, oraz na imię Jego Wysokości, w~
  którym występuje pauza pomiędzy sylabami, oznaczana przez <'> w~transkrypcji.
  Wynika to z~faktu, że Jego Wysokość pochodzi z~wyspy Rem, gdzie pojawiają
  się takie, unikatowe, elementy języka, z~uwagi na wpływ języków krajów
  ościennych.}