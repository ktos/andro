\chapter{Syntax}
\label{ch:syntax}

\andro, as mentioned earlier, is mostly head-final, but sometimes head-initial
language, and such incoherence is caused by its complicated history. In this
chapter, different type of sentence structures will be presented, so you will be
able to understand how different clauses are working in general.

\section{Basic sentence structures}
\label{sec:basic}

The most basic structure for declarative sentences is the SVO
(Subject-Verb-Object) sentence order.

\glossex{Mi pazi muchi.}{mi pazi much-i}{1SG like cat-PL}{``I like cats.''}

\subsection{Modal verbs and nominalization}

There is a very short list of modal, auxiliary verbs, used to express
likelihood, ability, permission, order, and obligation, especially in the case
of stating the fact. In any other case, nominalization is usually preferred. The
modal verbs are:

\begin{itemize}
    \item \randro{epi} -- to may,
    \item \randro{pazi} -- to like,
    \item \randro{kiruki} -- to can,
    \item \randro{vipini} -- to make someone do something.
\end{itemize}

\glossex{Mi kiruki feni.}{mi kiruki feni}{1SG can.AUX swim.PRS}{``I can swim.''}

\glossex{Pazi fesgai.}{pazi fesgai}{like.AUX read.PRS}{``I like reading.''}

Modal verbs are being put in the place of the regular verb, while moving the
action verb to the end of the sentence.

\glossex{Mi epi chet seysi.}{mi epi chet seysi}{1SG be.able.AUX here.LOC sit.PRS}{``I can sit here.''}

Conjugation, sentence order and operator markings are being used for the modal
verb, while the verb marking the action stays uninflected.

\glossex{Mi vi kiruket feni, abe va tayet.}{mi vi kiruke-t feni, abe va taye-t}{1SG IPFV can.AUX-PST swim.PRS but PFV forget-PST}{``I knew how to swim, but I forgot.''}

\glossex{Chet seysi mi epi?}{chet seysi mi epi}{here.LOC sit.PRS 1SG be.able.AUX}{``May I sit here?''}

\subsection{Nominalization and marking action}

Nominalization, mentioned in the morphology context, is achieved using the
\randro{na} particle, the nominalizer (\Nmlz{}). The nominalizer is changing the
verb into the noun, and allows using the actions as the accusative. Nominalizer
can be used in the suffix form, and usually is, but sometimes one may encounter
the head-initial form, after the verb, as a separate word.

\glossex{Egi dekit rekina.}{egi deki-t reki-na}{3SG decide-PST attack.V-NMLZ}{``He decided to attack.''}

\glossex{Ze esi gusto eni na.}{ze esi gusto eni na}{FUT be happy go NMLZ}{``I will be happy going.'' \\ ``I will be happy to go.''}

In case of modal verbs, nominalization must not be used.

\subsection{Negations}

Negation marker is a particle \randro{no}, which is used in
\textbf{head-initial} way in verb clauses, always after the verb.

\note{However, in some religious texts and formal contexts, \randro{no} particle
    may be found always at the end of the sentence. This is a remnant from the Old
    Nennekan language.}

\glossex{Mi ze karli no!}{mi ze karli no}{1SG FUT die NEG}{``I will not die!''}

\glossex{Ti bugi no.}{ti bugi no}{2SG lie NEG}{``You do not lie.''}

\glossex{Tori no dej́itos do!}{tori no dej́it-os do}{throw NEG weapon-PL IMP}{``Do not throw the weapon!''}

\glossex{Egi karlet no il rige.}{egi karle-t no il rige}{3SG kill-PST NEG 3SG.POSS lord}{``He did not kill his lord.''}

\subsection{Double negative}

In case of some adverbs which are expressing negative context, you may wonder
what about the double negative? In \andro both negative and double negative are
correct, and they are both expressing just the negative.

\glossex{Ze karli moĺi.}{ze karli moĺi}{FUT die never.ADV}{``I will never die!''}

\glossex{Ze karli no moĺi.}{ze karli no moĺi}{FUT die NEG never.ADV}{``I will never die!''}

\subsection{Negation marker for nouns}

The negation particle \randro{no} may be used to negate a noun. This is used for
a special construct ``instead of''.

\glossex{A femji inji no rujalaros femit}{a fem-ji inji no rujalar-os femi-t}{on branch-PL leave-PL instead.of people-PL hang-PST}{``On the branches there were people hanging, instead of leaves.''}

\subsection{Questions}

The basic sentence order for question is OSV (Object-Subject-Verb).

\glossex{Muchi ti pazi?}{much-i ti pazi}{cat-PL 2SG like}{``Do you like cats?''}

There is no question marker and very typical is that basic questions are being
asked with an intonation -- typically, a rising one on the verb.

\glossex{Il maŕie͞o egi koóli?}{il maŕie͞o egi koóli}{3SG.POSS spouse 3SG love}{``Does she love her husband?'' \\ ``Does he love his wife?'' \\ ``Does she love her spouse?''}

\note{Both \emph{egi} and \emph{marié͞o} may be used for all genders.}

\glossex{Il hinna egi esi?}{il hinna egi esi}{3SG.POSS girlfriend 3SG COP}{``Is her his girlfriend?''}

\glossex{Sotak epié karlet?}{sotak epié karle-t}{somebody 2SG.M.FRM kill-PST}{``Sir, have you killed someone?''}

\glossex{Je rujaler epié ze va karla͞i?}{Je rujaler epié ze va karla͞i}{DEM man 2SG.M.FRM FUT PFV kill}{``Sir, will you kill this man?''}

\glossex{Mi epié ze vi karla͞i?}{mi epié ze vi karla͞i}{1SG 2SG.M.FRM FUT IPFV kill}{``Sir, will you try to kill me?''}

\glossex{Il alye epiá veyt hetay?}{il alye epiá vey-t hetay}{3SG.POSS friend 2SG.F.FRM see-PST today}{``Have you seen your friend today, madam?''}

\glossex{Ti va fesgat?}{ti va fesga-t}{2SG PFV read-PST}{``Did you read it?''}

In the case of the modal verb, the basic verb at the end of the sentence is
still true.

\glossex{Chet mi epi seysi?}{chet mi epi seysi}{here 1SG be.able sit}{``May I sit here?''}

\glossex{Chet mi va epit seysi?}{chet mi va epi-t seysi}{here.LOC 1SG PFV be.able-PST sit}{``Was I able to sit here?''}

The emphasis particle, \randro{vay} may be used to place an emphasis to a
question:

\glossex{Vay tyi vapal ti kipeni?}{vay tyi vapal ti kipeni}{really.EMPH.FRM 2SG.POSS father 2SG model}{``Are you really imitating your father?''}

For the same purpose, \randro{elĺa} may be used, however is commonly perceived
as informal.

\glossex{Elĺa tyi patal ti kipeni?}{elĺa tyi patal ti kipeni}{really.EMPH.NFRM 2SG.POSS father.NFRM 2SG model}{``Are you really imitating your dad?''}

\subsection{Tag questions}

It may also be used in transformation from the declarative sentence into a
asking of the confirmation, similarly to the English ``right?'' or Polish
``prawda?''. Tag questions in \andro are perceived as a part of informal speech.

\glossex{Ti pazi muchi, elĺa?}{ti pazi much-i elĺa}{2SG like cat-PL really.Q.NFRM}{``You like cats, aren't you?''}

\subsection{Interrogatives}

There is a set of interrogative particles, which are used in direct asking
questions related to the posession, method, place, time or number.

\begin{itemize}
    \item \randro{chyi} -- whose,
    \item \randro{ko͞e} -- how, in what way,
    \item \randro{osor} -- why,
    \item \randro{so} -- what,
    \item \randro{somar} -- where,
    \item \randro{soter} -- who,
    \item \randro{voli} -- when,
    \item \randro{wodo} -- what way,
    \item \randro{yage} -- where to,
    \item \randro{yasu} -- where from,
    \item \randro{ali} -- how much, how many,
    \item \randro{chu͞i} -- which one,
    \item \randro{vaja} -- is is true that.
\end{itemize}

\glossex{Yasu ti iéni?}{yasu ti i-éni}{from.where.Q 2SG VEN-go}{``Where do you came from?''}

\glossex{Yasu ti iént?}{yasu ti i-én-t}{from.where.Q 2SG VEN-go-PST}{``Where did you came from?''}

\glossex{Ko͞e tyi ager nomi?}{ko͞e tyi ager nomi}{how.Q 2SG.POSS country to.name}{``How is your country named?''}

\glossex{Ko͞e ti seiti?}{ko͞e ti seiti}{how.Q 2SG feel}{``How do you feel?''}

\glossex{So tyi vahuryiáysi o ti famei?}{so tyi vahuryiáysi fo ti famei}{what.Q 2SG.POSS tattoo DAT 2SG mean}{``What does your tattoo mean for you?''}

\glossex{Wodo o Poleska mi ze cheri?}{wodo o Poleska mi ze cheri}{which.way.Q to Poland 1SG FUT cheri}{``Which way should I go to Poland?''}

\glossex{Mi eni o jan, per mi vibo vibi.}{eni o jan per mi vibo vibi}{1SG go to.LOC home in.order.to 1SG food eat}{``I am going home in order to eat food.''}

\glossex{Cherlok Holmes va choint sosbet kayetor esi no, e egi o jan eni permet.}{cherlok holmes va choin-t sosbet kayetor esi no e o jan eni perme-t}{Sherlock Holmes PFV state-PST suspect murderer be NEG and to home go allow-PST}{``Sherlock Holmes stated that suspect is not a murderer and allowed him to go home.''}

\glossex{Cherlok Holmes va choint sosbet kayetor esi no, abe egi va est - abe deíto va kopet lipe.}{cherlok holmes va choin-t sosbet kayetor esi no abe egi va est abe deíto va lope-t lipe}{Sherlock Holmes PFV state-PST suspect murderer be NEG but 3SG PFV be but weapon PFV hide-PST well}{``Sherlock Holmes stated that he was not a murderer -- but he was, but has hidden the weapon well.''}

In subordinate clauses, the verb may be omitted, if there is a possibility to
get it from the overall context.

\section{Case markers}
\label{sec:cases}

It is now our turn to introduce you to the most problematic part of \andro --
the vast variety of particles, which are performing different grammatical
functions.

In this section the markers for the cases will be introduced, while later in
this chapter you will see other particles for various grammatical functions.

Particles marking grammatical function of the noun phrase are
\textbf{head-final} and in some situations may and usually will be omitted, if
the sentence is understandable without.

\subsection{Accusative}

In the most cases accusative (\Acc{}) is not marked in any way and is directly
known because of the word order of the sentence.

\glossex{Mi veydi yasaji.}{mi veydi yasa-ji}{1SG see eel-PL}{``I see eels.''}

However, you may sometimes encounter the usage of the \randro{chu} particle for
accusative marking. This is used in three situations -- in regional variants, in
older texts, and for emphasis.

\glossex{Mi veydi chu yasaji.}{mi veydi chu yasa-ji}{1SG see ACC.EMPH eel-PL}{``I really see eels.''}

The problem is, that \randro{chu} may perform many other grammatical functions,
which will be described later in this chapter, so sometimes you have to infer
from the context what is the role of this particle in the particular sentence.

In \ardo, no special form of a pronoun is used when the pronoun is used in
accusative:

\glossex{Mi veydi ti!}{mi veydi ti}{1SG see 2SG}{``I see you!''}

However, in dialects, especially Nennekan, as a remnant of the Old Nennekan
language, a possessive variant of the second or third person pronouns is used in
the accusative:

\glossex{Mi veydi tyi!}{mi veydi tyi}{1SG see 2SG.ACC}{``I see you!''}

\subsection{Possessive}

The possessive (\Poss{}) particle, \randro{yi}, as mentioned in the morphology
section, is marking the possessive, and very commonly may be used as a suffix.

When used in standalone, it is adhering to the ``object yi possessor'' format,
similarly to the English ``of'':

\glossex{Vipetode yi muche.}{vipetode yi muche}{bowl POSS cat}{``A cat's bowl.'' \\ ``A bowl of the cat.''}

In the case of multiple possessive nesting very typical is to mix a suffix
version along with a standalone particle.

\glossex{Vipetode yi myi arśityi natali͞a}{vipetode yi myi arśit-yi natali͞a}{bowl POSS 1SG.POSS parent-POSS mother}{``A bowl belonging to my parent's mother.'}

(although \randro{vipetode yi natali͞a yi myi arśit} is of course equally
correct)

The suffix form is preferred in the~dialects of the West, in~which there are
strong influences of agglutinative Kairean languages.

You need to remember, that possessive marker is \textbf{head-final}, but the
suffix form is -- as the name suggests -- a suffix. So there is \randro{vipetode
    yi muche}, but \randro{mucheyi vipetode} -- be vary of this switch!

\subsection{Genitive}

Genitive (\Gen{}) case is another usually left unmarked.

\glossex{Myi keromamerey esi dowo yasaji.}{myi keromamerey esi dowo yasa-ji}{1SG.POSS hovercraft COP full eel-PL}{``My hovercraft is full of eels.''}

However, similarly to the accusative, \randro{chu} may be used, especially for
emphasis.

\glossex{Myi keromamerey esi dowo chu yasaji.}{myi keromamerey esi dowo chu yasa-ji}{1SG.POSS hovercraft COP full GEN.EMPH eel-PL}{``My hovercraft is full of eels.''}

You may encounter \randro{chu} as genitive marker in older texts.

\subsection{Dative}

Similarly to the accusative and genitive, dative (\Dat{}) is usually left
unmarked and the word order is taking care of showing what is the accusative and
what is dative.

\glossex{Mi jawirit ti chider.}{mi jawiri-t ti chider}{1SG steal-PST 2SG bike}{``I stole a bike from you!''}

However, in cases where there is a need for marking or for emphasis of a dative
case, a \randro{fo} particle may be used.

\glossex{Mi jawirit chider fo ti.}{mi jawiri-t chider fo ti}{1SG steal-PST bike DAT 2SG}{``I stole a bike \textbf{for} you.''}

% chu͞a

\subsection{Instrumentative}

Instrumentative (\Ins{}) marker, \randro{da}, is describing the usage of the
particular object to perform the action.

\glossex{Id́ak ostro da keja.}{id́a-k ostro da keja}{open-PST door INS key}{``I opened the door using a key.'' \\ ``I opened the door with a key.''}

\subsection{Comitative}

Comitative (\Com{}) marker, \randro{a} is describing the performing the action
along with some other animated being, so the situations where ``me and you'',
``you and your cat'' are doing something together.

\glossex{Mi labi a ti.}{mi labi a ti}{1SG play together.with.COM 2SG}{``I am playing with you.'' \\ ``Me and you are playing.'' \\ ``We are playing.''}

Comitative marker is a homophone of a Desert dialect locative marker \randro{a},
so they should not be used in the same sentence.

It may be omitted in some situations, or used as in emphasis along with
conjunction or plural numbers.

% a͞u jako COM

\glossex{Egyi vi aḿachagit a.}{egyi vi aḿachagi-t a}{3PL IPFV sleep-PST COM}{``They were sleeping with each other.'' \\ ``They were having sex with each other.''}

% one

\subsection{Locative}

Finally, to mark location, there is a set of locative (\Loc{}) particles marking
the position in both time and space. They are very similar to their English
counterparts.

\begin{itemize}
    \item \randro{in} -- in,
    \item \randro{on} -- on (physically),
    \item \randro{an} -- above,
    \item \randro{a} -- on (physically) (Desert dialect),
    \item \randro{ne͞a} -- close, near,
    \item \randro{ner} -- close, near (physically),
    \item \randro{churche} -- through,
    \item \randro{chet} -- here,
    \item \randro{dyet} -- behind
    \item \randro{huhu} -- after
    \item \randro{intay} -- in the day
    \item \randro{levi} -- in the left side, to the left,
    \item \randro{rivi} -- in the right side, to the right,
    \item \randro{si} -- outside,
    \item \randro{temu} -- temporal behind.
\end{itemize}

They are all \textbf{head-final}, and the most popular is \randro{in}, marking
generally everything.

\glossex{Mi esi in mibozor.}{mi esi in mibozor}{1SG COP in.LOC school}{``I am in the school.'' \\ ``I am a student.''}

\glossex{Mi esi in mibozor.}{mi esi in mibozor}{1SG COP in.LOC school}{``I am in the school.'' \\ ``I am physically in a school building.''}

\glossex{Mi loti in Nowaja.}{mi loti in nowaja}{1SG live in.LOC Nowaja}{``I live in Nowaja.''}

\randro{in} is also marking the direction:

\glossex{Mi vayareni in Ameriḱa.}{mi vayareni in amerika}{1SG travel in.LOC Nowaja}{``I travel to the United States.''}

As well as position in time:

\glossex{Seja ze haji in zetay}{seja ze haji zetay}{sun FUT shine in.LOC tomorrow}{``The sun will shine tomorrow.''}

\glossex{Myi sucha obi in nati naja.}{myi sucha obi in na-ti naja}{1SG.POSS work start.PRS in.LOC five-ORD hour}{``My work starts at five o'clock.''}

\glossex{In kati flosek ti sidstano?}{in ka-ti flosek ti sidstano}{in.LOC two-ORD second.month 2SG have.free.time}{``Are you free on the second day of the second month?''}

\note{The second month in And́royas calendar is \tandro{flosek}{the month of flowers}.}

However, in the case of time, \randro{in} is commonly omitted in the informal
speech, when not talking about a specific date.

\glossex{Seja ze haji zetay}{seja ze haji zetay}{sun FUT shine tomorrow}{``The sun will shine tomorrow.''}

The \randro{on} is used almost exclusively in the physical context, when
describing something is on top of something else or is/was going to the top of
something else. \randro{a} is its counterpart, used mostly in the Desert
dialect.

\glossex{Muchi ka͞ufit on tasek.}{muchi ka͞ufi-t on tasek}{kitten jump-PST on.LOC table}{``The kitten jumped onto the table.''}

\randro{ne͞a} and \randro{ner} are describing closeness. The distinction is that
\randro{ner} is very specific about the physical closeness, while \randro{ne͞a}
can be also used figuratively.

\glossex{Chido ingek a il vapal ner ostro.}{chido ingek a il vapal ner ostro}{child wait.PST on 3SG.POSS father near.physically door}{``The child waited at the door for her father.''}

\glossex{Ner aḱumaryi sina, chyi larima vidi taryo a sormo.}{ner aḱumar-yi sina, chyi larima vidi taryo a sormo}{near.physically river-POSS exit 3.INAN.POSS road turn sharp.ADV on east}{``Near the mouth of the river, its course turns sharply towards the East.''}

\glossex{Seysi chet ne͞a myi do.}{seysi chet ne͞a myi do}{sit here near 1SG.ACC IMP}{``Sit here by me.''}

\glossex{Yelonayi ati bove aśtot ne͞a meńi.}{yelona-yi a-ti bove aśto-t ne͞a meńi}{queue-POSS one-ORD boy stop-PST near entrance}{``The first boy in the line stopped at the entrance.''}

\subsection{Ablative}

The two ablative (\Abl{}) markers, \randro{get}, and \randro{a͞u}, ``from'', are
used in two separate contexts -- \randro{get} is

is used mostly in derivational
morphology, however you may use it in some contexts regarding changing of
physical location of an object, and simiarly to other case markers, may be used
for emphasis.

% a͞u jako ablative

\glossex{Situvi muche get tasek do!}{situvi muche get tasek do}{take cat ABL table IMP}{``Take the cat \textbf{from} the table!''}

% kone, ayge - abessive case
% a͞u - "about"

\section{Copula and topic marker}
\label{sec:copula}

Copula \tandro{esi}{to be} is a typical form of a declarative or interrogative
sentence when talking about a state.

\glossex{Ari esi wofo, koy vipetode yi muche.}{ari esi wofo koy vipetode yi muche}{sky COP color like bowl POSS cat}{``The sky has a color just like a cat's bowl.'}

\glossex{Ezimrukyi ardo͞em munerojila, Alyona yi Zaálta, esi aka kankuseros.}{Ezimruk-yi ardo-em muner-o-jila Alyona yi Zaálta esi a-ka kankuser-os}{Ezimruk-POSS high-SUPL office-ADJ-power council POSS nation COP one-two minister-PL}{``Ezimruk's highest authority, the National Council, is consisted of twelve ministers.''}

\glossex{So netila chu 2+2 esi?}{so netila chu 2+2 esi}{what result GEN two+two COP}{``What is the result of 2+2?''}

However, in typical \andro manner, copula may be omitted. This is especially
popular in dialects.

\glossex{Ari arso, koy vipetode yi muche.}{ari arso koy vipetode yi muche}{sky skyblue.ADJ like bowl POSS cat}{``The sky is blue, just like a cat's bowl.''}

\glossex{Hetay ari stobo.}{hetay ari stobo}{today sky gray}{``The sky is gray today.''}

In some dialects, the topic marker \Top{} \randro{ya} is used to mark what is
the topic of the sentence, and what is the state of the topic.

\glossex{Ati taúnin yi yeitto ya zosu intrise!}{ati taunin yi yeitto ya zosu intrise}{one-ORD part POSS story TOP very interesting}{``As for the first part of the story -- it is very interesting!'' \\ ``The first part of the story is very interesting!''}

The topic marker is very similar in usage to a conjunction, and also changing
the subordinate clause word order to SOV.

\glossex{Mi ya egla pazi.}{mi ya egla pazi}{1SG TOP 3SG like}{``As for me -- I like her.''}

Some scientists are saying there is no topic maker, because the usage of
\randro{ya} particle is exactly as the usage of the conjunction -- so it is just
a conjunction meaning ``as for...'' or ``regarding...'', commonly used with zero
copula.

\glossex{Ari ya stobo.}{ari ya stobo ø}{sky TOP gray COP}{``The sky is gray.''}

In \ardo, the topic marker is usually used only for emphasis.

\section{Pronouns}
\label{sec:pronouns}

Pronouns, especially \randro{mi}, may be omitted.

\glossex{Mi pazi muchi.}{mi pazi much-i}{1SG like cat-PL}{``I like cats.''}

\glossex{Pazi muchi.}{ø pazi much-i}{1SG like cat-PL}{``I like cats.''}

However, the same may be used in case of \randro{ti}, a second person pronoun:

\glossex{Pazi muchi.}{ø pazi much-i}{2SG like cat-PL}{``You like cats.''}

The ambiguity is solveable only by understanding the context of the sentence.

\subsection{Reflexive pronoun}

In case when describing a transitional verb action attempted by the subject on
the subject, a reflexive pronoun (\Refl{}) can be used. In this case,
\randro{chyi}, which is normally used as a interrogative pronoun ``whose?'' is
used as a reflexive.

\glossex{Egi rek chyi da tasek.}{egi rek-ø chyi da tasek}{3SG hit-PST REFL INS table}{``He hit the table.'' \\ ``He hit himself with a table.''}

\section{Phrase structures}
\label{sec:phrases}

\subsection{Adjective phrases}
\label{sec:adjectives}

The adjective is always \textbf{head-final}, before the noun, so the position of
the adjective directly references which noun it is defining.

\glossex{Mi pazi karié himji.}{mi pazi karié him-ji}{1SG like beautiful woman-PL}{``I like beautiful women.''}

\glossex{Mi pazi karié himji e chid kahokapataji.}{mi pazi karié him-ji e chid kahokapata-ji.}{1SG like beautiful woman-PL and fast aeroplane-PL}{``I like beautiful women and fast aeroplanes.''}

\glossex{Ruó muche machagi.}{ruó muche machagi}{black.ADJ cat sleep}{``The black cat is sleeping.''}

\glossex{Ruó muche e fasso pelir machagi.}{ruó muche e fasso pelir machagi}{black.ADJ cat and.CONJ brown dog sleep}{``The black cat and the brown dog are sleeping.''}

\glossex{Waril vayarji esi lipe aloser che karié sipalima.}{Waril vayar-ji esi lipe aloser che karié sipalima}{long.ADJ travel-PL COP good.ADJ source GEN beautiful story}{``Long travels are good source for a beautiful story.''}

There is no preferrence in order of adjectives, when there is more than one
describing a noun, you may use any order.

\glossex{Gruwe ruó pelir.}{gruwe ruó pelir}{big black dog}{``A big, black dog.''}

\glossex{Ruó gruwe pelir.}{ruó gruwe pelir}{black big dog}{``A big, black dog.''}

As mentioned in the \nameref{sec:morph-adjectives}, you can create an adjective
out of a verb, by changing the ending \xo{-i} into \xo{-o}.

\glossex{Muche alturi.}{muche alturi}{cat make.noise}{``A cat is making noise.''}

\glossex{Alturo muche.}{altur-o muche}{make.noise-ADJ cat}{``A noise-making cat.''}

\glossex{Altur faro muche.}{altur far-o muche}{noise make-ADJ cat}{``A noise-making cat''}

However, instead of this, an universal \randro{chu} particle or relative pronoun
(\Rel{}) may be used:

\glossex{Pakopo rujalar kant hetay ne͞a ji͞ari͞o in Lublin.}{pakop-o rujalar kan-t hetay ne͞a ji͞ari͞o in Lublin}{worry-ADJ man sing-PST today near.physically garden in.LOC Lublin}{``A worried man sang today near the garden in Lublin.''}

\glossex{Rujalar chu vi pakot kant hetay ne͞a ji͞ari͞o in Jechuf.}{rujalar chu vi pako-t kan-t hetay ne͞a ji͞ari͞o in Jechuf}{man REL IPFV worry-PST sing-PST today near.physically garden in Rzeszów}{``A man who was worrying sang today near the garden in Rzeszów.''}

\glossex{Muche chu altur vi fart.}{muche chu altur vi far-t}{cat REL noise IPFV make-PST}{``A cat who was making a noise.''}

\subsection{Adverbial phrases}
\label{sec:adverbs}

As mentioned in the \nameref{sec:morph-adjectives}, adjectives may describe not
only a noun, but also a verb, performing a role of an adverb, answering the
question ``how?''. In the case of adverbial phrases, adjectives are
\textbf{head-initial}, staying head-final in case of adjective phrases in the
same sentence.

\glossex{Egla yon dió.}{egla yon dió}{3SG.F yon-ø sudden.ADV}{``She looked back suddenly.''}

\glossex{Vireji friti paluk fo eveni vayaren.}{vire-ji friti paluk fo eveni vayaren}{star-PL shine identical.ADV DAT every traveler}{``The stars are shining equally for every traveller.''}

\glossex{Nontriso lanji fesgai gepo.}{no-ntris-o lan-ji fesgai waril}{NEG-interesting-ADJ book-PL read.PRS long.ADV}{``It takes a long time to read uninteresting books for me.''}

Adjectives are also used when answering to the \tandro{ko͞e}{how?} question:

\glossex{Ko͞e ti seiti? Lipe.}{ko͞e ti seiti lipe}{how.Q 2SG feel good.ADV}{``How do you feel? Good.''}

\note{There is no direct counterpart to ``how do you do?'' or ``how are you?''
    questions used as a greeting. \randro{Ko͞e ti seiti?} is not used in this way.}

\note{\randro{Ko͞e ti seiti?} in some regions is formed as \randro{Ko͞e ti avi?}
    -- ``How are you having yourself?''.}

\subsection{Regulars}

When describing regular activities, adverbial phrases are used.

\glossex{Mi fesgai relita.}{mi fesgai relita.ADV}{1SG read.PRS always}{``I always read.''}

\glossex{Mi koóli ti relita.}{mi koóli ti relita}{1SG love.PRS 2SG always}{``I love you forever.''}

Sometimes also the whole adjective phrase may be used as an adverbial phrase,
and in such a case it is after the verb it is describing just a regular
adverbial phrase.

\glossex{Mi fesgai recha tay.}{mi fesgai recha tay}{1SG read.PRS all.ADJ day}{``I read every day.''}

\section{Tense, aspect and voice markings}
\label{sec:markers}

\subsection{Tenses}

\subsection{Present tense}

The basic tense, described by the most basic form of the verb, is used to
describe what is actual at the time of saying or what is rather stale and
unchanging. It may be used for statements describing the reality.

Typically, the present tense is automatically marking the imperfective aspect
and the imperfective aspect marker is omitted. This of course may be ambigous.

\glossex{Ti fesgai.}{ti fesgai}{2SG read.PRS}{``You are reading.'' \\ ``You read.''}

\glossex{Che esi karié.}{che esi karié}{DEM.INAN COP beautiful}{``This thing is beautiful.''}

\glossex{Aḱame osupi.}{aḱame osupi}{rain fall.PRS.IPFV}{``It is raining.''}

\note{in case of the rain, it must be noted, that \randro{osupi} is mostly used
    for the rain, so even the form \randro{Osupi.} is a correct sentence
    describing that is is raining at the moment.}

\subsection{Future tense}

The future tense (\Fut{}) is marked by the head-final particle \randro{ze}. By
default, future tense is also marking the imperfective aspect. \andro does not
differentiate between close and far future, everything is marked in the same
way.

\glossex{Ti ze fesgai.}{ti ze fesgai}{2SG FUT read}{``You will read.'' \\ ``You will be reading.''}

\glossex{Hetay mi ze va vibi rome.}{hetay mi ze va vibi rome}{today 1SG FUT PFV eat meat}{``I will eat meat today.'' \\ ``I will be eating meat today.''}

% zeva

\subsection{Past tense}

The past tense (\Pst{}) is marked by the conjugated form of the verb. This is a
very different from a future tense, where verbs are not being conjugated.

\note{In \randro{eygepa andro} there were a future tense conjugation for verbs,
    usually ending with \xo{-p}.}

By default, the past tense is also marking the aspect as perfective, if no
aspect marked is used.

\subsection{Aspects}

Tenses and aspects are very interwined, so there is hard to show all examples in
one place. \andro is differentiating between two aspects: perfective and
imperfective.

The imperfective aspect (\Ipfv{}) is marked by the particle \randro{vi}. The
marker is ``default'' for the present and future tenses so it does not have to
be marked, but as usual it could for additional emphasis.

\glossex{Ti fesgai.}{ti fesgai}{2SG read.PRS}{``You are reading.''}

\glossex{Ti vi fesgai.}{ti vi fesgai}{2SG IPFV read.PRS}{``You are reading and still haven't finished.''}

\glossex{Ti vi fesgat.}{ti vi fesga-t}{2SG IPFV read-PST}{``You started to read in the past and still haven't finished.''}

\glossex{Mi vi koólet egi.}{mi vi koóle-t egi}{1SG IPFV love-PST 3SG}{``I started to love her and I still love her.''}

To mark the perfective (\Pfv{}) aspect, head-final \randro{va} particle is used.
Perfective aspect is default for the past tense, so it's usage in the past tense
is only for additional emphasis. In the present tense, it is usually not used,
however in some dialects there is an usage of perfective marker for a present
tense to emphase that the activity is being finished at the moment.

Perfective tense marker may be used in the future tense to emphase that the
activity will be started and finished in the future.

\glossex{Ti va fesgat.}{ti va fesga-t}{2SG PFV read-PST}{``You have read.''}

\glossex{Va koólet egla.}{va koóle-t egla}{PFV love-PST 3SG.F}{``I loved her.''}

\glossex{Egi palimit a mi.}{egi palimi-t a mi}{3SG talk-PST with.someone 1SG}{``He was talking to me.''}

\glossex{Ti ze va fesgai.}{ti ze va fesgai}{2SG FUT PFV read}{``You will read and finish reading.''}

\subsection{Passive voice}

To mark passive voice (\Pass{}), \randro{Ge} particle is marking the verb.

\glossex{Sotak ge karlet chu polno sotak.}{sotak ge karle-t chu polno sotak}{somebody PASS kill-PST ACC different somebody}{``Someone was killed by someone else.''}

\subsection{Imperative, Optative and Hortative constructs}

For marking an order, the imperative (\Imp{}) marker \randro{do} is used.
\randro{do}, as well as other imperative constructs has an unique property,
because it is always \textbf{at the end of the sentence}.

\glossex{Tori dej́itos do!}{tori dej́it-os do}{throw weapon-PL IMP}{``Throw your weapons!''}

However, giving a direct order is a very rare occurence, because it is seemed as
very impolite. Usually, in most cases, \randro{do} can be directly exchanged
with a particle \randro{hemi}, which will be used as a exhortative (\Exh{})
marker. \randro{hemi} and \randro{do} however are marking the speaker and the
listener in the different positions, which you may see in
\nameref{tab:imperative}.

\note{\randro{hemi} is also a verb, meaning ``to ask for something''. Similarly,
    \randro{ih́emi} may be used as a particle, which is even stronger, meaning ``to
    beg for something''.}

\glossex{Mudi mi fayse, hemi.}{mudi mi fayse hemi}{give 1SG drink please.HORT}{``Give me a drink, please.'' \\ ``Could you pass me a drink?''}

\glossex{Karla͞i no mi, ih́emi.}{karla͞i no mi ih́emi}{kill NEG 1SG beg.HORT}{``Do not kill me, I beg you.''}

\glossex{Inra͞i ti egi karlet no, ih́emi.}{inra͞i ti egi karlet no ih́emi}{tell 2SG 3SG kill-PST NEG beg.HORT}{``I beg you, tell you have not killed him.''}

Cohortative (\Chr{}) marker \randro{heme} is used if you would like to show that
the action should be performed by both speaker and the listener.

\note{In informal speech, \randro{heme} is sometimes elonged to a form of
    \randro{hemee}, \xm{xɛ.mɛ:}.}

\glossex{Zetay ze edihi heme!}{zetay ze edihi heme}{tomorrow FUT meet CHR}{``Let's meet tomorrow!''}

\glossex{Eni nafiye faysi hemee!}{eni nafiye faysi hemee}{go beer drink CHR}{``Let's go for a beer!'' \\ ``Let's go for beer drinking!''}

The optative (\Opt{}) and hortative (\Hort{}) markers \randro{vige} and
\randro{vage} may be used. Optative construct is a pretty formal construction
for saying ``it shall be done''.

\glossex{Vige hallo͞i nome yi Ori!}{Vige hallo͞i nome yi ori}{OPT praise name POSS god}{``The God's name shall be praised!''}

Optative is very formal, and usually used only in religious, formal contexts.
\randro{vige} is head-final and is usually not expecting the ``actual'' result.

\randro{vage}, on the other hand, is a bit more like typical hortative
construct, expecting the ``actual'' result, so one of the ways to say an order.
It is used in the same style as \randro{vige}, head-final.

\glossex{Vage fesgai.}{vage fesgai}{HORT read.PRS}{``Somebody should read it.'' \\ ``You should read it.''}

\glossex{Ti vage fesgai.}{ti vage fesgai}{2SG HORT read.PRS}{``You should read it.'' \\ ``Read it, please.''}

\glossex{Ti vage fesgai ja lana.}{ti vage fesgai ja lana}{2SG HORT read.PRS DEM book}{``You should read this book.''}

\glossex{Ti vage va fesgat ja lana.}{ti vage va fesga-t ja lana}{2SG HORT PFV read-PST DEM book}{``You should have read this book.''}

\note{\randro{vage} is not a direct order, but rather a kind of polite
    equivalent of order, where the speaker is showing their higher position to the
    listener. Especially used in school or at work when teacher or boss is ordering
    to do something to a student or subordinate.}

\begin{table}[]
    \caption{Different imperative constructs}
    \label{tab:imperative}
    \begin{tabular}{lll}
        \textbf{Formal} & \textbf{Informal} & \textbf{Power}      \\
        ih́emi           & ih́emi             & Beg (S<L)           \\
        hemi            & hemi              & Please (S=L or S<L) \\
        vage            & te                & Light order (S>L)   \\
        epi / do        & do                & Order (S>>L)        \\
    \end{tabular}
    \footnotetext{Marked relative ``positions'' of participants of the conversation: S -- speaker, L -- listener}
\end{table}

There are two another imperative constructs which may be useful. \randro{do} has
its informal equivalent, \randro{te}.

\glossex{Ti stopi mi sochi te!}{ti stopi mi sochi te}{2SG stop.PRS 1SG joke.PRS IMP.NFRM}{``Stop joking at me!''}

In very rare, formal contexts, verb \randro{epi}, ``to can'', may be used as a
formal imperative marker, equivalent in power to \randro{do}.

\glossex{Ti uneńi epi.}{ti uneńi epi}{2SG leave can.IMP.FRM}{``You may leave.'' \\ ``You are dismissed.''}

\note{This form is used currently in military or on the royal courts, in it's
    usage the speaker positions themself far higher than listener.}

\section{Demonstratives}
\label{sec:demonstratives}

\randro{je} and \randro{ja} are used when referring to any kind of animated
being. However, when referring to humans, usage of third-person pronoun is
required.

% heje

\glossex{Ja egla esi chase.}{ja egla esi chase}{DEM.F 3SG.F COP short}{``The woman over here is short.'' / ``This woman is short.''}

\glossex{Ja epiá amimer.}{ja egla amimer}{DEM.F 3SG.F.FRM famous}{``The lady over here is famous.'' / ``This lady is famous.''}

Second type of demonstrative pronouns \randro{je} and \randro{ja} is for
emphasis, and in this case they are mostly used along with a person's name or
surname, usually when the person described is not in the same place.

\glossex{Je Kasstor! Mi pazi no il nosacho dué!}{je kasstor mi pazi no il no-sacho due}{DEM.EMPH Kasstor 1SG like NEG 3SG.POSS NEG-true smile}{``That Kasstor! I don't like his false smile!''}


\subsection{Partykuła \emph{ja}}

Nie są stosowane rodzajniki określone lub nieokreślone, ale istnieje możliwość
wskazania na konkretny obiekt za pomocą partykuły \emph{ja}.

\glossex{Ja muche esi ruko}{Ja muche esi ruk-o}{DEM cat COP black-ADJ}{,,Ten kot jest czarny'' \\ ,,Ten konkretny kot jest czarny''}

Rzeczowniki nie ulegają odmianie przez przypadki, do oznaczania których używa
się głównie partykuł, patrz sekcję poświęconą szykowi zdania.

\section{Conditionals}
\label{sec:conditionals}

Czysty tryb przypuszczający (warunkowy, \textsc{cond}) realizowany jest za
pomocą partykuł \emph{miam} oraz \emph{vimi}.

\glossex{Miam mi va fesgat sepo͞e vimi ti diyu fari.}{miam mi va fesga-t sepo͞e vimi ti diyu fari}{if.COND 1SG PFV read-PST early.ADV then.COND 2SG something do}{,,Gdybym wcześniej przeczytał to ty byś coś zrobił.''}

\glossex{Miam mi ze va fesgai vimi ti che fari.}{miam mi ze va fesgai vimi ti fari}{if.COND 1SG FUT PFV read then.COND 2SG DEM do}{,,Gdy przeczytam, to zrób to.''}

Można zastosować partykułę \emph{abe}, aby określić warunek przeciwny:

\glossex{Miam egli ze avi deíto vimi ti lugiti abe reki! }{miam egli ze avi deíto vimi ti lugiti abe reki}{if.COND 3SG.M FUT have weapon then.COND 2SG escape but hit}{,,Jeśli on będzie miał broń, uciekaj, a w przeciwnym razie -- atakuj.''}

Partykuła \emph{vimi} może być pomijana, o ile zastosuje się szyk zdania podrzędnego:

\glossex{Miam ze va fesgai, mi jeoza chu ti ze dari.}{Miam ze va fesgai, mi jeoza chu ti ze dari}{if.COND FUT PFV read 1SG candy ACC 2SG FUT give}{,,Jeśli to przeczytasz, to dam ci cukierka.'' \\ ,,Gdy to przeczytasz, to dam ci cukierka.''}

W odniesieniu do czynności zakończonych będzie to miało znaczenie ,,skoro-to'',
jednak alternatywnie w tej roli można również stosować partykułę \emph{imin}.

\glossex{Miam ze fesgat, jeoza chu ti.}{Miam ze fesga-t, jeoza chu ti}{if.COND PFV read-PST candy ACC 2SG}{,,Skoro przeczytałeś, cukierek dla ciebie''}

\glossex{Imin ze fesgat, jeoza baljezi.}{imin ze fesga-t, jeoza baljezi}{because PFV read-PST candy receive.PRS}{,,Skoro przeczytałeś, dostajesz teraz cukierka.''}

Sama partykuła \emph{vimi} służy, w szyku pytającym, do zadawania pytań ,,czy
nie powinniśmy czegoś zrobić?''.

\glossex{Enitya vimi fesgai alive ?}{enitya vimi fesgai alive}{instruction COND read before.ADV}{,,Nie powinieneś wcześniej przeczytać instrukcji?''}

\section{Conjuctions, subordinal clauses and relative pronouns}
\label{sec:conjunctions}

The basic word order for a subordinate clause is SOV (Subject-Object-Verb).
Subordinate clauses are clauses which are attached to the base clause with one
of the particles:

\begin{itemize}
    \item \randro{voli} -- when, until,
    \item \randro{imin} -- because,
    \item \randro{abe} -- but,
    \item \randro{abejar} -- however,
    \item \randro{pama} -- at that time,
    \item \randro{per} -- in order to.
\end{itemize}

Similarly, the subordinate clause can also be attached with the conjunction
particle:

\begin{itemize}
    \item \randro{e} -- and,
    \item \randro{yen} -- or,
    \item \randro{leyfe} -- exclusive or,
    \item \randro{zor} -- exclusive or.
\end{itemize}

\subsection{Conjunctions}

% abe, abegam, abejar, anko, azo, chane͞o, feban, imin, kihile, leyfe/zor, oban, pama, per, soki, sokisito, vajar, miam..vimi, yen



\subsection{Relative pronouns}
When building more complex sentences with subsentences, there may be a need to
talk about the objects in the parent sentence. To refer to that, relative
(\Rel{}) pronouns may be used. In \andro, most demonstrative pronouns may be
used as relative pronouns, apart from \randro{je} and \randro{ja}, in case of
which third person personal pronouns would be preferred (or sometimes omitted at
all, see syntax examples).

However, there is one generic relative pronoun which may be used for inanimate
objects: \randro{cheí} -- \Rel{}.\Inan{}. Almost a homonym of \randro{chei}
(\Tpl{}.\Inan{}), differentiated only by accent, \randro{cheí} is used only in a
relative context.

\section{Comparatives}
\label{sec:comparatives}

% dabeyo, dachu, azo, anis, alive, ji, key, o/a͞u, razi, tar, zosu, zosuvi

\section{Diminutives, emphasis, familiarity and formality}
\label{sec:diminutives}

\subsection{Emphasis and vocativity}

% de, e͞a, ey, live, owa, raratako, rizu͞o, seladi, vay, yeé

\subsection{Honorification, names and surnames}

% mal

W zależności od regionu możliwe jest, że użytkownicy języka będą bardzo
wyczuleni na kwestie grzecznościowe. Typowym tego typu elementem jest niechęć do
stosowania operatora trybu rozkazującego \emph{do} na rzecz \emph{hemi}, ale
bardzo często możesz również spotkać się z~honoryfikatorami służącymi do
odpowiedniego zwracania się do innych osób.

W And́royas typowym jest używanie pierwszego imienia w~odniesieniu do rozmówcy,
lub podczas określania osoby, chyba, że jest to niemożliwe do jednoznacznej
identyfikacji, wtedy używa się pełnego imienia i~nazwiska. Z drugiej jednak
strony mieszkańcy Cesarstwa są bardzo dumni ze swojej rodziny i~swojego rodowego
nazwiska, stąd przedstawiajac się często to podkreślą przedstawiając się pełnym
imieniem oraz nazwiskiem.

\glossex{Mi nomi Eryus mal Edoraril.}{mi nomi Eryus mal Edoraril}{1SG name.PRS Eryus of.family Edoraril}{,,Nazywam się Eryus mal Edoraril.''}

Warto tutaj zwrócić uwagę, że imiona i~nazwiska w~Cesarstwie są rozdzielane
partykułą \emph{mal}, oznaczającą ,,z rodziny'', np. \emph{Koolder mal
    Erlehirni} to Koolder z rodziny Erlehirni. Czasami możliwe jest, że dzieci
dziedziczą nazwiska po obojgu rodziców, stąd występują nazwiska łączone
łącznikiem, takie jak \emph{Alya mal Arkai-Valor}. Istnieją również także
oznaczane za pomocą łącznika gałęzie rodów, które dziś stały się zwykłymi
nazwiskami, np. \emph{Nimu͞e mal Hetasi-Hi}, co oznacza ród Hetasi i jego gałąź
Hi.

\note{Gałęzie rodowe, i co za tym idzie, ich oznaczenia w nazwiskach pojawiały
    się w~sytuacji, kiedy nazwisko rodowe przechodziło tylko na pierwsze dziecko,
    natomiast kolejne dzieci uzyskiwały nazwiska z~określeniem gałęzi. Zwyczaj ten
    zanikł prawie całkowicie około VII wieku po Zjednoczeniu.}
\skipline

Używa się zaimków osobowych \emph{epié} i~\emph{epiá}, które odpowiadają mniej
więcej polskim ,,pan'' i~,,pani''. Używa się ich w~odniesieniu do obcych osób
albo osób stojących wyżej w~hierarchii, albo w~sytuacji, kiedy nie znamy imienia
osoby, do której chcemy się zwrócić. Bardzo często można napotkać ich stosowanie
w postaci przyrostków z~łącznikiem, w~stosunku do imienia, np. mówiąc o kimś
wyżej w~hierarchii możemy powiedzieć \emph{Koolder-epié}. W~podobny sposób
można określać czyjąś funkcję, np. \emph{Furu-falazera} -- dowódca Furu. Czasami
można napotkać formę \emph{falazera-epiá} -- pani dowódca. W~taki sposób można
używać słów takich jak \emph{falazer} (dowódca), \emph{kachister} (nauczyciel),
\emph{meneder} (lekarz) i~innych.

\note{\emph{epié} i~\emph{epiá} praktycznie nie są używane w Republice Nennek,
    gdzie preferowane jest zwracanie się imieniem, zaimkiem ogólnym \emph{egi} lub
    zależnymi od płci \emph{egli/egla} lub ewentualnie przyrostkiem \emph{-gam},
    jeżeli naprawdę chce się podkreślić swoją niższą pozycję wobec rozmówcy.}

\note{W momencie gdy dwoje rozmówców będzie traktować się nawzajem z identycznym
    poziomem grzeczności, będą się do siebie zwracać nawzajem ukazując swoją niższą
    pozycję, np. nawzajem tytułować siebie z przyrostkiem \emph{-epié}.}
\skipline

Z drugiej strony, możliwe jest, że rozmówca będzie traktować drugą osobę jako
osobę od niego niższą statusem, ukazując swoją wyższą pozycję. Jest to
oczywiście niegrzeczne i stąd bardzo rzadko spotykane. Przyrostkami takimi mogą
być rzeczownik \emph{pezawe} (,,gorszy człowiek'') lub wręcz rzeczownik
\emph{zam} (dosłownie ,,śmieć''). Bardzo pogardliwe, spotkane w sytuacji i
próbach zastraszenia rozmówcy.

\subsection{Diminutives}

Oczywiście, w codziennych sytuacjach osoby sobie bliskie nie będą używały
określeń stricte formalnych -- do babci raczej wnukowie zwrócą się
\emph{chancha} niż \emph{gruchana}, jeśli są z nią blisko, a do swoich rodziców
\emph{mama} i \emph{patal} bardziej niż \emph{natali͞a} oraz \emph{vapal}.

W podobny sposób stosowane są często zdrobienia i partykuły lub rzeczowniki
z~nimi związane, takie jak \emph{myi}, który może być stosowany jako przyrostek
zdrabniający (\textsc{dim}), np. \emph{pelir-myi} -- ,,mój pieseczek'', czy też
\emph{koól}, stosowany np. \emph{Alya-koóla} -- ,,kochana Alya'', stosowany w
odniesieniu do osoby darzonej uczuciem.

Istnieje również słabsza wersja \emph{koól}, \emph{arey}, przyimek stosowany do
określenia sympatii do drugiej osoby. Może być również stosowany do określenia
sympatii do rzeczy, bez określenia jej posiadania, w przeciwieństwie do
\emph{myi}.

\subsection{Titles}

Jako monarchia, Cesarstwo posiadało szereg tytułów szlacheckich (np. \emph{lir},
czy \emph{eber}) jednakże od czasów początku Trzeciego Cesarstwa zostały
wycofane z użytku. Wciąż jednak, w ekstremalnie formalnym języku można stosować
te określenia jako przyrostek funkcyjny (na wzór np.~\emph{Koolder-epié} --
\emph{Koolder-lir}), jednak nie są stosowane w drugiej osobie (\textsc{2SG}), a
zamiast tego stosuje się \emph{arḱer/arḱera} dla podkreślenia formalności.

\note{\emph{arḱer/arḱera} nie obowiązuje przy zwracaniu się do członków
    rodziny cesarskiej oraz rodzin królów i książąt, do których należy zwracać
    się zaimkiem \emph{rige/rigea} (,,władca'' / ,,władczyni'')) w mniej
    formalnych sytuacjach i tytułem w bardziej formalnych.}

\note{W przypadku osoby pełniącej funkcję \emph{ajor} również stosuje się tytuł
    jako zaimek w drugiej osobie.}

Określenie \emph{arḱer} stosowane było również do zwracania się do osób
pełniących niektóre funkcje, np. burmistrza lub zarządcy miasta, często w formie
przyrostku funkcyjnego, obecnie jest to bardzo rzadkie.

Lista tytułów szlacheckich:

\begin{itemize}
    \item \emph{kyige/kyige͞a} -- król/królowa,
    \item \emph{kigeje/kigeje͞a} -- książę/księżna/księżniczka (tytuły Rodziny
          Cesarskiej)
    \item \emph{ajor/ajora} -- obecnie jest to funkcja administracyjna w
          Cesarstwie, oznaczająca ,,gubernatora'', osobę zarządzającą regionem
          administracyjnym,
    \item \emph{lir/lira} -- hrabia/hrabina,
    \item \emph{eber/ebera} -- baron/baronessa,
    \item \emph{arḱer/arḱera} -- ogólna forma ,,lord'' (,,lady'').
\end{itemize}

\subsection{Honorary phrases}

W odróżnieniu od kultur ziemskich, w And́royas nie przyjęła się koncepcja
zwrotów honorowych, np. ,,Jego Najjaśniejsza Wysokość Książę Lichtensteinu''
raczej byłby tytułowany po prostu \emph{Kigeje yi Lihtenchuteyn}, względnie jak
każdy inny tytuł jako przyrostek, np. \emph{Henrik-lihtenchuteynyikigeje}.

Dla podkreślenia ważności osoby o której się mówi, lub do której się zwraca, w
wysoce formalnych zwrotach stosuje się przyrostek \emph{-epié} w stosunku do
funkcji, np. \emph{diosever-epié} -- ,,pan sierżant'', ,,panie sierżancie''.

\note{W odniesieniu do głów państw raczej powinno używać się formy \emph{arḱer},
    np. w stosunku do królów, książąt, prezydentów czy Cesarzowej.}

Istnieje jeden zwrot honorowy stosowany do dzisiaj, \emph{ardo arḱeji} --
,,wysocy panowie'', stosowany do grupy wysoko postawionych osób.

\subsection{The Royal Family}

W przypadku rodziny cesarskiej używa się określeń \emph{eyger} oraz
\emph{eygera}, np. \emph{Fayfnira-eygera} -- cesarzowa Fayfnira, ale i~takich
jak \emph{and́royasyikigje͞a} (księżniczka And́royas -- tj. siostry
cesarza/cesarzowej), czy \emph{and́royasyikigeje} (książę And́royas -- bracia
panującego).

Z kolei dzieci panującego często określane są tytułami \emph{icheryikigeje},
\emph{icheryikigeje͞a} lub \emph{icheryihima} -- dosłownie książę lub
księżniczka krwi. Oprócz tego istnieje przyrostek \emph{-hima}, stosowany często
na Wschodzie w~stosunku do całej żeńskiej strony rodu panującego, poza
Cesarzową, który z kolei na Zachodzie często jest używany zamiennie z
\emph{-hina} jako "panna" dla kobiety niezamężnej.

Małżonek panującego może posiadać tytuł zarówno równorzędny -- np. \emph{eyger},
ale i~na przykład \emph{eygeryikigeje}, dosłownie ,,książę cesarstwa'' lub
,,książę cesarzowej''. Dokładne zasady są zależne od aktualnej sytuacji
politycznej.

Stąd aktualnie (w momencie pisania tej książki), mamy:

\begin{itemize}
    \item \textbf{Katia-eygera mal Arkai}\\ \xt{ˈka.ti.a ˈɛj.gɛ.ra ˈmal ˈar.ka.i},\\
          Cesarzowa Katia mal Arkai,
    \item \textbf{So'tak-eygeryikigeje mal Valor}\\ \xt{ˈsɔ|.tak ˈɛj.gɛ.rʏ.ki.gɛ.ʐɛ
              ˈmal ˈva.lɔr},\\ Ksiażę Cesarzowej, So'tak mal Valor,
    \item \textbf{Alya-icheryikigeje͞a mal Arkai-Valor}\\\xt{ˈal.ja i.ʈ͡ʂe.rʏ.ki.gɛ.ʐɛa ˈmal
              ˈar.ka.i-va.lɔr},\\ Księżniczka Krwi, Alya mal Arkai-Valor,
    \item \textbf{Niva-and́royasyikigeje͞a mal Arkai}\\\xt{ˈni.va an.ˈdrɔ.ja.sʏ.ki.gɛ.ʐɛa
              mal ˈar.ka.i},\\ Księżniczka And́royas, Niva mal Arkai,
    \item \textbf{Karra-and́royasyikigeje͞a mal Arkai}\\\xt{ˈkar.ra an.ˈdrɔ.ja.sʏ.ki.gɛ.ʐɛa
              mal ˈar.ka.i},\\ Księżniczka And́royas, Karra mal Arkai,
    \item \textbf{Jaida-and́royasyikigeje͞a mal Arkai}\\\xt{ˈʐa.i.da an.ˈdrɔ.ja.sʏ.ki.gɛ.ʐɛa
              mal ˈar.ka.i},\\ Księżniczka And́royas, Jaida mal Arkai.
\end{itemize}

\note{Należy tutaj zwrócić uwagę na wymowę imienia Jej Wysokości, w~której
    głoski /i/ oraz /a/ nie zlewają się w~/ia/, oraz na imię Jego Wysokości, w~
    którym występuje pauza pomiędzy sylabami, oznaczana przez <'> w~transkrypcji.
    Wynika to z~faktu, że Jego Wysokość pochodzi z~wyspy Rem, gdzie pojawiają
    się takie, unikatowe, elementy języka, z~uwagi na wpływ języków krajów
    ościennych.}

\section{Misc}

% layko
