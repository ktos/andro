\chapter{Background}
\label{ch:background}

\begin{flushright}\small
    \randro{\larger Ora͞iyihaye͞o ze ge uf́riti moĺi.}\\
    ``The light of the gods will never be extinguished.''
\end{flushright}\bigskip

\section{Language and its speakers}

Furu mal Klaji, a well-known linguist and one of my best teachers said one day
that \andro is absolutely hideous: it could be classified as an isolating and
analytical language, with markers for tense or mood, but there is a little of
inflection, for number or grammatical class, as well as there are polymorphemic
words and derivational morphology. Most of the particles are head-initial, while
some of them are head-final. Phonotactics is extremely complex and there are
nine writing systems.

\note{She later added that it's all because the language reflects the culture,
    and the culture reflects the language. In the case of our country -- the
    amalgamation of many regions, beliefs and concepts, all united by a single
    vision of moving humanity further. }\skipline

The \andro language is spoken by about 1.5 billion of speakers, living on the
Maŕid, the Earth-like planet in an extrasolar planetary system around a Sun-like
star designated by Earth's astronomers as HIP 109378. The Maŕid's sun, called
\randro{Seja} is about 69 light years from Earth.

\note{The people of Maŕid and Earth are the same species, yet there is no
    information available at the moment on how they appeared so far from
    Earth.}
\skipline

\andro is one of the official languages of the country And́royas, officially
known as \tandro{Eygeride yi And́royas}{Empire of And́royas}. And́royas is the
largest and most populated country on Maŕid, with about 1.1 billion citizens.

\section{Historical background}

The language \andro is sometimes also known as the And́royasan language
(\tandro{Inrat yi And́royas}{language of And́royas}), as the And́royas was a
birthplace of the language, however it is spoken also far beyond the borders of
And́royas, and was adapted as a official language in at least 5 other countries.

Its history is a bit complex -- it is derived from the Royal And́royasan
language (also known as Imperial or simply Royal Language) (\tandro{eygepa
    inrat}{royal language}). The Imperial And́royasan was partially artificial -- it
was artificially constructed language based on the pidgin used by traders
in~contacts between the Kingdom of Nennek, and~the countries of the West. It~was
proposed as a secondary official language after the establishment of the Second
Empire in the 286 year after the Unification. Combining features of Old Nennekan
language and several Kairenese, Keilian, and~Turanian languages it seemed as an
ideal candidate for the language for the newly united countries.

It was, however, quite widely criticized because of the existence of a very
strong connection with the~Nennekan language, which caused some difficulties,
especially for the speakers of other languages of the Second Empire. One of the
mostly known criticism of the \randro{eygepa andro} was inability to write it in
a syllabic Chizelic script. However, after the fall of the Second Empire in year
401 it was already widespread as the language of the upper classes and~became a
tool of universal communication, and~then it evolved like any natural language.

Modern \andro differs from the Imperial language in, among other things, the
loss of the form of the future tense in favor of the particle \randro{ze}, a
change in the syntax of particles \randro{do} and \randro{hemi}, the loss of the
vowel \xm{y}, the geminate consonants \xm{rr} and \xm{pp} and~a number of other
phonetic changes.

Because of political and cultural influence of the Empire, especially since 7th
century, it became widespread around the Maŕid, becoming the commonly known
foreign language. Since about 8th century, the Republic of Nennek is no longer
teaching the Nennekan language in~schools, having replaced it only
and~exclusively with \andro.

\section{Dialects}

In this book, the main emphasis is on the literary form of \andro, known
as \tandro{ardo andro}{high andro}, yet due to huge number of speakers, it is
not absolutely uniform.

At the moment, 6 main dialects are recognized, apart from \randro{ardo andro}.

\subsection{Nennekan}

The Nennekan dialect, has a lot less level of formality (see also
\nameref{sec:diminutives}) and extensive usage of the \randro{egi} pronoun,
which is a \Tsg{} pronoun independent of grammatical gender (class). In the
Nennekan dialect, typical formal pronouns \randro{epié} or \randro{epiá} are
not used commonly, as well as there is no usage of honorific suffix
\randro{-gam}.

Nennnekan dialect speakers are common in~Rem, Amesra and northern Vagyr.

\subsection{Western}
The Western dialect, sometimes divided into smaller three: Lidenian, Lononian,
Izolan, is preferring usage of possessive suffix \randro{-yi} instead of
\randro{yi} particle. Similarly, some other particles are used in the form of
the suffix, for example \randro{ze} particle, future tense marker, may be used
in that form, so instead of the regular word order:

\glossex{Ti ze vibi.}{ti ze vibi}{2SG FUT eat}{``You will eat.''}

the markers are used as suffixes:

\glossex{Ti vibize.}{ti vibi-ze}{2SG eat-FUT}{``You will eat.''}

There is also a lot of regional words.

The Western dialect is very popular in Lono, but also Rilla, Dorel and the whole
of Western Islands.

\subsection{Northern}

The Northern dialect, also known as Ellan dialect, is differentiated by
extensive usage of \xt{ʃ} phoneme instead of \xt{ʐ}.

\subsection{Desert}

The Desert dialect, also known as Dorelian, has an extensive own dictionary.
Instead of \xt{r} the \xt{ʁ} phoneme is used, and the aspiration appears --
\xt{g} and \xt{k} before vowels are pronounced as \xt{ɡʱ} and \xt{kʱ},
respectively. One of the interesting elements of Desert dialect is the
\randro{nodi} pronoun, the ``exclusive we''.

Very popular in~Agava and~Istapa.

\subsection{Southern}

The Southern (Nomisran) dialect is using long vowels (e.g. \xm{a:}) instead of
hiatus with accent change. Therefore, \randro{baán} is more \xt{ba:n} than
\xt{ba.ˈan}.

There is a lot of another sound changes: \xt{ɨ} used instead of \xt{ʏ}, \xt{ʒ}
instead of \xt{ʐ}, and sometimes \xt{h} instead of \xt{x}.

Southern dialect is common southern Vagyr.

\subsection{South-Eastern}

The South-Eastern dialect (Papitian) has an extensive regional dictionary.
Similarly to the Southern, \xt{ʐ} and \xt{x} are realized as \xt{ʒ} and \xt{h},
but also sometimes \xt{ʏ} is \xt{ji}.

Most commonly met in Maddo.

\subsection{Reman}

In the Reman dialect the most outstanding feature is the pause: \xt{|}, very
commonly used instead of hiatus. This particular feature is also heavily visible
in regional names, see \randro{So'tak}, \xt{sɔ|.tak}.

As the name suggest, very popular on the Greater Rem and Lesser Rem, as well as
Smaller Islands.
