\section{Language and its speakers}

\section{Historical background}

\section{Dialects}

Jak wspomniano, ten słownik skupia się na standardowej formie języka, jednak już
mogłeś zauważyć, że wielokrotnie wspominamy o dialektach. Obecnie wyróżnia się 6
głównych dialektów oprócz języka standardowego:

\begin{itemize}
    \item dialekt nennecki, wyróżnia się znacznie mniejszym poziomem
    formalności, w szczególności stosowaniem zaimka \emph{egi} zamiast
    formalnych \emph{epié/epiá}, a czasami nawet zamiast odmiennych przez
    rodzaje gramatyczne \emph{egli/egla}; w dialekcie tym raczej też nie stosuje
    się honoryfikatora \emph{-gam},
    \item dialekt zachodni (lideński, lonoński, izolański), w~którym preferowany
    jest przyrostek dzierżawczy \emph{-yi} zamiast partykuły \emph{yi}, w
    podobny sposób niektóre inne partykuły są używane jako przyrostki, np.
    partykuła czasu przyszłego \emph{ze} -- \emph{Ti ze vibi.} jest często
    zastępowane \emph{Ti vibize.}, występuje też dużo własnego regionalnego
    słownictwa,
    \item dialekt północny (ellański) wyróżniający się pojawianiem się fonemu
    \xt{ʃ} w miejsce \xt{ʐ},
    \item dialekt pustynny (dorelski), z szeregiem własnego słownictwa, użyciem
    \xt{ʁ} zamiast \xt{r} oraz pojawianiem się aspiracji -- \xt{g} oraz \xt{k}
    przed samogłoskami są wymawiane jako \xt{ɡʱ} oraz \xt{kʱ}, odpowiednio;
    jedną z ciekawostek jest również \emph{nodi}, ekskluzywne "my",
    \item dialekt południowy (nomisrański), stosujący długie samogłoski (np.
    /a:/) zamiast rozziewu ze zmianą akcentu (stąd \emph{baán} to raczej
    \xt{ba:n} niż \xt{ba.ˈan}) oraz \xt{ɨ} zamiast \xt{ʏ}, \xt{ʒ} zamiast \xt{ʐ}
    i czasem \xt{h} zamiast \xt{x},
    \item dialekt południowo-wschodni (papityjski), z szeregiem własnego
    słownictwa, oraz wymową \xt{ʐ} i \xt{x} jako \xt{ʒ} oraz \xt{h}, czasami
    także \xt{ʏ} jako \xt{ji},
	\item dialekt remański, z występowaniem pauzy \xt{|} w niektórych słowach,
	w~szczególności spotykane jest to w imionach -- patrz \emph{So'tak}
	\xt{sɔ|.tak}.
\end{itemize}

Dialekt nennecki najpopularniejszy jest w regonie Nennek, ale i pojawia się
w~Rem, Amesrze lub północnym Vagyr. Dialekt zachodni najpopularniejszy jest
w~Lono, ale i Rilli, Dorel oraz w~regionie Wysp Zachodnich. Dialekt pustynny
przede wszystkim można spotkać w~Agavie i~Istapie. Dialekt południowy to domena
południowego Vagyr, natomiast południowo-wschodni najczęściej występuje w Maddo.