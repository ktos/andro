\newpage

\section{Preface}

\begin{spacing}{1.1}

    I am very glad you are reading this book! My name is \emph{Koolder mal
        Erlehirni} and this book is a new project we started over two years ago. At that
    moment it is mostly just a dictionary, but it will be accompanied with second
    book, describing the rules and grammar of \emph{andro} language, known as
    ,,Andro Language Reference Guide''.

    Usage of this book does require some knowledge in linguistics. We will start
    with description of conventions used in the dictionary.

    In this book only the transcription (romanization) is used, so you don't have to
    know anything about our alphabets or syllabaries.

    \bigskip

    Good luck!

    \section[Abbreviations and format]{Abbreviations and formatting}

    Every word in this dictionary is presented to you in a standarized form, for
    example:

    \dictword{isdar}[ˈis.dar]
    \dictterm{n}{half}

    The bold text is a text in a transcription. In this dictionary, the
    transcription Ziri is used. The next one is a phonemic representation using the
    symbols from the International Phonetic Alphabet, IPA. As you may notice, this
    dictionary is not using the full phonetic/allophonic descriptions, mostly
    because the Andro language is not that strict and you may hear a bit different
    phonemes in different dialects.

    The next thing written in \emph{italics} is the abbreviation of what part of the
    speech is represented by the dictionary term:

    \begin{table}[h]
        \begin{tabular}{ll}
            \emph{n}    & noun      \\
            \emph{v}    & verb      \\
            \emph{adj}  & adjective \\
            \emph{pro}  & pronoun   \\
            \emph{part} & particle
        \end{tabular}
    \end{table}

    Depending on the dictionary term you will see additional markings:

    \dictword{archit}[ˈar.ʈ͡ʂit]
    \dictterm{n}{(\textsc{pl} architji \xm{ˈar.ʈ͡ʂit.ʐi}) (\textsc{fem} archita \xm{ˈar.ʈ͡ʂi.ta}) ancestor}
    \skipline

    Markings with \textsc{small caps} are used to present different grammatical
    features. \Pl{} (plural) is showing you what is the plural form of the
    noun, as nouns in Andro are changing with grammatical number.

    Marking \Fem{} shows you what is the form in the feminine grammatical
    gender, as nouns in Andro have assigned grammatical gender. You will also notice
    some nouns marked with (\Fem{}), which are only existing in feminine
    grammatical gender, for example:

    \dictwordb{hu͞ekapa}[ˈxuɛ.ka.pa]
    \dictterm{n}{(\textsc{fem}) computer}
    \skipline

    Both markings, for plural and feminine form, are optional.

    In case of verbs, you will see the marking \Pst{}, showing the past form
    of the verb, as verbs conjugate against tenses.

    \dictwordb{kuanti}[ˈku.an.ti]
    \dictterm{v}{(\textsc{pst} kuant [ˈku.aŋt]) to hunt}
    \skipline

    In case of adjectives, up to two markings may be presented: \Comp{} and
    \Supl{}, presenting comparative and superlative form, respectively.

    \dictwordb{wa͞ime}[ˈwai.mɛ]
    \dictterm{adj}{(\textsc{comp} wa͞ime͞a [ˈwai.mɛa], \textsc{supl} wa͞ime͞am [ˈwai.mɛam]) wide}
    \skipline

    The rest of used abbreviations, e.g. \Dem{}, \Inan{}, \Rel{}, \Refl{}, \Tsg{},
    \Ssg{}, \Tpl{}, \Fpl{}, \Fsg{}, \Poss{}, \Ins{}, \Emph{}, \Imp{}, \Acc{},
    \Gen{}, \Dat{}, \Abl{}, \Loc{}, \Voc{}, \Nfrm{}, \Imp{}, \Cond{}, \Top{} are
    described in the following section.

    \note{Finally, you may encounter text noted like this. These are usually additional notes about cultural aspects.}
    \skipline

    \printglosses[style=list]

\end{spacing}