\newpage

\section{Przedmowa}

Cieszę się, że czytasz ten słownik! Nazywam się Koolder mal Erlehirni i ta książka to owoc wytężonej pracy w okresie prawie dwóch ostatnich lat. Przygotowałem ten dokument w taki sposób, aby oprócz bycia tylko słownikiem zawierał również szereg uwag dotyczących samego języka androyasańskiego, jak również różnorodne notatki dotyczące kultury naszego kraju, które mogą być przydatne w zrozumieniu tekstów, które będziesz tłumaczyć.

Część z tych materiałów wymaga pewnego przygotowania lingwistycznego, jednak nie jest to absolutnie wymagane.

Słownik zaczyna się od wyjaśnienia części pojęć i oznaczeń, które będziesz spotykać dalej w treści, oraz z krótkiego opisu historii języka, wprowadzenia do jego ortografii, fonetyki i gramatyki.

Aby uniknąć dodatkowych problemów, nigdzie w tekście nie zostanie przedstawione pismo, którym się posługują ci, dla których ten język jest językiem ojczystym – zamiast tego wszędzie stosowana będzie transkrypcja Ziri. W miarę obycia z transkrypcją zdasz sobie sprawę, że dokładnie odzwierciedla nasze pismo, więc w dalszej kolejności będziesz w stanie przejść do bardziej zaawansowanych materiałów dotyczących języka androyasańskiego.

Powodzenia!

\section[Skróty i oznaczenia]{Skróty i oznaczenia używane w~słowniku}

Każde słowo, które napotkasz w słowniku, jest zapisane dokładnie w tym samym formacie.

Tekst pogrubiony oznacza słowo w transkrypcji. Następnie zapisana jest wymowa za pomocą standardowego Międzynarodowego Alfabetu Fonetycznego, IPA. Kolejny będzie skrót, który oznacza z jaką częścią mowy masz do czynienia:

\begin{table}[h]
\begin{tabular}{ll}
\emph{n}    & rzeczownik           \\
\emph{v}    & czasownik            \\
\emph{adj}  & przymiotnik          \\
\emph{pro}  & przyimek albo zaimek \\
\emph{part} & partykuła           
\end{tabular}
\end{table}

W zależności od części mowy napotkasz kolejne oznaczenia. W przypadku rzeczownika możesz napotkać na przykład takie hasło:

\dictwordb{archit}[ˈar.t͡ʂit]
\dictterm{n}{(\textsc{pl} architji) (\textsc{fem} archita) przodek}
\skipline

Oznaczenie \textsc{pl} określa tutaj jaka jest forma mnoga danego rzeczownika. Oznaczenie \textsc{fem} określa jaka jest forma w rodzaju żeńskim. Możesz napotkać również słowa pozbawione formy mnogiej oraz takie, które mają oznaczenie (\textsc{fem}) i występują tylko w rodzaju żeńskim, na przykład:

\dictwordb{hu͞ekapa}[ˈxuɛ.ka.pa]
\dictterm{n}{(\textsc{fem}) komputer}
\skipline

W przypadku czasownika, możesz zobaczyć oznaczenie \textsc{pst}, które pokazuje formę czasu przeszłego dla czasownika.

\dictwordb{kuanti}[ˈku.an.ti]
\dictterm{v}{(\textsc{pst} kuant) polować}
\skipline

Dla przymiotników przedstawiane są formy \textsc{comp} oraz \textsc{supl}, które oznaczają odpowiednio stopień wyższy i najwyższy. Oczywiście nie wszystkie przymiotniki będą posiadały takie formy, stąd te oznaczenia nie będą występować zawsze.

\dictwordb{wa͞ime}[ˈwai.mɛ]
\dictterm{adj}{(\textsc{comp} wa͞ime͞a, \textsc{supl} wa͞ime͞am) szeroki}
\skipline

Partykuły i przyimki nie posiadają zazwyczaj dodatkowych skrót\-owców w~swoich opisach.

\note{Napotkasz również tekst zapisany w taki sposób. Są to dodatkowe uwagi na temat poprzedzającego słowa, definiujące jego źródłosłów lub kontekst kulturowy. Zwracaj uwagę na takie opisy, gdyż oprócz ciekawych informacji możesz napotkać sposób użycia w mowie formalnej lub potocznej.}
\skipline