\newpage

\section{A}
\begin{multicols}{2}

\dictword{a}[ˈa]
\dictterm{part}{na czymś (fizycznie -- coś leży na czymś)}
\note{Zazwyczaj używane tylko w dialekcie pustynnym. Patrz \emph{on}. Wykorzystywane
w związkach frazeologicznych.}

\dictword{a}[ˈa]
\dictterm{n}{jeden, jedynka}

\dictword{abara}[ˈa.ba.ra]
\dictterm{n}{(\textsc{pl} abarji [ˈa.bar.ʐi]) (\textsc{fem}) góra, szczyt}

\dictword{abe}[ˈa.bɛ]
\dictterm{part}{ale}

\dictword{achuke}[ˈa.t͡ʂu.kɛ]
\dictterm{n}{(\textsc{pl} achukos [ˈa.t͡ʂu.kɔs]) (\textsc{fem} achuka [ˈa.t͡ʂu.ka]) mityczna istota, "posłaniec", luźny odpowiednik anioła}

\dictword{ad́aki}[a.ˈda.ki]
\dictterm{v}{(\textsc{pst} ad́ak [a.ˈdak]) zamykać}

\dictword{ad́ako}[a.ˈda.kɔ]
\dictterm{adj}{zamknięty}

\dictword{a͞edar}[ˈaɛ.dar]
\dictterm{n}{półtora}

\dictword{agea}[ˈa.gɛ.a]
\dictterm{n}{(\textsc{pl} ageji) (\textsc{fem}) sałata}

\dictword{ager}[ˈa.gɛr]
\dictterm{n}{(\textsc{pl} agejos [a.gɛ.ʐɔs]) teren, terytorium, kraj}

\dictword{agerer}[ˈa.gɛ.rɛr]
\dictterm{n}{(\textsc{pl} agereros [ˈa.gɛ.rɛ.rɔs]) (\textsc{fem} agerera [ˈa.gɛ.rɛ.ra]) obywatel}

\dictword{agun}[ˈa.gun]
\dictterm{n}{(\textsc{pl} agunos [ˈa.gu.nɔs]) wybrzeże}

\dictword{aguna}[ˈa.gu.na]
\dictterm{n}{(\textsc{pl} agunaji [ˈa.gu.na.ʐi]) (\textsc{fem}) plaża}

\dictword{ah́emi}[a.ˈxɛ.mi]
\dictterm{v}{(\textsc{pst} ah́em [ˈa.xɛm]) przepraszać}

\dictword{akaro}[ˈa.ka.rɔ]
\dictterm{adj}{czerwony}

\dictword{akaro}[ˈa.ka.rɔ]
\dictterm{n}{czerwień}

\dictword{aḱajami}[a.ˈka.ʐa.mi]
\dictterm{v}{(\textsc{pst} aḱajamat [a.ˈka.ʐa.mat]) przebaczać}

\dictword{aḱu͞a}[a.ˈkua]
\dictterm{n}{(\textsc{pl} aḱu͞aji [a.ˈkua.ʐi]) (\textsc{fem}) woda}

\dictword{aḱu͞ame}[a.ˈkua.mɛ]
\dictterm{n}{(\textsc{pl} aḱu͞ameji [a.ˈkua.mɛ.ʐi]) (\textsc{fem}) deszcz}

\dictword{aḱumar}[a.ˈku.mar]
\dictterm{n}{(\textsc{pl} aḱumaros [a.ˈku.ma.rɔs]) rzeka}

\dictword{alha}[ˈal.xa]
\dictterm{n}{(\textsc{pl} alhaji [ˈal.xa.ʐi]) (\textsc{fem}) wartość}

\dictword{ali}[ˈa.li]
\dictterm{part}{ile}

\dictword{alima}[ˈa.li.ma]
\dictterm{n}{(\textsc{pl} alimji [ˈa.lim.ʐi]) (\textsc{fem}) pierś}

\dictword{aliona}[ˈa.li.ɔ.na]
\dictterm{n}{(\textsc{pl} alionji [ˈa.li.ɔn.ʐi]) (\textsc{fem}) rada}

\dictword{aloser}[ˈa.lɔ.sɛr]
\dictterm{n}{(\textsc{pl} aloseros [ˈa.lɔ.sɛ.rɔs]) źródło}

\dictword{alt́arevi}[al.ˈta.rɛ.vi]
\dictterm{n}{(\textsc{pl} alt́arevis [al.ˈta.rɛ.vis]) alternatywa, opcja}

\dictword{altur}[ˈal.tur]
\dictterm{n}{(\textsc{pl} alturos [ˈal.tu.rɔs]) hałas}

\dictword{alturi}[ˈal.tu.ri]
\dictterm{v}{(\textsc{pst} altur [ˈal.tur]) hałasować}

\dictword{alye}[ˈal.jɛ]
\dictterm{n}{(\textsc{pl} alyes [ˈal.jɛs]) (\textsc{fem} alya [ˈal.ja]) przyjaciel, przyjaciółka}

\dictword{aḿachager}[a.ˈma.t͡ʂa.gɛr]
\dictterm{n}{(\textsc{pl} aḿachageros [a.ˈma.t͡ʂa.gɛ.rɔs]) (\textsc{fem} aḿachagera [a.ˈma.t͡ʂa.gɛ.ra]) kochanek, kochanka}
\note{Używane tylko w~kontekście seksualnym. W~miłości romantycznej patrz \emph{koóler}.}

\dictword{aḿachagi}[a.ˈma.t͡ʂa.gi]
\dictterm{v}{(\textsc{pst} aḿachaget [a.ˈma.t͡ʂa.gɛt]) przesypiać, przespać; \emph{aḿachagi a} -- przespać się z~kimś; \emph{aḿachagi chu} -- przespać coś (np. jakieś wydarzenie)}

\dictword{aḿaridi}[a.ˈma.ri.di]
\dictterm{v}{(\textsc{pst} aḿarit [a.ˈma.rit]) przemieszczać}

\dictword{amde}[ˈam.dɛ]
\dictterm{adj}{(\textsc{comp} amde͞a [ˈam.dɛa], \textsc{supl} amdem [ˈam.dɛm]) prosty, łatwy}
\note{Może również być wykorzystane jako rzeczownik oznaczający prostą drogę.}

\dictword{aḿiko}[a.ˈmi.kɔ]
\dictterm{n}{przeziębienie}

\dictword{aḿiko}[a.ˈmi.kɔ]
\dictterm{adj}{przeziębiony, przechłodzony}

\dictword{amimer}[ˈa.mi.mɛr]
\dictterm{adj}{sławny}

\dictword{an}[ˈan]
\dictterm{part}{nad, ponad (czymś)}

\dictword{ańakta}[a.ˈnak.ta]
\dictterm{n}{(\textsc{pl} ańaktoji [a.ˈnak.tɔ.ʐi]) (\textsc{fem}) sytuacja}

\dictword{And́royas}[an.ˈdrɔ.jas]
\dictterm{n}{Nazwa własna określająca kraj: Cesarstwo And́royas, \emph{Eygeride yi And́royas}.}

\dictword{anis}[ˈa.nis]
\dictterm{part}{prawie}

\dictword{anko}[ˈan.kɔ]
\dictterm{part}{jeszcze}

\dictword{anper}[ˈan.pɛr]
\dictterm{adj}{(\textsc{comp} anpere [ˈan.pɛ.rɛ], \textsc{supl} anperem [ˈan.pɛ.rɛm]) mokry}

\dictword{anpi}[ˈan.pi]
\dictterm{v}{(\textsc{pst} anpit [ˈan.pit]) pluć}

\dictword{antoi}[ˈan.tɔ.i]
\dictterm{v}{(\textsc{pst} antot [ˈan.tɔt]) pachnieć}

\dictword{antrom}[ˈan.trɔm]
\dictterm{n}{(\textsc{pl} antromos [ˈan.trɔ.mɔs]) jaskinia}

\dictword{anwer}[ˈan.wɛr]
\dictterm{n}{(\textsc{pl} anweros [ˈan.wɛ.rɔs]) (\textsc{fem} anwera [ˈan.wɛ.ra]) wybraniec}

\dictword{anwi}[ˈan.wi]
\dictterm{v}{(\textsc{pst} anwot [ˈan.wɔt]) wybierać}

\dictword{anvi}[ˈan.vi]
\dictterm{v}{(\textsc{pst} anvot [ˈan.vɔt]) woleć, preferować}

\dictword{apim}[ˈa.pim]
\dictterm{n}{(\textsc{pl} apimis [ˈa.pi.mis]) (\textsc{fem} apima [ˈa.pi.ma]) jednostka miary, jednostka w~wojsku, pojedyńcza rzecz}

\dictword{archit}[ˈar.t͡ʂit]
\dictterm{n}{(\textsc{pl} architji [ˈar.t͡ʂit.ʐi]) (\textsc{fem} archita [ˈar.t͡ʂi.ta]) przodek}

\dictword{ared}[ˈa.rɛd]
\dictterm{n}{(\textsc{fem}) góra (kierunek)}

\dictword{ardo}[ˈar.dɔ]
\dictterm{adj}{(\textsc{comp} ardo͞e [ˈar.dɔɛ], \textsc{supl} ardo͞em [ˈar.dɔɛm]) wysoki, wysoko, górny}

\dictword{ari}[ˈa.ri]
\dictterm{n}{niebo}

\dictword{aribant}[ˈa.ri.baŋt]
\dictterm{n}{(\textsc{pl} aribanji [ˈa.ri.ban.ʐi]) tęcza}

\dictword{ariunhika}[ˈa.ri.un.xi.ka]
\dictterm{n}{(\textsc{pl} ariunhikiji [ˈa.ri.un.xi.ki.ʐi]) asteroida}

\dictword{arḱami}[ar.ˈka.mi]
\dictterm{v}{(\textsc{pst} arḱam [ar.ˈkam]) rozkazywać}

\dictword{arḱamo}[ar.ˈka.mɔ]
\dictterm{adj}{rozkazujący}

\dictword{armed}[ˈar.mɛd]
\dictterm{n}{(\textsc{pl} armedos [ˈar.mɛ.dɔs]) kamień}

\dictword{arśit}[ar.ˈsit]
\dictterm{n}{(\textsc{pl} arśiji [ar.ˈsi.ʐi]) (\textsc{fem} arśita [ar.ˈsi.ta]) ojciec, rodzic, matka}
\note{Jako ,,matka'' i~,,ojciec'' używane najczęściej w~odniesieniu nie do biologicznego rodzica, gdzie używa się określeń \emph{vapal} oraz \emph{natali͞a}. Używane między innymi w~odniesieniu do rodziców adoptowanych lub postaci religijnych, np. bogów.}

\dictword{arso}[ˈar.sɔ]
\dictterm{adj}{błękitny, jasnoniebieski}

\dictword{arso}[ˈar.sɔ]
\dictterm{n}{błękit, kolor słonecznego nieba}

\dictword{aruj́ar}[a.ru.ˈʐar]
\dictterm{n}{(\textsc{pl} aruj́aros [a.ru.ˈja.rɔs]) (\textsc{fem} aruj́ara [a.ru.ˈja.ra]) szlachcic}

\dictword{arýer}[ar.ˈjɛr]
\dictterm{n}{piach, piasek}

\dictword{aśeysi}[a.ˈsɛj.si]
\dictterm{v}{(\textsc{pst} aśet [a.ˈsɛt]) przesiadać}

\dictword{aśtopi}[a.ˈstɔ.pi]
\dictterm{v}{(\textsc{pst} aśtot [a.ˈstɔt]) przestawać, zatrzymywać}

\dictword{aspen}[ˈas.pɛn]
\dictterm{n}{(\textsc{pl} aspenji [as.pɛn.ʐi]) moździerz}

\dictword{ati}[ˈa.ti]
\dictterm{adj}{pierwszy, pierwsze}

\dictword{a͞u}[ˈau]
\dictterm{part}{,,z'' lub ,,o''}

\dictword{avek}[ˈa.vɛk]
\dictterm{adj}{przestarzały}

\dictword{av́eyrali}[a.ˈvɛj.ra.li]
\dictterm{v}{(\textsc{pst} av́eyralit [a.ˈvɛj.ra.lit]) przegapić}
\note{Używane w~kontekście, że podmiot tak skupiał się na czymś, że umknął
mu pewien ważny fakt.}

\dictword{avi}[ˈa.vi]
\dictterm{v}{(\textsc{pst} afit [ˈa.fit]) mieć}

\dictword{ayro}[ˈaj.rɔ]
\dictterm{adj}{zielony}

\dictword{ayro}[ˈaj.rɔ]
\dictterm{n}{zieleń}

\dictword{aysi}[ˈaj.si]
\dictterm{v}{(\textsc{pst} aysit [ˈaj.sit]) oznaczyć, naznaczyć}

\dictword{aysi}[ˈaj.si]
\dictterm{n}{(\textsc{pl} aysiji [ˈaj.si.ʐi]) znamię, znak}

\dictword{azo}[ˈa.zɔ]
\dictterm{part}{także}

\end{multicols}

\newpage
\section{B}
\begin{multicols}{2}

\dictword{baaji}[ˈba.a.ʐi]
\dictterm{v}{(\textsc{pst} baajet [ˈba.a.ʐɛt]) beczeć (owca)}
\note{Słowo to stanowi jeden z~nielicznych wyjątków, kiedy przy rozziewie <aa> akcent jest nadal na pierwszą sylabę.}

\dictword{baán}[ba.ˈan]
\dictterm{n}{(\textsc{pl} baániji [ba.ˈa.ni.ʐi]) czas; \emph{vek baán} -- stare czasy, dawne dzieje}

\dictword{bajeko}[ˈba.ʐɛ.kɔ]
\dictterm{adj}{(\textsc{comp} bajeke͞a [ˈba.ʐɛ.kɛa], \textsc{supl} bajeke͞am [ˈba.ʐɛ.kɛam]) inny, obcy}

\dictword{bajeker}[ˈba.ʐɛ.kɛr]
\dictterm{n}{(\textsc{pl} bajekeros [ˈba.ʐɛ.kɛ.rɔs]) obcy ludzie, inni ludzie}

\dictword{bakizo}[ˈba.ki.zɔ]
\dictterm{adj}{następny, kolejny}

\dictword{baljeji}[ˈbal.ʐɛ.ʐi]
\dictterm{v}{(\textsc{pst} baljejet [ˈbal.ʐɛ.jɛt]) otrzymać, otrzymywać, dostawać}

\dictword{ban}[ˈban]
\dictterm{n}{(\textsc{pl} bani [ˈba.ni]) odpowiednik sekundy, ok. 0,45 sekundy}
\note{Używane również jako krótki odcinek czasu: ,,chwila'', ,,moment'', ,,chwilka''.}

\dictword{bar}[ˈbar]
\dictterm{n}{(\textsc{pl} baros [ˈba.rɔs]) odległość ramion ludzkich -- starodawna jednostka miar}

\dictword{bara}[ˈba.ra]
\dictterm{n}{(\textsc{pl} baraji [ˈba.ra.ʐi]) (\textsc{fem}) róża}

\dictword{bargul}[ˈbar.gul]
\dictterm{n}{(\textsc{pl} barguji [ˈbar.gu.ʐi]) rodzaj potrawy z ciasta wypełnionego nadzieniem; pierogi}
\note{Patrz też: \emph{hitler}. Potrawy te różnią się pochodzeniem i możliwymi farszami - w przypadku \emph{bargul} częściej stosowane były farsze słodkie niż słone. Obecnie zatarła się jednak różnica pomiędzy tymi określeniami.}

\dictword{bayor}[ˈba.jɔr]
\dictterm{n}{(\textsc{pl} bayoros [ˈba.jɔ.rɔs]) robak}

\dictword{beísta}[bɛ.ˈi.sta]
\dictterm{n}{(\textsc{pl} beístos [bɛ.ˈi.stɔs]) (\textsc{fem}) bestia}

\dictword{bejet́i}[bɛ.ʐɛ.ˈti]
\dictterm{v}{(\textsc{pst} bejet [ˈbɛ.ʐɛt]) trwać}

\dictword{beúsma}[bɛ.ˈus.ma]
\dictterm{n}{(\textsc{pl} beúsmiji) (\textsc{fem}) legenda}

\dictword{beúsmo}[bɛ.ˈus.mɔ]
\dictterm{adj}{legendarny}

\dictword{bev́an}[bɛ.ˈvan]
\dictterm{n}{(\textsc{pl} bev́anos [bɛ.ˈva.nɔs]) nasienie}

\dictword{bevetsi}[ˈbɛ.vɛt.si]
\dictterm{v}{(\textsc{pst} bevetsit [ˈbɛ.vɛt.sit]) motywować}

\dictword{bevetsito}[ˈbɛ.vɛt.si.tɔ]
\dictterm{adj}{zmoty\-wowany}

\dictword{bevetso}[ˈbɛ.vɛt.sɔ]
\dictterm{adj}{motywujący}

\dictword{beykar}[ˈbɛj.kar]
\dictterm{n}{(\textsc{pl} beykaros [ˈbɛj.ka.rɔs]) wąż, gad}
\note{W tym samym konktekście można spotkać również słowo \emph{hebo}.}

\dictword{bezena}[ˈbɛ.zɛ.na]
\dictterm{n}{(\textsc{pl} bezenji [ˈbɛ.zɛn.ʐi]) (\textsc{fem}) pewność}

\dictword{bezeni}[ˈbɛ.zɛ.ni]
\dictterm{v}{(\textsc{pst} bezent [ˈbɛ.zɛŋt]) zapewniać}

\dictword{beleti}[ˈbɛ.lɛ.ti]
\dictterm{v}{(\textsc{pst} blet [ˈblɛt]) gadać, plotkować}

\dictword{biri}[ˈbi.ri]
\dictterm{v}{(\textsc{pst} birit [ˈbi.rit]) moczyć}

\dictword{bloy}[ˈblɔj]
\dictterm{n}{krew}

\dictword{bo͞ar}[ˈbɔ.ar]
\dictterm{n}{(\textsc{pl} boars [ˈbɔ.a.rɔs]) ręka}

\dictword{bochada}[ˈbɔ.t͡ʂa.da]
\dictterm{n}{(\textsc{fem}) nadzieja}

\dictword{bodarga}[ˈbɔ.dar.ga]
\dictterm{n}{(\textsc{pl} bodargiji [ˈbɔ.dar.gi.ʐi]) (\textsc{fem}) żołądek}

\dictword{boluka}[ˈbɔ.lu.ka]
\dictterm{n}{(\textsc{pl} bolukaji [ˈbɔ.lu.ka.ʐi]) (\textsc{fem}) oddział (wojskowy), pułk,
obecnie odpowiednik pułku, tj. około 1000 żołnierzy}

\dictword{boso}[ˈbɔ.sɔ]
\dictterm{adj}{nagi, rozebrany}

\dictword{bove}[ˈbɔ.vɛ]
\dictterm{n}{(\textsc{pl} bovos [ˈbɔ.vɔs]) chłopiec}

\dictword{bowa}[ˈbɔ.wa]
\dictterm{n}{(\textsc{pl} bowaji [ˈbɔ.wa.ʐi]) (\textsc{fem}) jajo, jajko}

\dictword{brasi}[ˈbra.si]
\dictterm{v}{(\textsc{pst} brasit [ˈbra.sit]) kupić, kupować}

\dictword{brat}[ˈbrat]
\dictterm{n}{(\textsc{pl} bratos [ˈbra.tɔs]) brat}

\dictword{brateva}[ˈbra.tɛ.va]
\dictterm{n}{(\textsc{pl} bratevaji [ˈbra.tɛ.va.ʐi]) (\textsc{fem}) kompania (wojskowa), obecnie
zgrupowanie wojskowe około 500 ludzi}

\dictword{bucho}[ˈbu.t͡ʂɔ]
\dictterm{n}{wola}

\dictword{budar}[ˈbu.dar]
\dictterm{n}{okrucieństwo}

\dictword{budaro}[ˈbu.da.rɔ]
\dictterm{adj}{(\textsc{comp} budare͞a [ˈbu.da.rɛa], \textsc{supl} budare͞am [ˈbu.da.rɛam]) okrutny}

\dictword{bugi}[ˈbu.gi]
\dictterm{v}{(\textsc{pst} buget [ˈbu.gɛt]) leżeć, kłaść; również w stosunku do miejsc i budynków}
\note{W formie czasu przeszłego w~starych tekstach można również spotkać \emph{bukt} oraz \emph{bugt}, czasami używane w~nieformalnym języku.}

\dictword{buruku}[ˈbu.ru.ku]
\dictterm{n}{(\textsc{pl} burukos [ˈbu.ru.kɔs]) biuro}

\dictword{buut}[ˈbu.ut]
\dictterm{n}{(\textsc{pl} buutji [ˈbu.ut.ʐi]) głowa}

\end{multicols}
\newpage
\section{CH}
\begin{multicols}{2}

\dictword{cha}[ˈt͡ʂa]
\dictterm{n}{sześć, szóstka}

\dictword{chabara}[ˈt͡ʂa.ba.ra]
\dictterm{n}{(\textsc{pl} chabaros [ˈt͡ʂa.ba.rɔs]) (\textsc{fem}) szyja}

\dictword{chalebe}[ˈt͡ʂa.lɛ.bɛ]
\dictterm{n}{(\textsc{pl} chalebos [ˈt͡ʂa.lɛ.bɔs]) wieczór}

\dictword{chanche}[ˈt͡ʂan.t͡ʂɛ]
\dictterm{n}{(\textsc{pl} chanches [ˈt͡ʂan.t͡ʂɛs]) (\textsc{fem} chancha [ˈt͡ʂan.t͡ʂa]) dziadek, babcia}
\note{Najczęściej używane jako zdrobnienie w~języku potocznym i~nieformalnym.}

\dictword{chant}[ˈt͡ʂaŋt]
\dictterm{adj}{równy}

\dictword{chantui}[ˈt͡ʂan.tu.i]
\dictterm{v}{(\textsc{pst} chantut [ˈt͡ʂan.tut]) porównywać}

\dictword{char}[ˈt͡ʂar]
\dictterm{n}{(\textsc{pl} charji [ˈt͡ʂar.ʐi]) dar}
\note{\emph{oraichar} oznacza dosłownie dar bogów, błogosławieństwo, natomiast \emph{kayćhar} oznacza dar nocy, przekleństwo.}

\dictword{chase}[ˈt͡ʂa.sɛ]
\dictterm{adj}{(\textsc{comp} chase͞a [ˈt͡ʂa.sɛa], \textsc{supl} chase͞am [ˈt͡ʂa.sɛam]) niski}

\dictword{chati}[ˈt͡ʂa.ti]
\dictterm{adj}{szósty}

\dictword{chaweri}[ˈt͡ʂa.wɛ.ri]
\dictterm{v}{(\textsc{pst} chawerit [ˈt͡ʂa.wɛ.rit]) dziać (się)}

\dictword{che}[ˈt͡ʂɛ]
\dictterm{part}{to}
\note{\emph{che} może być stosowany jako zaimek, ale wyłącznie do obiektów nieożywionych; \emph{che recha} -- to wszystko, tylko tyle, to i~już, nic więcej}

\dictword{cheí}[t͡ʂɛ.ˈi]
\dictterm{pro}{te rzeczy}

\dictword{chelika}[ˈt͡ʂɛ.li.ka]
\dictterm{n}{(\textsc{fem}) litość}

\dictword{cheliko}[ˈt͡ʂɛ.li.kɔ]
\dictterm{adj}{litościwy}

\dictword{chelikokone}[ˈt͡ʂɛ.li.kɔ.kɔ.nɛ]
\dictterm{adj}{bezlitostny}
\note{Jedno ze słów, które można stworzyć poprzez dodanie suffiksu -kone do przymiotnika.}

\dictword{cheri}[ˈt͡ʂɛ.ri]
\dictterm{v}{(\textsc{pst} cherit [ˈt͡ʂɛ.rit]) jeździć, jechać}

\dictword{chet}[ˈt͡ʂɛt]
\dictterm{part}{tutaj}

\dictword{chetna}[ˈt͡ʂɛt.na]
\dictterm{n}{(\textsc{pl} chetiji [ˈt͡ʂɛ.ti.ʐi]) (\textsc{fem}) czoło}

\dictword{chetni}[ˈt͡ʂɛt.ni]
\dictterm{v}{(\textsc{pst} chetnit [ˈt͡ʂɛt.nit]) stawić czoło, przeciwstawić się, pokonać przeszkodę, pokonywać przeszkody}
\note{Można również stosować \emph{varsi onvojibo} -- stanąć naprzeciw.}

\dictword{cheryot}[ˈt͡ʂɛr.jɔt]
\dictterm{n}{(\textsc{pl} cheryotos [ˈt͡ʂɛr.jɔ.tɔs]) pojazd}
\note{Dawniej: rydwan lub wóz konny, obecnie każdy pojazd.}

\dictword{chey}[ˈt͡ʂɛj]
\dictterm{pro}{(należące do) tych rzeczy}

\dictword{chicho}[ˈt͡ʂi.t͡ʂɔ]
\dictterm{n}{(\textsc{pl} chichos [ˈt͡ʂi.t͡ʂɔs]) poziomka}

\dictword{chid}[ˈt͡ʂid]
\dictterm{adj}{(\textsc{comp} chide [ˈt͡ʂi.dɛ], \textsc{supl} chidem [ˈt͡ʂi.dɛm]) szybki}

\dictword{chidi}[ˈt͡ʂi.di]
\dictterm{v}{(\textsc{pst} chidit [ˈt͡ʂi.dit]) spieszyć się}

\dictword{chidirbi}[ˈt͡ʂi.dir.bi]
\dictterm{v}{(\textsc{pst} chidirbit [ˈt͡ʂi.dir.bit]) wymiotować}

\dictword{chido}[ˈt͡ʂi.dɔ]
\dictterm{n}{(\textsc{pl} chidos [ˈt͡ʂi.dɔs]) dziecko}

\dictword{chifaro}[ˈt͡ʂi.fa.rɔ]
\dictterm{adj}{(\textsc{comp} chifare [ˈt͡ʂi.fa.rɛ], \textsc{supl} chifarem [ˈt͡ʂi.fa.rɛm]) młody}

\dictword{chikai}[ˈt͡ʂi.ka.i]
\dictterm{v}{(\textsc{pst} chikat [ˈt͡ʂi.kat]) śmiać (się)}

\dictword{chivebo}[ˈt͡ʂi.vɛ.bɔ]
\dictterm{adj}{biedny}

\dictword{chiwi}[ˈt͡ʂi.wi]
\dictterm{v}{(\textsc{pst} chiwit [ˈt͡ʂi.wit]) pisać}

\dictword{chiwo}[ˈt͡ʂi.dɔ]
\dictterm{n}{(\textsc{pl} chiwos [ˈt͡ʂi.wɔs]) pismo, zapiski}

\dictword{choester}[ˈt͡ʂɔ.ɛs.tɛr]
\dictterm{n}{(\textsc{pl} choesterji [ˈt͡ʂɔ.ɛs.tɛr.ʐi]) czołg}

\dictword{choesti}[ˈt͡ʂɔ.ɛs.ti]
\dictterm{v}{(\textsc{pst} choeset [ˈt͡ʂɔ.ɛs.ɛt]) czołgać (się)}

\dictword{choini}[ˈt͡ʂɔ.i.ni]
\dictterm{v}{(\textsc{pst} choint [ˈt͡ʂɔ.iŋt]) stwierdzać autorytatywnie, uważać (że), potwierdzać}

\dictword{choinji}[ˈt͡ʂɔ.in.ʐi]
\dictterm{n}{(\textsc{pl} choinjis [ˈt͡ʂɔ.in.ʐis]) potwierdzenie}

\dictword{choke}[ˈt͡ʂɔ.kɛ]
\dictterm{adj}{(\textsc{comp} choke͞a [ˈt͡ʂɔ.kɛa], \textsc{supl} choke­­͞am [ˈt͡ʂɔ.kɛam]) miły}

\dictword{chu}[ˈt͡ʂu]
\dictterm{part}{przez, który, z~powodu, dla}
\note{Używany również jako operator dopełnienia.}

\dictword{chu͞a}[ˈt͡ʂua]
\dictterm{part}{która}
\note{Przestarzały. Operator dopełnienia dla rodzaju żeńskiego.}

\dictword{chu͞i}[ˈt͡ʂui]
\dictterm{part}{który (liczebnikowo)}

\dictword{churche}[ˈt͡ʂur.t͡ʂɛ]
\dictterm{part}{przez, via}
\note{Używany w~toastach, np. \emph{churche ti} oznacza \emph{za ciebie}, a~
\emph{churche Eygeride} to \emph{za Cesarstwo}.}

\dictword{chutojer}[ˈt͡ʂu.tɔ.ʐɛr]
\dictterm{n}{(\textsc{pl} chutojerji [ˈt͡ʂu.tɔ.ʐɛr.ʐi]) system, układ}
\note{Na przykład \emph{Solyi chutojer} -- Układ Słoneczny.}

\dictword{chuwure}[ˈt͡ʂu.wu.rɛ]
\dictterm{n}{(\textsc{pl} chuwuros [ˈt͡ʂu.wu.rɔs]) autobus}

\dictword{chyi}[ˈt͡ʂʏ]
\dictterm{pro}{(należący do) tej rzeczy (\textsc{3SG.GEN} lub \textsc{3SG.POSS})}

\dictword{chyi}[ˈt͡ʂʏ]
\dictterm{part}{czyj, kogo, czego}
\note{Również używany jako zaimek zwrotny ,,się''.}

\end{multicols}
\newpage
\section{D}
\begin{multicols}{2}

\dictword{dabeyo}[ˈda.bɛ.jɔ]
\dictterm{part}{więcej}

\dictword{dachu}[ˈda.t͡ʂu]
\dictterm{part}{mniej}

\dictword{dant}[ˈdaŋt]
\dictterm{n}{(\textsc{pl} dantos [ˈdaŋt.ɔs]) penis}

\dictword{dari}[ˈda.ri]
\dictterm{v}{(\textsc{pst} daret [ˈda.rɛt]) dawać}

\dictword{daze}[ˈda.zɛ]
\dictterm{n}{(\textsc{pl} dazos [ˈda.zɔs]) usta}

\dictword{dazer}[ˈda.zɛr]
\dictterm{n}{(\textsc{pl} dazeros [ˈda.zɛ.rɔs]) pocałunek}

\dictword{dazi}[ˈda.zi]
\dictterm{v}{(\textsc{pst} dazit [ˈda.zit]) całować}

\dictword{dechu}[ˈdɛ.t͡ʂu]
\dictterm{part}{tylko}

\dictword{dehuzer}[ˈdɛ.xu.zɛr]
\dictterm{n}{(\textsc{pl} dehuzerji [ˈdɛ.xu.zɛr.ʐi]) (\textsc{fem} dehuzera [ˈdɛ.xu.zɛ.ra]) badacz}

\dictword{dehuzerida}[ˈdɛ.xu.zɛ.ri.da]
\dictterm{n}{(\textsc{pl} dehuzeridji [ˈdɛ.xu.zɛ.rid.ʐi]) (\textsc{fem}) laboratorium}

\dictword{dehuzi}[ˈdɛ.xu.zi]
\dictterm{v}{(\textsc{pst} dehuzet [ˈdɛ.xu.zɛt]) badać}

\dictword{deíto}[dɛ.ˈi.tɔ]
\dictterm{n}{(\textsc{pl} dej́itos [dɛ.ˈʐi.tɔs]) broń}

\dictword{deítokar}[dɛ.ˈi.tɔ.kar]
\dictterm{n}{obrona}

\dictword{deítokari}[dɛ.ˈi.tɔ.ka.ri]
\dictterm{v}{(\textsc{pst} deítokar [dɛ.ˈi.tɔ.kar]) bronić}

\dictword{dejesu}[ˈdɛ.jɛ.su]
\dictterm{n}{jelec}

\dictword{deki}[ˈdɛ.ki]
\dictterm{v}{(\textsc{pst} dekit [ˈdɛ.kit]) decydować}

\dictword{deku}[ˈdɛ.ku]
\dictterm{n}{(\textsc{pl} dekji [ˈdɛk.ʐi]) decyzja}

\dictword{demet}[ˈdɛ.mɛt]
\dictterm{n}{(\textsc{pl} demetos [ˈdɛ.mɛ.tɔs]) robak}

\dictword{desaho}[ˈdɛ.sa.xɔ]
\dictterm{adj}{specjalny}

\dictword{desvi}[ˈdɛs.vi]
\dictterm{v}{(\textsc{pst} desvit [ˈdɛs.vit]) rosnąć}

\dictword{di͞amant}[ˈdia.maŋt]
\dictterm{n}{(\textsc{pl} diamantijis [ˈdi.a.maŋt.i.ʐis]) diament}

\dictword{diía}[di.ˈi.a]
\dictterm{n}{(\textsc{pl} diíji [di.ˈi.ʐi]) (\textsc{fem}) chmura}

\dictword{diosa}[ˈdi.ɔ.sa]
\dictterm{n}{(\textsc{pl} dioji [ˈdi.ɔ.ʐi]) (\textsc{fem}) bitwa}

\dictword{dioseva}[ˈdi.ɔ.sɛ.va]
\dictterm{n}{(\textsc{pl} diosevaji [ˈdi.ɔ.sɛ.va.ʐi]) (\textsc{fem}) batalion}

\dictword{diras}[ˈdi.ras]
\dictterm{n}{seks, stosunek seksualny}

\dictword{diter}[ˈdi.tɛr]
\dictterm{part}{tam}

\dictword{divai}[ˈdi.va.i]
\dictterm{v}{(\textsc{pst} divat [ˈdi.vat]) operować, używać czegoś, kierować czymś}

\dictword{divayer}[ˈdi.va.jɛr]
\dictterm{n}{(\textsc{pl} divayeros [ˈdi.va.jɛ.rɔs]) (\textsc{fem} divayera [ˈdi.va.jɛ.ra]) operator, użytkownik, kierowca}

\dictword{diyu}[ˈdi.ju]
\dictterm{part}{coś}

\dictword{do}[ˈdɔ]
\dictterm{part}{partykuła -- operator trybu rozkazującego, patrz też \emph{hemi}}

\dictword{doevi}[ˈdɔ.ɛ.vi]
\dictterm{v}{(\textsc{pst} do͞evit [ˈdɔɛ.vit]) skrzypieć}

\dictword{doevo}[ˈdɔ.ɛ.vɔ]
\dictterm{adj}{skrzypiący}

\dictword{doki}[ˈdɔ.ki]
\dictterm{n}{(\textsc{pl} dokiji [ˈdɔ.ki.ʐi]) miecz}
\note{Dawniej: broń w~ogólności.}

\dictword{dokofa}[ˈdɔ.kɔ.fa]
\dictterm{n}{waga}

\dictword{dokofe}[ˈdɔ.kɔ.fɛ]
\dictterm{adj}{(\textsc{comp} dokofe͞a [ˈdɔ.kɔ.fɛa], \textsc{supl} dokofe͞am [ˈdɔ.kɔ.fɛam]) ciężki}

\dictword{doloze}[ˈdɔ.lɔ.zɛ]
\dictterm{n}{(\textsc{pl} dolozeji [ˈdɔ.lɔ.zɛ.ʐi]) wzrost}

\dictword{dolozi}[ˈdɔ.lɔ.zi]
\dictterm{v}{(\textsc{pst} dolozet [ˈdɔ.lɔ.zɛt]) wzrastać}

\dictword{domi}[ˈdɔ.mi]
\dictterm{v}{(\textsc{pst} domit [ˈdɔ.mit]) szeptać}

\dictword{doroos}[ˈdɔ.rɔ.ɔs]
\dictterm{n}{(\textsc{pl} doroosi [ˈdɔ.rɔ.ɔ.si]) hipopotam}

\dictword{douge}[ˈdɔ.u.gɛ]
\dictterm{n}{(\textsc{pl} doukos [ˈdɔ.u.kɔs]) dąb}

\dictword{dowo}[ˈdɔ.wɔ]
\dictterm{adj}{(\textsc{comp} dowo͞a [ˈdɔ.wɔa], \textsc{supl} dowoam [ˈdɔ.wɔam]) pełny }

\dictword{dozo}[ˈdɔ.zɔ]
\dictterm{adj}{(\textsc{comp} doze͞a [ˈdɔ.zɛa], \textsc{supl} doze͞am [ˈdɔ.zɛam]) brudny }

\dictword{drebi}[ˈdrɛ.bi]
\dictterm{v}{(\textsc{pst} dret [ˈdrɛt]) rozdrabniać, kruszyć}

\dictword{dyet}[ˈdjɛt]
\dictterm{part}{za}

\end{multicols}
\newpage
\section{E}
\begin{multicols}{2}

\dictword{e}[ˈɛ]
\dictterm{part}{,,i'' lub ,,oraz''}

\dictword{e͞a}[ˈɛa]
\dictterm{part}{formalny zwrot określający odbiorcę -- ,,proszę pana'' lub ,,proszę pani''}
\note{Używany również jako forma grzecznego zwrócenia uwagi -- ,,przepraszam pana''.}

\dictword{echo}[ˈɛ.t͡ʂɔ]
\dictterm{n}{(\textsc{pl} echos [ˈɛ.t͡ʂɔs]) echo}

\dictword{edihi}[ˈɛ.di.xi]
\dictterm{v}{(\textsc{pst} edihit [ˈɛ.di.xit]) spotykać}

\dictword{egi}[ˈɛ.gi]
\dictterm{pro}{on, ona (\textsc{3SG})}
\note{Zaimek egi jest niezależny od płci. Spotykany zazwyczaj tylko w~dialekcie Republiki Nennek.}

\dictword{ego͞i}[ˈɛ.gɔi]
\dictterm{pro}{oni, one (\textsc{3PL})}

\dictword{egyi}[ˈɛ.gʏ]
\dictterm{pro}{ich, ichnie (\textsc{3PL.GEN} lub \textsc{3SG.POSS})}

\dictword{e͞iger}[ˈɛi.gɛr]
\dictterm{n}{(\textsc{pl} e͞igeros [ˈɛi.gɛ.rɔs]) (\textsc{fem} e͞igera [ɛi.gɛ.ra]) cesarz}

\dictword{e͞igeride}[ˈɛi.gɛ.ri.dɛ]
\dictterm{n}{(\textsc{pl} e͞igerides [ˈɛi.gɛ.ri.dɛs]) cesarstwo, imperium}

\dictword{egli}[ˈɛ.gli]
\dictterm{pro}{on (\textsc{3SG})}

\dictword{egla}[ˈɛ.gla]
\dictterm{pro}{ona (\textsc{3SG-FEM})}

\dictword{ekenin}[ˈɛ.kɛ.nin]
\dictterm{n}{(\textsc{pl} ekenos [ˈɛ.kɛ.nɔs]) odkrycie (naukowe), wniosek}

\dictword{elĺa}[ɛl.ˈla]
\dictterm{part}{serio, doprawdy}
\note{Również używane z~niedowierzaniem i~strachem, w~odniesieniu do czynności której nie zamierza się wykonać.}

\dictword{eḿarudo}[ɛ.ˈma.ru.dɔ]
\dictterm{n}{(\textsc{pl} emarudojis [ˈɛ.ma.ru.dɔ.ʐis]) szmaragd, kolor szmaragdowy}

\dictword{eḿarudo}[ɛ.ˈma.ru.dɔ]
\dictterm{adj}{szmaragdowy}

\dictword{eni}[ˈɛ.ni]
\dictterm{v}{(\textsc{pst} ent [ˈɛŋt]) iść, chodzić}

\dictword{entepi}[ˈɛn.tɛ.pi]
\dictterm{v}{(\textsc{pst} entepit [ˈɛŋt.ɛ.pit]) ruszać}

\dictword{entepo}[ˈɛn.tɛ.pɔ]
\dictterm{adj}{ruchomy, ruszający się}

\dictword{epadot}[ˈɛ.pa.dɔt]
\dictterm{n}{(\textsc{pl} epadotos [ˈɛ.pa.dɔ.tɔs]) karabin}

\dictword{epi}[ˈɛ.pi]
\dictterm{v}{(\textsc{pst} epit [ˈɛ.pit]) móc, \emph{epi rori} -- wydawać się}

\dictword{epiá}[ɛ.pi.ˈa]
\dictterm{pro}{pani (ona)}

\dictword{epié}[ɛ.pi.ˈɛ]
\dictterm{pro}{pan (on)}
\note{Zaimki epié oraz epiá są bardzo formalne. Używa się ich tylko w~języku formalnym, w
odniesieniu do obcych osób albo osób stojących wyżej w~hierarchii, albo w~sytuacji, kiedy nie
znamy imienia osoby, do której chcemy się zwrócić. Mogą być używane zarówno jako 
3SG, jak i~2SG. W~niektórych dialektach stosowane również jako przyrostki,
w szczególności do funkcji np. \emph{falazer-epié} -- ,,pan dowódca''.}
\note{Uwaga: zaimków tych nie stosuje się w~powszechnej mowie w~niektórych regionach Republiki Nennek.}

\dictword{epil}[ˈɛ.pil]
\dictterm{pro}{pana, pani, pański}

\dictword{erchigu}[ˈɛr.t͡ʂi.gu]
\dictterm{n}{(\textsc{pl} erchigus [ˈɛr.t͡ʂi.gus]) (\textsc{fem} erchiga [ˈɛr.t͡ʂi.ga]) prezes, prezydent, (nieformalnie) szef}
\note{Prezydent kraju (związkowego) to \emph{agaryierchigu}.}

\dictword{erokwa}[ˈɛ.rɔk.wa]
\dictterm{n}{(\textsc{pl} erokwaji) (\textsc{fem}) kura, kurczak}

\dictword{erokwyirome}[ˈɛ.rɔk.wʏ.rɔ.mɛ]
\dictterm{n}{(\textsc{pl} erokwyiromes) drób, mięso z kurczaka}

\dictword{eshu}[ˈɛs.xu]
\dictterm{n}{(\textsc{pl} eshus [ˈɛs.xus]) drzwi}

\dictword{esi}[ˈɛ.si]
\dictterm{v}{(\textsc{pst} est [ˈɛ.sɛt]) być; \emph{esi gir on} -- być złym na coś, kogoś}

\dictword{espero}[ˈɛs.pɛ.rɔ]
\dictterm{n}{(\textsc{pl} esperos [ˈɛs.pɛ.rɔs]) miasto}

\dictword{eswai}[ˈɛs.wa.i]
\dictterm{v}{(\textsc{pst} eswat [ˈɛs.wat]) wrzeszczeć}

\dictword{eveni}[ˈɛ.vɛ.ni]
\dictterm{part}{każdy}

\dictword{ey}[ˈɛj]
\dictterm{part}{nieformalny zwrot określający odbiorcę -- ej, ty!}

\dictword{eygepa}[ˈɛj.gɛ.pa]
\dictterm{adj}{cesarski}

\end{multicols}
\newpage
\section{F}
\begin{multicols}{2}

\dictword{faája}[fa.ˈa.ʐa]
\dictterm{n}{(\textsc{pl} faájis [fa.ˈa.ʐis]) (\textsc{fem}) ładunek}

\dictword{fahojo}[ˈfa.xɔ.ʐɔ]
\dictterm{adj}{(\textsc{comp} fahoje [ˈfa.xɔ.ʐɛ], \textsc{supl} fahojem [ˈfa.xɔ.ʐɛm]) głośny}

\dictword{fajer}[ˈfa.ʐɛr]
\dictterm{n}{(\textsc{pl} faji [ˈfa.ʐi]) zwierzę}

\dictword{falarzo}[ˈfa.lar.zɔ]
\dictterm{n}{(\textsc{pl} falarzos [ˈfa.lar.zɔs]) dowództwo}

\dictword{falazer}[ˈfa.la.zɛr]
\dictterm{n}{(\textsc{pl} falazeros [ˈfa.la.zɛr.ɔs]) (\textsc{fem} falzera [ˈfal.zɛ.ra]) dowódca}

\dictword{falazeya}[ˈfa.la.zɛ.ja]
\dictterm{n}{(\textsc{pl} falazeyaji [ˈfa.la.zɛ.ja.ʐi]) (\textsc{fem}) pluton (wojskowy), obecnie
odpowiednik plutonu, tj. około 25 ludzi}

\dictword{falazi}[ˈfa.la.zi]
\dictterm{v}{(\textsc{pst} falazet [ˈfa.la.zɛt]) dowodzić}

\dictword{falazo}[ˈfa.la.zɔ]
\dictterm{adj}{dowodzący}

\dictword{faliko}[ˈfa.li.kɔ]
\dictterm{adj}{(\textsc{comp} falike͞a [ˈfa.li.kɛa], \textsc{supl} falike͞am [ˈfa.li.kɛam]) spokojny}

\dictword{famei}[ˈfa.mɛ.i]
\dictterm{v}{(\textsc{pst} famet [ˈfa.mɛt]) znaczyć (mieć znaczenie)}

\dictword{famet}[ˈfa.mɛt]
\dictterm{n}{znaczenie}

\dictword{farapak}[ˈfa.ra.pak]
\dictterm{n}{(\textsc{pl} farapakos) kapusta}

\dictword{fari}[ˈfa.ri]
\dictterm{v}{(\textsc{pst} faret [ˈfa.rɛt]) robić, pracować}

\dictword{faro}[ˈfa.rɔ]
\dictterm{adj}{działający, pracujący, operacyjny}

\dictword{farta}[ˈfar.ta]
\dictterm{n}{(\textsc{pl} farteji [ˈfar.tɛ.ʐi]) (\textsc{fem}) praca (pejoratywnie), robota}

\dictword{fatabey}[ˈfa.ta.bɛj]
\dictterm{n}{(\textsc{pl} fatabeyos [ˈfa.ta.bɛ.jɔs]) serce}

\dictword{fasso}[ˈfas.sɔ]
\dictterm{adj}{brązowy}

\dictword{fasso}[ˈfas.sɔ]
\dictterm{n}{kolor brązowy}

\dictword{fays}[ˈfaj.sɔ]
\dictterm{adj}{pijany}

\dictword{faysi}[ˈfaj.si]
\dictterm{v}{(\textsc{pst} faysit [ˈfaj.sit]) pić}

\dictword{feni}[ˈfɛ.ni]
\dictterm{v}{(\textsc{pst} fenit [ˈfɛ.nit]) pływać}

\dictword{fesgai}[ˈfɛs.ga.i]
\dictterm{v}{(\textsc{pst} fesgat [ˈfɛs.gat]) czytać}

\dictword{feśidara}[fɛ.ˈsi.da.ra]
\dictterm{n}{(\textsc{pl} feśidarji [fɛ.ˈsi.dar.ʐi]) (\textsc{fem}) obecność}

\dictword{feśidari}[fɛ.ˈsi.da.ri]
\dictterm{v}{(\textsc{pst} feśidaret [fɛ.ˈsi.da.rɛt]) być obecnym}

\dictword{fida}[ˈfi.da]
\dictterm{adj}{(\textsc{comp} fide͞a [ˈfi.dɛa], \textsc{supl} fide͞am [ˈfi.dɛam]) wierny}

\dictword{fideye}[ˈfi.dɛ.jɛ]
\dictterm{adj}{(\textsc{comp} fideye͞a [ˈfi.dɛ.jɛa], \textsc{supl} fideye͞am [ˈfi.dɛ.jɛam]) lekki}

\dictword{filacha}[ˈfi.la.t͡ʂa]
\dictterm{n}{(\textsc{pl} filachaji) (\textsc{fem}) orzech, orzeszek}

\dictword{filet}[ˈfi.lɛt]
\dictterm{n}{(\textsc{pl} filetos [ˈfi.lɛ.tɔs]) pole}

\dictword{filu͞a}[ˈfi.lua]
\dictterm{n}{(\textsc{fem}) torba, walizka, opakowanie}

\dictword{finti}[ˈfin.ti]
\dictterm{v}{(\textsc{pst} fint [ˈfiŋt]) szukać}

\dictword{flo}[ˈflɔ]
\dictterm{n}{(\textsc{pl} flos [ˈflɔs]) kwiat}

\dictword{fo}[ˈfɔ]
\dictterm{part}{dla (kogoś); \emph{fo tyi lipet} -- dla twojego dobra}

\dictword{fok}[ˈfɔk]
\dictterm{n}{(\textsc{pl} fokos [ˈfɔ.kɔs]) nazwa drobnej waluty używanej w Cesarstwie, ,,grosz''}

\dictword{for}[ˈfɔr]
\dictterm{n}{(\textsc{pl} foros [ˈfɔ.rɔs]) nazwa waluty używanej w Cesarstwie, ,,złoty''}
\note{Etymologia tego słowa jest jasna -- pochodzi od \emph{forn}, czyli faktycznie oznacza ,,złoty''. Symbolem waluty Cesarstwa używanym w transkrypcji jest \emph{Ƒ}.}

\dictword{fori}[ˈfɔ.ri]
\dictterm{v}{(\textsc{pst} forit [ˈfɔ.rit]) płacić, \emph{fori fo} -- płacić za kogoś lub za coś, \emph{fori o} -- płacić komuś}

\dictword{forn}[ˈfɔrn]
\dictterm{n}{złoto}

\dictword{fozeni}[ˈfɔ.zɛ.ni]
\dictterm{v}{(\textsc{pst} fozenit [ˈfɔ.zɛ.nit]) pożyczać, pożyczyć}
\note{Używa się kolejności osoba-przedmiot, więc \emph{mi va fozenit ti myi loesa} to ,,ja pożyczyłem ci moją lalkę'', ale \emph{mi va fozenit mi loesa} ,,pożyczyłem lalkę od kogoś'', \emph{mi va fozenit myi loesa} ,,pożyczyłem komuś lalkę''.}

\dictword{fozenya}[ˈfɔ.zɛn.ja]
\dictterm{n}{(\textsc{pl} fozeniji [ˈfɔ.zɛ.ni.ʐi]) (\textsc{fem}) pożyczka (np. pieniężna), kredyt}

\dictword{friti}[ˈfri.ti]
\dictterm{v}{(\textsc{pst} frit [ˈfrit]) świecić, płonąć}

\dictword{furet}[ˈfu.rɛt]
\dictterm{n}{(\textsc{pl} furetos [ˈfu.rɛ.tɔs]) wicher}

\dictword{futomamerey}[ˈfu.tɔ.ma.mɛ.rɛj]
\dictterm{n}{(\textsc{pl} futomamereyos [ˈfu.tɔ.ma.mɛ.rɛ.jɔs]) poduszkowiec}

\dictword{futon}[ˈfu.tɔn]
\dictterm{n}{(\textsc{pl} futonji [ˈfu.tɔn.ʐi]) wiatr}

\end{multicols}
\newpage
\section{G}
\begin{multicols}{2}

\dictword{gasoma}[ˈga.sɔ.ma]
\dictterm{n}{(\textsc{pl} gasomas) fasola}

\dictword{ge}[ˈgɛ]
\dictterm{part}{partykuła -- operator strony biernej}

\dictword{gen}[ˈgɛn]
\dictterm{n}{świt}

\dictword{geolo}[ˈgɛ.ɔ.lɔ]
\dictterm{n}{(\textsc{pl} geolos [ˈgɛ.ɔ.lɔs]) straż}

\dictword{geoli}[ˈgɛ.ɔ.li]
\dictterm{v}{(\textsc{pst} geolit [ˈgɛ.ɔ.lit]) pilnować}

\dictword{geolor}[ˈgɛ.ɔ.lɔr]
\dictterm{n}{(\textsc{pl} geoloros [ˈgɛ.ɔ.lɔ.rɔs]) (\textsc{fem} geolora [ˈgɛ.ɔ.lɔ.ra]) strażnik}

\dictword{gepita}[ˈgɛ.pi.ta]
\dictterm{n}{(\textsc{pl} gepitaji [ˈgɛ.pi.ta.ʐi]) (\textsc{fem}) okropność}

\dictword{gepito}[ˈgɛ.pi.tɔ]
\dictterm{adj}{(\textsc{comp} gepite͞a [ˈgɛ.pi.tɛa], \textsc{supl} gepite͞am [ˈgɛ.pi.tɛam]) okropny, ohydny}

\dictword{gepo}[ˈgɛ.pɔ]
\dictterm{adj}{(\textsc{comp} gepe [ˈgɛ.pɛ], \textsc{supl} gepem [ˈgɛ.pɛm]) wolny, powolny}

\dictword{gesbita}[ˈgɛs.bi.ta]
\dictterm{n}{(\textsc{pl} gesbitaji [ˈgɛs.bi.ta.ʐi]) (\textsc{fem}) choroba}

\dictword{gesbiti}[ˈgɛs.bi.ti]
\dictterm{v}{(\textsc{pst} gesbit [ˈgɛs.bit]) chorować}

\dictword{get}[ˈgɛt]
\dictterm{part}{od}

\dictword{getumi}[ˈgɛ.tu.mi]
\dictterm{v}{(\textsc{pst} getum [ˈgɛ.tum]) poddać}

\dictword{gir}[ˈgir]
\dictterm{adj}{zły, zepsuty, źle}

\dictword{girefari}[ˈgi.rɛ.fa.ri]
\dictterm{v}{(\textsc{pst} girefaret [ˈgi.rɛ.fa.rɛt]) psuć, uszkadzać}

\dictword{giriti}[ˈgi.ri.ti]
\dictterm{v}{(\textsc{pst} girit [ˈgi.rit]) pozdrawiać}

\dictword{girzuta}[ˈgir.zu.ta]
\dictterm{n}{(\textsc{pl} girzutaji) (\textsc{fem}) sałatka}

\dictword{gont}[ˈgɔŋt]
\dictterm{n}{(\textsc{pl} gontos [ˈgɔŋt.ɔs]) duch, dusza}

\dictword{graku͞a}[ˈgra.kua]
\dictterm{n}{(\textsc{pl} grakuji [ˈgra.ku.ʐi]) (\textsc{fem}) morze}

\dictword{grama}[ˈgra.ma]
\dictterm{n}{(\textsc{pl} gramajis [ˈgra.ma.ʐis]) (\textsc{fem}) trawa}

\dictword{gruchan}[ˈgru.t͡ʂan]
\dictterm{n}{(\textsc{pl} gruchos [ˈgru.t͡ʂɔs]) (\textsc{fem} gruchana [ˈgru.t͡ʂa.na]) dziadek, babcia}
\note{Bardziej formalne. Patrz również \emph{chanche}.}

\dictword{gruwa}[ˈgru.wa]
\dictred{gruwe}

\dictword{gruwe}[ˈgru.wɛ]
\dictterm{adj}{(\textsc{comp} gruwe͞a [ˈgru.wɛa], \textsc{supl} gruwe͞am [ˈgru.wɛam]) duży, wielki, olbrzymi}
\note{W niektórych dialektach najwyższy stopień (\textsc{supl}) to \emph{gruwa}.}

\dictword{gufedo}[ˈgu.fɛ.dɔ]
\dictterm{adj}{(\textsc{comp} gufede͞o [ˈgu.fɛ.dɛɔ], \textsc{supl} gufede͞om [ˈgu.fɛ.dɛɔm]) brzydki}

\dictword{gunt}[ˈguŋt]
\dictterm{n}{(\textsc{pl} guntji [ˈguŋt.ʐi]) śmiech}

\dictword{gusen}[ˈgu.sɛn]
\dictterm{n}{(\textsc{pl} guseji [ˈgu.sɛ.ʐi]) tekst}
\note{Wyjątek: słowo zakończone \emph{-i}, które nie jest czasownikiem.}

\dictword{gusta}[ˈgu.sta]
\dictterm{n}{(\textsc{pl} gusteji [ˈgu.stɛ.ʐi]) (\textsc{fem}) uciecha, szczęście}

\dictword{gusti}[ˈgus.ti]
\dictterm{v}{(\textsc{pst} gust [ˈgu.sɛt]) cieszyć się; \emph{gusti dyet} -- cieszyć się z~powodu czegoś}

\dictword{gusto}[ˈgu.stɔ]
\dictterm{adj}{szczęśliwy}

\end{multicols}
\newpage
\section{H}
\begin{multicols}{2}

\dictword{hallo͞i}[ˈxal.lɔi]
\dictterm{v}{(\textsc{pst} hallot [ˈxal.lɔt]) chwalić}

\dictword{halloya}[ˈxal.lɔ.ja]
\dictterm{n}{chwała, świętość}

\dictword{hawa}[ˈxa.wa]
\dictterm{n}{szczek (psa)}

\dictword{hawi}[ˈxa.wi]
\dictterm{v}{(\textsc{pst} hawet [ˈxa.wɛt]) szczekać}

\dictword{haye͞o}[ˈxa.jɛɔ]
\dictterm{n}{(\textsc{pl} hayois [ˈxa.jɔ.is]) światło}

\dictword{he}[ˈxɛ]
\dictterm{part}{tutaj}
\note{Przestarzałe. Patrz \emph{chet}.}

\dictword{hede͞a}[ˈxɛ.dɛa]
\dictterm{n}{(\textsc{pl} hedeji [ˈxɛ.dɛ.ʐi]) (\textsc{fem}) armata}

\dictword{hedera}[ˈxɛ.dɛ.ra]
\dictterm{n}{(\textsc{pl} hederji [ˈxɛ.dɛr.ʐi]) (\textsc{fem}) armia, siły zbrojne}
\note{And́royasyi Hedera -- Armia Androyas}

\dictword{hemi}[ˈxɛ.mi]
\dictterm{v}{(\textsc{pst} hemet [ˈxɛ.mɛt]) prosić}

\dictword{hemi}[ˈxɛ.mi]
\dictterm{part}{partykuła -- operator trybu przypuszczająco-proszącego (ekshortatywnego) w zdaniu, patrz również \emph{do}}

\dictword{henita}[ˈxɛ.ni.ta]
\dictterm{n}{(\textsc{pl} henitas [ˈxɛ.ni.tas]) (\textsc{fem}) stopa}

\dictword{hetay}[ˈxɛ.taj]
\dictterm{part}{dzisiaj, dziś}

\dictword{hifara}[ˈxi.fa.ra]
\dictterm{n}{(\textsc{pl} hifaras [ˈxi.fa.ras]) (\textsc{fem}) plecy}

\dictword{hima}[ˈxi.ma]
\dictterm{n}{(\textsc{pl} himji [ˈxim.ʐi]) (\textsc{fem}) dziewczyna, panna, panienka}
\note{Dawniej także: księżniczka.}

\dictword{hinaja}[ˈxi.na.ʐa]
\dictterm{n}{(\textsc{pl} hinaji [ˈxi.na.ʐi]) (\textsc{fem}) kobieta, pani}
\note{Spotykana jest również forma \emph{hinata}.}

\dictword{hirni}[ˈxir.ni]
\dictterm{v}{(\textsc{pst} hirnet [ˈxir.nɛt]) produkować}

\dictword{hitache}[ˈxi.ta.t͡ʂɛ]
\dictterm{n}{(\textsc{pl} hitachis [ˈxi.ta.t͡ʂis]) fotografia, zdjęcie}

\dictword{hitler}[ˈxit.lɛr]
\dictterm{n}{(\textsc{pl} hitleji [ˈxit.lɛ.ʐi]) rodzaj potrawy z ciasta wypełnionego farszem; pierogi}

\note{\emph{Hitler} robi się z ciasta makaronowego i wypełniane są nadzieniem zależnym od rejonu, z którego pochodzą. Mogą być pieczone, gotowane w wodzie, gotowane na parze lub smażone, najczęstszym farszem są ziemniaki, ser lub różne rodzaje i mieszanki mięs. \emph{Hitler} wywodzą się z rejonu Betuli, ale od początku Cesarstwa stały się popularnym daniem na całym jego obszarze. }

\note{Z uwagi na konotacje tego słowa z nazwiskiem z historii Ziemi, można stosować zamiennie słowo \emph{bargul}.}

\dictword{hojo}[ˈxɔ.ʐɛ]
\dictterm{adj}{(\textsc{comp} hoje͞o [xɔ.ʐɛ.ɔ], \textsc{supl} hojem [xɔ.ʐɛm]) mało, mniej, najmniej}

\dictword{huarn}[ˈxu.arn]
\dictterm{n}{(\textsc{pl} huarneji [ˈxu.ar.nɛ.ʐi]) słowo}

\dictword{hu͞ekapa}[ˈxuɛ.ka.pa]
\dictterm{n}{(\textsc{fem}) komputer}
\note{Skrót. Patrz \emph{hu͞erjikapatana}.}

\dictword{hu͞era}[ˈxuɛ.ra]
\dictterm{n}{(\textsc{pl} hu͞erji [ˈxuɛr.ʐi]) (\textsc{fem}) liczba, numer}

\dictword{hu͞eri}[ˈxuɛ.ri]
\dictterm{v}{(\textsc{pst} hueret [ˈxu.ɛ.rɛt]) liczyć}

\dictword{hu͞erjikapatana}[ˈxuɛr.ʐi.ka.pa.ta.na]
\dictterm{n}{(\textsc{pl} hu͞erjikapatanis [ˈxuɛr.ʐi.ka.pa.ta.nis]) (\textsc{fem}) komputer}
\note{Dosłownie: liczby-maszyna. Jest to jedno z najdłuższych słów w języku
andro i stąd najczęściej stosuje się skróconą formę, \emph{hu͞ekapa}.}

\dictword{hu͞eva}[ˈxuɛ.va]
\dictterm{n}{(\textsc{pl} hu͞evaji [ˈxuɛ.va.ʐi]) (\textsc{fem}) ryba}

\dictword{hufay}[ˈxu.faj]
\dictterm{n}{(\textsc{pl} hufayos [ˈxu.fa.jɔs]) tłuszcz}

\dictword{huhu}[ˈxu.xu]
\dictterm{part}{po}

\dictword{hukajek}[ˈxu.ka.ʐɛk]
\dictterm{n}{(\textsc{pl} hukajos [ˈxu.ka.ʐɔs]) sposób}

\dictword{husawa}[ˈxu.sa.wa]
\dictterm{n}{(\textsc{pl} husawaji [ˈxu.sa.wa.ʐi]) (\textsc{fem}) prostytutka, kurwa}
\note{Wykorzystywane czasami również jako przekleństwo.}

\end{multicols}
\newpage
\section{I}
\begin{multicols}{2}

\dictword{ib́iri}[i.ˈbi.ri]
\dictterm{v}{(\textsc{pst} ib́irit [i.ˈbi.rit]) oddawać mocz, sikać}

\dictword{ichera}[ˈi.t͡ʂɛ.ra]
\dictterm{n}{(\textsc{fem}) krew}

\dictword{id́ak}[i.ˈdak]
\dictterm{n}{(\textsc{pl} id́akos [i.ˈda.kɔs]) otwór}

\dictword{id́aki}[i.ˈda.ki]
\dictterm{v}{(\textsc{pst} id́ak [i.ˈdak]) otwierać}

\dictword{id́ako}[i.ˈda.kɔ]
\dictterm{adj}{otwarty}

\dictword{iéni}[i.ˈɛ.ni]
\dictterm{v}{(\textsc{pst} iént [i.ˈɛŋt]) przychodzić, przybyć, przybywać; \emph{iéni a͞u} -- pochodzić (z)}

\dictword{iényihuarn}[i.ˈɛ.nʏ.xu.arn]
\dictterm{n}{(\textsc{pl} iényihuarneji [i.ˈɛ.nʏ.xu.ar.nɛ.ʐi]) hasło}

\dictword{ifrit}[ˈi.frit]
\dictterm{n}{(\textsc{pl} ifritos [ˈi.fri.tɔs]) ogień}

\dictword{ifriti}[ˈi.fri.ti]
\dictterm{adj}{ognisty, zapalający}

\dictword{if́riti}[i.ˈfri.ti]
\dictterm{v}{(\textsc{pst} if́iret [i.ˈfi.rɛt]) zapalać, zaświecać}

\dictword{ihalfo}[ˈi.xal.fɔ]
\dictterm{adj}{pomidorowy}

\dictword{ihalfu}[ˈi.xal.fu]
\dictterm{n}{(\textsc{pl} ihalfus) pomidor}

\dictword{ih́emi}[i.ˈxɛ.mi]
\dictterm{v}{(\textsc{pst} ih́emet [i.ˈxɛ.mɛt]) błagać}

\dictword{iḱoperi}[i.ˈkɔ.pɛ.ri]
\dictterm{v}{(\textsc{pst} iḱoperet [i.ˈkɔ.pɛ.rɛt]) przykrywać}

\dictword{il}[ˈil]
\dictterm{pro}{jego, jej (\textsc{3SG.GEN} lub \textsc{3SG.POSS})}

\dictword{iḿaheyo}[i.ˈma.xɛ.jɔ]
\dictterm{adj}{przyciągający, pociągający}

\dictword{iḿahi}[i.ˈma.xi]
\dictterm{v}{(\textsc{pst} iḿahit [i.ˈma.xit]) przyciągać}

\dictword{iḿami}[i.ˈma.mi]
\dictterm{v}{(\textsc{pst} iḿant [i.ˈmaŋt]) przynosić}

\dictword{imin}[ˈi.min]
\dictterm{part}{bo, ponieważ}

\dictword{imṕigetin}[im.ˈpi.gɛ.tin]
\dictterm{n}{(\textsc{pl} imṕigetos [im.ˈpi.gɛ.tɔs]) (\textsc{fem} imṕigeta [im.ˈpi.gɛ.ta]) służący}

\dictword{imṕigetira}[im.ˈpi.gɛ.ti.ra]
\dictterm{n}{(\textsc{pl} imṕigetiros [im.ˈpi.gɛ.ti.rɔs]) (\textsc{fem}) służba (np. utrzymania ruchu)}

\dictword{imṕigi}[im.ˈpi.gi]
\dictterm{v}{(\textsc{pst} imṕigit [im.ˈpi.git]) służyć}

\dictword{imvi}[ˈim.vi]
\dictterm{v}{(\textsc{pst} imvit [ˈim.vit]) bać (się)}

\dictword{imzitin}[ˈim.zi.tin]
\dictterm{n}{(\textsc{pl} imzitiji [ˈim.zi.ti.ʐi]) prom (zarówno łódź, jak i~prom kosmiczny)}

\dictword{in}[ˈin]
\dictterm{part}{w (czymś)}

\dictword{iner}[ˈi.nɛr]
\dictterm{n}{(\textsc{pl} inerji [ˈi.nɛr.ʐi]) głos}

\dictword{ingeki}[ˈin.gɛ.ki]
\dictterm{v}{(\textsc{pst} ingek [ˈin.gɛk]) czekać}

\dictword{init}[ˈi.nit]
\dictterm{n}{(\textsc{pl} inji [ˈin.ʐi]) liść}

\dictword{inmeú}[in.mɛ.ˈu]
\dictterm{part}{natychmiast}
\note{W niektórych dialektach używa się określenia \emph{immeú} /im:.ɛ.ˈu/.}

\dictword{inra͞i}[ˈin.rai]
\dictterm{v}{(\textsc{pst} inrat [ˈin.rat]) mówić}
\note{Dawniej, przede wszystkim w~dialekcie zachodnim, używane było bardzo często słowo \emph{parlai}.}

\dictword{inrat}[ˈin.rat]
\dictterm{n}{(\textsc{pl} inratji [ˈin.rat.ʐi]) mowa, język}

\dictword{insupe}[ˈin.su.pɛ]
\dictterm{n}{(\textsc{pl} insupejos [ˈin.su.pɛ.ʐɔs]) inspektor}

\dictword{intay}[ˈin.taj]
\dictterm{part}{w dniu}

\dictword{intrise}[ˈin.tri.sɛ]
\dictterm{adj}{(\textsc{comp} intrise͞a [ˈin.tri.sɛa], \textsc{supl} intrise͞am [ˈin.tri.sɛam]) interesujące}
\note{Pochodzi od rdzenia \emph{inret} z~przesunięciem, dosłownie ,,mówi się o tym, to jest ciekawe''.}

\dictword{inure}[ˈi.nu.rɛ]
\dictterm{adj}{(\textsc{comp} inure͞a [ˈi.nu.rɛa], \textsc{supl} inure͞am [ˈi.nu.rɛam]) zwykły, standardowy, typowy}

\dictword{inuri}[ˈi.nu.ri]
\dictterm{v}{(\textsc{pst} inurit [ˈi.nu.rit]) przyzwyczajać; \emph{inuri a} -- robić coś zwykle, być przyzwyczajonym do czegoś, być nawykłym do czegoś}


\dictword{isan}[ˈi.san]
\dictterm{n}{dziedzictwo}
\note{Pojęcie to wywodzi się od rytualnego, zdobionego, zwierzęcego rogu, \emph{isanyiisdat}. Róg taki, jako rodowy przedmiot i~symbol nestora rodu był przekazywany z~pokolenia na pokolenie. Zwyczaje związane z~\emph{isanyiisdat} zanikły na obszarze Nennek w~czasie Epoki Niezależności, gdzieniegdzie jednak kultywowane były aż do rozpadu Pierwszego Cesarstwa.}

\dictword{isdam}[ˈis.dam]
\dictterm{n}{(\textsc{pl} isdamos [ˈis.da.mɔs]) połączenie, zjednoczenie}

\dictword{isdama͞i}[ˈis.da.mai]
\dictterm{v}{(\textsc{pst} isdat [ˈis.dat]) łączyć, dodawać}

\dictword{isdar}[ˈis.dar]
\dictterm{n}{połowa}

\dictword{isdara͞i}[ˈis.da.rai]
\dictterm{v}{(\textsc{pst} isdaret [ˈis.da.rɛt]) dzielić, łamać, ale również podzielać (zdanie)}

\dictword{isdat}[ˈis.dat]
\dictterm{n}{(\textsc{pl} isdatos [ˈis.da.tɔs]) róg zwierzęcy}

\dictword{isdara}[ˈis.da.ra]
\dictterm{n}{(\textsc{pl} isdarji [ˈis.dar.ʐi]) (\textsc{fem}) odpowiednik stopnia dywizji,
zgrupowanie wojsk lądowych liczące obecnie około 10 tysięcy żołnierzy}

\dictword{isin}[ˈi.sin]
\dictterm{n}{(\textsc{pl} isinos [ˈi.si.nɔs]) ząb}

\dictword{iskabar}[ˈis.ka.bar]
\dictterm{n}{(\textsc{pl} iskabaros [ˈis.ka.ba.rɔs]) postęp}

\dictword{isnaja}[ˈis.na.ʐa]
\dictterm{n}{(\textsc{pl} isnaji [ˈis.na.ʐi]) (\textsc{fem}) szkło}

\dictword{issude}[ˈis.su.dɛ]
\dictterm{n}{(\textsc{pl} issudos [ˈis.su.dɔs]) plan}

\dictword{issudera}[ˈis.su.dɛ.ra]
\dictterm{n}{(\textsc{pl} issuderji [ˈis.su.dɛr.ʐi]) (\textsc{fem}) strategia}

\dictword{issudero}[ˈis.su.dɛ.rɔ]
\dictterm{adj}{strategiczny}

\dictword{issu͞et}[ˈis.suɛ.tɛ]
\dictterm{n}{(\textsc{pl} issu͞ete [ˈis.suɛ.tɛ]) pieniądz}

\dictword{issu͞i}[ˈis.sui]
\dictterm{v}{(\textsc{pst} issut [ˈis.sut]) planować}

\dictword{isturea}[ˈis.tu.rɛ.a]
\dictterm{n}{(\textsc{pl} istureji [ˈis.tu.rɛ.ʐi]) (\textsc{fem}) podatek}

\dictword{itonal}[ˈi.tɔ.nal]
\dictterm{n}{(\textsc{pl} itonalos [ˈi.tɔ.na.lɔs]) bilet}

\end{multicols}
\newpage
\section{J}
\begin{multicols}{2}

\dictword{ja}[ˈʐa]
\dictterm{part}{ten konkretny, ten, to, ta}

\dictword{jaimi}[ˈʐa.i.mi]
\dictterm{v}{(\textsc{pst} jaint [ˈʐa.iŋt]) budować}

\dictword{jan}[ˈʐan]
\dictterm{n}{(\textsc{pl} janos [ˈʐa.nɔs]) dom}

\dictword{jarani}[ˈʐa.ra.ni]
\dictterm{v}{(\textsc{pst} jarant [ˈʐa.raŋt]) rozwiązywać (problem, zagadkę), ale także:
rozwiązywać sznury, więzy}

\dictword{jari}[ˈʐa.ri]
\dictterm{v}{(\textsc{pst} jaret [ˈʐa.rɛt]) toczyć}

\dictword{jaro}[ˈʐa.rɔ]
\dictterm{n}{(\textsc{pl} jaroan [ˈʐa.rɔ.an]) ucho}

\dictword{jawir}[ˈʐa.wir]
\dictterm{n}{(\textsc{pl} jawirji [ˈʐa.wir.ʐi]) kradzież}

\dictword{jawiri}[ˈʐa.wi.ri]
\dictterm{v}{(\textsc{pst} jawirit [ˈʐa.wi.rit]) kraść}

\dictword{jawiror}[ˈʐa.wi.rɔr]
\dictterm{n}{(\textsc{pl} jawiroros [ˈʐa.wi.rɔ.rɔs]) złodziej}

\dictword{je}[ˈʐɛ]
\dictterm{pro}{przyimek określony, wskazanie na konrketny obiekt}

\dictword{jeji}[ˈʐɛ.ʐi]
\dictterm{n}{(\textsc{pst} jejit [ˈʐɛ.ʐit]) klęczeć, klękać, uklęknąć}
\note{\emph{jeji do} znaczy dosłownie ,,padnij na kolana''.}

\dictword{jeluche}[ˈʐɛ.lu.t͡ʂɛ]
\dictterm{n}{(\textsc{pl} jeluches [ˈʐɛ.lu.t͡ʂɛs]) zamiar, cel, intencja}

\dictword{jenosa}[ˈʐɛ.nɔ.sa]
\dictterm{n}{(\textsc{pl} jenosaji [ˈʐɛ.nɔ.sa.ʐi]) (\textsc{fem}) drużyna, w~szczególności
drużyna jako oddział wojskowy, grupa około 5 żołnierzy}

\dictword{jerya}[ˈʐɛr.ja]
\dictterm{n}{(\textsc{pl} jeryeniji [ˈʐɛr.jɛ.ni.ʐi]) (\textsc{fem}) przecinek}
\note{Używane w liczebnikach, w postaci \emph{a jerya rasa} -- 1,83.}

\dictword{jeya}[ˈʐɛ.ja]
\dictterm{n}{(\textsc{pl} jeye [ˈʐɛ.jɛ]) (\textsc{fem}) kolano}

\dictword{ji}[ˈʐi]
\dictterm{part}{coraz}

\dictword{ji͞ari͞o}[ˈʐia.riɔ]
\dictterm{n}{(\textsc{pl} ji͞ari͞os [ˈʐia.riɔs]) ogród}

\dictword{jiki}[ˈʐi.ki]
\dictterm{v}{(\textsc{pst} jikit [ˈʐi.kit]) biegać}

\dictword{jo}[ˈʐɔ]
\dictterm{part}{Partykuła używana przy liczebnikach. Oznacza "oraz" i jest używana czasami zamiast systemu pozycyjnego.}

\dictword{jubo}[ˈʐu.bɔ]
\dictterm{adj}{fioletowy}

\dictword{jubo}[ˈʐu.bɔ]
\dictterm{n}{fiolet}

\end{multicols}
\newpage
\section{K}
\begin{multicols}{2}

\dictword{ka}[ˈka]
\dictterm{n}{dwa, dwójka}
\note{Dla każdego liczebnika można wykorzystać określenie \emph{razi}, na przykład
\emph{ka razi} oznacza dwukrotnie, dwa razy.}

\dictword{kabama}[ˈka.ba.ma]
\dictterm{n}{(\textsc{pl} kabamji [ˈka.bam.ʐi]) (\textsc{fem}) bomba}

\dictword{kabami}[ˈka.ba.mi]
\dictterm{v}{(\textsc{pst} kabamit [ˈka.ba.mit]) bombardować, wysadzać}

\dictword{kabuler}[ˈka.bu.lɛr]
\dictterm{n}{(\textsc{pl} kabulos [ˈka.bu.lɔs]) wybuch}

\dictword{kache}[ˈka.t͡ʂɛ]
\dictterm{n}{(\textsc{pl} kachu [ˈka.t͡ʂu]) skrzydło}

\dictword{kachibarer}[ˈka.t͡ʂi.ba.rɛr]
\dictterm{n}{(\textsc{pl} kachibareros [ˈka.t͡ʂi.ba.rɛ.rɔs]) uniwersytet}

\dictword{kachisi}[ˈka.t͡ʂi.si]
\dictterm{v}{(\textsc{pst} kachisit [ˈka.t͡ʂi.sit]) uczyć}

\dictword{kachister}[ˈka.t͡ʂi.stɛr]
\dictterm{n}{(\textsc{pl} kachistos [ˈka.t͡ʂi.stɔs]) (\textsc{fem} kachistera [ˈka.t͡ʂi.stɛ.ra]) nauczyciel}

\dictword{kaéti}[ka.ˈɛ.ti]
\dictterm{v}{(\textsc{pst} kaét [ka.ˈɛt]) słuchać}

\dictword{kahi}[ˈka.xi]
\dictterm{v}{(\textsc{pst} kahit [ˈka.xit]) latać}

\dictword{kahobeykar}[ˈka.xɔ.bɛj.kar]
\dictred{trakon}

\dictword{kahokapata}[ˈka.xɔ.ka.pa.ta]
\dictterm{n}{(\textsc{pl} kahokapataji [ˈka.xɔ.ka.pa.ta.ʐi]) (\textsc{fem}) samolot}

\dictword{kahomamurte}[ˈka.xɔ.ma.mur.tɛ]
\dictterm{n}{(\textsc{pl} kahomamurtos [ˈka.xɔ.ma.mur.tɔs]) lotnisko, port lotniczy}

\dictword{kahuna}[ˈka.xu.na]
\dictterm{n}{(\textsc{pl} kahunos [ˈka.xun.ɔs]) (\textsc{fem}) ptak}

\dictword{kaj́et}[ka.ˈʐɛt]
\dictterm{n}{słuch}

\dictword{kalajim}[ˈka.la.ʐim]
\dictred{kali}

\dictword{kali}[ˈka.li]
\dictterm{adj}{(\textsc{comp} kalji [ˈkal.ʐi], \textsc{supl} kaljim [ˈkal.ʐim]) czysty}
\note{W dialekcie zachodnim najwyższy stopień (\textsc{supl}) ma formę \emph{kalajim}.}

\dictword{kalir}[ˈka.lir]
\dictterm{n}{(\textsc{pl} kalirji [ˈka.lir.ʐi]) ideał}

\dictword{kaliri͞o}[ˈka.lir.iɔ]
\dictterm{adj}{idealny, optymalny}

\dictword{kamari}[ˈka.ma.ri]
\dictterm{v}{(\textsc{pst} kamarit [ˈka.ma.rit]) nie rozumieć}

\dictword{kamat}[ˈka.mat]
\dictterm{n}{(\textsc{pl} kamatji [ˈka.mat.ʐi]) kłamstwo}

\dictword{kami}[ˈka.mi]
\dictterm{v}{(\textsc{pst} kamat [ˈka.mat]) kłamać}

\dictword{kamiber}[ˈka.mi.bɛr]
\dictterm{n}{(\textsc{pl} kamiberos [ˈka.mi.bɛ.rɔs]) kontroler}

\dictword{kamibi}[ˈka.mi.bi]
\dictterm{v}{(\textsc{pst} kamibit [ˈka.mi.bit]) kontrolować}

\dictword{kant}[ˈkaŋt]
\dictterm{n}{(\textsc{pl} kantos [ˈkan.tɔs]) śpiew}

\dictword{kantar}[ˈkan.tar]
\dictterm{n}{(\textsc{pl} kantaros [ˈkan.ta.rɔs]) (\textsc{fem} kantara [ˈkan.ta.ra]) śpiewak}

\dictword{kanti}[ˈkan.ti]
\dictterm{v}{(\textsc{pst} kant [ˈkaŋt]) śpiewać}

\dictword{kapata}[ˈka.pa.ta]
\dictterm{n}{(\textsc{pl} kapatji [ˈka.pat.ʐi]) (\textsc{fem}) maszyna}

\dictword{kapatana}[ˈka.pa.ta.na]
\dictterm{n}{(\textsc{pl} kapatanis [ˈka.pa.ta.nis]) (\textsc{fem}) maszyna}
\note{Forma używana dawniej. Obecnie raczej \emph{kapata}, w~niektórych słowach
ewoluowało wręcz do członu \emph{-kapa}, patrz \emph{hu͞ekapa}.}

\dictword{kapi}[ˈka.pi]
\dictterm{v}{(\textsc{pst} kapit [ˈka.pit]) rozumieć}

\dictword{karié}[ka.ri.ˈɛ]
\dictterm{adj}{(\textsc{comp} karié͞a [ka.ri.ˈɛa], \textsc{supl} karié͞am [ka.ri.ˈɛam]) ładny, piękny}

\dictword{karla}[ˈkar.la]
\dictterm{n}{(\textsc{fem}) śmierć, patrz również \emph{karla͞i}.}

\dictword{karla͞i}[ˈkar.lai]
\dictterm{v}{(\textsc{pst} karlet [ˈkar.lɛt]) zabijać}
\note{Dosłownie: zadawać śmierć, dawniej w~formie czasu przeszłego również \emph{karl}.}

\dictword{karlar}[ˈkar.lar]
\dictterm{n}{(\textsc{pl} karlaros [ˈkar.la.rɔs]) zmarły}

\dictword{karlayter}[ˈkar.laj.tɛr]
\dictterm{n}{(\textsc{pl} karlayteros [ˈkar.laj.tɛ.rɔs]) zabójca}

\dictword{karlayti}[ˈkar.laj.ti]
\dictterm{n}{(\textsc{pl} karlaytiji [ˈkar.laj.ti.ʐi]) zabójstwo}

\dictword{karli}[ˈkar.li]
\dictterm{v}{(\textsc{pst} karet [ˈka.rɛt]) umierać}

\dictword{kati}[ˈka.ti]
\dictterm{adj}{drugi}

\dictword{katia}[ˈka.ti.a]
\dictterm{adj}{druga}
\note{Bardzo często jest używany jako imię żeńskie, zarówno w~formie \emph{Katia}, jak i~\emph{Katya} lub \emph{Kati͞a}.}

\dictword{katnisi}[ˈkat.ni.si]
\dictterm{v}{(\textsc{pst} katnit [ˈkat.nit]) podwajać}

\dictword{ka͞ufi}[ˈkau.fi]
\dictterm{v}{(\textsc{pst} ka͞ufit [ˈkau.fit]) skakać}

\dictword{kay}[ˈkaj]
\dictterm{n}{(\textsc{pl} kajis [ˈka.ʐis]) (\textsc{fem}) noc}

\dictword{kayćhar}[kaj.ˈt͡ʂar]
\dictterm{n}{(\textsc{pl} kayćharji [kaj.ˈt͡ʂar.ʐi]) klątwa}
\note{Dosłownie: Dar Nocy.}

\dictword{kayer}[ˈka.jɛr]
\dictterm{adj}{(\textsc{comp} kayere [ˈka.jɛ.rɛ], \textsc{supl} kayerem [ˈka.jɛ.rɛm]) zły, gorszy}

\dictword{kayeto}[ˈka.jɛ.tɔ]
\dictterm{n}{(\textsc{pl} kayetos [ˈka.jɛ.tɔs]) grzech, zły uczynek}

\dictword{kayetor}[ˈka.jɛ.tɔr]
\dictterm{n}{(\textsc{pl} kayetoros [ˈka.jɛ.tɔ.rɔs]) (\textsc{fem} kayetora [ˈka.jɛ.tɔ.ra]) przestępca, grzesznik, winowajca, złoczyńca}

\dictword{kayri}[ˈkaj.ri]
\dictterm{n}{zło}

\dictword{ke͞amkey}[ˈkɛam.kɛj]
\dictterm{n}{(\textsc{pl} ke͞amkeji [ˈkɛam.kɛ.ʐi]) narkotyk, lek, środek leczniczy}

\dictword{kean}[ˈkɛ.an]
\dictterm{n}{(\textsc{pl} keani [ˈkɛ.a.ni]) korzeń}

\dictword{kebuér}[kɛ.bu.ˈɛr]
\dictterm{n}{(\textsc{pl} kebuérji [kɛ.bu.ˈɛr.ʐi]) pałac}

\dictword{keja}[ˈkɛ.ʐa]
\dictterm{n}{(\textsc{pl} keja͞o [ˈkɛ.ʐaɔ]) (\textsc{fem}) klucz}

\dictword{kejauri}[ˈkɛ.ʐa.u.ri]
\dictterm{v}{(\textsc{pst} kejauret [ˈkɛ.ʐa.u.rɛt]) wykluczać}

\dictword{kelay}[ˈkɛl.aj]
\dictterm{n}{(\textsc{pl} kelayos [ˈkɛ.la.jɔs]) język}
\note{\emph{kelayidazer}: pocałunek z~języczkiem.}

\dictword{kelayidazer}[ˈkɛ.la.ʏ.da.zɛr]
\dictred{kelay}

\dictword{keli}[ˈkɛ.li]
\dictterm{v}{(\textsc{pst} kelet [ˈkɛ.lɛt]) lizać}
\note{Dosłownie: pracować językiem.}

\dictword{kelsan}[ˈkɛl.san]
\dictterm{n}{(\textsc{pl} kelsanos [ˈkɛl.sa.nɔs]) porządek}

\dictword{kelsani}[ˈkɛl.sa.ni]
\dictterm{v}{(\textsc{pst} kelsant [ˈkɛl.saŋt]) porządkować}

\dictword{kerle}[ˈkɛr.lɛ]
\dictterm{n}{(\textsc{pl} kerles [ˈkɛr.lɛs]) warownia, zamek}

\dictword{kervo}[ˈkɛr.vɔ]
\dictterm{n}{(\textsc{pl} kervos [ˈkɛr.vɔs]) mózg}

\dictword{kest}[ˈkɛ.stɛ]
\dictterm{n}{(\textsc{pl} kestos [ˈkɛs.tɔs]) pytanie}

\dictword{kesti}[ˈkɛs.ti]
\dictterm{v}{(\textsc{pst} kestet [ˈkɛs.tɛt]) pytać}

\dictword{key}[ˈkɛj]
\dictterm{part}{jak}
\note{Wykorzystywany przy porównywaniu: coś jest jak coś innego.}

\dictword{kigeje}[ˈki.gɛ.ʐɛ]
\dictterm{n}{(\textsc{pl} kigejos [ˈki.gɛ.ʐɔs]) (\textsc{fem} kigeje͞a [ˈki.gɛ.ʐɛa]) książę, księżniczka}

\dictword{kigejerid}[ˈki.gɛ.ʐɛ.rid]
\dictterm{n}{(\textsc{pl} kigejeridos [ˈki.gɛ.ʐɛ.ri.dɔs]) księstwo}

\dictword{kihile}[ˈki.xi.lɛ]
\dictterm{part}{być może, możliwe więc}

\dictword{kihilo}[ˈki.xi.lɔ]
\dictterm{adj}{(\textsc{comp} kihilo͞e [ˈki.xi.lɔɛ]) możliwy}
\note{Nie posiada najwyższego stopnia.}

\dictword{kiiwai}[ˈki:.wa.i]
\dictterm{v}{(\textsc{pst} kiiwat [ˈki:.wat]) witać}

\dictword{kiiway}[ki:.ˈwaj]
\dictterm{n}{(\textsc{pl} kiiwaji [ˈki:.wa.ʐi]) powitanie}
\note{Również forma typowego, \textsc{supl} niezbyt formalnego powitania.}

\dictword{kipen}[ˈki.pɛn]
\dictterm{n}{(\textsc{pl} kipenos [ˈki.pɛ.nɔs]) wzór}

\dictword{kipeni}[ˈki.pɛ.ni]
\dictterm{v}{(\textsc{pst} kipenit [ˈki.pɛ.nit]) wzorować; \emph{kipeni a} -- wzorować się na kimś, na czymś}

\dictword{kiro}[ˈki.rɔ]
\dictterm{adj}{żółty}

\dictword{kiro}[ˈki.rɔ]
\dictterm{n}{kolor żółty}

\dictword{kiruki}[ˈki.ru.ki]
\dictterm{v}{(\textsc{pst} kiruket [ˈki.ru.kɛt]) umieć}

\dictword{klik}[ˈklik]
\dictterm{n}{(\textsc{pl} kliki [ˈkli.ki]) odpowiednik minuty, ok. 45 sekund}

\dictword{koange}[ˈkɔ.an.gɛ]
\dictterm{n}{(\textsc{pl} koanges [ˈkɔ.an.gɛs]) wątroba}

\dictword{kobato}[ˈkɔ.ba.tɔ]
\dictterm{adj}{(\textsc{comp} kobato͞e [ˈkɔ.ba.tɔɛ], \textsc{supl} kobato͞em [ˈkɔ.ba.tɔɛm]) słaby}

\dictword{kobovi}[ˈkɔ.bɔ.vi]
\dictterm{v}{(\textsc{pst} kobovit [ˈkɔ.bɔ.vit]) walczyć}

\dictword{kobovir}[ˈkɔ.bɔ.vir]
\dictterm{n}{(\textsc{pl} koboviros [ˈkɔ.bɔ.vi.rɔs]) (\textsc{fem} kobovira [ˈkɔ.bɔ.vi.ra]) żołnierz}

\dictword{koda}[ˈkɔ.da]
\dictterm{n}{(\textsc{pl} kodiji [ˈkɔ.di.ʐi]) (\textsc{fem}) ogon}

\dictword{ko͞e}[ˈkɔɛ]
\dictterm{part}{jak, w~jaki sposób}
\note{Stosowany tylko w pytaniu - aby powiedzieć "coś jak coś" używa się \emph{koy}.}

\dictword{kokis}[ˈkɔ.kis]
\dictterm{n}{(\textsc{pl} kokijis [ˈkɔ.ki.ʐis]) brak, niedobór}

\dictword{kokoro}[ˈkɔ.kɔ.rɔ]
\dictterm{n}{(\textsc{pl} kokoros [ˈkɔ.kɔ.rɔs]) serce}

\dictword{kone}[ˈkɔ.nɛ]
\dictterm{part}{bez (czegoś)}
\note{Może pojawiać się jako sufiks słowa, np. chelikakone - bez litości.}

\dictword{koólen}[kɔ.ˈɔl.ɛn]
\dictterm{n}{miłość romantyczna, miłość rodzicielska}

\dictword{koóler}[kɔ.ˈɔ.lɛr]
\dictterm{n}{(\textsc{pl} koólerji [kɔ.ˈɔ.lɛr.ʐi]) (\textsc{fem} koólera [kɔ.ˈɔ.lɛ.ra]) kochanek, obiekt miłości}
\note{W sytuacji, w~której nie chcemy określać płci, używa się rodzaju męskiego. 
Słowo to dotyczy głównie miłości romantycznej, bo kochanek w~sensie seksualnym to
\emph{aḿachager}.}

\dictword{koóli}[kɔ.ˈɔ.li]
\dictterm{v}{(\textsc{pst} koólet [kɔ.ˈɔ.lɛt]) kochać}

\dictword{koperi}[ˈkɔ.pɛ.ri]
\dictterm{v}{(\textsc{pst} kopet [ˈkɔ.pɛt]) kryć, ukrywać}

\dictword{korebi}[ˈkɔ.rɛ.bi]
\dictterm{v}{(\textsc{pst} kret [ˈkrɛt]) ufać}

\dictword{korno}[ˈkɔr.nɔ]
\dictterm{n}{(\textsc{pl} kornos [ˈkɔr.nɔs]) róg, narożnik, dziób ptaka, wydłużone zakończenie przedmiotu, przednia część kadłuba statku}
\note{Ale nie róg zwierzęcy, patrz \emph{isdat}.}

\dictword{kovi}[ˈkɔ.vi]
\dictterm{part}{tuż, ledwo}

\dictword{kower}[ˈkɔ.wɛr]
\dictterm{n}{północ (kierunek)}

\dictword{koy}[ˈkɔj]
\dictterm{part}{jakoś, jakby, jakiś, jak}

\dictword{kuanti}[ˈku.an.ti]
\dictterm{v}{(\textsc{pst} kuant [ˈku.aŋt]) polować}

\dictword{kuka͞o}[ˈku.kaɔ]
\dictterm{n}{(\textsc{pl} kukaos [ˈku.kaɔs]) (\textsc{fem}) cud}

\dictword{kukaó}[ku.ka.ˈɔ]
\dictterm{adj}{(\textsc{comp} kukao͞e [ˈku.ka.ɔɛ], \textsc{supl} kukao͞em [ˈku.ka.ɔɛm]) cudowny}

\dictword{kuosi}[ˈku.ɔ.si]
\dictterm{v}{(\textsc{pst} kuoset [ˈku.ɔ.sɛt]) żądać}

\dictword{kutut}[ˈku.tut]
\dictterm{n}{(\textsc{pl} kutos [ˈku.tɔs]) but}

\dictword{kyestyo}[ˈkjɛs.tjɔ]
\dictterm{n}{(\textsc{pl} kyesyos [ˈkjɛs.jɔs]) krokodyl}

\dictword{kyige}[ˈkʏ.gɛ]
\dictterm{n}{(\textsc{pl} kyigos [ˈkʏ.gɔs]) (\textsc{fem} kyige͞a [ˈkʏ.gɛa]) król, królowa}
\note{Słowo to jest jednym z~nielicznych, w~którym głoska /ʏ/ nie występuje w~kontekście określenia posiadania czegoś. Słowo to pochodzi wprost z~języka staronenneckiego.}

\dictword{kyigerid}[ˈkʏ.gɛ.rid]
\dictterm{n}{(\textsc{pl} kyigeridos [ˈkʏ.gɛ.ri.dɔs]) królestwo}

\end{multicols}
\newpage
\section{L}
\begin{multicols}{2}

\dictword{laák}[la.ˈak]
\dictterm{n}{(\textsc{pl} laákji [la.ˈak.ʐi]) drzewo}

\dictword{labi}[ˈla.bi]
\dictterm{v}{(\textsc{pst} labit [ˈla.bit]) bawić się}

\dictword{laḱan}[la.ˈkan]
\dictterm{n}{(\textsc{pl} laḱanji [la.ˈkan.ʐi]) las}

\dictword{laḱer}[la.ˈkɛr]
\dictterm{n}{(\textsc{pl} laḱerji [la.ˈkɛr.ʐi]) (\textsc{fem}) park}

\dictword{lamiji}[ˈla.mi.ʐit]
\dictterm{v}{(\textsc{pst} lamijit [ˈla.mi.ʐit]) oddychać}

\dictword{laoche}[ˈla.ɔ.t͡ʂɛ]
\dictterm{n}{(\textsc{pl} laoches [ˈla.ɔ.t͡ʂɛs]) gratulacje}

\dictword{laochi}[ˈla.ɔ.t͡ʂi]
\dictterm{v}{(\textsc{pst} lachit [ˈla.t͡ʂit]) gratulować}

\dictword{lana}[ˈla.na]
\dictterm{n}{(\textsc{pl} lanji [ˈlan.ʐi]) (\textsc{fem}) książka, księga}

\dictword{lannake}[ˈlan.na.kɛ]
\dictterm{n}{(\textsc{pl} lannakaji [ˈlan.na.ka.ʐi]) zdanie}

\dictword{lanomaŕida}[la.nɔ.ma.ˈri.da]
\dictterm{n}{(\textsc{pl} lanomaŕidi [la.nɔ.ma.ˈri.di]) (\textsc{fem}) biblioteka}

\dictword{larima}[ˈla.ri.ma]
\dictterm{n}{(\textsc{pl} larimji [ˈla.rim.ʐi]) (\textsc{fem}) droga}

\dictword{larka}[ˈlar.ka]
\dictterm{n}{(\textsc{pl} larkeji [ˈlar.kɛ.ʐi]) (\textsc{fem}) mewa}

\dictword{layko}[ˈlaj.kɔ]
\dictterm{part}{jako; \emph{fari layko} -- pracować jako}

\dictword{layni}[ˈlaj.ni]
\dictterm{v}{(\textsc{pst} laynit [ˈlaj.nit]) sprawiać przyjemność}

\dictword{layneya}[ˈlaj.nɛ.ja]
\dictterm{n}{(\textsc{pl} layniji [ˈlaj.nit]) (\textsc{fem}) przyjemność}

\dictword{leknulgura}[ˈlɛk.nul.gu.ra]
\dictterm{n}{(\textsc{pl} leknulgureji [ˈlɛk.nul.gu.rɛ.ʐi]) (\textsc{fem}) restauracja}

\dictword{lekrok}[ˈlɛk.rɔk]
\dictterm{n}{(\textsc{pl} lekrokji [ˈlɛk.rɔk.ʐi]) toaleta (miejsce)}

\dictword{legeni}[ˈlɛ.gɛ.ni]
\dictterm{v}{(\textsc{pst} legent [ˈlɛ.gɛŋt]) gryźć}

\dictword{lehu}[ˈlɛ.xu]
\dictterm{part}{wcale}

\dictword{lejekli}[ˈlɛ.ʐɛk.li]
\dictterm{v}{(\textsc{pst} lejeklit [ˈlɛ.ʐɛk.lit]) kontynuować}

\dictword{lese}[ˈlɛ.sɛ]
\dictterm{n}{(\textsc{pl} leseji) (\textsc{fem} lesa) owca, jagnię, baran}

\dictword{leseyirome}[ˈlɛ.sɛ.ʏ.rɔ.mɛ]
\dictterm{n}{(\textsc{pl} leseyiromes) jagnięcina}

\dictword{levi}[ˈlɛ.vi]
\dictterm{part}{z lewej, lewa strona}

\dictword{levisi}[ˈlɛ.vi.si]
\dictterm{v}{(\textsc{pst} leviset [ˈlɛ.vi.sɛt]) tracić, odejmować}

\dictword{levo}[ˈlɛ.vɔ]
\dictterm{adj}{(\textsc{comp} levo͞e [ˈlɛ.vɔɛ], \textsc{supl} levo͞em [ˈlɛ.vɔɛm]) lewy}

\dictword{leyfe}[ˈlɛj.fɛ]
\dictterm{part}{albo}

\dictword{liche}[ˈli.t͡ʂɛ]
\dictterm{n}{(\textsc{pl} liches [ˈli.t͡ʂɛs]) prawo (w sądownictwie)}

\dictword{liina}[ˈli:.na]
\dictterm{n}{(\textsc{pl} liínji [li.ˈin.ʐi]) (\textsc{fem}) fala}

\dictword{lipe}[ˈli.pɛ]
\dictterm{adj}{(\textsc{comp} lipe͞a [ˈli.pɛa], \textsc{supl} lipe͞am [ˈli.pɛam]) dobry, lepszy, najlepszy}
\note{\emph{lipe kay}, \emph{lipe tay}, \emph{lipe zevoyim} to typowe pozdrowienia
i przywitania -- dobrej nocy, dobrego dnia, dobrego poranka.}

\dictword{lipet}[ˈli.pɛt]
\dictterm{n}{(\textsc{pl} lipetos [ˈli.pɛ.tɔs]) dobro}

\dictword{lipi}[ˈli.pi]
\dictterm{v}{(\textsc{pst} lipet [ˈli.pɛt]) ulepszać}

\dictword{lisavi}[ˈli.sa.vi]
\dictterm{n}{radio}

\dictword{liteta}[ˈli.tɛ.ta]
\dictterm{n}{(\textsc{pl} litetji [ˈli.tɛt.ʐi]) (\textsc{fem}) noga}

\dictword{live}[ˈli.vɛ]
\dictterm{n}{przód}

\dictword{live}[ˈli.vɛ]
\dictterm{part}{naprzód, do przodu}
\note{Używane również jako okrzyk zachęcający, także w wojsku.}

\dictword{loder}[ˈlɔ.dɛr]
\dictterm{adj}{(\textsc{comp} lodere [ˈlɔ.dɛ.rɛ], \textsc{supl} loderem [ˈlɔ.dɛ.rɛm]) oszczędny}

\dictword{loesa}[ˈlɔ.ɛ.sa]
\dictterm{n}{(\textsc{pl} loeseji [ˈlɔ.ɛ.sɛ.ʐi]) (\textsc{fem}) lalka}

\dictword{lohuin}[ˈlɔ.xu.in]
\dictterm{n}{(\textsc{pl} lohuinos [ˈlɔ.xu.i.nɔs]) kult, wiara}

\dictword{lolaru}[ˈlɔ.la.ru]
\dictterm{adj}{optymistyczny}

\dictword{lont}[ˈlɔŋt]
\dictterm{n}{(\textsc{pl} lontos [ˈlɔn.tɔs]) odległość}

\dictword{lonte}[ˈlɔn.tɛ]
\dictterm{adj}{(\textsc{comp} lote͞a [ˈlɔ.tɛa], \textsc{supl} lontem [ˈlɔn.tɛm]) daleki, odległy}

\dictword{lontoyner}[ˈlɔŋt.ɔj.nɛr]
\dictterm{n}{(\textsc{pl} lontoyneros [ˈlɔn.tɔj.nɛ.rɔs]) telefon}
\note{Dosłownie: odległość-głos.}

\dictword{lot}[ˈlɔt]
\dictterm{n}{życie}

\dictword{loti}[ˈlɔ.ti]
\dictterm{v}{(\textsc{pst} lot [ˈlɔt]) żyć, mieszkać}

\dictword{lonroki}[ˈlɔn.rɔ.ki]
\dictterm{v}{(\textsc{pst} lonrokit [ˈlɔn.rɔ.kit]) uprawiać (czynność); \emph{diras lonroki} - uprawiać seks}

\end{multicols}
\newpage
\section{M}
\begin{multicols}{2}

\dictword{ma}[ˈma]
\dictterm{n}{siedem, siódemka}

\dictword{machagi}[ˈma.t͡ʂa.gi]
\dictterm{v}{(\textsc{pst} machaget [ˈma.t͡ʂa.gɛt]) spać}

\dictword{machi}[ˈma.t͡ʂi]
\dictterm{v}{(\textsc{pst} machit [ˈma.t͡ʂit]) pchać}

\dictword{maesli}[ˈma.ɛs.li]
\dictterm{v}{(\textsc{pst} maeslit [ˈma.ɛs.lit]) życzyć}

\dictword{maesto}[ˈma.ɛs.tɔ]
\dictterm{n}{(\textsc{pl} maestos [ˈma.ɛs.tɔs]) życzenia}

\dictword{mahetan}[ˈma.xɛ.tan]
\dictterm{n}{(\textsc{pl} mahetanos [ˈma.xɛ.ta.nɔs]) pociąg (pojazd)}

\dictword{mahi}[ˈma.xi]
\dictterm{v}{(\textsc{pst} mahet [ˈma.xɛt]) ciągnąć}

\dictword{mal}[ˈmal]
\dictterm{part}{partykuła określająca przynależność do określonej rodziny}
\note{W~tradycyjnej formie zapisu nazwiska zawsze poprzedza się nazwę rodu 
partykułą \emph{mal}, na przykład \emph{Koolder mal Erlehirni}.}

\dictword{maldora}[ˈmal.dɔ.ra]
\dictterm{n}{(\textsc{pl} maldoros [ˈmal.dɔ.rɔs]) (\textsc{fem}) rodzina}

\dictword{maldoran}[ˈmal.dɔ.ran]
\dictterm{n}{(\textsc{pl} maldoranos [ˈmal.dɔ.ra.nɔs]) sąd, rozprawa sądowa}
\note{Określenie to pochodzi od faktu, że przed wiekami sprawy sądowe były rozpatrywane
wewnątrz rodzin, które to potem przedstawiały jaki wyrok został podjęty.}

\dictword{maldoro}[ˈmal.dɔ.rɔ]
\dictterm{adj}{rodzinny}

\dictword{maltorn}[ˈmal.tɔrn]
\dictterm{n}{dom rodzinny}
\note{Słowo \emph{maltorn} pochodzi od słów \emph{mal} (rodzina) oraz \emph{torn}, który 
jest wrostkiem często oznaczającym coś związanego z~murami, twierdzami -- aczkolwiek
samo \emph{torn} pochodzi od \emph{tori} (rzucać), ale jest często stosowane
w nazwach miast.}

\dictword{mama}[ˈma.ma]
\dictterm{n}{(\textsc{pl} mamis [ˈma.mis]) (\textsc{fem}) matka}
\note{Nieformalne. Patrz \emph{natali͞a}.}

\dictword{mamer}[ˈma.mɛr]
\dictterm{n}{(\textsc{pl} mamerji [ˈma.mɛr.ʐi]) nosiciel, transport}

\dictword{mami}[ˈma.mi]
\dictterm{v}{(\textsc{pst} mamet [ˈma.mɛt]) nosić}

\dictword{mamurte}[ˈma.mur.tɛ]
\dictterm{n}{(\textsc{pl} mamurtos [ˈma.mur.tɔs]) port morski lub śródlądowy}

\dictword{mani}[ˈma.ni]
\dictterm{v}{(\textsc{pst} mant [ˈmaŋt]) gubić}

\dictword{mantreja}[ˈman.trɛ.ʐa]
\dictterm{n}{(\textsc{pl} mantreji [ˈman.trɛ.ʐi]) (\textsc{fem}) lista}

\dictword{mapar}[ˈma.par]
\dictterm{part}{żaden}

\dictword{mar}[ˈmar]
\dictterm{n}{mąż}

\dictword{maŕida}[ma.ˈri.da]
\dictterm{n}{(\textsc{pl} maŕidi [ma.ˈri.di]) (\textsc{fem}) ziemia (jako obszar posiadany),
miejsce, ląd}

\dictword{maŕiello}[ma.ˈri.ɛl.lɔ]
\dictterm{n}{(\textsc{pl} mariéllos [ma.ri.ˈɛl.lɔs]) małżeństwo}

\dictword{maŕie͞o}[ma.ˈri.ɛɔ]
\dictterm{n}{(\textsc{pl} marié͞os [ma.ri.ˈɛɔs]) (\textsc{fem} marié͞a [ma.ri.ˈɛa]) małżonek, małżonka}

\dictword{maritari}[ˈma.ri.ta.ri]
\dictterm{v}{(\textsc{pst} maritat [ˈma.ri.tat]) żenić}

\dictword{marri}[ˈmar.ri]
\dictterm{v}{(\textsc{pst} maret [ˈma.rɛt]) dmuchać, wiać (w odniesieniu do wiatru)}

\dictword{maruni}[ˈma.ru.ni]
\dictterm{v}{(\textsc{pst} marunit [ˈma.ru.nit]) nucić}

\dictword{masuti}[ˈma.su.ti]
\dictterm{v}{(\textsc{pst} masut [ˈma.sut]) potrzebować}

\dictword{mati}[ˈma.ti]
\dictterm{adj}{siódmy}

\dictword{mede}[ˈmɛ.dɛ]
\dictterm{adj}{środkowy, leżący pomiędzy}

\dictword{medi}[ˈmɛ.di]
\dictterm{n}{(\textsc{pl} medis [ˈmɛ.dis]) środek}

\dictword{mejeri}[ˈmɛ.ʐɛ.ri]
\dictterm{v}{(\textsc{pst} mejer [ˈmɛ.ʐɛr]) niszczyć; \emph{okupao mejeri} -- pozbywać się problemu}

\dictword{menate}[ˈmɛ.na.tɛ]
\dictterm{adj}{(\textsc{comp} menate͞a [ˈmɛ.na.tɛa], \textsc{supl} menate͞am [ˈmɛ.na.tɛam]) żyzny}

\dictword{menede}[ˈmɛ.nɛ.dɛ]
\dictterm{adj}{(\textsc{comp} menede͞a [ˈmɛ.nɛ.dɛa], \textsc{supl} menede͞am [ˈmɛ.nɛ.dɛam]) zdrowy}
\note{Spotykana jest forma również forma \emph{menedo}.}

\dictword{menede}[ˈmɛ.nɛ.dɛ]
\dictterm{n}{zdrowie}

\dictword{meneder}[ˈmɛ.nɛ.dɛr]
\dictterm{n}{(\textsc{pl} menederos [ˈmɛ.nɛ.dɛ.rɔs]) (\textsc{fem} menedera [ˈmɛ.nɛ.dɛ.ra]) lekarz}

\dictword{menedi}[ˈmɛ.nɛ.di]
\dictterm{v}{(\textsc{pst} menet [ˈmɛ.nɛt]) leczyć, zdrowieć}

\dictword{menoje}[ˈmɛ.nɔ.jɛ]
\dictterm{adj}{(\textsc{comp} menoje͞a [ˈmɛ.nɔ.ʐɛa], \textsc{supl} menoje͞am [ˈmɛ.nɔ.ʐɛam]) tani}

\dictword{meter}[ˈmɛ.tɛr]
\dictterm{n}{(\textsc{pl} meterdi [ˈmɛ.tɛr.di]) metr (ziemska jednostka miary)}

\dictword{mi}[ˈmi]
\dictterm{pro}{ja}

\dictword{miam}[ˈmi.am]
\dictterm{part}{gdyby}

\dictword{mibali}[ˈmi.ba.li]
\dictterm{v}{(\textsc{pst} mibat [ˈmi.bat]) używać}

\dictword{mibozor}[ˈmi.bɔ.zɔr]
\dictterm{n}{(\textsc{pl} mibozoris [ˈmi.bɔ.zɔ.ris]) (\textsc{fem} mibozora [ˈmi.bɔ.zɔ.ra]) szkoła}
\note{W rodzaju żeńskim jest używana dla szkół żeńskich.}

\dictword{mife͞a}[ˈmi.fɛa]
\dictterm{n}{(\textsc{pl} mife͞aji [ˈmi.fɛa.ʐi]) (\textsc{fem}) fraza, zwrot}

\dictword{mika}[ˈmi.ka]
\dictterm{n}{(\textsc{pl} mikji [ˈmik.ʐi]) (\textsc{fem}) lód}

\dictword{miker}[ˈmi.kɛr]
\dictterm{n}{śnieg}

\dictword{miki}[ˈmi.ki]
\dictterm{v}{(\textsc{pst} mik [ˈmik]) zamarzać, mrozić, zamrażać, chłodzić}

\dictword{miki}[ˈmi.ki]
\dictterm{adj}{(\textsc{comp} mikia [ˈmi.ki.a], \textsc{supl} mikiam [ˈmi.ki.am]) zimny}

\dictword{mikyimarida}[ˈmi.kʏ.ma.ri.da]
\dictterm{n}{(\textsc{fem}) zima}
\note{Dosłownie: kraina lodu.}

\dictword{mireisro}[ˈmi.rɛ.is.rɔ]
\dictterm{adj}{(\textsc{comp} miroisre͞a [ˈmi.rɔ.is.rɛa], \textsc{supl} miroisre͞am [ˈmi.rɔ.is.rɛam]) prawdopodobny}
\note{Może być stosowany na końcu zdania jako partykuła ,,prawdopodobnie''.}

\dictword{miruja}[ˈmi.ru.ʐa]
\dictterm{n}{(\textsc{pl} miruji [ˈmi.ru.ʐi]) (\textsc{fem}) żona}

\dictword{missani}[ˈmis.sa.ni]
\dictterm{n}{(\textsc{pl} missanis [ˈmis.sa.nis]) wiadomość}

\dictword{missi}[ˈmis.si]
\dictterm{v}{kontaktować się}

\dictword{mo}[ˈmɔ]
\dictterm{part}{nic, nikt}

\dictword{moḱey}[mɔ.ˈkɛj]
\dictterm{part}{nijak}

\dictword{moḱoy}[mɔ.ˈkɔj]
\dictterm{part}{jakikolwiek}

\dictword{moĺi}[mɔ.ˈli]
\dictterm{part}{nigdy}

\dictword{moroza}[ˈmɔ.rɔ.za]
\dictterm{n}{(\textsc{pl} moroji [ˈmɔ.rɔ.ʐi]) (\textsc{fem}) piłka}

\dictword{mosno}[ˈmɔs.nɔ]
\dictterm{adj}{(\textsc{comp} mosne [ˈmɔs.nɛ], \textsc{supl} mosne͞am [ˈmɔs.nɛam]) dziki}

\dictword{mośo}[mɔ.ˈsɔ]
\dictterm{part}{nigdzie}

\dictword{mosti}[ˈmɔs.ti]
\dictterm{v}{(\textsc{pst} moset [ˈmɔ.sɛt]) tworzyć}

\dictword{mot́er}[mɔ.ˈtɛr]
\dictterm{part}{nikt}

\dictword{moýage}[mɔ.ˈja.gɛ]
\dictterm{part}{donikąd}

\dictword{moýasu}[mɔ.ˈja.su]
\dictterm{part}{znikąd}

\dictword{moyi}[ˈmɔ.ʏ]
\dictterm{part}{niczyj}

\dictword{muche}[ˈmu.t͡ʂɛ]
\dictterm{n}{(\textsc{pl} mucho [ˈmu.t͡ʂɔ]) (\textsc{fem} mucha [ˈmu.t͡ʂa]) kot, kotka}

\dictword{muchi}[ˈmu.t͡ʂi]
\dictterm{n}{(\textsc{pl} muchyo [ˈmu.t͡ʂjɔ]) (\textsc{fem} muchya [ˈmu.t͡ʂja]) kotek, kociaczek}
\note{Zawołanie \emph{mućhi, mućhi} oznacza mniej więcej ,,kici, kici''.}

\dictword{muner}[ˈmu.nɛr]
\dictterm{n}{(\textsc{pl} muneris [ˈmu.nɛ.ris]) urząd, stanowisko; zarówno jako urząd prezydenta \emph{agaryierchigu muner}, jak i urząd pocztowy \emph{yurchero muner}}

\dictword{musa}[ˈmu.sa]
\dictterm{n}{(\textsc{pl} musi [ˈmu.si]) mysz}

\dictword{myi}[ˈmʏ]
\dictterm{pro}{mój (\textsc{1SG.GEN} lub \textsc{1SG.POSS})}
\note{Zaimki myi oraz tyi zazwyczaj zapisywane są w~postaci ideogramu, czasami nawet w~tekście zapisanym alfabetem fonetycznym.}

\end{multicols}
\newpage
\section{N}
\begin{multicols}{2}

\dictword{na}[ˈna]
\dictterm{n}{pięć, piątka}

\dictword{naári}[na.ˈa.ri]
\dictterm{part}{wkrótce, niedługo}

\dictword{na͞epal}[ˈnaɛ.pal]
\dictterm{n}{gładkość}

\dictword{na͞epi}[ˈnaɛ.pi]
\dictterm{v}{(\textsc{pst} na͞et [ˈnaɛt]) gładzić, wygładzać}

\dictword{na͞epo}[ˈnaɛ.pɔ]
\dictterm{adj}{(\textsc{comp} na͞epo͞a [ˈnaɛ.pɔa], \textsc{supl} na͞epo͞am [ˈnaɛ.pɔam]) gładki, wygładzony}

\dictword{nafit}[ˈna.fit]
\dictterm{n}{(\textsc{pl} nafito [ˈna.fi.tɔ]) piwnica}

\dictword{nafiye}[ˈna.fi.ja]
\dictterm{n}{(\textsc{pl} nafiyes [ˈna.fi.jɛs]) (\textsc{fem} nafiya [ˈna.fi.ja]) piwo}
\note{W niektórych dialektach nie występuje w~rodzaju męskim.}

\dictword{naja}[ˈna.ʐa]
\dictterm{n}{(\textsc{pl} naji [ˈna.ʐi]) (\textsc{fem}) odpowiednik godziny, ok. 75 minut}

\dictword{nanal}[ˈna.nal]
\dictterm{n}{(\textsc{pl} nanalos [ˈna.na.lɔs]) śmigło}

\dictword{narli}[ˈnar.li]
\dictterm{v}{(\textsc{pst} narlet [ˈnar.lɛt]) ściskać}

\dictword{naroje}[ˈna.rɔ.ʐɛ]
\dictterm{n}{(\textsc{pl} narojes [ˈna.rɔ.ʐɛs]) (\textsc{fem} naroja [ˈna.rɔ.ʐa]) ciało}

\dictword{naruo}[ˈna.ru.ɔ]
\dictterm{n}{(\textsc{pl} naruos [ˈna.ru.ɔs]) brzuch}

\dictword{nasecho}[ˈna.sɛ.t͡ʂɔ]
\dictterm{n}{(\textsc{pl} nasechos [ˈna.sɛ.t͡ʂɔs]) szpital}

\dictword{natale}[ˈna.ta.lɛ]
\dictterm{n}{narodziny}

\dictword{natali}[ˈna.ta.li]
\dictterm{v}{(\textsc{pst} natal [ˈna.tal]) rodzić}

\dictword{natali͞a}[ˈna.ta.lia]
\dictterm{n}{(\textsc{pl} natalji [ˈna.tal.ʐi]) (\textsc{fem}) matka}

\dictword{nati}[ˈna.ti]
\dictterm{n}{piąty}

\dictword{navicha}[ˈna.vi.t͡ʂa]
\dictterm{n}{(\textsc{fem}) próba}

\dictword{ne͞a}[ˈnɛa]
\dictterm{part}{obok, blisko}

\dictword{nebesi}[ˈnɛ.bɛ.si]
\dictterm{n}{(\textsc{pl} nebesis [ˈnɛ.bɛ.sis]) (\textsc{fem} nebesa [ˈnɛ.bɛ.sa]) policjant}
\note{Pochodzi od skrótu \emph{NBS} (\emph{Noworu͞a Buruku chu Sekupao} -- Cywilne (niewojskowe) Biuro Bezpieczeństwa), który oznacza federalne jednostki policyjne. Rzeczownik powstały od tej nazwy przylgnął jednak do oficerów wszystkich sił policyjnych.}

\dictword{neb́i͞a}[nɛ.ˈbia]
\dictterm{n}{(\textsc{fem}) mgła}

\dictword{ner}[ˈnɛr]
\dictterm{adj}{(\textsc{comp} nera [ˈnɛ.ra], \textsc{supl} neram [ˈnɛ.ram]) bliski}

\dictword{niger}[ˈni.gɛr]
\dictterm{pro}{tamto, tamta rzecz}
\note{Używany rzadko, zwykle w dialekcie południowo-wschodnim. Raczej używane jest \emph{je}, \emph{niger} dotyczy
rzeczy, które nie są w zasięgu wzroku.}

\dictword{nihitu}[ˈni.xi.tu]
\dictterm{n}{(\textsc{pl} nihitus [ˈni.xi.tus]) strefa, obszar}

\dictword{nimu}[ˈni.mu]
\dictterm{n}{magia, czary}

\dictword{nimu͞er}[ˈni.mu.ɛr]
\dictterm{n}{(\textsc{pl} nimu͞erji [ˈni.muɛr.ʐi]) (\textsc{fem} nimu͞e [ˈni.muɛ]) czarodziej, czarodziejka}
\note{Istnieje również słowo \emph{nimu͞era}, oznaczające wiedźmę, używane stricte pejoratywnie. Nie istnieje pejoratywne określenie mężczyzny czarodzieja, raczej zamiast tego używa się określenia \emph{rukolaner}, dosłownie ,,czarnoksiężnik'', które jednak powstało w~czasach znacznie późniejszych niż \emph{nimu͞era}.}

\dictword{nimu͞era}[ˈni.muɛ.ra]
\dictred{nimu͞er}

\dictword{nimui}[ˈni.mu.i]
\dictterm{v}{czarować}

\dictword{nirudo}[ˈni.ru.dɔ]
\dictterm{n}{(\textsc{pl} nirudos [ˈni.ru.dɔs]) punkt}

\dictword{niyi}[ˈni.ʏ]
\dictterm{pro}{nasze (\textsc{1PL.GEN} lub \textsc{1PL.POSS})}

\dictword{no}[ˈnɔ]
\dictterm{part}{nie}

\dictword{nodi}[ˈnɔ.di]
\dictterm{pro}{my ekskluzywne}
\note{Oznacza ``my, ale nie włącznie z~tobą''. Spotykany najczęściej tylko 
w dialekcie pustynnym.}

\dictword{nodowo}[ˈnɔ.dɔ.wɔ]
\dictterm{adj}{pusty}

\dictword{nodyi}[ˈnɔ.dʏ]
\dictterm{pro}{nasze (ekskluzywne)}

\dictword{noḱarlay}[nɔ.ˈkar.laj]
\dictterm{adj}{taki, którzy nie może umrzeć, nieśmiertelny}

\dictword{noḱarlaj}[nɔ.kar.ˈlaj]
\dictterm{n}{(\textsc{pl} noḱarji [nɔ.ˈkar.ʐi]) (\textsc{fem} noḱarla [nɔ.ˈkar.la]) nieśmiertelny człowiek}

\dictword{nokihilo}[ˈnɔ.ki.xi.lɔ]
\dictterm{adj}{niemożliwy}

\dictword{nolipe}[ˈnɔ.li.pɛ]
\dictterm{adj}{zły, niedobry}

\dictword{nome}[ˈnɔ.mɛ]
\dictterm{n}{(\textsc{pl} nomis [ˈnɔ.mis]) imię}

\dictword{nomi}[ˈnɔ.mi]
\dictterm{v}{(\textsc{pst} nomet [ˈnɔ.mɛt]) nazywać, nazywać się, być nazywanym}

\dictword{noni}[ˈnɔ.ni]
\dictterm{pro}{my}

\dictword{noneper}[ˈnɔ.nɛ.pɛr]
\dictterm{adj}{(\textsc{comp} nonepere [ˈnɔ.nɛ.pɛ.rɛ], \textsc{supl} noneperem [ˈnɔ.nɛ.pɛ.rɛm]) suchy}

\dictword{noril}[ˈnɔ.ril]
\dictterm{adj}{(\textsc{comp} norile͞a [ˈnɔ.ri.lɛa], \textsc{supl} norile͞am [ˈnɔ.ri.lɛam]) krótki}

\dictword{nosaper}[ˈnɔ.sa.pɛr]
\dictterm{adj}{nieznany}

\dictword{notaerno}[ˈnɔ.ta.ɛr.nɔ]
\dictterm{adj}{niezadowolony}

\dictword{notopai}[ˈnɔ.tɔ.pa.i]
\dictterm{adj}{nieskończony}

\dictword{novek}[ˈnɔ.vɛk]
\dictterm{n}{młodość, nowość}

\dictword{noveka}[ˈnɔ.vɛ.ka]
\dictterm{n}{(\textsc{fem}) wiosna}

\dictword{nowa}[ˈnɔ.vɛ.ka]
\dictterm{n}{(\textsc{pl} nojewas [ˈnɔ.ʐɛ.was]) (\textsc{fem}) uroda}

\dictword{nowirklo}[ˈnɔ.wir.klɔ]
\dictterm{adj}{(\textsc{comp} nowirkle [ˈnɔ.wir.klɛ], \textsc{supl} nowirklem [ˈnɔ.wir.klɛm]) nieważny, nieistotny}

\dictword{noyeji}[ˈnɔ.jɛ.ʐi]
\dictterm{v}{(\textsc{pst} noyejit [ˈnɔ.jɛ.ʐit]) oślepiać, tracić wzrok, ślepnąć}
\note{\emph{mi noyji} -- ja ślepnę, \emph{mi noyji ti} -- ja ciebie oślepiam}

\dictword{noworu͞a}[ˈnɔ.wɔ.rua]
\dictterm{adj}{cywilny, niewojskowy, pokojowy}

\dictword{nozewo}[ˈnɔ.zɛ.wɔ]
\dictterm{adj}{(\textsc{comp} nozewe [ˈnɔ.zɛ.wɛ], \textsc{supl} nozewem [ˈnɔ.zɛ.wɛm]) chudy}

\dictword{nu͞er}[ˈnuɛr]
\dictterm{adj}{(\textsc{comp} nu͞ere [ˈnuɛ.rɛ], \textsc{supl} nuerem [ˈnu.ɛ.rɛm]) nowy}

\dictword{nu͞eyaŕitay}[nuɛ.ja.ˈri.taj]
\dictterm{n}{dzień nowego roku}
\note{Dosłownie: Nowy-Rok-Dzień. Święto obchodzone 1. dnia miesiąca novek.}

\dictword{nurachi}[ˈnu.ra.t͡ʂi]
\dictterm{v}{(\textsc{pst} nurachit [ˈnu.ra.t͡ʂit]) spodziewać}

\end{multicols}
\newpage
\section{O}
\begin{multicols}{2}

\dictword{o}[ˈɔ]
\dictterm{part}{do, dla, niż (przy porównywaniu)}

\dictword{obi}[ˈɔ.bi]
\dictterm{v}{(\textsc{pst} obit [ˈɔ.bit]) zaczynać}

\dictword{obo}[ˈɔ.bɔ]
\dictterm{n}{(\textsc{pl} obos [ˈɔ.bɔs]) początek}

\dictword{ofari}[ˈɔ.fa.ri]
\dictterm{v}{(\textsc{pst} ofarit [ˈɔ.fa.rit]) zaczynać (coś robić)}

\dictword{okwat}[ˈɔk.wat]
\dictterm{n}{(\textsc{pl} okwatos) (\textsc{fem} okwata) wieprz, świnia}

\dictword{okwatyirome}[ˈɔk.wa.tʏ.rɔ.mɛ]
\dictterm{n}{(\textsc{pl} okwatyiromes) wieprzowina}

\dictword{okupao}[ˈɔ.ku.pa.ɔ]
\dictterm{n}{(\textsc{pl} okupaos [ˈɔ.ku.pa.ɔs]) problem}

\dictword{on}[ˈɔn]
\dictterm{part}{na (czymś), na temat, o (kimś, czymś), na (coś)}

\dictword{onvojibo}[ˈɔn.vɔ.ʐi.bɔ]
\dictterm{part}{z naprzeciw, naprzeciwko}

\dictword{opeka}[ˈɔ.pɛ.ka]
\dictterm{n}{(\textsc{pl} opekaji [ˈɔ.pɛ.ka.ʐi]) (\textsc{fem}) wyspa}

\dictword{oraićhar}[ˈɔ.ra.i.t͡ʂar]
\dictterm{n}{dar bogów, błogosławieństwo}

\dictword{ore}[ˈɔ.rɛ]
\dictterm{part}{teraz}

\dictword{oreno}[ˈɔ.rɛ.nɔ]
\dictterm{adj}{pomarańczowy}

\dictword{oreno}[ˈɔ.rɛ.nɔ]
\dictterm{n}{kolor pomarańczowy}

\dictword{ori}[ˈɔ.ri]
\dictterm{n}{(\textsc{pl} ora͞i [ˈɔ.rai]) (\textsc{fem} oriá [ɔ.ri.ˈa]) bóg, bogini}

\dictword{osake}[ˈɔ.sa.kɛ]
\dictterm{part}{wiele, wielu, dużo}
\note{Stosowane zamiennie z \emph{vyele}.}

\dictword{osfuta}[ˈɔs.fu.ta]
\dictterm{n}{(\textsc{pl} osfutiji [ˈɔs.fu.ti.ʐi]) (\textsc{fem}) rakieta}

\dictword{osik}[ˈɔ.sik]
\dictterm{n}{(\textsc{pl} osiki [ˈɔ.si.ki]) nos}

\dictword{ośiki}[ɔ.ˈsi.ki]
\dictterm{v}{(\textsc{pst} ośik [ɔ.ˈsik]) czuć węchem, wąchać}

\dictword{osor}[ˈɔs.ɔr]
\dictterm{part}{dlaczego}

\dictword{ossa}[ˈɔs.sa]
\dictterm{n}{(\textsc{pl} osji [ˈɔs.ʐi]) (\textsc{fem}) kość}

\dictword{ostro}[ˈɔs.trɔ]
\dictterm{n}{drzwi}

\dictword{ostrolo}[ˈɔs.trɔ.lɔ]
\dictterm{n}{(\textsc{pl} ostrolos [ˈɔs.trɔ.lɔs]) furtka, drzwiczki}

\dictword{osupi}[ˈɔ.su.pi]
\dictterm{v}{(\textsc{pst} osupet [ˈɔ.su.pɛt]) upadać, upaść, padać (deszcz)}

\dictword{owa}[ˈɔ.wa]
\dictterm{part}{och!}

\dictword{oyo}[ˈɔ.jɔ]
\dictterm{n}{(\textsc{pl} ojios [ˈɔ.ʐi.ɔs]) oko}

\dictword{oyosenaja}[ˈɔ.jɔ.sɛ.na.ʐa]
\dictterm{n}{(\textsc{pl} oyosenaji [ˈɔ.jɔ.sɛ.na.ʐi]) (\textsc{fem}) okulary}

\dictword{oyotori}[ˈɔ.jɔ.tɔ.ri]
\dictterm{v}{(\textsc{pst} oyotorit [ˈɔ.jɔ.tɔ.rit]) rzucać spojrzenia, podglądać, podejrzewać}

\dictword{ozoi}[ˈɔ.zɔ.i]
\dictterm{v}{(\textsc{pst} ozot [ˈɔ.zɔt]) trzymać}

\dictword{ozeyo}[ˈɔ.zɛ.jɔ]
\dictterm{adj}{(\textsc{comp} ozeye [ˈɔ.zɛ.jɛ], \textsc{supl} ozeyem [ˈɔ.zɛ.jɛm]) trudny, trudniejszy, najtrudniejszy}

\end{multicols}
\newpage
\section{P}
\begin{multicols}{2}

\dictword{pajako}[ˈpa.ʐa.kɔ]
\dictterm{n}{zapłata, pensja}

\dictword{pakopi}[ˈpa.kɔ.pi]
\dictterm{v}{(\textsc{pst} pakot [ˈpa.kɔt]) martwić}

\dictword{pakopo}[ˈpa.kɔ.pɔ]
\dictterm{adj}{zmartwiony}

\dictword{palimi}[ˈpa.li.mi]
\dictterm{v}{(\textsc{pst} palimit [ˈpa.li.mit]) rozmawiać}

\dictword{paluk}[ˈpa.luk]
\dictterm{adj}{taki sam, identyczny}

\dictword{pama}[ˈpa.ma]
\dictterm{part}{wtedy}

\dictword{parlai}[ˈpar.la.i]
\dictred{inra͞i}

\dictword{patal}[ˈpa.tal]
\dictterm{n}{(\textsc{pl} patalos [ˈpa.ta.lɔs]) tata, tatuś}
\note{Używane zazwyczaj w~nieformalnym kontekście. Patrz \emph{vapal}.}

\dictword{pa͞uri}[ˈpau.ri]
\dictterm{v}{(\textsc{pst} paurit [ˈpa.u.rit]) bać}

\dictword{payte}[ˈpaj.tɛ.a]
\dictterm{adj}{(\textsc{comp} payte͞a [ˈpaj.tɛa], \textsc{supl} payteam [ˈpaj.tɛam]) męczący}

\dictword{payti}[ˈpaj.ti]
\dictterm{v}{(\textsc{pst} payet [ˈpa.jɛt]) męczyć}

\dictword{pazer}[ˈpa.zɛr]
\dictterm{adj}{(\textsc{comp} pazere [ˈpa.zɛ.rɛ], \textsc{supl} pazerem [ˈpa.zɛ.rɛm]) lubiany}

\dictword{pazero}[ˈpa.zɛ.rɔ]
\dictterm{adj}{ulubiony}

\dictword{pazi}[ˈpa.zi]
\dictterm{v}{(\textsc{pst} pazit [ˈpa.zit]) lubić}

\dictword{peayo}[ˈpɛ.a.jɔ]
\dictterm{n}{(\textsc{pl} peayos [ˈpɛ.a.jɔs]) stacja, przystanek}

\dictword{pebedar}[ˈpɛ.bɛ.dar]
\dictterm{n}{(\textsc{pl} pebedaros [ˈpɛ.bɛ.da.rɔs]) warzywo}

\dictword{pebedarer}[ˈpɛ.bɛ.da.rɛr]
\dictterm{n}{(\textsc{pl} pebedareros [ˈpɛ.bɛ.da.rɛ.rɔs]) (\textsc{fem} pebedarera [ˈpɛ.bɛ.da.rɛ.ra]) wegetarianin}

\dictword{pebedarero}[ˈpɛ.bɛ.da.rɛ.rɔ]
\dictterm{adj}{wegetariański}

\dictword{pelir}[ˈpɛ.lir]
\dictterm{n}{(\textsc{pl} peliros [ˈpɛ.li.rɔs]) (\textsc{fem} pelira [ˈpɛ.li.ra]) pies}

\dictword{pe͞oi}[ˈpɛɔ.i]
\dictterm{v}{(\textsc{pst} peot [ˈpɛ.ɔt]) słyszeć}

\dictword{per}[ˈpɛr]
\dictterm{part}{aby, żeby}

\dictword{perma}[ˈpɛr.ma]
\dictterm{n}{(\textsc{pl} permos [ˈpɛr.mɔs]) (\textsc{fem}) pozwolenie, przepustka}

\dictword{permi}[ˈpɛr.mi]
\dictterm{v}{(\textsc{pst} permet [ˈpɛr.mɛt]) pozwalać}

\dictword{pesulo}[ˈpɛ.su.lɔ]
\dictterm{adj}{różowy}

\dictword{pesulo}[ˈpɛ.su.lɔ]
\dictterm{n}{kolor różowy, róż (kosmetyk)}

\dictword{petode}[ˈpɛ.tɔ.dɛ]
\dictterm{n}{(\textsc{pl} petodes [ˈpɛ.tɔ.dɛs]) pojemnik}

\dictword{pevi}[ˈpɛ.vi]
\dictterm{v}{(\textsc{pst} pevit [ˈpɛ.vit]) karmić}

\dictword{pimer}[ˈpi.mɛr]
\dictterm{n}{(\textsc{pl} pimeros [ˈpi.mɛ.rɔs]) rzecz}

\dictword{pinchi}[ˈpin.t͡ʂi]
\dictterm{v}{(\textsc{pst} pinchet [ˈpin.t͡ʂɛt]) lśnić}

\dictword{pipe}[ˈpi.pɛ]
\dictterm{n}{(\textsc{pl} pipes [ˈpi.pɛs]) włos}

\dictword{pukes}[ˈpu.kɛs]
\dictterm{n}{(\textsc{pl} pukesos [ˈpu.kɛ.sɔs]) kolor}

\dictword{pukese}[ˈpu.kɛ.sɔ]
\dictterm{adj}{kolorowy}

\dictword{putari}[ˈpu.ta.ri]
\dictterm{v}{(\textsc{pst} putarit [ˈpu.ta.rit]) wyjaśniać}

\dictword{pyotka}[ˈpjɔt.ka]
\dictterm{n}{(\textsc{pl} pyotkaji [ˈpjɔt.ka.ʐi]) (\textsc{fem}) most}

\end{multicols}
\newpage
\section{R}
\begin{multicols}{2}

\dictword{ra}[ˈra]
\dictterm{n}{dziewięć, dziewiątka}

\dictword{rabei}[ˈra.bɛ.i]
\dictterm{v}{(\textsc{pst} rabet [ˈra.bɛt]) wierzyć}

\dictword{radi}[ˈra.di]
\dictterm{v}{(\textsc{pst} radit [ˈra.dit]) sprzedać, sprzedawać}

\dictword{ra͞e}[ˈraɛ]
\dictterm{n}{(\textsc{pl} ra͞os [ˈraɔs]) koło, krąg}

\dictword{rahioko}[ˈra.xi.ɔ.kɔ]
\dictterm{adj}{wyraźny}

\dictword{rajiri}[ˈra.ʐi.ri]
\dictterm{v}{(\textsc{pst} rajit [ˈra.ʐit]) krzyczeń}

\dictword{ranor}[ˈra.nɔr]
\dictterm{adj}{(\textsc{comp} ranore [ˈra.nɔ.rɛ], \textsc{supl} ranorem [ˈra.nɔ.rɛm]) poprawny}

\dictword{ranori}[ˈra.nɔ.ri]
\dictterm{v}{(\textsc{pst} ranoret [ˈra.nɔ.rɛt]) poprawiać}

\dictword{ranse}[ˈran.sɔ]
\dictterm{n}{(\textsc{pl} ransos [ˈran.sɔs]) rodzaj, gatunek}

\dictword{ra͞o}[ˈraɔ]
\dictterm{adj}{okrągły, dookoła}
\note{Używany również do określenia szacowanej wartości: ,,około''.}

\dictword{ra͞oi}[ˈraɔ.i]
\dictterm{v}{(\textsc{pst} raot [ˈra.ɔt]) zaokrąglać, okrążać}

\dictword{raratako}[ˈra.ra.ta.kɔ]
\dictterm{part}{wystarczy, dość}

\dictword{raratako}[ˈra.ra.ta.kɔ]
\dictterm{adj}{wystarczający}

\dictword{rati}[ˈra.ti]
\dictterm{adj}{dziewiąty}

\dictword{ray}[ˈraj]
\dictterm{n}{(\textsc{pl} raji [ˈra.ʐi]) (\textsc{fem}) piorun, błyskawica}

\dictword{razi}[ˈra.zi]
\dictterm{part}{określa wielokrotność, używane w~połączeniu z~liczebnikiem, na
przykład \emph{ka razi} -- dwukrotnie}

\dictword{recha}[ˈrɛ.t͡ʂa]
\dictterm{part}{wszyscy, wszystko, wszystkie}

\dictword{rechakwa}[ˈrɛ.t͡ʂak.wa]
\dictterm{n}{(\textsc{pl} rechakwaji [ˈrɛ.t͡ʂak.wa.ʐi]) (\textsc{fem}) ocean}

\dictword{rechaḿaresen}[rɛ.t͡ʂa.ˈma.rɛ.sɛn]
\dictterm{n}{Wszechświat}

\dictword{rede}[ˈrɛ.dɛ]
\dictterm{part}{ponownie, znów}

\dictword{redo͞a}[ˈrɛ.dɔa]
\dictterm{part}{kilka, niewielu, niewiele, kilku}

\dictword{regu}[ˈrɛ.gu]
\dictterm{adj}{(\textsc{comp} regu͞e [ˈrɛ.guɛ], \textsc{supl} regu͞em [ˈrɛ.guɛm]) ciemny}

\dictword{regu͞o}[ˈrɛ.guɔ]
\dictterm{n}{(\textsc{pl} reguos [ˈrɛ.gu.ɔs]) ciemność}

\dictword{reki}[ˈrɛ.ki]
\dictterm{v}{(\textsc{pst} rek [ˈrɛk]) uderzyć, uderzać}

\dictword{relita}[ˈrɛ.li.ta]
\dictterm{part}{zawsze}

\dictword{reso}[ˈrɛ.sɔ]
\dictterm{part}{wszędzie}

\dictword{ret}[ˈrɛt]
\dictterm{n}{(\textsc{pl} retos [ˈrɛ.tɔs]) sieć}

\dictword{retal}[ˈrɛ.tal]
\dictterm{n}{(\textsc{pl} retalji [ˈrɛ.tal.ʐi]) dach}

\dictword{reter}[ˈrɛt.ɛr]
\dictterm{n}{(\textsc{pl} reteros [ˈrɛ.tɛ.rɔs]) powrót}

\dictword{retto͞i}[ˈrɛt.tɔi]
\dictterm{v}{(\textsc{pst} rettot [ˈrɛt.tɔt]) wracać}

\dictword{rewapo}[ˈrɛ.wa.pɔ]
\dictterm{adj}{(\textsc{comp} rewape͞a [ˈrɛ.wa.pɛa], \textsc{supl} rewape͞am [ˈrɛ.wa.pɛam]) wąski}

\dictword{reylina}[ˈrɛj.li.na]
\dictterm{n}{(\textsc{fem}) jesień}

\dictword{rigati}[ˈri.ga.ti]
\dictterm{v}{dziękować; także po prostu ,,dziękuję''}

\dictword{rige}[ˈri.gɛ]
\dictterm{n}{(\textsc{pl} rigeos [ˈri.gɛ.ɔs]) (\textsc{fem} rigea [ˈri.gɛ.a]) władca, pan, lord}
\note{W rodzaju żeńskim słowo to uzyskuje końcówkę <ea>, jednak nie wymawia się
tego jako [ˈri.gɛa], a jako [ˈri.gɛ.a].}

\dictword{rigei}[ˈri.gɛ.i]
\dictterm{v}{rządzić, władać}

\dictword{rigeypa}[ˈri.gɛj.pa]
\dictterm{n}{(\textsc{fem}) władza, moc}

\dictword{riner}[ˈri.nɛr]
\dictterm{n}{(\textsc{pl} rineros [ˈri.nɛ.rɔs]) (\textsc{fem} rinera [ˈri.nɛ.ra]) posłaniec}

\dictword{riper}[ˈri.pɛr]
\dictterm{n}{(\textsc{pl} ripiji [ˈri.pi.ʐi]) (\textsc{fem}) rozpruwacz}

\dictword{ripi}[ˈri.pi]
\dictterm{v}{(\textsc{pst} ripet [ˈri.pɛt]) pruć, rozpruwać}

\dictword{ripijer}[ˈri.pi.ʐɛr]
\dictterm{n}{(\textsc{pl} ripijeros [ˈri.pi.ʐɛ.rɔs]) granat}

\dictword{rissi}[ˈris.si]
\dictterm{v}{(\textsc{pst} risset [ˈris.sɛt]) kopać}

\dictword{rivi}[ˈri.vi]
\dictterm{part}{z prawej, prawa strona}

\dictword{rivisi}[ˈri.vi.si]
\dictterm{v}{(\textsc{pst} rivisit [ˈri.vi.sit]) wzbogacać się, zyskiwać, dodawać}

\dictword{rizo}[ˈri.zɔ]
\dictterm{adj}{(\textsc{comp} rizo͞e [ˈri.zɔɛ], \textsc{supl} rizo͞em [ˈri.zɔɛm]) dokładny}

\dictword{rizu͞o}[ˈri.zuɔ]
\dictterm{part}{dokładnie tak, w~rzeczy samej}

\dictword{ro͞al}[ˈrɔal]
\dictterm{adj}{(\textsc{comp} ro͞ale [ˈrɔa.lɛ], \textsc{supl} ro͞alem [ˈrɔa.lɛm]) bogaty}

\dictword{rome}[ˈrɔ.mɛ]
\dictterm{n}{(\textsc{pl} romes [ˈrɔ.mɛs]) mięso}

\dictword{rori}[ˈrɔ.ri]
\dictterm{v}{(\textsc{pst} rorit [ˈrɔ.rit]) myśleć; \emph{epi rori} -- wydawać się}

\dictword{rosoi}[ˈrɔ.sɔ.i]
\dictterm{v}{(\textsc{pst} rosoit [ˈrɔ.sɔ.it]) znikać, zniknąć}

\dictword{rufaler}[ˈru.fa.lɛr]
\dictterm{n}{(\textsc{pl} rufaleros [ˈru.fa.lɛ.rɔs]) (\textsc{fem} rufalera [ˈru.fa.lɛ.ra]) rolnik}

\dictword{rujalar}[ˈru.ʐa.lar]
\dictterm{n}{(\textsc{pl} rujalaros [ˈru.ʐa.la.rɔs]) (\textsc{fem} rujalara [ˈru.ʐa.la.ra]) mężczyzna, kobieta, człowiek, ludzie}

\dictword{rukila}[ˈru.ki.la]
\dictterm{n}{(\textsc{pl} rukilaji [ˈru.ki.la.ʐi]) (\textsc{fem}) kosmetyk}

\dictword{ruko}[ˈru.kɔ]
\dictterm{adj}{czarny}

\dictword{ruko}[ˈru.kɔ]
\dictterm{n}{czerń}

\dictword{rukolaner}[ˈru.kɔ.la.nɛr]
\dictterm{n}{czarnoksieżnik, zły czarodziej, czarownik}
\note{Patrz również: \emph{nimu͞er}.}

\dictword{rumar}[ˈru.mar]
\dictterm{n}{(\textsc{pl} rumaros [ˈru.ma.rɔs]) (\textsc{fem} rumara [ˈru.ma.ra]) demon}

\dictword{ruyanso}[ˈru.jan.sɔ]
\dictterm{adj}{brutalny}

\end{multicols}
\newpage
\section{S}
\begin{multicols}{2}

\dictword{sa}[ˈsa]
\dictterm{n}{trzy, trójka}

\dictword{saboti}[ˈsa.bɔ.ti]
\dictterm{v}{(\textsc{pst} sabot [ˈsa.bɔt]) pokazywać, prezentować, przedstawiać}

\dictword{sacha}[ˈsa.t͡ʂa]
\dictterm{n}{(\textsc{fem}) prawda}

\dictword{sacho}[ˈsa.t͡ʂɔ]
\dictterm{adj}{prawdziwy}

\dictword{safayo}[ˈsa.fa.jɔ]
\dictterm{n}{(\textsc{pl} safajis [ˈsa.fa.ʐis]) szafir (kolor), szafir (kamień)}

\dictword{safayo}[ˈsa.fa.jɔ]
\dictterm{adj}{szafirowy}

\dictword{sage}[ˈsa.gɛ]
\dictred{saperyer}

\dictword{sagerya}[ˈsa.gɛ.rja]
\dictred{saperya}

\dictword{sainsi}[ˈsa.in.si]
\dictterm{v}{znać}

\dictword{saiperi}[ˈsa.i.pɛ.ri]
\dictterm{v}{(\textsc{pst} saiperit [ˈsa.i.pɛ.rit]) pamiętać}

\dictword{samicha}[ˈsa.mi.t͡ʂa]
\dictterm{n}{(\textsc{pl} samichoji [ˈsa.mi.t͡ʂɔ.ʐi]) (\textsc{fem}) ocena}

\dictword{samichi}[ˈsa.mi.t͡ʂi]
\dictterm{v}{(\textsc{pst} samichit [ˈsa.mi.t͡ʂit]) oceniać}

\dictword{saperi}[ˈsa.pɛ.ri]
\dictterm{v}{(\textsc{pst} saperit [ˈsa.pɛ.rit]) wiedzieć}

\dictword{saperya}[ˈsa.pɛr.ja]
\dictterm{n}{(\textsc{fem}) wiedza}
\note{W szczególności: wiedza tajemna.}
\note{Również: \emph{sagerya}.}

\dictword{saperyer}[ˈsa.pɛr.jɛr]
\dictterm{n}{(\textsc{pl} saperji [ˈsa.pɛr.ʐi]) (\textsc{fem} saperyera [ˈsa.pɛr.jɛ.ra]) mędrzec, wiedzący}
\note{Również spotykane są formy \emph{sage} lub \emph{zage}, zwłaszcza w~starszych tekstach.}

\dictword{sati}[ˈsa.ti]
\dictterm{adj}{trzeci}

\dictword{sa͞ućhit}[sau.ˈt͡ʂit]
\dictterm{n}{(\textsc{pl} sa͞ućhitos [sau.ˈt͡ʂi.tɔs]) paznokieć}

\dictword{sa͞uk}[ˈsauk]
\dictterm{adj}{(\textsc{comp} sauke [ˈsa.u.kɛ], \textsc{supl} saukem [ˈsa.u.kɛm]) smutny}

\dictword{say}[ˈsaj]
\dictterm{n}{(\textsc{pl} sajis [ˈsa.ʐis]) (\textsc{fem}) grzmot}

\dictword{saykar}[ˈsaj.kar]
\dictterm{n}{przekleństwo nie skierowane na określoną postać, w~stylu: a~niech to wszystko szlag trafi}

\dictword{seiti}[ˈsɛ.i.ti]
\dictterm{v}{czuć}

\dictword{Seja}[ˈsɛ.ʐa]
\dictterm{n}{(\textsc{fem}) gwiazda układu Maŕid}

\dictword{sejitramo}[ˈsɛ.ʐi.tra.mɔ]
\dictterm{n}{zachód słońca}

\dictword{seḱier}[sɛ.ˈki.ɛr]
\dictterm{n}{kwarc}

\dictword{sekot}[ˈsɛ.kɔt]
\dictterm{n}{(\textsc{pl} sekotos [ˈsɛ.kɔ.tɔs]) (\textsc{fem} sekota [ˈsɛ.kɔ.ta]) cień}

\dictword{sekupa}[ˈsɛ.ku.pa]
\dictterm{adj}{(\textsc{comp} sekupe [ˈsɛ.ku.pɛ], \textsc{supl} sekupem [ˈsɛ.ku.pɛm]) bezpieczny}

\dictword{sekupao}[ˈsɛ.ku.pa.ɔ]
\dictterm{n}{bezpieczeństwo}

\dictword{sepo}[ˈsɛ.pɔ]
\dictterm{adj}{(\textsc{comp} sepo͞e [ˈsɛ.pɔɛ], \textsc{supl} sepo͞em [ˈsɛ.pɔɛm]) wcześnie}

\dictword{seysei}[ˈsɛj.sɛ.i]
\dictterm{v}{(\textsc{pst} seyset [ˈsɛj.sɛt]) poczuć, odczuwać}
\note{Istnieje wyrażenie \emph{Seysei choke chu ti}, oznaczające dosłownie
,,dzięki tobie poczułem się miło'', używane jako odpowiednik ,,miło mi było ciebie
poznać''.}

\dictword{seysi}[ˈsɛj.si]
\dictterm{v}{(\textsc{pst} set [ˈsɛt]) siadać, siedzieć}

\dictword{siboti}[ˈsi.bɔ.ti]
\dictterm{v}{(\textsc{pst} sibot [ˈsi.bɔt]) tłumaczyć, wyjaśniać}

\dictword{sichemi}[ˈsi.t͡ʂɛ.mi]
\dictterm{v}{(\textsc{pst} sichent [ˈsi.t͡ʂɛŋt]) wybaczać, przebaczać}

\dictword{sićheri}[si.ˈt͡ʂɛ.ri]
\dictterm{v}{(\textsc{pst} sićherit [si.ˈt͡ʂɛ.rit]) wyjeżdżać, odjeżdzać}

\dictword{sideti}[ˈsi.dɛ.ti]
\dictterm{v}{(\textsc{pst} sidet [ˈsi.dɛt]) boleć}

\dictword{sidowi}[ˈsi.dɔ.wi]
\dictterm{v}{(\textsc{pst} sidowet [ˈsi.dɔ.wɛt]) napełniać, wypełniać}

\dictword{sifriti}[ˈsi.fri.ti]
\dictterm{v}{gasić}

\dictword{sii}[ˈsi:]
\dictterm{part}{jeśli, skoro, jeżeli}

\dictword{siki}[ˈsi.ki]
\dictterm{v}{(\textsc{pst} sik [ˈsik]) myć}

\dictword{sikima}[ˈsi.ki.ma]
\dictterm{n}{(\textsc{pl} sikimiji [ˈsi.ki.mi.ʐi]) (\textsc{fem}) łazienka}

\dictword{simosti}[ˈsi.mɔs.ti]
\dictterm{v}{(\textsc{pst} simoset [ˈsi.mɔ.sɛt]) wytwarzać, utwarzać, wytworzyć, utworzyć, stwarzać, stworzyć}

\dictword{sindo}[ˈsin.dɔ]
\dictterm{adj}{(\textsc{comp} sinde [ˈsin.dɛ], \textsc{supl} sindem [ˈsin.dɛm]) powolny, niemrawy, niezdarny}

\dictword{sini}[ˈsi.ni]
\dictterm{v}{wychodzić}

\dictword{siniko}[ˈsi.ni.kɔ]
\dictterm{adj}{(\textsc{comp} siniko͞e [ˈsi.ni.kɔɛ], \textsc{supl} siniko͞em [ˈsi.ni.kɔɛm]) wspaniały}

\dictword{sipalima}[ˈsi.pa.li.ma]
\dictterm{n}{(\textsc{pl} sipalimji [ˈsi.pa.lim.ʐi]) (\textsc{fem}) opowieść, opowiadanie}

\dictword{sipalimi}[ˈsi.pa.li.mi]
\dictterm{v}{opowiadać}

\dictword{sitora}[ˈsi.tɔ.ra]
\dictterm{n}{(\textsc{pl} sitorji [ˈsi.tɔr.ʐi]) (\textsc{fem}) wyrzutnia}

\dictword{siro}[ˈsi.rɔ]
\dictterm{adj}{bezproblemowy, jasny}

\dictword{siro}[ˈsi.rɔ]
\dictterm{adj}{biały}

\dictword{siro}[ˈsi.rɔ]
\dictterm{n}{biel}

\dictword{siryazo}[ˈsir.ja.zɔ]
\dictterm{n}{(\textsc{pl} siryazos [ˈsir.ja.zɔs]) światłość, jasność}

\dictword{sisuchi}[ˈsi.su.t͡ʂi]
\dictterm{v}{(\textsc{pst} sisuchit [ˈsi.su.t͡ʂit]) wyłączać}

\dictword{siteli}[ˈsi.tɛ.li]
\dictterm{v}{wycierać}

\dictword{situvi}[ˈsi.tu.vi]
\dictterm{v}{(\textsc{pst} situvet [ˈsi.tu.vɛt]) zabierać}

\dictword{sivachi}[ˈsi.va.t͡ʂi]
\dictterm{v}{(\textsc{pst} sivachit [ˈsi.va.t͡ʂit]) wydrapać}

\dictword{siyurcha}[ˈsi.jur.t͡ʂa]
\dictterm{n}{(\textsc{pl} siyurchos [ˈsi.jur.t͡ʂɔs]) (\textsc{fem}) wysyłka, transmisja wychodząca}

\dictword{siyurchi}[ˈsi.jur.t͡ʂi]
\dictterm{v}{(\textsc{pst} siyurchet [ˈsi.jur.t͡ʂɛt]) wysyłać}

\dictword{so}[ˈsɔ]
\dictterm{part}{co}

\dictword{soel}[ˈsɔ.ɛl]
\dictterm{n}{(\textsc{pl} soelji) (\textsc{fem} soela) wół, krowa, byk}

\dictword{soelyirome}[ˈsɔ.ɛ.lʏ.rɔ.mɛ]
\dictterm{n}{(\textsc{pl} soelyiromes) wołowina}

\dictword{soki}[ˈsɔ.ki]
\dictterm{part}{więc, zatem}

\dictword{somak}[ˈsɔ.mak]
\dictterm{part}{gdzieś}

\dictword{somar}[ˈsɔ.mar]
\dictterm{part}{gdzie}

\dictword{sormi}[ˈsɔr.mi]
\dictterm{v}{(\textsc{pst} sormit [ˈsɔr.mit]) wschodzić (o słońcu)}

\dictword{sormo}[ˈsɔr.mɔ]
\dictterm{n}{(\textsc{pl} sormiji [ˈsɔr.mi.ʐi]) wschód (kierunek), wschód słońca}

\dictword{sosbet}[ˈsɔs.bɛt]
\dictterm{n}{(\textsc{pl} sosbetos [ˈsɔs.bɛ.tɔs]) (\textsc{fem} sosbeta [ˈsɔs.bɛ.ta]) podejrzany}

\dictword{sotak}[ˈsɔ.tak]
\dictterm{part}{ktoś}

\dictword{soter}[ˈsɔ.tɛr]
\dictterm{part}{kto}

\dictword{stoamjor}[ˈstɔ.am.ʐɔr]
\dictterm{n}{(\textsc{pl} stoamji [ˈstɔ.am.ʐi]) program (komputerowy), algorytm, przepis}

\dictword{stobo}[ˈstɔ.bɔ]
\dictterm{adj}{szary}

\dictword{stobo}[ˈstɔ.bɔ]
\dictterm{n}{szarość}

\dictword{stopi}[ˈstɔ.pi]
\dictterm{v}{(\textsc{pst} stopet [ˈstɔ.pɛt]) zatrzymywać}

\dictword{suasi}[ˈsu.a.si]
\dictterm{v}{(\textsc{pst} suat [ˈsu.at]) ssać}

\dictword{sucha}[ˈsu.t͡ʂa]
\dictterm{n}{(\textsc{pl} suchiji [ˈsu.t͡ʂi.ʐi]) (\textsc{fem}) praca}

\dictword{suchi}[ˈsu.t͡ʂi]
\dictterm{v}{(\textsc{pst} suchit [ˈsu.t͡ʂit]) włączać}

\dictword{sul}[ˈsul]
\dictterm{n}{sól}

\dictword{sulga}[ˈsul.ga]
\dictterm{n}{(\textsc{pl} sulgi [ˈsul.gi]) (\textsc{fem}) pomoc}

\dictword{sulgi}[ˈsul.gi]
\dictterm{v}{(\textsc{pst} sulgit [ˈsul.git]) pomagać}

\dictword{supirit}[ˈsu.pi.rit]
\dictterm{n}{(\textsc{pl} supiritos [ˈsu.pi.ri.tɔs]) zmiana}

\dictword{supiriti}[ˈsu.pi.ri.ti]
\dictterm{v}{(\textsc{pst} supirit [ˈsu.pi.rit]) zmieniać}

\dictword{supunto}[ˈsu.pun.tɔ]
\dictterm{n}{(\textsc{pl} supuntos [ˈsu.pun.tɔs]) kopia (czegoś)}

\dictword{suyer}[ˈsu.jɛr]
\dictterm{n}{(\textsc{pl} suyeros [ˈsu.jɛ.rɔs]) (\textsc{fem} suyera [ˈsu.jɛ.ra]) marynarz}

\dictword{suyerot}[ˈsu.jɛ.rɔt]
\dictterm{n}{(\textsc{pl} suyerotos [ˈsu.jɛ.rɔ.tɔs]) statek}

\end{multicols}
\newpage
\section{T}
\begin{multicols}{2}

\dictword{ta}[ˈta]
\dictterm{n}{cztery, czwórka}

\dictword{taerno}[ˈta.ɛr.nɔ]
\dictterm{adj}{zadowolony}

\dictword{tager}[ˈta.gɛr]
\dictterm{n}{(\textsc{pl} tagerji [ˈta.gɛr.ʐi])  oszustwo}

\dictword{tagerer}[ˈta.gɛ.rɛr]
\dictterm{n}{(\textsc{pl} tagereros [ˈta.gɛ.rɛ.rɔs]) (\textsc{fem} tagerera [ˈta.gɛ.rɛ.ra]) oszust}

\dictword{tageri}[ˈta.gɛ.ri]
\dictterm{v}{(\textsc{pst} tagerit [ˈta.gɛ.rit]) oszukiwać}

\dictword{takari}[ˈta.ka.ri]
\dictterm{v}{(\textsc{pst} takarit [ˈta.ka.rit]) pojawiać się}

\dictword{tamoki}[ˈta.mɔ.ki]
\dictterm{v}{(\textsc{pst} tamok [ˈta.mɔk]) odwiedzać, wizytować}

\dictword{tamoko}[ˈta.mɔ.kɔ]
\dictterm{n}{odwiedziny, wizyta}

\dictword{tanoni}[ˈta.nɔ.ni]
\dictterm{v}{(\textsc{pst} tanot [ˈta.nɔt]) zarabiać}

\dictword{tar}[ˈtar]
\dictterm{part}{zbyt}

\dictword{taris}[ˈta.ris]
\dictterm{n}{(\textsc{pl} taris [ˈta.ris]) ostrze}

\dictword{tasek}[ˈta.sɛk]
\dictterm{n}{(\textsc{pl} taskos [ˈtas.kɔs]) stół}

\dictword{tati}[ˈta.ti]
\dictterm{adj}{czwarty}

\dictword{taúnin}[ta.ˈu.nin]
\dictterm{n}{(\textsc{pl} taúninos [ta.ˈu.ni.nɔs]) plaster (mięsa), kromka (chleba), kawałek ciasta, część, fragment}

\dictword{tay}[ˈtaj]
\dictterm{n}{(\textsc{pl} tajis [ˈta.ʐis]) (\textsc{fem}) dzień, data}

\dictword{tayeti}[ˈta.jɛ.ti]
\dictterm{v}{(\textsc{pst} tayet [ˈta.jɛt]) zapominać}

\dictword{tayizigen}[ˈta.ʏ.zi.gɛn]
\dictterm{n}{(\textsc{pl} tajizigenji [ˈta.ʐi.zi.gɛn.ʐi]) kalendarz}
\note{Dosłownie: zbiór dni.}

\dictword{tebatte}[ˈtɛ.bat.tɛ]
\dictterm{n}{(\textsc{fem}) powinność}
\note{Może być używane również jako potwierdzenie: ,,tak, powinieneś to zrobić''.}

\dictword{tebumar}[ˈtɛ.bu.mar]
\dictterm{n}{(\textsc{pl} tebumaros [ˈtɛ.bu.ma.rɔs]) kod, szyfr}

\dictword{tedima}[ˈtɛ.di.ma]
\dictterm{n}{(\textsc{pl} tedimji [ˈtɛ.dim.ʐi]) (\textsc{fem}) pokusa}

\dictword{tekyu}[ˈtɛk.ju]
\dictterm{n}{technologia}

\dictword{teli}[ˈtɛ.li]
\dictterm{v}{(\textsc{pst} telet [ˈtɛ.lɛt]) trzeć}

\dictword{tepa}[ˈtɛ.pa]
\dictterm{adj}{ciepły}

\dictword{tepa}[ˈtɛ.pa]
\dictterm{n}{(\textsc{fem}) ciepło}

\dictword{tesiko}[ˈtɛ.si.kɔ]
\dictterm{adj}{wspólny}
\note{Także: razem.}

\dictword{teviti}[ˈtɛ.vi.ti]
\dictterm{v}{(\textsc{pst} tevit [ˈtɛ.vit]) chcieć}

\dictword{teyamlara}[ˈtɛ.jam.la.ra]
\dictterm{n}{(\textsc{pl} teyamlarji [ˈtɛ.jam.lar.ʐi]) (\textsc{fem}) harmonogram}

\dictword{ti}[ˈti]
\dictterm{pro}{ty}

\dictword{ti͞ocho}[ˈtiɔ.t͡ʂɔ]
\dictterm{adj}{świeży}

\dictword{tira}[ˈti.ra]
\dictterm{n}{(\textsc{pl} tirji [ˈtir.ʐi]) (\textsc{fem}) łza}

\dictword{tiri}[ˈti.ri]
\dictterm{v}{(\textsc{pst} tirit [ˈti.rit]) płakać}

\dictword{tochoi}[ˈtɔ.t͡ʂɔ.i]
\dictterm{v}{(\textsc{pst} tochot [ˈtɔ.t͡ʂɔt]) dźgać}

\dictword{toi}[ˈtɔ.i]
\dictterm{pro}{wy}

\dictword{toleze}[ˈtɔ.lɛ.zɛ]
\dictterm{n}{(\textsc{pl} tolezeji [ˈtɔ.lɛ.zɛ.ʐi]) wysiłek}

\dictword{tope}[ˈtɔ.pɛ]
\dictterm{n}{(\textsc{pl} topeji [ˈtɔ.pɛ.ʐi]) koniec}

\dictword{topi}[ˈtɔ.pi]
\dictterm{v}{(\textsc{pst} topet [ˈtɔ.pɛt]) kończyć}

\dictword{tori}[ˈtɔ.ri]
\dictterm{v}{(\textsc{pst} toret [ˈtɔ.rɛt]) rzucać}

\dictword{tosaper}[ˈtɔ.sa.pɛr]
\dictterm{n}{(\textsc{pl} tosaperji [ˈtɔ.sa.pɛr.ʐi]) (\textsc{fem} tosapera [ˈtɔ.sa.pɛ.ra]) oficer}

\dictword{tot́ukcha}[tɔ.ˈtuk.t͡ʂa]
\dictterm{n}{(\textsc{pl} tot́ukji [tɔ.ˈtuk.ʐi]) (\textsc{fem}) proch, popiół}

\dictword{trakon}[ˈtra.kɔn]
\dictterm{n}{(\textsc{pl} trakones [ˈtra.kɔ.nɛs]) (\textsc{fem} trakona [ˈtra.kɔ.na]) smok}
\note{W dialekcie zachodnim do dziś popularniejsze jest słowo \emph{kahobeykar}, dosłownie ,,latający wąż''.}

\dictword{tramo}[ˈtra.mɔ]
\dictterm{n}{(\textsc{pl} tramos [ˈtra.mɔs]) zachód (kierunek), zmierzch}

\dictword{tramo͞a}[ˈtra.mɔa]
\dictterm{adj}{zachodni}

\dictword{tranki}[ˈtran.ki]
\dictterm{v}{(\textsc{pst} tranek [ˈtra.nɛk]) uspokajać}

\dictword{tranok}[ˈtra.nɔk]
\dictterm{n}{pokój, spokój}

\dictword{tuichir}[ˈtu.i.t͡ʂir]
\dictterm{n}{(\textsc{pl} tuichiros [ˈtu.i.t͡ʂi.rɔs]) obraz}

\dictword{tutege}[ˈtu.tɛ.gɛ]
\dictterm{n}{(\textsc{pl} tuteges [ˈtu.tɛ.gɛs]) sport}

\dictword{tuteger}[ˈtu.tɛ.gɛr]
\dictterm{n}{(\textsc{pl} tutegeros [ˈtu.tɛ.gɛ.rɔs]) (\textsc{fem} tutegera [ˈtu.tɛ.gɛ.ra]) sportowiec}

\dictword{tuvi}[ˈtu.vi]
\dictterm{v}{(\textsc{pst} tuvit [ˈtu.vit]) brać}

\dictword{tyi}[ˈtʏ]
\dictterm{pro}{twój (\textsc{2SG.GEN} lub \textsc{2SG.POSS})}

\dictword{tyoi}[ˈtjɔ.i]
\dictterm{pro}{wasze (\textsc{2PL.GEN} lub \textsc{2PL.POSS})}

\end{multicols}
\newpage
\section{U}
\begin{multicols}{2}

\dictword{uf́riti}[u.ˈfri.ti]
\dictterm{v}{(\textsc{pst} uf́rit [u.ˈfrit]) palić, spalić, wypalić}

\dictword{uḱachisi}[u.ˈka.t͡ʂi.si]
\dictterm{v}{uczyć się}

\dictword{ultik}[ˈul.tik]
\dictterm{adj}{ostatni}

\dictword{un}[ˈun]
\dictterm{part}{pod}

\dictword{uńakaro}[u.ˈna.ka.rɔ]
\dictterm{adj}{podczerwony}

\dictword{uńakaro}[u.ˈna.ka.rɔ]
\dictterm{n}{podczerwień}

\dictword{unarel}[ˈu.na.rɛl]
\dictterm{n}{(\textsc{pl} unarelji [ˈu.na.rɛl.ʐi] uścisk}

\dictword{uńarli}[u.ˈnar.li]
\dictterm{v}{(\textsc{pst} uńarlit [u.ˈnar.lit]) uściskać}

\dictword{uńatali}[u.ˈna.ta.li]
\dictterm{v}{(\textsc{pst} uńatal [u.ˈna.tal]) urodzić (się)}

\dictword{unatalit́ay}[u.na.ta.li.ˈtaj]
\dictterm{n}{(\textsc{pl} unatalit́aji [ˈu.na.ta.li.ˈta.ʐi]) urodziny}

\dictword{uneńi}[u.nɛ.ˈni]
\dictterm{v}{(\textsc{pst} uneńit [u.nɛ.ˈnit]) odchodzić, uchodzić}

\dictword{uneńit}[u.nɛ.ˈnit]
\dictterm{n}{(\textsc{pl} uneńitji [u.nɛ.ˈnit.ʐi]) podejście}

\dictword{uneya}[ˈu.nɛ.ja]
\dictterm{n}{(\textsc{pl} uneyji [ˈu.nɛj.ʐi]) (\textsc{fem}) informacja, dana}

\dictword{uneyatao}[ˈu.nɛ.ja.ta.ɔ]
\dictterm{n}{dochodzenie, zbieranie informacji}

\dictword{unuracho}[ˈu.nu.ra.t͡ʂɔ]
\dictterm{adj}{niespodzie\-wany}

\dictword{uśeysi}[u.ˈsɛj.si]
\dictterm{v}{usiąść}

\dictword{ut́opi}[u.ˈtɔ.pi]
\dictterm{v}{(\textsc{pst} ut́opet [u.ˈtɔ.pɛt]) ukończyć}

\dictword{uýubi}[u.ˈju.bi]
\dictterm{v}{(\textsc{pst} uýut [u.ˈjut]) ogłuchnąć, stracić słuch, ale również: 
\emph{ge uýut} -- zostać uciszonym}

\end{multicols}
\newpage
\section{V}
\begin{multicols}{2}

\dictword{va}[ˈva]
\dictterm{part}{partykuła -- operator aspektu dokonanego}

\dictword{vachei}[ˈva.t͡ʂɛ.i]
\dictterm{v}{(\textsc{pst} vachet [ˈva.t͡ʂɛt]) ranić}

\dictword{vacheta}[ˈva.t͡ʂɛ.ta]
\dictterm{n}{(\textsc{pl} vachetji [ˈva.t͡ʂɛt.ʐi]) (\textsc{fem}) rana, zadrapanie}

\dictword{vachi}[ˈva.t͡ʂi]
\dictterm{v}{(\textsc{pst} vachit [ˈva.t͡ʂit]) drapać}

\dictword{vacho}[ˈva.t͡ʂɔ]
\dictterm{adj}{ranny}

\dictword{vagafra}[ˈva.ga.fra]
\dictterm{n}{(\textsc{pl} vagafraji [ˈva.ga.fra.ʐi]) (\textsc{fem}) narzędzie}

\dictword{vage}[ˈva.gɛ]
\dictterm{part}{partykuła wyrażajaca operator trybu ekshortatywnego, zachęcającego do wykonania czynności }

\dictword{vahur}[ˈva.xur]
\dictterm{n}{skóra}

\dictword{vahuryiáysi}[va.xu.rʏ.ˈaj.si]
\dictterm{n}{(\textsc{pl} vahuryiaysiji [ˈva.xur.ji.aj.si.ʐi]) tatuaż}
\note{Dosłownie: skórne znamię.}

\dictword{vaja}[ˈva.ʐa]
\dictterm{part}{czy}

\dictword{vajati}[ˈva.ʐa.ti]
\dictterm{v}{(\textsc{pst} vajat [ˈva.ʐat]) narzekać; \emph{vajati on} -- narzekać na}

\dictword{vajar}[ˈva.ʐar]
\dictterm{part}{a może, czy może jednak}

\dictword{vajikoy}[ˈva.ʐi.kɔj]
\dictterm{part}{czyjkolwiek}

\dictword{val}[ˈval]
\dictterm{n}{(\textsc{pl} valos) (\textsc{fem} vala) syn, córka}
\note{Dawniej używane było również jako partykuła w imionach: \emph{Keyer val 
Dorigan} - Keyer, syn Dorigana. Ciekawostką jest to, że forma ta wyglądała 
identycznie dla kobiet - \emph{Keyla val Dorigan}. Używana było tylko jeśli
rodzic (najczęściej ojciec) był powszechnie znaną osobistością.}

\dictword{valor}[ˈva.lɔr]
\dictterm{adj}{(\textsc{comp} valora [ˈva.lɔ.ra], \textsc{supl} valoram [ˈva.lɔ.ram]) silny}

\dictword{vapal}[ˈva.pal]
\dictterm{n}{(\textsc{pl} vapalos [ˈva.pa.lɔs]) ojciec}
\note{Bardziej formalne niż \emph{patal}.}

\dictword{valor}[ˈva.lɔr]
\dictterm{n}{(\textsc{pl} valoros [ˈva.lɔ.rɔs]) zbroja}

\dictword{varsi}[ˈvar.si]
\dictterm{v}{(\textsc{pst} vajat [ˈva.ʐat]) wstawać, wstać, stać, stawać}

\dictword{vati}[ˈva.ti]
\dictterm{v}{(\textsc{pst} vat [ˈvat]) patrzeć}

\dictword{vay}[ˈvaj]
\dictterm{part}{partykuła wzmacniająca pytania ,,czy''}

\dictword{vayara}[ˈva.ja.ra]
\dictterm{n}{(\textsc{pl} vayarji [ˈva.jar.ʐi]) (\textsc{fem}) podróż}

\dictword{vayaren}[ˈva.ja.rɛn]
\dictterm{n}{(\textsc{pl} vayarenjos [ˈva.ja.rɛn.ʐɔs]) (\textsc{fem} vayarena [ˈva.ja.rɛ.na]) podróżnik}

\dictword{vayareni}[ˈva.ja.rɛ.ni]
\dictterm{v}{podróżować}

\dictword{vayaro}[ˈva.ja.rɔ]
\dictterm{adj}{podróżujący}
\note{Bardziej popularna forma, ale w~dialekcie pustynnym pojawia się określenie
\emph{vyage}.}

\dictword{vayi}[ˈva.ji]
\dictterm{part}{czyjś}

\dictword{vechiret}[ˈvɛ.t͡ʂi.rɛt]
\dictterm{adj}{(\textsc{comp} vechirede [ˈvɛ.t͡ʂi.rɛ.dɛ], \textsc{supl} vechiredem [ˈvɛ.t͡ʂi.rɛ.dɛm]) głupi, głupszy, najgłupszy}

\dictword{vechireder}[ˈvɛ.t͡ʂi.rɛ.dɛr]
\dictterm{n}{(\textsc{pl} vechiredeji [ˈvɛ.t͡ʂi.rɛ.dɛ.ʐi]) (\textsc{fem} vechiredera [ˈvɛ.t͡ʂi.rɛ.dɛ.ra]) głupiec, głupek}

\dictword{vechiredo}[ˈvɛ.t͡ʂi.rɛ.dɔ]
\dictterm{n}{głupota}

\dictword{veder}[ˈvɛ.dɛr]
\dictterm{n}{(\textsc{pl} vederos [ˈvɛ.dɛ.rɔs]) (\textsc{fem} vedera [ˈvɛ.dɛ.ra]) rzemieślnik (dosł. zginacz)}

\dictword{vedi}[ˈvɛ.di]
\dictterm{v}{giąć, zginać}

\dictword{vejo}[ˈvɛ.ʐɔ]
\dictterm{adj}{niebieski, granatowy}

\dictword{vejo}[ˈvɛ.ʐɔ]
\dictterm{n}{kolor niebieski, granatowy}

\dictword{vek}[ˈvɛk]
\dictterm{adj}{(\textsc{comp} vekka [ˈvɛk.ka], \textsc{supl} vyekka [ˈvjɛk.ka]) stary}

\dictword{vek}[ˈvɛk]
\dictterm{n}{starość}

\dictword{verayo}[ˈvɛ.ra.jɔ]
\dictterm{adj}{zaszczytny, zaszczycony}

\dictword{vetfar}[ˈvɛt.far]
\dictterm{n}{kunszt, sztuka}

\dictword{veychali}[ˈvɛj.t͡ʂa.li]
\dictterm{v}{(\textsc{pst} veychalit [ˈvɛj.t͡ʂa.lit]) obserwować}

\dictword{veychali͞a}[ˈvɛj.t͡ʂa.lia]
\dictterm{n}{(\textsc{pl} veychalyes [ˈvɛj.t͡ʂal.jɛs]) (\textsc{fem}) obserwacja, sztuka obserwacji, wynik obserwacji}

\dictword{veydi}[ˈvɛj.di]
\dictterm{v}{(\textsc{pst} veyet [ˈvɛ.jɛt]) widzieć, zobaczyć}

\dictword{veydili}[ˈvɛj.di.li]
\dictterm{v}{oglądać}

\dictword{veydili}[ˈvɛj.di.li]
\dictterm{v}{(\textsc{pst} veydit [ˈvɛj.dit]) oglądać}

\dictword{veydilo}[ˈvɛj.di.lɔ]
\dictterm{adj}{(\textsc{comp} veydilo͞e [ˈvɛj.di.lɔɛ], \textsc{supl} veydilo͞em [ˈvɛj.di.lɔɛm]) widoczny, widzialny}

\dictword{veyos}[ˈvɛj.ɔs]
\dictterm{n}{wzrok}

\dictword{veyrali}[ˈvɛj.ra.li]
\dictterm{v}{(\textsc{pst} veyralit [ˈvɛj.ra.lit]) gapić się}

\dictword{veysamichi}[ˈvɛj.sa.mi.t͡ʂi]
\dictterm{v}{(\textsc{pst} veysamichit [ˈvɛj.sa.mi.t͡ʂit]) oceniać wzrokiem}

\dictword{vi}[ˈvi]
\dictterm{part}{partykuła -- operator aspektu niedokonanego}

\dictword{vibi}[ˈvi.bi]
\dictterm{v}{jeść}

\dictword{vibo}[ˈvi.bɔ]
\dictterm{n}{jedzenie}

\dictword{vicho}[ˈvi.t͡ʂɔ]
\dictterm{adj}{częsty}

\dictword{vige}[ˈvi.gɛ]
\dictterm{part}{partykuła -- formalny operator ekshortatywny ,,niech się stanie''}
\note{Istnieje szereg idiomów używanych wraz z~operatorem \emph{vige}:
\emph{vige choke tay} -- miłego dnia, \emph{vige menede} -- na zdrowie,
\emph{vige sekupa vayara} -- bezpiecznej podróży (dosłownie tłumaczy się na: 
,,niech twoja podróż będzie bezpieczna'').}

\dictword{viki}[ˈvi.ki]
\dictterm{v}{(\textsc{pst} vik [ˈvik]) ciąć}

\dictword{vimi}[ˈvi.mi]
\dictterm{part}{partykuła -- operator trybu przypuszczającego}

\dictword{vinal}[ˈvi.nal]
\dictterm{n}{chleb}

\dictword{vipetode}[ˈvi.pɛ.tɔ.dɛ]
\dictterm{n}{(\textsc{pl} vipetodes [ˈvi.pɛ.tɔ.dɛs]) miska}
\note{Dosłownie: pojemnik na jedzenie.}

\dictword{vipini}[ˈvi.pi.ni]
\dictterm{v}{musieć, zmuszać, zmusić, przymusić}

\dictword{vir}[ˈvir]
\dictterm{adj}{(\textsc{comp} virre [ˈvir.rɛ], \textsc{supl} virrem [ˈvir.rɛm]) mały}

\dictword{viran}[ˈvi.ran]
\dictterm{n}{(\textsc{pl} viranos) groch, groszek}

\dictword{vireja}[ˈvi.rɛ.ʐa]
\dictterm{n}{(\textsc{pl} vireji [ˈvi.rɛ.ʐi]) gwiazda}

\dictword{viri}[ˈvi.ri]
\dictterm{part}{trochę}

\dictword{vitokar}[ˈvi.tɔ.kar]
\dictterm{n}{(\textsc{pl} vitokaros [ˈvi.tɔ.ka.rɔs]) tarcza, pancerz, osłona}

\dictword{vitokaro}[ˈvi.tɔ.ka.rɔ]
\dictterm{adj}{pancerny, opancerzony}

\dictword{voese}[ˈvɔ.ɛ.sɛ]
\dictterm{adj}{spragniony (domyślnie: napoju, wody), ale również może być 
spragniony czegokolwiek innego -- używa się tutaj partykuły \emph{chu}, 
np. \emph{voese chu koóln} -- spragniony miłości}

\dictword{voesi}[ˈvɔ.ɛ.si]
\dictterm{v}{(\textsc{pst} voesit [ˈvɔ.ɛ.sit]) pragnąć}

\dictword{vojibo}[ˈvɔ.ʐi.bɔ]
\dictterm{adj}{przeciwny}

\dictword{vojiboge}[ˈvɔ.ʐi.bɔ.gɛ]
\dictterm{part}{przeciw (komuś)}

\dictword{voli}[ˈvɔ.li]
\dictterm{part}{kiedy; \emph{o voli} -- do kiedy}

\dictword{volu}[ˈvɔ.lu]
\dictterm{part}{kiedyś}

\dictword{volumo}[ˈvɔ.lu.mɔ]
\dictterm{part}{kiedykolwiek}

\dictword{votepo}[ˈvɔ.tɛ.pɔ]
\dictterm{adj}{(\textsc{comp} votepo͞e [ˈvɔ.tɛ.pɔɛ], \textsc{supl} votepo͞em [ˈvɔ.tɛ.pɔɛm]) gorący}

\dictword{vyage}[ˈvja.gɛ]
\dictterm{adj}{podróżujący}
\note{Raczej używane tylko w~dialekcie pustynnym. Patrz \emph{vayaro}}.

\dictword{vyajeto}[ˈvju.ʐɛ.tɔ]
\dictterm{adj}{osłupiały}

\dictword{vyele}[ˈvjɛ.lɛ]
\dictterm{adj}{(\textsc{comp} vyelo͞e [ˈvjɛ.lɔɛ], \textsc{supl} vyelo͞em [ˈvjɛ.lɔɛm]) wiele, dużo, więcej, najwięcej}

\end{multicols}
\newpage
\section{W}
\begin{multicols}{2}

\dictword{wa}[ˈwa]
\dictterm{n}{zero, pustka, nicość, brak}

\dictword{wa͞ime}[ˈwai.mɛ]
\dictterm{adj}{(\textsc{comp} wa͞ime͞a [ˈwai.mɛa], \textsc{supl} wa͞ime͞am [ˈwai.mɛam]) szeroki}

\dictword{waleva}[ˈwa.lɛ.va]
\dictterm{n}{(\textsc{pl} waleveji [ˈwa.lɛ.vɛ.ʐi]) (\textsc{fem}) obietnica}

\dictword{walevi}[ˈwa.lɛ.vi]
\dictterm{v}{(\textsc{pst} walevet [ˈwa.lɛ.vɛt]) obiecywać}

\dictword{waril}[ˈwa.ril]
\dictterm{adj}{(\textsc{comp} warile͞a [ˈwa.ri.lɛa], \textsc{supl} warile͞am [ˈwa.ri.lɛam]) długi}

\dictword{warila}[ˈwa.ri.la]
\dictterm{n}{(\textsc{fem}) długość}

\dictword{warilo}[ˈwa.ri.lɔ]
\dictterm{adj}{wzdłuż}

\dictword{wesazi}[ˈwɛ.sa.zi]
\dictterm{v}{(\textsc{pst} wesazit [ˈwɛ.sa.zit]) stawać się, zmieniać się}

\dictword{wesi}[ˈwɛ.si]
\dictterm{v}{(\textsc{pst} weset [ˈwɛ.sɛt]) błyszczeć}

\dictword{wiéni}[wi.ˈɛ.ni]
\dictterm{v}{(\textsc{pst} wiént [wi.ˈɛŋt]) podchodzić}

\dictword{wifori}[ˈwi.fɔ.ri]
\dictterm{v}{(\textsc{pst} wiforit [ˈwi.fɔ.rit]) kosztować}

\dictword{wiforo}[ˈwi.fɔ.rɔ]
\dictterm{adj}{(\textsc{comp} wiforo͞e [ˈwi.fɔ.rɔɛ], \textsc{supl} wiforo͞em [ˈwi.fɔ.rɔɛm]) drogi, kosztowny}

\dictword{wiḱelsani}[wi.ˈkɛl.sa.ni]
\dictterm{v}{(\textsc{pst} wiḱelsant [wi.ˈkɛl.saŋt]) podporządkować się}

\dictword{wirklo}[ˈwir.klɔ]
\dictterm{adj}{ważny}

\dictword{wo͞an}[ˈwɔ.an]
\dictterm{n}{(\textsc{pl} woaji [ˈwɔ.a.ʐi]) pióro}

\dictword{wochu}[ˈwɔ.t͡ʂu]
\dictterm{part}{który? która?}

\dictword{wodo}[ˈwɔ.dɔ]
\dictterm{part}{którędy}

\dictword{wofo}[ˈwɔ.fɔ]
\dictterm{n}{(\textsc{pl} wofiji [ˈwɔ.fi.ʐi]) kolor}
\note{Przestarzałe. Częściej używa się \emph{pukes}.}

\dictword{wora}[ˈwɔ.ra]
\dictterm{n}{(\textsc{pl} worji [ˈwɔr.ʐi]) (\textsc{fem}) wojna}

\dictword{worar}[ˈwɔ.rar]
\dictterm{n}{(\textsc{pl} worjes [ˈwɔr.ʐɛs]) (\textsc{fem} worara [ˈwɔ.ra.ra]) wojownik}

\dictword{woru͞a}[ˈwɔ.rua]
\dictterm{adj}{wojenny, wojskowy}

\dictword{wozidi}[ˈwɔ.zi.di]
\dictterm{n}{przestrzeń}

\end{multicols}
\newpage
\section{Y}
\begin{multicols}{2}

\dictword{ya}[ˈja]
\dictterm{n}{osiem, ósemka}

\dictword{yabor}[ˈja.bɔr]
\dictterm{n}{nadgarstek}

\dictword{yage}[ˈja.gɛ]
\dictterm{part}{dokąd}

\dictword{yalito}[ˈja.li.tɔ]
\dictterm{n}{(\textsc{pl} yalitos [ˈja.li.tɔs]) peron, terminal lotniskowy}

\dictword{yar}[ˈjar]
\dictterm{n}{(\textsc{pl} yarji [ˈjar.ʐi]) rok}

\dictword{yasa͞o}[ˈja.saɔ]
\dictterm{n}{(\textsc{pl} yasaji [ˈja.sa.ʐi]) węgorz}

\dictword{yasu}[ˈja.su]
\dictterm{part}{skąd}

\dictword{yati}[ˈja.ti]
\dictterm{adj}{ósmy}

\dictword{ye}[ˈjɛ]
\dictterm{part}{tak, tja (nieformalne)}

\dictword{yeécho}[jɛ.ˈɛ.t͡ʂɔ]
\dictterm{part}{oczywiście}

\dictword{yen}[ˈjɛn]
\dictterm{part}{lub}

\dictword{yeitto}[ˈjɛ.it.tɔ]
\dictterm{n}{(\textsc{pl} yeitji [ˈjɛ.it.ʐi]) historia, historyjka}

\dictword{yeki}[ˈjɛ.ki]
\dictterm{v}{szyć}

\dictword{yeloya}[ˈjɛ.lɔ.ja]
\dictterm{n}{(\textsc{pl} yeloji [ˈjɛ.lɔ.ʐi]) (\textsc{fem}) lina}

\dictword{yi}[ˈʏ]
\dictterm{pro}{przyimek dzierżawczy}

\dictword{yis}[ˈʏs]
\dictterm{part}{tak}

\dictword{yobar}[ˈjɔ.bar]
\dictterm{n}{(\textsc{pl} yobarji [ˈjɔ.bar.ʐi]) patyk, drewienko, chrust}

\dictword{yoker}[ˈjɔ.kɛt]
\dictterm{n}{(\textsc{pl} yoketos [ˈjɔ.kɛ.tɔs]) guzik, przycisk, klawisz}

\dictword{yoni}[ˈjɔ.ni]
\dictterm{v}{(\textsc{pst} yon [ˈjɔn]) odwracać}

\dictword{yoya}[ˈjɔ.ja]
\dictterm{adj}{(\textsc{comp} yoye͞a [ˈjɔ.jɛa], \textsc{supl} yoye͞am [ˈjɔ.jɛam]) późny, późno}

\dictword{yube}[ˈju.bɛ]
\dictterm{n}{cisza}

\dictword{yubi}[ˈju.bi]
\dictterm{v}{(\textsc{pst} yut [ˈjut]) ściszać, uciszać}

\dictword{yurcher}[ˈjur.t͡ʂɛr]
\dictterm{n}{(\textsc{pl} yurcheros [ˈjur.t͡ʂɛ.rɔs]) list pocztowy, list elektroniczny, e-mail}

\dictword{yurchera}[ˈjur.t͡ʂɛ.ra]
\dictterm{n}{ (\textsc{fem}) poczta}
\note{Oficjalnie: urząd pocztowy \emph{yurchero muner}.}

\dictword{yurchero}[ˈjur.t͡ʂɛ.rɔ]
\dictterm{adj}{pocztowy}

\dictword{yurcheromamer}[ˈjur.t͡ʂɛ.rɔ.ma.mɛr]
\dictterm{n}{(\textsc{pl} yurcheromameros [ˈjur.t͡ʂɛ.rɔ.ma.mɛ.rɔs]) (\textsc{fem} yurcheromamera [ˈjur.t͡ʂɛ.rɔ.ma.mɛ.ra]) listonosz}

\dictword{yurchi}[ˈjur.t͡ʂi]
\dictterm{v}{(\textsc{pst} yurchet [ˈjur.t͡ʂɛt]) słać, wysyłać (w szczególności pocztą)}

\end{multicols}
\newpage
\section{Z}
\begin{multicols}{2}

\dictword{zaálta}[za.ˈal.ta]
\dictterm{n}{(\textsc{pl} zaáltos [ˈza.al.tɔs]) (\textsc{fem}) lud, naród}

\dictword{zaámey}[za.ˈa.mɛj]
\dictterm{n}{(\textsc{pl} zaámeji [ˈza.ˈa.mɛ.ʐi]) (\textsc{fem} zaámeya [ˈza.a.mɛ.ja]) szmata (pejoratywnie), brudna rzecz}
\note{Używane jako pejoratywne określenie osoby, w~szczególności osoby, która
sypia z~wieloma innymi osobami. W~starożytności niewolnicy musieli używać
w stosunku do siebie określenia \emph{je zaámey} lub wręcz \emph{che zaámey}
zamiast zaimka \emph{mi}.}

\dictword{zage}[ˈza.gɛ]
\dictred{saperyer}

\dictword{zaibi}[ˈza.i.bi]
\dictterm{v}{(\textsc{pst} zaibit [ˈza.i.bit]) czyścić}

\dictword{zaimi}[ˈza.i.mi]
\dictterm{v}{(\textsc{pst} zaimit [ˈza.i.mit]) żegnać}
\note{Używane również jako pożegnanie, w~mowie potocznej skracane często do
\emph{zay} -- ,,pa'', ,,pa-pa''.}

\dictword{zajan}[ˈza.ʐan]
\dictterm{n}{(\textsc{pl} zajanji [ˈza.ʐan.ʐi]) bank}

\dictword{zay}[ˈzaj]
\dictterm{part}{żegnaj, pa, papa}
\note{Patrz również \emph{zaimi}.}

\dictword{ze}[ˈzɛ]
\dictterm{part}{partykuła -- operator czasu przyszłego}

\dictword{ze͞ijaki}[ˈzɛi.ʐa.ki]
\dictterm{adj}{wrażliwy, podatny na zranienie}

\dictword{zeku͞a}[ˈzɛ.kua]
\dictterm{n}{(\textsc{pl} zekuji [ˈzɛ.ku.ʐi]) (\textsc{fem}) jezioro}

\dictword{zetay}[ˈzɛ.taj]
\dictterm{part}{jutro}

\dictword{zeuna}[ˈzɛ.u.na]
\dictterm{n}{(\textsc{pl} zeunji [ˈzɛ.un.ʐi]) (\textsc{fem}) kosmos}

\dictword{zeuye}[ˈzɛ.u.jɛ]
\dictterm{adj}{(\textsc{comp} zeuye͞a [ˈzɛ.u.jɛa], \textsc{supl} zeuye͞am [ˈzɛ.u.jɛam]) gruby}

\dictword{zeuyerot}[ˈzɛ.u.jɛ.rɔt]
\dictterm{n}{(\textsc{pl} zeuyerotes [ˈzɛ.u.jɛ.rɔ.tɛs]) statek kosmiczny, pojazd kosmiczny}

\dictword{zevoyim}[ˈzɛ.vɔ.ʏm]
\dictterm{n}{(\textsc{pl} zevoyimos [ˈzɛ.vɔ.ji.mɔs]) poranek, ranek}

\dictword{zezana}[ˈzɛ.za.na]
\dictterm{n}{(\textsc{pl} zezanji [ˈzɛ.zan.ʐi]) (\textsc{fem}) pistolet}

\dictword{zigen}[ˈzi.gɛn]
\dictterm{n}{(\textsc{pl} zigenji [ˈzi.gɛn.ʐi]) zbiór}

\dictword{zihu}[ˈzi.xu]
\dictterm{n}{(\textsc{pl} zihus [ˈzi.xus]) owoc}

\dictword{zimeba}[ˈzi.mɛ.ba]
\dictterm{n}{(\textsc{pl} zimebiji [ˈzi.mɛ.bi.ʐi]) (\textsc{fem}) osłona}

\dictword{zimebi}[ˈzi.mɛ.bi]
\dictterm{v}{(\textsc{pst} zimebit [ˈzi.mɛ.bit]) ratować, chronić}

\dictword{zime͞osuper}[ˈzi.mɛɔ.su.pɛr]
\dictterm{n}{(\textsc{pl} zime͞osuperos [ˈzi.mɛɔ.su.pɛ.rɔs]) spadochron}

\dictword{zodichi}[ˈzɔ.di.t͡ʂi]
\dictterm{v}{(\textsc{pst} zodichit [ˈzɔ.di.t͡ʂit]) gwizdać}

\dictword{zir}[ˈzir]
\dictterm{n}{(\textsc{pl} zirji [ˈzir.ʐi]) (\textsc{fem}) jaskółka}

\dictword{zo͞ar}[ˈzɔar]
\dictterm{n}{(\textsc{pl} zo͞aros [ˈzɔa.rɔs]) ogon}

\dictword{zokem}[ˈzɔ.kɛm]
\dictterm{n}{(\textsc{pl} zokemos [ˈzɔ.kɛ.mɔs]) metal}

\dictword{zor}[ˈzɔr]
\dictterm{part}{albo}

\dictword{zoreche}[ˈzɔ.rɛ.t͡ʂɛ]
\dictterm{adj}{(\textsc{comp} zoreche͞a [ˈzɔ.rɛ.t͡ʂɛa], \textsc{supl} zoreche͞am [ˈzɔ.rɛ.t͡ʂɛam]) głodny}

\dictword{zosu}[ˈzɔ.su]
\dictterm{part}{bardzo}

\dictword{zosuvi}[ˈzɔ.su.vi]
\dictterm{part}{bardziej}

\end{multicols}